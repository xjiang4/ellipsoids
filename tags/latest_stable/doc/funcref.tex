\section{ellipsoid Methods}
{\Large {\tt dimension}} - returns the dimension of the space in which
the ellipsoid is defined and the rank of its shape matrix.

Parameters:
\begin{itemize}
\item {\tt E} - single ellipsoid or array of ellipsoids.
\end{itemize}

Returns:
\begin{itemize}
\item {\tt n} - dimension of the space.
\item {\tt r} - rank of the ellipsoid's shape matrix. This output parameter
is optional.
\end{itemize}

Example:
{\tt \begin{verbatim}
>> E1 = ellipsoid;
>> A = [3 1; 0 1; -2 1]; E2 = ellipsoid([1; -1; 1], A*A');
>> E3 = ellipsoid(eye(2));
>> E4 = ellipsoid(0);
>> [n, r] = dimension([E1 E2; E3 E4])

n =

   0     3
   2     1

r =

   0     2
   2     0
\end{verbatim} }

\newpage

{\Large {\tt display}} - displays the details of the ellipsoid object.

Parameters:
\begin{itemize}
\item {\tt E} - ellipsoid or array of ellipsoids.
\end{itemize}
This function is rarely used explicitely. It is called automatically
for the object {\tt E} when semicolon does not terminate the statement.

\newpage

{\Large {\tt distance}} - computes the distance from the given ellipsoid
to the specified object - vector, ellipsoid, hyperplane or polytope.

Parameters:
\begin{itemize}
\item {\tt E} - ellipsoid or array of ellipsoids.
\item {\tt X} - in case of vectors, it is a matrix, whose columns represent
the vectors to which the distance is measured, number of columns must match
the number of ellipsoids in the array {\tt E};\\
in case of ellipsoids, it is array of ellipsoids whose size must match the
size of {\tt E};\\
in case of hyperplanes, it is array of hyperplanes whose size must match the
size of {\tt E};\\
in case of polytopes, it is polytope array whose length must match
the number of ellipsoids in the array {\tt E}.
\item {\tt F} - this flag if set to $1$ indicates that the distance should
be computed in the metric of ellipsoids in {\tt E}. This parameter is optional,
its default value is $0$, and the distance is computed in the Euclidean metric.
This parameter makes no difference for the distance computation
between ellipsoids and polytopes.
\end{itemize}

Returns:
\begin{itemize}
\item {\tt D} - distance or array of distances.
\end{itemize}

Example:
{\tt \begin{verbatim}
>> E = ellipsoid([-2; -1], [4 -1; -1 1]);
>> V = [1 1; 1 -1; -1 1; -1 -1]';
>> distance(E, V)

ans =

     2.3428    1.0855    1.3799    -0.4402
\end{verbatim} }

\newpage

{\Large {\tt double}} - returns parameters of the ellipsoid, its center
and shape matrix.

Parameters:
\begin{itemize}
\item {\tt E} - single ellipsoid.
\end{itemize}

Returns:
\begin{itemize}
\item {\tt q} - center of the ellipsoid. It is optional output parameter.
\item {\tt Q} - shape matrix of the ellipsoid.
\end{itemize}

\newpage

{\Large {\tt ellintersection\_ia}} - computes maximum volume ellipsoid that
is contained in the intersection of given ellipsoids.

Parameters:
\begin{itemize}
\item {\tt EE} - array of given ellipsoids.
\end{itemize}

Returns:
\begin{itemize}
\item {\tt I} - resulting maximum volume internal ellipsoid.
\end{itemize}

\newpage

{\Large {\tt ellipsoid}} - constructor for the ellipsoid object.
If called without parameters, returns empty ellipsoid.

Parameters:
\begin{itemize}
\item {\tt q} - center of the ellipsoid, vector in ${\bf R}^n$.
This parameter is optional, and if omitted, it is assumed that the ellipsoid
is centered at the origin.
\item {\tt Q} - shape matrix of the ellipsoid, matrix in ${\bf R}^{n\times n}$,
symmetric positive semidefinite.
\end{itemize}
Returns:
\begin{itemize}
\item {\tt E} - object of type {\tt ellipsoid class}.
\end{itemize}
Example:
{\tt \begin{verbatim}
>> E = ellipsoid([1 0 -1 6]', 9*eye(4));
\end{verbatim} }
creates a ball of radius $3$ in ${\bf R}^4$ centered at $[1 ~ 0 ~ -1 ~ 6]^T$.

\newpage

{\Large {\tt ellunion\_ea}} - computes minimum volume ellipsoid that
contains the union of given ellipsoids.

Parameters:
\begin{itemize}
\item {\tt EE} - array of given ellipsoids.
\end{itemize}

Returns:
\begin{itemize}
\item {\tt U} - resulting minimum volume enclosing ellipsoid.
\end{itemize}

\newpage

{\Large {\tt eq}} - overloaded operator {\tt '=='},
it checks if two ellipsoids are equal.

Parameters:
\begin{itemize}
\item {\tt E1} - first ellipsoid.
\item {\tt E2} - second ellipsoid.
\end{itemize}

Returns $1$ if two ellipsoids are equal, $0$ - otherwise.

Example:
{\tt \begin{verbatim}
>> E = ellipsoid([-2; -1], [4 -1; -1 1]);
>> E == [E ellipsoid(eye(2))]

ans =

     1     0
\end{verbatim} }

\newpage

{\Large {\tt ge}}, {\Large {\tt gt}} - checks if the first ellipsoid is bigger
than the second one.

Parameters:
\begin{itemize}
\item {\tt E1} - first ellipsoid.
\item {\tt E2} - second ellipsoid.
\end{itemize}

Returns $1$ if {\tt E1} contains {\tt E2} when both have the same center,
$0$ - otherwise.

Example:
{\tt \begin{verbatim}
>> E > E

ans =

     1
\end{verbatim} }
illustrates the fact that an ellipsoid is always bigger than itself.

\newpage

{\Large {\tt hpintersection}} - computes the ellipsoid which results from
intersection of given ellipsoid with given hyperplane.

Parameters:
\begin{itemize}
\item {\tt E} - ellipsoid or array of ellipsoids.
\item {\tt H} - hyperplane or array of hyperplanes.
\end{itemize}
If {\tt E} and {\tt H} are arrays, then their sizes must match.

Returns:
\begin{itemize}
\item {\tt I} - ellipsoid or array of ellipsoids that are intersections of
ellipsoids in {\tt E} with hyperplanes in {\tt H}.
\end{itemize}

Example:
{\tt \begin{verbatim}
>> E = ellipsoid([-2; -1], [4 -1; -1 1]);
>> H = [hyperplane([0 -1; -1 0]', 1); hyperplane([0 1; 1 0]', 1)];
>> I = hpintersection(E, H)

I =
2x2 array of ellipsoids.
\end{verbatim} }

\newpage

{\Large {\tt intersect}} - checks if the union or intersection of ellipsoids
intersects given ellipsoid, hyperplane or polytope.

Parameters:
\begin{itemize}
\item {\tt E} - ellipsoid or array of ellipsoids whose union or intersection
is considered.
\item {\tt X} - can be single or array of ellipsoids, hyperplanes or polytopes.
\item {\tt s} - if {\tt 'u'}, {\tt E} should be treated as union of ellipsoids,
if {\tt 'i'}, {\tt E} should be treated as intersection of ellipsoids.
This parameter is optional, its default value is {\tt 'u'}.
\end{itemize}

Returns $-1$ in case parameter {\tt s} is set to {\tt 'i'} and the
intersection of ellipsoids in {\tt E} is empty,\\
$0$ if the union or intersection of ellipsoids in {\tt E} does not intersect
the object in {\tt X},\\
$1$ if the union or intersection of ellipsoids in {\tt E} and the object
in {\tt X} have nonempty intersection.

Example:
{\tt \begin{verbatim}
>> E1 = ellipsoid([-2; -1], [4 -1; -1 1]);
>> E2 = E1 + [5; 5];
>> H  = hyperplane([1; -1]);
>> intersect([E1 E2], H)

ans =

     1

>> intersect([E1 E2], H, 'i')

ans =

    -1
\end{verbatim} }
Here two ellipsoids {\tt E1} and {\tt E2} do not intersect but both are
intersected by hyperplane {\tt H}.

\newpage

{\Large {\tt intersection\_ea}} - computes the external ellipsoidal
approximation of the intersection of the ellipsoid with given ellipsoid,
halfspace or polytope.

Parameters:
\begin{itemize}
\item {\tt E} - ellipsoid or array of ellipsoids.
\item {\tt X} - can be single or array of ellipsoids, hyperplanes or polytopes.
\end{itemize}
If {\tt E} and {\tt X} are arrays, then their sizes must match.

Returns:
\begin{itemize}
\item {\tt EA} - ellipsoid or array of ellipsoids that externally approximate
the intersection.
\end{itemize}

Example:
{\tt \begin{verbatim}
>> E1 = ellipsoid([-2; -1], [4 -1; -1 1]);
>> E2 = E1 + [5; 5];
>> B  = ell_unitball(2);
>> EA = intersection_ea([E1 E2], B)

EA =
1x2 array of ellipsoids.
\end{verbatim} }

\newpage

{\Large {\tt intersection\_ia}} - computes the internal ellipsoidal
approximation of the intersection of the ellipsoid with given ellipsoid,
halfspace or polytope.

Parameters:
\begin{itemize}
\item {\tt E} - ellipsoid or array of ellipsoids.
\item {\tt X} - can be single or array of ellipsoids, hyperplanes or polytopes.
\end{itemize}
If {\tt E} and {\tt X} are arrays, then their sizes must match.

Returns:
\begin{itemize}
\item {\tt IA} - ellipsoid or array of ellipsoids that internally approximate
the intersection.
\end{itemize}

Example:
{\tt \begin{verbatim}
>> E1 = ellipsoid([-2; -1], [4 -1; -1 1]);
>> E2 = E1 + [5; 5];
>> B  = ell_unitball(2);
>> IA = intersection_ia([E1 E2], B)

IA =
1x2 array of ellipsoids.
\end{verbatim} }

\newpage

{\Large {\tt inv}} - inverts the shape matrix of the ellipsoid if it is
nonsingular.

Parameters:
\begin{itemize}
\item {\tt E} - ellipsoid or array of ellipsoids.
\end{itemize}

Returns:
\begin{itemize}
\item {\tt I} - ellipsoid or array of ellipsoids with inverted shape matrices.
\end{itemize}

\newpage

{\Large {\tt isbaddirection}} - checks if ellipsoidal approximations of
the geometric difference of two ellipsoids can be computed for given
directions.

Parameters:
\begin{itemize}
\item {\tt E1} - first ellipsoid.
\item {\tt E2} - second ellipsoid.
\item {\tt L} - matrix whose columns are direction vectors that need to be
checked.
\end{itemize}

Returns $1$ if given direction is bad and ellipsoidal approximation cannot
be computed for it, $0$ - otherwise.

Example:
{\tt \begin{verbatim}
>> E = ellipsoid([-2; -1], [4 -1; -1 1]);
>> B = 3*ell_unitball(2);
>> L = [1 0; 1 1; 0 1; -1 1]';
>> isbaddirection(B, E, L)

ans =

     0     1     1     0}
\end{verbatim} }
means that for vectors $[1 ~ 1]^T$ and $[0 ~ 1]^T$ the ellipsoidal approximation
of the geometric difference ${\tt B}\dot{-}{\tt E}$ cannot be computed.

\newpage

{\Large {\tt isdegenerate}} - checks if given ellipsoid is degenerate.

Parameters:
\begin{itemize}
\item {\tt E} - ellipsoid or array of ellipsoids.
\end{itemize}

Returns $1$ if the ellipsoid is degenerate, $0$ - otherwise.

\newpage

{\Large {\tt isempty}} - checks if given ellipsoid is an empty object.

Parameters:
\begin{itemize}
\item {\tt E} - ellipsoid or array of ellipsoids.
\end{itemize}

Returns $1$ if empty, $0$ - otherwise.

\newpage

{\Large {\tt isinside}} - checks if the intersection of ellipsoids contains
the union or intersection of given ellipsoids or polytopes.

Parameters:
\begin{itemize}
\item {\tt E} - ellipsoid or array of ellipsoids whose intersection contains
or not the union or intersection of other given objects.
\item {\tt X} - can be array of ellipsoids or polytopes whose union or
intersection belongs or not to the intersection {\tt E}.
\item {\tt s} - if {\tt 'u'}, {\tt X} should be treated as union,
if {\tt 'i'}, {\tt X} should be treated as intersection. This parameter
is optional, its default value is {\tt 'u'}.
\end{itemize}

Returns $-1$ if parameter {\tt s} is set to {\tt 'i'} and the intersection
of ellipsoids or polytopes in {\tt X} is empty,\\
$0$ - if the intersection {\tt E} does not cover {\tt X},\\
$1$ - if {\tt E} contains {\tt X}.

Example:
{\tt \begin{verbatim}
>> E = ellipsoid([-2; -1], [4 -1; -1 1]);
>> B = ell_unitball(2);
>> isinside(E, [E B], 'i')

ans =

     1
\end{verbatim} }
illustrates the fact that any ellipsoid contains its intersection with
another ellipsoid.

\newpage

{\Large {\tt isinternal}} - checks if the union or intersection of ellipsoids
contains given vectors.

Parameters:
\begin{itemize}
\item {\tt E} - ellipsoid or array of ellipsoids whose intersection contains
or not given vectors or polytopes.
\item {\tt X} - can be matrix whose columns represent vectors to be checked.
\item {\tt s} - if {\tt 'u'}, {\tt E} should be treated as union,
if {\tt 'i'}, {\tt E} should be treated as intersection. This parameter
is optional, its default value is {\tt 'u'}.
\end{itemize}

Returns $1$ if vector in {\tt X} belongs to union or intersection {\tt E},
$0$ - otherwise.

Example:
{\tt \begin{verbatim}
>> E1 = ellipsoid([-2; -1], [4 -1; -1 1]);
>> E2 = E1 + [5; 5];
>> isinternal([E1 E2], [2 7; -1 4], 'i')

ans =

     0     0

>> isinternal([E1 E2], [2 7; -1 4])

ans =

     1     1
\end{verbatim} }
Here the intersection of {\tt E1} and {\tt E2} is empty, and thus, cannot
contain any points. The union, on the other hand, contains centers
of both ellipsoids.

\newpage

{\Large {\tt le}}, {\Large {\tt lt}} - checks if the second ellipsoid is bigger
than the first one.

Parameters:
\begin{itemize}
\item {\tt E1} - first ellipsoid.
\item {\tt E2} - second ellipsoid.
\end{itemize}

Returns $1$ if {\tt E2} contains {\tt E1} when both have the same center,
$0$ - otherwise.

This operation is the mirror of {\tt ge}, {\tt gt}.

\newpage

{\Large {\tt maxeig}} - returns the biggest eigenvalue of the ellipsoid.

Parameters:
\begin{itemize}
\item {\tt E} - ellipsoid or array of ellipsoids.
\end{itemize}

Returns:
\begin{itemize}
\item {\tt a} - largest eigenvalue or array of largest eigenvalues of
ellipsoids in {\tt E}.
\end{itemize}

\newpage

{\Large {\tt mineig}} - returns the smallest eigenvalue of the ellipsoid.

Parameters:
\begin{itemize}
\item {\tt E} - ellipsoid or array of ellipsoids.
\end{itemize}

Returns:
\begin{itemize}
\item {\tt a} - smallest eigenvalue or array of smallest eigenvalues of
ellipsoids in {\tt E}.
\end{itemize}

\newpage

{\Large {\tt minkdiff}} - computes and plots the geometric difference of
two ellipsoids in 2D and 3D.

Parameters:
\begin{itemize}
\item {\tt E1} - first ellipsoid.
\item {\tt E2} - second ellipsoid.
\item {\tt o} - options structure whose fields describe how the geometric
difference must be plotted. The fields of this structure are
\begin{itemize}
\item {\tt show\_all} - if set to $1$, also displays the ellipsoids
{\tt E1} and {\tt E2};
\item {\tt newfigure} - if set to $1$, each plot command will open a new
figure window;
\item {\tt fill} - if set to $1$, the resulting set and the ellipsoids,
if plotted in 2D, will be filled with color;
\item {\tt color} - specifies the color of the plot in the RGB format:
{\tt [x y z]};
\item {\tt shade} - the level of transparency for 3D plots, takes values
between $0$ and $1$ ($0$ - transparent, $1$ - opaque).
\end{itemize}
This parameter is optional.
\end{itemize}

Returns:
\begin{itemize}
\item {\tt x} - center of the resulting set.
\item {\tt X} - matrix whose columns represent the boundary points of the
resulting set. The number of points is defined by parameters
{\tt plot2d\_grid} for 2D plots and {\tt plot3d\_grid} for 3D plots of the
global {\tt ellOptions} structure.
\end{itemize}
Both output parameters are optional.

\newpage

{\Large {\tt minkdiff\_ea}} - computes external ellipsoidal approximations
of the geometric difference of two ellipsoids of arbitrary dimension
for given directions, if these directions are not bad.

Parameters:
\begin{itemize}
\item {\tt E1} - first ellipsoid.
\item {\tt E2} - second ellipsoid.
\item {\tt L} - matrix whose columns specify the directions for which
the approximations should be computed.
\end{itemize}

Returns:
\begin{itemize}
\item {\tt EA} - array of computed external ellipsoids. Can be empty, if
all the directions specified in {\tt L} are bad.
\end{itemize}

Example:
{\tt \begin{verbatim}
>> E = ellipsoid([-2; -1], [4 -1; -1 1]);
>> B = 3*ell_unitball(2);
>> L = [1 0; 1 1; 0 1; -1 1]';
>> EA = minkdiff_ea(B, E, L)

EA =
1x2 array of ellipsoids.
\end{verbatim} }
The resulting array {\tt EA} contains only two ellipsoids because two
of the four directions specified in {\tt L} are bad.

\newpage

{\Large {\tt minkdiff\_ia}} - computes internal ellipsoidal approximations
of the geometric difference of two ellipsoids of arbitrary dimension
for given directions, if these directions are not bad.

Parameters:
\begin{itemize}
\item {\tt E1} - first ellipsoid.
\item {\tt E2} - second ellipsoid.
\item {\tt L} - matrix whose columns specify the directions for which
the approximations should be computed.
\end{itemize}

Returns:
\begin{itemize}
\item {\tt EA} - array of computed internal ellipsoids. Can be empty, if
all the directions specified in {\tt L} are bad.
\end{itemize}

Example:
{\tt \begin{verbatim}
>> E = ellipsoid([-2; -1], [4 -1; -1 1]);
>> B = 3*ell_unitball(2);
>> L = [1 0; 1 1; 0 1; -1 1]';
>> IA = minkdiff_ia(B, E, L)

IA =
1x2 array of ellipsoids.
\end{verbatim} }
The resulting array {\tt IA} contains only two ellipsoids because two
of the four directions specified in {\tt L} are bad.

\newpage

{\Large {\tt minkmp}} - computes and plots geometric (Minkowski) sum of the
geometric difference of two ellipsoids and the geometric sum of $n$ ellipsoids
in 2D or 3D.

Parameters:
\begin{itemize}
\item {\tt E0} - first ellipsoid.
\item {\tt E} - second ellipsoid.
\item {\tt EE} - array of ellipsoids whose geometric sum needs to be computed.
\item {\tt o} - options structure whose fields describe how the geometric
difference must be plotted. The fields of this structure are
\begin{itemize}
\item {\tt show\_all} - if set to $1$, also displays the ellipsoids
{\tt E1} and {\tt E2};
\item {\tt newfigure} - if set to $1$, each plot command will open a new
figure window;
\item {\tt fill} - if set to $1$, the resulting set and the ellipsoids,
if plotted in 2D, will be filled with color;
\item {\tt color} - specifies the color of the plot in the RGB format:
{\tt [x y z]};
\item {\tt shade} - the level of transparency for 3D plots, takes values
between $0$ and $1$ ($0$ - transparent, $1$ - opaque).
\end{itemize}
This parameter is optional.
\end{itemize}

Returns:
\begin{itemize}
\item {\tt x} - center of the resulting set.
\item {\tt X} - matrix whose columns represent the boundary points of the
resulting set. The number of points is defined by parameters
{\tt plot2d\_grid} for 2D plots and {\tt plot3d\_grid} for 3D plots of the
global {\tt ellOptions} structure.
\end{itemize}
Both output parameters are optional.

\newpage

{\Large {\tt minkmp\_ea}} - computes external ellipsoidal approximations
of the geometric (Minkowski) sum of the geometric difference of two ellipsoids
and the geometric sum of $n$ ellipsoids.

Parameters:
\begin{itemize}
\item {\tt E0} - first ellipsoid.
\item {\tt E} - second ellipsoid.
\item {\tt EE} - array of ellipsoids whose geometric sum needs to be computed.
\item {\tt L} - matrix whose columns specify the directions for which
the approximations should be computed.
\end{itemize}

Returns:
\begin{itemize}
\item {\tt EA} - array of computed external ellipsoids. Can be empty, if
all the directions specified in {\tt L} are bad.
\end{itemize}

Example:
{\tt \begin{verbatim}
>> E = ellipsoid([-2; -1], [4 -1; -1 1]);
>> B = ell_unitball(2);
>> L = [1 0; 1 1; 0 1; -1 1]';
>> EA = minkmp_ea(3*B, E, [B E] L)

EA =
1x2 array of ellipsoids.
\end{verbatim} }
The resulting array {\tt EA} contains only two ellipsoids because two
of the four directions specified in {\tt L} are bad.

\newpage

{\Large {\tt minkmp\_ia}} - computes internal ellipsoidal approximations
of the geometric (Minkowski) sum of the geometric difference of two ellipsoids
and the geometric sum of $n$ ellipsoids.

Parameters:
\begin{itemize}
\item {\tt E0} - first ellipsoid.
\item {\tt E} - second ellipsoid.
\item {\tt EE} - array of ellipsoids whose geometric sum needs to be computed.
\item {\tt L} - matrix whose columns specify the directions for which
the approximations should be computed.
\end{itemize}

Returns:
\begin{itemize}
\item {\tt IA} - array of computed internal ellipsoids. Can be empty, if
all the directions specified in {\tt L} are bad.
\end{itemize}

Example:
{\tt \begin{verbatim}
>> E = ellipsoid([-2; -1], [4 -1; -1 1]);
>> B = ell_unitball(2);
>> L = [1 0; 1 1; 0 1; -1 1]';
>> IA = minkmp_ia(3*B, E, [B E] L)

IA =
1x2 array of ellipsoids.
\end{verbatim} }
The resulting array {\tt IA} contains only two ellipsoids because two
of the four directions specified in {\tt L} are bad.

\newpage

{\Large {\tt minkpm}} - computes and plots geometric (Minkowski) difference of
the geometric sum of ellipsoids and a single ellipsoid in 2D or 3D.

Parameters:
\begin{itemize}
\item {\tt EE} - array of ellipsoids whose geometric sum needs to be computed.
\item {\tt E} - single ellipsoid.
\item {\tt o} - options structure whose fields describe how the geometric
difference must be plotted. The fields of this structure are
\begin{itemize}
\item {\tt show\_all} - if set to $1$, also displays the ellipsoids
{\tt E1} and {\tt E2};
\item {\tt newfigure} - if set to $1$, each plot command will open a new
figure window;
\item {\tt fill} - if set to $1$, the resulting set and the ellipsoids,
if plotted in 2D, will be filled with color;
\item {\tt color} - specifies the color of the plot in the RGB format:
{\tt [x y z]};
\item {\tt shade} - the level of transparency for 3D plots, takes values
between $0$ and $1$ ($0$ - transparent, $1$ - opaque).
\end{itemize}
This parameter is optional.
\end{itemize}

Returns:
\begin{itemize}
\item {\tt x} - center of the resulting set.
\item {\tt X} - matrix whose columns represent the boundary points of the
resulting set. The number of points is defined by parameters
{\tt plot2d\_grid} for 2D plots and {\tt plot3d\_grid} for 3D plots of the
global {\tt ellOptions} structure.
\end{itemize}
Both output parameters are optional.

\newpage

{\Large {\tt minkpm\_ea}} - computes external ellipsoidal approximations
of the geometric (Minkowski) difference of the geometric sum of ellipsoids
and a single ellipsoid.

Parameters:
\begin{itemize}
\item {\tt EE} - array of ellipsoids whose geometric sum needs to be computed.
\item {\tt E} - single ellipsoid.
\item {\tt L} - matrix whose columns specify the directions for which
the approximations should be computed.
\end{itemize}

Returns:
\begin{itemize}
\item {\tt EA} - array of computed external ellipsoids. Can be empty, if
all the directions specified in {\tt L} are bad.
\end{itemize}

Example:
{\tt \begin{verbatim}
>> E1 = ellipsoid([2; -1], [9 -5; -5 4]);
>> E2 = ellipsoid([-2; -1], [4 -1; -1 1]);
>> B = ell_unitball(2);
>> L = [1 0; 1 1; 0 1; -1 1]';
>> EA = minkpm_ea([B E1], E2 L)

EA =
1x4 array of ellipsoids.
\end{verbatim} }

\newpage

{\Large {\tt minkpm\_ia}} - computes internal ellipsoidal approximations
of the geometric (Minkowski) difference of the geometric sum of ellipsoids
and a single ellipsoid.

Parameters:
\begin{itemize}
\item {\tt EE} - array of ellipsoids whose geometric sum needs to be computed.
\item {\tt E} - single ellipsoid.
\item {\tt L} - matrix whose columns specify the directions for which
the approximations should be computed.
\end{itemize}

Returns:
\begin{itemize}
\item {\tt IA} - array of computed internal ellipsoids. Can be empty, if
all the directions specified in {\tt L} are bad.
\end{itemize}

Example:
{\tt \begin{verbatim}
>> E1 = ellipsoid([2; -1], [9 -5; -5 4]);
>> E2 = ellipsoid([-2; -1], [4 -1; -1 1]);
>> B = ell_unitball(2);
>> L = [1 0; 1 1; 0 1; -1 1]';
>> IA = minkpm_ia([B E1], E2 L)

IA =
1x2 array of ellipsoids.
\end{verbatim} }
The resulting array {\tt IA} contains only two ellipsoids because two
of the four directions specified in {\tt L} are bad.

\newpage

{\Large {\tt minksum}} - computes and plots the geometric sum of finite
number of ellipsoids in 2D and 3D.

Parameters:
\begin{itemize}
\item {\tt E} - array of ellipsoids whose geometric sum needs to be computed.
\item {\tt o} - options structure whose fields describe how the geometric
sum must be plotted. The fields of this structure are
\begin{itemize}
\item {\tt show\_all} - if set to $1$, also displays the ellipsoids in the
array {\tt E};
\item {\tt newfigure} - if set to $1$, each plot command will open a new
figure window;
\item {\tt fill} - if set to $1$, the resulting set and the ellipsoids,
if plotted in 2D, will be filled with color;
\item {\tt color} - specifies the color of the plot in the RGB format:
{\tt [x y z]};
\item {\tt shade} - the level of transparency for 3D plots, takes values
between $0$ and $1$ ($0$ - transparent, $1$ - opaque).
\end{itemize}
This parameter is optional.
\end{itemize}

Returns:
\begin{itemize}
\item {\tt x} - center of the resulting set.
\item {\tt X} - matrix whose columns represent the boundary points of the
resulting set. The number of points is defined by parameters
{\tt plot2d\_grid} for 2D plots and {\tt plot3d\_grid} for 3D plots of the
global {\tt ellOptions} structure.
\end{itemize}
Both output parameters are optional.

\newpage

{\Large {\tt minksum\_ea}} - computes external ellipsoidal approximations
of the geometric sum of finite number of ellipsoids of arbitrary dimension
for given directions.

Parameters:
\begin{itemize}
\item {\tt E} - array of ellipsoids whose geometric sum needs to be
approximated.
\item {\tt L} - matrix whose columns specify the directions for which
the approximations should be computed.
\end{itemize}

Returns:
\begin{itemize}
\item {\tt EA} - array of computed external ellipsoids whose length is the same
as the number of columns of matrix {\tt L}.
\end{itemize}

Example:
{\tt \begin{verbatim}
>> E = ellipsoid([-2; -1], [4 -1; -1 1]);
>> B = ell_unitball(2);
>> L = [1 0; 1 1; 0 1; -1 1]';
>> EA = minkdiff_ea([E B inv(E)] L)

EA =
1x4 array of ellipsoids.
\end{verbatim} }

\newpage

{\Large {\tt minksum\_ia}} - computes internal ellipsoidal approximations
of the geometric sum of finite number of ellipsoids of arbitrary dimension
for given directions.

Parameters:
\begin{itemize}
\item {\tt E} - array of ellipsoids whose geometric sum needs to be
approximated.
\item {\tt L} - matrix whose columns specify the directions for which
the approximations should be computed.
\end{itemize}

Returns:
\begin{itemize}
\item {\tt IA} - array of computed internal ellipsoids whose length is the same
as the number of columns of matrix {\tt L}.
\end{itemize}

Example:
{\tt \begin{verbatim}
>> E = ellipsoid([-2; -1], [4 -1; -1 1]);
>> B = ell_unitball(2);
>> L = [1 0; 1 1; 0 1; -1 1]';
>> IA = minkdiff_ia([E B inv(E)] L)

IA =
1x4 array of ellipsoids.
\end{verbatim} }

\newpage

{\Large {\tt minus}} - overloaded operator {\tt '-'}.

Parameters:
\begin{itemize}
\item {\tt E} - ellipsoid or array of ellipsoids defined in ${\bf R}^n$.
\item {\tt b} - vector in ${\bf R}^n$ or matrix in ${\bf R}^{n\times m}$,
where $m$ equals the number of ellipsoids in the array {\tt E}.
\end{itemize}

Returns:
\begin{itemize}
\item {\tt E1} - ellipsoid or array of ellipsoids with same shapes as {\tt E},
but with centers shifted by vectors in {\tt -b}.
\end{itemize}

Example:
{\tt \begin{verbatim}
>> E  = [ellipsoid([-2; -1], [4 -1; -1 1]) ell_unitball(2)];
>> E1 = E - [1; 1];
>> E1(1)

ans =

Center:
    -3
    -2

Shape:
     4    -1
    -1     1

Nondegenerate ellipsoid in R^2.

>> E1(2)

ans =

Center:
    -1
    -1

Shape:
     1     0
     0     1

Nondegenerate ellipsoid in R^2.
\end{verbatim} }

\newpage

{\Large {\tt move2origin}} - moves given ellipsoids to the origin.

Parameters:
\begin{itemize}
\item {\tt E} - ellipsoid or array of ellipsoids.
\end{itemize}

Returns:
\begin{itemize}
\item {\tt O} - the same ellipsoid or array of ellipsoids as {\tt E},
but all centered at the origin.
\end{itemize}

Example:
{\tt \begin{verbatim}
>> E = ellipsoid([-2; -1], [4 -1; -1 1]);
>> O = move2origin(E)

O =

Center:
     0
     0

Shape:
     4    -1
    -1     1

Nondegenerate ellipsoid in R^2.
\end{verbatim} }

\newpage

{\Large {\tt mtimes}} - overloaded operator {\tt '*'}.

Parameters:
\begin{itemize}
\item {\tt A} - scalar, or matrix in ${\bf R}^{m\times n}$.
\item {\tt E} - ellipsoid or array of ellipsoids defined in ${\bf R}^n$.
\end{itemize}

Returns:
\begin{itemize}
\item {\tt E1} - ellipsoid or array of ellipsoids resulting from linear
transformation of {\tt E}.
\end{itemize}

Example:
{\tt \begin{verbatim}
>> E = ellipsoid([-2; -1], [4 -1; -1 1]);
>> A = [0 1; -1 0];
>> A*E

ans =

Center:
    -1
     2

Shape:
     1     1
     1     4

Nondegenerate ellipsoid in R^2.
\end{verbatim} }

\newpage

{\Large {\tt ne}} - overloaded operator {\tt '\~{ }='},
it checks if two ellipsoids are not equal.

Parameters:
\begin{itemize}
\item {\tt E1} - first ellipsoid.
\item {\tt E2} - second ellipsoid.
\end{itemize}

Returns $1$ if two ellipsoids are not equal, $0$ - otherwise.

Example:
{\tt \begin{verbatim}
>> E = ellipsoid([-2; -1], [4 -1; -1 1]);
>> E \~{ }= [E ellipsoid(eye(2))]

ans =

     0     1
\end{verbatim} }

\newpage

{\Large {\tt parameters}} - returns parameters of the ellipsoid, its center
and shape matrix.

Parameters:
\begin{itemize}
\item {\tt E} - single ellipsoid.
\end{itemize}

Returns:
\begin{itemize}
\item {\tt q} - center of the ellipsoid. It is optional output parameter.
\item {\tt Q} - shape matrix of the ellipsoid.
\end{itemize}

{\bf Remark.} This function is obsolete. Use {\tt double} instead.

\newpage

{\Large {\tt plot}} - plots ellipsoids in 1D, 2D and 3D.

Parameters:
\begin{itemize}
\item {\tt E} - ellipsoid or array of ellipsoids.
\item {\tt c} - specifies the color:
\begin{itemize}
\item {\tt 'b'} - blue;
\item {\tt 'c'} - cyan;
\item {\tt 'g'} - green;
\item {\tt 'k'} - black;
\item {\tt 'm'} - magenta;
\item {\tt 'r'} - red;
\item {\tt 'w'} - white;
\item {\tt 'y'} - yellow.
\end{itemize}
This parameter is optional.
\item {\tt o} - options structure whose fields describe how the ellipsoids
must be plotted. The fields of this structure are
\begin{itemize}
\item {\tt newfigure} - if set to $1$, each plot command will open a new
figure window;
\item {\tt fill} - if set to $1$, the resulting set and the ellipsoids,
if plotted in 2D, will be filled with color;
\item {\tt width} - specifies line width for 1D and 2D plots;
\item {\tt color} - specifies the color of the plot in the RGB format:
{\tt [x y z]};
\item {\tt shade} - the level of transparency for 3D plots, takes values
between $0$ and $1$ ($0$ - transparent, $1$ - opaque).
\end{itemize}
This parameter is optional.
\end{itemize}

Returns: None.

\newpage

{\Large {\tt plus}} - overloaded operator {\tt '+'}.

Parameters:
\begin{itemize}
\item {\tt E} - ellipsoid or array of ellipsoids defined in ${\bf R}^n$.
\item {\tt b} - vector in ${\bf R}^n$ or matrix in ${\bf R}^{n\times m}$,
where $m$ equals the number of ellipsoids in the array {\tt E}.
\end{itemize}

Returns:
\begin{itemize}
\item {\tt E1} - ellipsoid or array of ellipsoids with same shapes as {\tt E},
but with centers shifted by vectors in {\tt b}.
\end{itemize}

Example:
{\tt \begin{verbatim}
>> E  = [ellipsoid([-2; -1], [4 -1; -1 1]) ell_unitball(2)];
>> E1 = E + [1; 1];
>> E1(1)

ans =

Center:
    -1
     0

Shape:
     4    -1
    -1     1

Nondegenerate ellipsoid in R^2.

>> E1(2)

ans =

Center:
     1
     1

Shape:
     1     0
     0     1

Nondegenerate ellipsoid in R^2.
\end{verbatim} }

\newpage

{\Large {\tt polar}} - computes polars for ellipsoids which contain the origin.

Parameters:
\begin{itemize}
\item {\tt E} - ellipsoid or array of ellipsoids.
\end{itemize}

Returns:
\begin{itemize}
\item {\tt P} - polar ellipsoid or array of polar ellipsoids for those
in {\tt E}.
\end{itemize}

For ellipsoids that do not contain the origin, this function returns empty
ellipsoid.

Example:
{\tt \begin{verbatim}
>> E = ellipsoid([4 -1; -1 1]);
>> polar(E) == inv(E)

ans =

     1
\end{verbatim} }
illustrates the fact that polar set of an ellipsoid centered at the origin
equals to inverse of this ellipsoid.

\newpage

{\Large {\tt projection}} - computes projection of ellipsoids onto given
orthogonal basis.

Parameters:
\begin{itemize}
\item {\tt E} - ellipsoid or array of ellipsoids defined in ${\bf R}^n$.
\item {\tt B} - matrix in ${\bf R}^{n\times m}$ whose columns are orthogonal.
\end{itemize}

Returns:
\begin{itemize}
\item {\tt P} - array of projected ellipsoids.
\end{itemize}

Example:
{\tt \begin{verbatim}
>> E = ellipsoid([-2; -1; 4], [4 -1 0; -1 1 0; 0 0 9]);
>> B = [0 1 0; 0 0 1]';
>> P = projection(E, B)

P =

Center:
    -1
     4

Shape:
     1     0
     0     9

Nondegenerate ellipsoid in R^2.
\end{verbatim} }

\newpage

{\Large {\tt rho}} - computes the support function of the ellipsoids for
given directions and corresponding boundary points.

Parameters:
\begin{itemize}
\item {\tt E} - ellipsoid or array of ellipsoids defined in ${\bf R}^n$.
\item {\tt L} - vector in ${\bf R}^n$ if {\tt E} is an array of ellipsoids,
otherwise, if {\tt E} is a single ellipsoid, it can be matrix in
${\bf R}^{n\times m}$ whose columns represent the directions for which
the support function needs to be computed.
\end{itemize}

Returns:
\begin{itemize}
\item {\tt R} - array of support function values.
\item {\tt X} - matrix whose columns are boundary points corresponding
to the directions in {\tt L}. This output parameter is optional.
\end{itemize}

\newpage

{\Large {\tt shape}} - has the same functionality as {\tt mtimes} but
modifies only the shape matrix of the ellipsoid leaving its center as is.

Parameters:
\begin{itemize}
\item {\tt A} - scalar, or matrix in ${\bf R}^{m\times n}$.
\item {\tt E} - ellipsoid or array of ellipsoids defined in ${\bf R}^n$.
\end{itemize}

Returns:
\begin{itemize}
\item {\tt E1} - ellipsoid or array of ellipsoids resulting from modification
of the shape matrices of ellipsoids in {\tt E}.
\end{itemize}

Example:
{\tt \begin{verbatim}
>> E = ellipsoid([-2; -1], [4 -1; -1 1]);
>> A = [0 1; -1 0];
>> shape(E, A)

ans =

Center:
    -2
    -1

Shape:
     1     1
     1     4

Nondegenerate ellipsoid in R^2.
\end{verbatim} }

\newpage

{\Large {\tt trace}} - computes trace of given ellipsoids.

Parameters:
\begin{itemize}
\item {\tt E} - ellipsoid or array of ellipsoids.
\end{itemize}

Returns:
\begin{itemize}
\item {\tt T} - array of trace values whose size matches the size of {\tt E}.
\end{itemize}

Example:
{\tt \begin{verbatim}
>> E = ellipsoid([4 -1; -1 1]);
>> B = ell_unitball(2);
>> T = trace([E B])

V =

    5     2
\end{verbatim} }

\newpage


{\Large {\tt uminus}} - overloaded operation unitary minus.

Parameters:
\begin{itemize}
\item {\tt E} - ellipsoid or array of ellipsoids.
\end{itemize}

Returns:
\begin{itemize}
\item {\tt E1} - array of the same ellipsoids as in {\tt E}, whose centers
are multilied by $-1$.
\end{itemize}

Example:
{\tt \begin{verbatim}
>> E = -ellipsoid([-2; -1], [4 -1; -1 1])

E =

Center:
     2
     1

Shape:
     4    -1
    -1     1

Nondegenerate ellipsoid in R^2.
\end{verbatim} }

\newpage

{\Large {\tt volume}} - computes volume of given ellipsoids.

Parameters:
\begin{itemize}
\item {\tt E} - ellipsoid or array of ellipsoids.
\end{itemize}

Returns:
\begin{itemize}
\item {\tt V} - array of volume values whose size matches the size of {\tt E}.
\end{itemize}

Example:
{\tt \begin{verbatim}
>> E = ellipsoid([4 -1; -1 1]);
>> B = ell_unitball(2);
>> V = volume([E B])

V =

    5.4414     3.1416
\end{verbatim} }
















\newpage













\section{hyperplane Methods}
{\Large {\tt contains}} - checks if the hyperplanes contain given vectors.

Parameters:
\begin{itemize}
\item {\tt H} - hyperplane or array of hyperplanes.
\item {\tt X} - matrix whose columns represent the vectors needed to be checked.
The number of columns must match the number of hyperplanes in {\tt H}.
\end{itemize}

Returns $1$ if the vector in {\tt X} belongs to the hyperplane in {\tt H},
$0$ - otherwise.

Example:
{\tt \begin{verbatim}
>> H = hyperplane([-1; 1]);
>> X = [100 -1 2; 100 1 2];
>> contains(H, X)

ans =

     1     0     1
\end{verbatim} }

\newpage

{\Large {\tt dimension}} - returns the dimension of the space in which
the hyperplane is defined.

Parameters:
\begin{itemize}
\item {\tt H} - hyperplane or array of hyperplanes.
\end{itemize}

Returns:
\begin{itemize}
\item {\tt D} - array of dimension values of the same size as {\tt H}.
\end{itemize}

Example:
{\tt \begin{verbatim}
>> H1 = hyperplane([-1; 1]);
>> H2 = hyperplane([-1; 1; 8; -2; 3], 7);
>> H3 = hyperplane([1; 2; 0], -1);
>> D  = dimension([H1 H2 H3]);

D =

   2     5     3
\end{verbatim} }

\newpage

{\Large {\tt display}} - displays the details of the hyperplane object.

Parameters:
\begin{itemize}
\item {\tt H} - hyperplane or array of hyperplanes.
\end{itemize}
This function is rarely used explicitely. It is called automatically
for the object {\tt H} when semicolon does not terminate the statement.

\newpage

{\Large {\tt double}} - returns parameters of the hyperplane object,
its normal and scalar.

Parameters:
\begin{itemize}
\item {\tt H} - hyperplane object.
\end{itemize}

Parameters:
\begin{itemize}
\item {\tt v} - normal vector.
\item {\tt c} - scalar. This output parameter is optional.
\end{itemize}

\newpage

{\Large {\tt eq}} - overloaded operator {\tt '=='}.
it checks if two hyperplanes are equal.

Parameters:
\begin{itemize}
\item {\tt H1} - first hyperplane.
\item {\tt H2} - second hyperplane.
\end{itemize}

Returns $1$ if two hyperplanes are equal, $0$ - otherwise.

Example:
{\tt \begin{verbatim}
>> H1 = hyperplane([-1; 1]);
>> H2 = hyperplane([-1; 1; 8; -2; 3], 7);
>> H3 = hyperplane([1; 2; 0], -1);
>> H2 == [H1 H2 H3]

ans =

     0     1     0
\end{verbatim} }

\newpage

{\Large {\tt hyperplane}} - constructor for the hyperplane object.
If called without parameters, returns empty hyperplane.

Parameters:
\begin{itemize}
\item {\tt V} - vector in ${\bf R}^n$ or matrix in ${\bf R}^{n\times m}$ whose
columns define the normals to the hyperplanes that need to be created.
\item {\tt C} - scalar value, or, in case {\tt V} has $m$ columns, it can be
array with $m$ values.
\end{itemize}

Returns:
\begin{itemize}
\item {\tt H} - hyperplane, or, if {\tt V} has more than one column,
array of hyperplanes.
\end{itemize}

Example:
{\tt \begin{verbatim}
>> V = [1 1 1; 1 1 1];
>> C = [1 -5 0];
>> H = hyperplane(V, C);
\end{verbatim} }
defines three parallel hyperplanes and returns them in the array {\tt H}.

\newpage

{\Large {\tt isempty}} - checks if the hyperplane object is empty.

Parameters:
\begin{itemize}
\item {\tt H} - hyperplane or array of hyperplanes.
\end{itemize}

Returns: $1$ if the hyperplane object is empty, $0$ - otherwise.

\newpage

{\Large {\tt isparallel}} - checks if the hyperplanes are parallel.

Parameters:
\begin{itemize}
\item {\tt H1} - first hyperplane.
\item {\tt H2} - second hyperplane.
\end{itemize}

Returns $1$ if two hyperplanes are equal, $0$ - otherwise.

Example:
{\tt \begin{verbatim}
>> H = hyperplane([-1 1; 1 1; 1 1], [2 1 0]);
>> isparallel(H, H(2))

ans =

     0     1     1
\end{verbatim} }

\newpage

{\Large {\tt ne}} - overloaded operator {\tt '\~{ }='},
it checks if two hyperplanes are not equal.

Parameters:
\begin{itemize}
\item {\tt H1} - first hyperplane.
\item {\tt H2} - second hyperplane.
\end{itemize}

Returns $1$ if two hyperplanes are not equal, $0$ - otherwise.

Example:
{\tt \begin{verbatim}
>> H1 = hyperplane([-1; 1]);
>> H2 = hyperplane([-1; 1; 8; -2; 3], 7);
>> H3 = hyperplane([1; 2; 0], -1);
>> [H2 H1 H3] == [H1 H2 H3]

ans =

     0     0     1
\end{verbatim} }

\newpage

{\Large {\tt parameters}} - returns parameters of the hyperplane object,
its normal and scalar.

Parameters:
\begin{itemize}
\item {\tt H} - hyperplane object.
\end{itemize}

Parameters:
\begin{itemize}
\item {\tt v} - normal vector.
\item {\tt c} - scalar. This output parameter is optional.
\end{itemize}

{\bf Remark.} This function is obsolete. Use {\tt double} instead.

\newpage

{\Large {\tt plot}} - plots hyperplanes in 2D and 3D.

Parameters:
\begin{itemize}
\item {\tt H} - hyperplane or array of hyperplanes.
\item {\tt c} - specifies the color:
\begin{itemize}
\item {\tt 'b'} - blue;
\item {\tt 'c'} - cyan;
\item {\tt 'g'} - green;
\item {\tt 'k'} - black;
\item {\tt 'm'} - magenta;
\item {\tt 'r'} - red;
\item {\tt 'w'} - white;
\item {\tt 'y'} - yellow.
\end{itemize}
This parameter is optional.
\item {\tt o} - options structure whose fields describe how the hyperplanes
must be plotted. The fields of this structure are
\begin{itemize}
\item {\tt newfigure} - if set to $1$, each plot command will open a new
figure window;
\item {\tt size} - length of the line segment in 2D or square diagonal in 3D.
\item {\tt center} - center of the line segment in 2D or square diagonal in 3D.
\item {\tt width} - specifies line width for 2D plots;
\item {\tt color} - specifies the color of the plot in the RGB format:
{\tt [x y z]};
\item {\tt shade} - the level of transparency for 3D plots, takes values
between $0$ and $1$ ($0$ - transparent, $1$ - opaque).
\end{itemize}
This parameter is optional.
\end{itemize}

Returns: None.

\newpage

{\Large {\tt uminus}} - overloaded operator unitary minus. It does not change
the hyperplane, and affects only the halfspace this hyperplane defines.

Parameters:
\begin{itemize}
\item {\tt H} - hyperplane or array of hyperplanes.
\end{itemize}

Returns:
\begin{itemize}
\item {\tt H1} - array of the same hyperplanes as in {\tt H} whose normals and
scalars are multiplied by $-1$.
\end{itemize}

Example:
{\tt \begin{verbatim}
>> H = -hyperplane([-1; 1], 1)

H =

Normal:
     1
    -1

Shift:
    -1

Hyperplane in R^2.
\end{verbatim} }
















\newpage

















\section{linsys Methods}
{\Large {\tt dimension}} - returns dimensions of state, input, output and
disturbance input spaces.

Parameters:
\begin{itemize}
\item {\tt LSYS} - linear system or array of linear systems.
\end{itemize}

Returns:
\begin{itemize}
\item {\tt N} - state space dimension.
\item {\tt I} - dimension of the input space.
\item {\tt O} - dimension of the output space.
\item {\tt D} - dimension of the disturbance input space.
\end{itemize}

\newpage

{\Large {\tt display}} - displays the details of the linear system object.

Parameters:
\begin{itemize}
\item {\tt LSYS} - linear system or array of linear systems.
\end{itemize}
This function is rarely used explicitely. It is called automatically
for the object {\tt LSYS} when semicolon does not terminate the statement.

\newpage

{\Large {\tt hasdisturbance}} - checks if given linear system is the system
with disturbance.

Parameters:
\begin{itemize}
\item {\tt LSYS} - linear system or array of linear systems.
\end{itemize}

Returns $1$ if it is system with disturbance, $0$ - otherwise.

\newpage

{\Large {\tt hasnoise}} - checks if given linear system has noise at the
output.

Parameters:
\begin{itemize}
\item {\tt LSYS} - linear system or array of linear systems.
\end{itemize}

Returns $1$ if it is system with noise, $0$ - otherwise.

\newpage

{\Large {\tt isdiscrete}} - checks if given linear system is discrete-time.

Parameters:
\begin{itemize}
\item {\tt LSYS} - linear system or array of linear systems.
\end{itemize}

Returns $1$ if the system is discrete-time, $0$ - otherwise.

\newpage

{\Large {\tt isempty}} - checks if given linear system is an empty object.

Parameters:
\begin{itemize}
\item {\tt LSYS} - linear system or array of linear systems.
\end{itemize}

Returns $1$ if {\tt LSYS} is an empty object, $0$ - otherwise.

\newpage

{\Large {\tt islti}} - checks if given linear system is time invariant.

Parameters:
\begin{itemize}
\item {\tt LSYS} - linear system or array of linear systems.
\end{itemize}

Returns $1$ if the system is time invariant, $0$ - otherwise.

\newpage

{\Large {\tt linsys}} - constructor for the linear system object.
If called without parameters, creates an empty object.

Parameters:
\begin{itemize}
\item {\tt A} - matrix $A\in{\bf R}^{n\times n}$, can be symbolic if it
depends on time ($t$ - in continuous case, $k$ - in discrete case).
\item {\tt B} - matrix $B\in{\bf R}^{n\times m}$, can by symbolic if it
depends on time.
\item {\tt U} - defines control bounds: it can be ellipsoid object with
dimension $m$; or structure with fields {\tt center} - symbolic vector,
and {\tt shape} - symbolic matrix, if the ellipsoidal bounds depend on time;
or, if the control is fixed, it can be single vector of type {\it double}
or symbolic.
\item {\tt G} - matrix $G\in{\bf R}^{n\times d}$, can by symbolic if it
depends on time. This parameter is optional.
\item {\tt V} - defines disturbance bounds: it can be ellipsoid object with
dimension $d$; or structure with fields {\tt center} - symbolic vector,
and {\tt shape} - symbolic matrix, if the ellipsoidal bounds depend on time;
or, if the control is fixed, it can be single vector of type {\it double}
or symbolic. This parameter is optional.
\item {\tt C} - matrix $C\in{\bf R}^{o\times n}$, can by symbolic if it
depends on time. This parameter is optional.
\item {\tt W} - defines noise bounds: it can be ellipsoid object with
dimension $o$; or structure with fields {\tt center} - symbolic vector,
and {\tt shape} - symbolic matrix, if the ellipsoidal bounds depend on time;
or, if the control is fixed, it can be single vector of type {\it double}
or symbolic. This parameter is optional.
\item {\tt s} - if set to {\tt 'd'}, indicates that the system is discrete-time.
\end{itemize}

Returns:
\begin{itemize}
\item {\tt LSYS} - linear system object.
\end{itemize}

Example:
{\tt \begin{verbatim}
>> A = {'0' '1 + cos(pi*k/2)'; '-2' '0'};
>> B = [0; 1];
>> U = ellipsoid(4);
>> G = [1; 0];
>> V = 1/(k+1);
>> C = [1 0];
>> lsys = linsys(A, B, U, G, V, C, [], 'd');
\end{verbatim} }

defines the following affine discrete-time system:
\begin{eqnarray*}
\left[\begin{array}{c}
x_1[k+1]\\
x_2[k+1]\end{array}\right] & = & \left[\begin{array}{cc}
0 & 1 + \cos(\frac{\pi k}{2})\\
-2 & 0\end{array}\right] + \left[\begin{array}{c}
0\\
1\end{array}\right]u[k] + \left[\begin{array}{c}
\frac{1}{k+1}\\
0\end{array}\right], ~~~ -2\leq u[k]\leq2\\
y[k] & = & [1 ~~ 0]\left[\begin{array}{c}
x_1[k]\\
x_2[k]\end{array}\right], ~~~ k\geq0.
\end{eqnarray*}























\newpage


















\section{reach Methods}
{\Large {\tt cut}} - extracts a segment of the reach tube from the given start
time to the given end time.

Parameters:
\begin{itemize}
\item {\tt RS} - reach set object.
\item {\tt T} - time interval of interest in the form {\tt [t1 t2]}.
If {\tt T} is a single scalar, then the reach set for this particular time
value is returned.
\end{itemize}

Returns:
\begin{itemize}
\item {\tt CRS} - the resulting reach set object.
\end{itemize}

\newpage

{\Large {\tt dimension}} - returns the dimension of the reach set and
the dimension of the state space for which the reach set was originally
computed.

Parameters:
\begin{itemize}
\item {\tt RS} - reach set object.
\end{itemize}

Returns:
\begin{itemize}
\item {\tt d} - dimension of the reach set.
\item {\tt n} - dimension of the state space for which the reach set was
originally computed. Values of {\tt d} and {\tt n} can be different if
the reach set object is a result of {\tt projection} operation.
This parameter is optional.
\end{itemize}

\newpage

{\Large {\tt display}} - displays the details of the reach set object.

Parameters:
\begin{itemize}
\item {\tt RS} - reach set object.
\end{itemize}
This function is rarely used explicitely. It is called automatically
for the object {\tt LSYS} when semicolon does not terminate the statement.

\newpage

{\Large {\tt evolve}} - computes further evolution in time of the already
existing reach set.

Parameters:
\begin{itemize}
\item {\tt CRS} - existing reach set.
\item {\tt T} - new time horizon.
\item {\tt LSYS} - linear system object that describes new dynamics.
This parameter is optional.
\end{itemize}

Returns:
\begin{itemize}
\item {\tt RS} - reach set object.
\end{itemize}

\newpage

{\Large {\tt get\_center}} - returns trajectory of the center of the reach set.

Parameters:
\begin{itemize}
\item {\tt RS} - reach set object.
\end{itemize}

Returns:
\begin{itemize}
\item {\tt X} - matrix whose columns represent the position of the center
at times specified in the second output parameter, {\tt T}.
\item {\tt T} - array of time values at which the trajectory of the reach set
center is evaluated. This output parameter is optional.
\end{itemize}
The number of columns in {\tt X} and values in {\tt T}
is specified by the parameter
{\tt time\_grid} of the {\tt ellOptions} structure.

\newpage

{\Large {\tt get\_directions}} - returns the trajectories of the direction
vectors for which the ellipsoidal approximations were computed.

Parameters:
\begin{itemize}
\item {\tt RS} - reach set object.
\end{itemize}

Returns:
\begin{itemize}
\item {\tt X} - array of cells where each cell represents a trajectory
of the direction vector, for which the ellipsoidal approximations of the
reach set were computed.
\item {\tt T} - array of time values at which these trajectories
are evaluated. This output parameter is optional.
\end{itemize}
The number of columns in cells of {\tt X} and values in {\tt T} is specified by
the parameter {\tt time\_grid} of the {\tt ellOptions} structure.

\newpage

{\Large {\tt get\_ea}} - returns array of ellipsoids that represent the
external approximation of the reach set.

Parameters:
\begin{itemize}
\item {\tt RS} - reach set object.
\end{itemize}

Returns:
\begin{itemize}
\item {\tt E} - array of ellipsoids that externally approximate the reach set.
The intersection of ellipsoids in the given column of this array is
the external approximation of the reach set at the time spacified by the
corresponding value in the array {\tt T}.
\item {\tt T} - array of time values at which the approximations
are evaluated. This output parameter is optional.
\end{itemize}
The number of columns in {\tt E} and values in {\tt T} is specified by
the parameter {\tt time\_grid} of the {\tt ellOptions} structure.

\newpage

{\Large {\tt get\_goodcurves}} - returns the trajectories along which
the ellipsoidal approximations are touching the actual reach set.

Parameters:
\begin{itemize}
\item {\tt RS} - reach set object.
\end{itemize}

Returns:
\begin{itemize}
\item {\tt X} - array of cells where each cell represents a good trajectory
along which one of the computed ellipsoidal approximations touches the
reach set.
\item {\tt T} - array of time values at which the good trajectories
are evaluated. This output parameter is optional.
\end{itemize}
The number of columns in cells of {\tt X} and values in {\tt T} is specified by
the parameter {\tt time\_grid} of the {\tt ellOptions} structure.

\newpage

{\Large {\tt get\_ia}} - returns array of ellipsoids that represent the
internal approximation of the reach set.

Parameters:
\begin{itemize}
\item {\tt RS} - reach set object.
\end{itemize}

Returns:
\begin{itemize}
\item {\tt E} - array of ellipsoids that internally approximate the reach set.
The union of ellipsoids in the given column of this array is
the internal approximation of the reach set at the time specified by the
corresponding value in the array {\tt T}.
\item {\tt T} - array of time values at which the approximations
are evaluated. This output parameter is optional.
\end{itemize}
The number of columns in {\tt E} and values in {\tt T} is specified by
the parameter {\tt time\_grid} of the {\tt ellOptions} structure.

\newpage

{\Large {\tt get\_system}} - returns the linear system object for which
the reach set was computed.

Parameters:
\begin{itemize}
\item {\tt RS} - reach set object.
\end{itemize}

Returns:
\begin{itemize}
\item {\tt LSYS} - the linear system object, which was used for the reach
set computation.
\end{itemize}

\newpage

{\Large {\tt intersect}} - checks if the external or internal approximation
of the reach set intersects with given ellipsoids, hyperplanes or
polytopes.

Parameters:
\begin{itemize}
\item {\tt RS} - reach set object.
\item {\tt X} - can be array of ellipsoids, hyperplanes or polytopes.
\item {\tt s} - if set to {\tt 'i'}, indicates that internal approximation
should be checked, if set to {\tt 'e'} - external. This input parameter
is optional, its default value is {\tt 'e'}.
\end{itemize}

Returns $1$ if the intersection is nonempty, $0$ - otherwise.

\newpage

{\Large {\tt iscut}} - checks if the given reach set object resulted
from {\tt cut} operation.

Parameters:
\begin{itemize}
\item {\tt RS} - reach set object.
\end{itemize}

Returns $1$ if {\tt RS} is a result of {\tt cut} operation, $0$ - otherwise.

\newpage

{\Large {\tt isempty}} - checks if the given reach set object is empty.

Parameters:
\begin{itemize}
\item {\tt RS} - reach set object.
\end{itemize}

Returns $1$ if it is an empty object, $0$ - otherwise.

\newpage

{\Large {\tt isprojection}} - checks if the given reach set is a result
of {\tt projection} operation.

Parameters:
\begin{itemize}
\item {\tt RS} - reach set object.
\end{itemize}

Returns $1$ if {\tt RS} is a projection, $0$ - otherwise.

\newpage

{\Large {\tt plot\_ea}} - plots external approximation of the reach set
in 2D and 3D.

Parameters:
\begin{itemize}
\item {\tt RS} - reach set object.
\item {\tt c} - specifies the color:
\begin{itemize}
\item {\tt 'b'} - blue;
\item {\tt 'c'} - cyan;
\item {\tt 'g'} - green;
\item {\tt 'k'} - black;
\item {\tt 'm'} - magenta;
\item {\tt 'r'} - red;
\item {\tt 'w'} - white;
\item {\tt 'y'} - yellow.
\end{itemize}
This parameter is optional.
\item {\tt o} - options structure whose fields describe how the reach set
must be plotted. The fields of this structure are
\begin{itemize}
\item {\tt width} - specifies line width for 2D plots;
\item {\tt fill} - if set to $1$, the set, if plotted in 2D,
will be filled with color;
\item {\tt color} - specifies the color of the plot in the RGB format:
{\tt [x y z]};
\item {\tt shade} - the level of transparency for 3D plots, takes values
between $0$ and $1$ ($0$ - transparent, $1$ - opaque).
\end{itemize}
This parameter is optional.
\end{itemize}

Returns: None.

\newpage

{\Large {\tt plot\_ia}} - plots internal approximation of the reach set
in 2D and 3D.

Parameters:
\begin{itemize}
\item {\tt RS} - reach set object.
\item {\tt c} - specifies the color:
\begin{itemize}
\item {\tt 'b'} - blue;
\item {\tt 'c'} - cyan;
\item {\tt 'g'} - green;
\item {\tt 'k'} - black;
\item {\tt 'm'} - magenta;
\item {\tt 'r'} - red;
\item {\tt 'w'} - white;
\item {\tt 'y'} - yellow.
\end{itemize}
This parameter is optional.
\item {\tt o} - options structure whose fields describe how the reach set
must be plotted. The fields of this structure are
\begin{itemize}
\item {\tt width} - specifies line width for 2D plots;
\item {\tt fill} - if set to $1$, the set, if plotted in 2D,
will be filled with color;
\item {\tt color} - specifies the color of the plot in the RGB format:
{\tt [x y z]};
\item {\tt shade} - the level of transparency for 3D plots, takes values
between $0$ and $1$ ($0$ - transparent, $1$ - opaque).
\end{itemize}
This parameter is optional.
\end{itemize}

Returns: None.

\newpage

{\Large {\tt projection}} - projects the reach set onto the given orthogonal
basis.

Parameters:
\begin{itemize}
\item {\tt RS} - reach set object with dimension $n$.
\item {\tt B} - matrix in ${\bf R}^{n\times m}$ whose columns are orthogonal,
they represent the basis vectors.
\end{itemize}

Returns:
\begin{itemize}
\item {\tt PRS} - reach set object with the projected reach set.
\end{itemize}

\newpage

{\Large {\tt reach}} - constructor for the reach set object and the main
function that computes the reach set.

Parameters:
\begin{itemize}
\item {\tt LSYS} - the linear system object with state space dimension $n$.
\item {\tt X0} - ellipsoid object with dimension $n$, it represents
the set of initial conditions.
\item {\tt L0} - matrix in ${\bf R}^{n\times m}$ whose columns represent the
directions for which the new approximations need to be computed.
\item {\tt T} - time interval in the form {\tt [t0 t1]}. If {\tt t1} is
smaller than {\tt t0}, then the backward reach set should be computed.
If {\tt T} is a scalar, then it is assumed that {\tt t0 = 0}.
\item {\tt o} - options structure with fields
\begin{itemize}
\item {\tt approximation} - if set to $0$, then only external approximation
is computed, if set to $1$, then only internal approximation is computed,
if set to $2$ (default value), then both approximations are computed.
\item {\tt save\_all} - if set to $0$ (default value), then the intermediate
calculation data should not be saved, $1$ indicates the opposite.
\item {\tt minmax} - for discrete-time systems, if set to $0$ (default value),
then maxmin reach set must be computed, $1$ indicates that minmax reach set is
to be computed; for continuous-time systems - ignored.
\end{itemize}
This parameter is optional.
\end{itemize}

Returns:
\begin{itemize}
\item {\tt RS} - reach set object.
\end{itemize}

\newpage

{\Large {\tt refine}} - adds new approximations for the specified directions
to the given reach set object, thus improving the overall approximation.

Parameters:
\begin{itemize}
\item {\tt RS} - reach set object with dimension $n$. This object must
contain the intermediate calculation data, otherwise, the refinement
is not possible.
\item {\tt L0} - matrix in ${\bf R}^{n\times m}$ whose columns represent the
directions for which the new approximations need to be computed.
\item {\tt o} - options structure with fields
\begin{itemize}
\item {\tt approximation} - if set to $0$, then only external approximation
is computed, if set to $1$, then only internal approximation is computed,
if set to $2$ (default value), then both approximations are computed.
\item {\tt save\_all} - if set to $0$, then the intermediate calculation
data should not be saved, $1$ (default value) indicates the opposite.
\end{itemize}
This parameter is optional.
\end{itemize}

Returns:
\begin{itemize}
\item {\tt RRS} - reach set object with refined approximation.
\end{itemize}
















\newpage





















\section{Miscellaneous Functions}
{\Large {\tt ell\_enclose}} - computes minimum volume ellipsoid that
contains given vectors.

Parameters:
\begin{itemize}
\item {\tt V} - matrix whose columns represent the vectors to be enclosed.
\end{itemize}

Returns:
\begin{itemize}
\item {\tt E} - enclosing ellipsoid.
\end{itemize}

\newpage

{\Large {\tt ell\_inv}} - computes matrix inverse treating ill-conditioned matrices properly.

Parameters:
\begin{itemize}
\item {\tt A} - non-singular square matrix.
\end{itemize}

Returns:
\begin{itemize}
\item {\tt I} - inverse of matrix {\tt A}.
\end{itemize}

\newpage

{\Large {\tt ell\_simdiag}} - computes the orthogonal transformation matrix
that simultaneously diagonalizes two symmetric matrices.

Parameters:
\begin{itemize}
\item {\tt A} - symmetric positive definite matrix in ${\bf R}^{n\times n}$.
\item {\tt B} - symmetric positive semidefinite matrix in ${\bf R}^{n\times n}$.
\end{itemize}

Returns:
\begin{itemize}
\item {\tt T} - orthogonal matrix in ${\bf R}^{n\times n}$, such that
$TAT^T$ is identity matrix, and $TBT^T$ is diagonal martrix.
\end{itemize}

\newpage

{\Large {\tt ell\_unitball}} - creates the ellipsoid object that represents
a unit ball of the given dimension.

Parameters:
\begin{itemize}
\item {\tt n} - dimension of the space in which the unit ball is defined.
\end{itemize}

Returns:
\begin{itemize}
\item {\tt B} - ellipsoid object with identity shape matrix centered at the
origin.
\end{itemize}

\newpage

{\Large {\tt ell\_valign}} - computes the orthogonal matrix that aligns two
vectors.

Parameters:
\begin{itemize}
\item {\tt w} - vector in ${\bf R}^n$.
\item {\tt v} - vector in ${\bf R}^n$ that needs to be rotated to be parallel
to {\tt w}.
\end{itemize}

Returns:
\begin{itemize}
\item {\tt S} - orthogonal matrix in ${\bf R}^{n\times n}$, such that
$Sv = \frac{\|v\|_2}{\|w\|_2}w$.
\end{itemize}

\newpage

{\Large {\tt hyperplane2polytope}} - converts array of hyperplanes of the
same dimension into the polytope object of the Multi-Parametric Toolbox.

Parameters:
\begin{itemize}
\item {\tt H} - array of hyperplanes, all of which must have the same dimension.
\end{itemize}

Returns:
\begin{itemize}
\item {\tt P} - polytope object as defined in the Multi-Parametric Toolbox.
\end{itemize}

\newpage

{\Large {\tt polytope2hyperplane}} - converts the polytope object of the
Multi-Parametric Toolbox into the array of hyperplanes.

Parameters:
\begin{itemize}
\item {\tt P} - polytope object as defined in the Multi-Parametric Toolbox.
\end{itemize}

Returns:
\begin{itemize}
\item {\tt H} - array of hyperplanes.
\end{itemize}



