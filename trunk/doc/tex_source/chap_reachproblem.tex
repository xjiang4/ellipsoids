\subsection{Systems without disturbances}\label{subsec_sysnodist}
\todo[inline]{Added this subsection}
Consider a general continuous-time
\begin{equation}
\dot{x}(t) = f(t, x, u),
\label{ctds1}
\end{equation}
or discrete-time dynamical system
\begin{equation}
x(t+1) = f(t, x, u),
\tag*{(\ref{ctds1}d)}
\label{dtds1}
\end{equation}
wherein $t$ is time\footnote{In discrete-time case $t$ assumes integer values.},
$x\in{\bf R}^n$ is the state, $u\in{\bf R}^m$ is the control,
and $f$ is a measurable vector function taking values in
${\bf R}^n$.\footnote{We are being general when giving the basic definitions.
However, it is  important to understand that for any specific
\emph{continuous-time} dynamical system it must be determined whether
the solution exists and is unique, and in which class of solutions
these conditions are met.
Here we shall assume that function $f$ is such that the solution of the differential equation (\ref{ctds1})
exists and is unique in Fillipov sense.
This allows the right-hand side to be discontinuous.
For discrete-time systems this problem does not exist.}
The control values $u(t, x(t))$ are restricted
to a closed compact control set $\UU(t)\subset{\bf R}^m$.
An \emph{open-loop} control does not depend on the state, $u=u(t)$; for a \emph{closed-loop} control, $u=u(t, x(t))$.

\bd[Reach set]
The (forward) reach set $\XX(t, t_0, x_0)$ at time $t>t_0$ from the initial position
$(t_0, x_0)$ is the set of all states $x(t)$ reachable at time $t$ by
system (\ref{ctds1}), or \ref{dtds1}, with $x(t_0)=x_0$
through all possible controls
$u(\tau, x(\tau))\in\UU(\tau)$, $t_0\leq\tau< t$.
For a given set of initial states $\XX_0$, the reach set $\XX(t, t_0, \XX_0)$ is
\[ \XX(t, t_0, \XX_0) = \bigcup_{x_0\in\XX_0}\XX(t, t_0, x_0). \]
\label{def_olrs}
\ed
Here are two facts about forward reach sets.
\begin{enumerate}

\item  $\XX(t, t_0, \XX_0)$ is the same for open-loop and closed-loop control.

\item  $\XX(t, t_0, \XX_0)$ satisfies the semigroup property,
\begin{equation}
\XX(t, t_0, \XX_0) = \XX(t, \tau, \XX(\tau, t_0, \XX_0)), \;\;\;
t_0\leq \tau< t.
\label{semigroup}
\end{equation}
\end{enumerate}

For linear systems
\begin{equation}
f(t, x, u) = A(t)x(t) + B(t)u,
\label{linearrhs}
\end{equation}
with matrices $A(t)$ in ${\bf R}^{n\times n}$ and $B(t)$ in ${\bf R}^{m\times n}$.
For continuous-time linear system the state transition matrix is
\[ \dot{\Phi}(t, t_0) = A(t)\Phi(t, t_0), \;\;\; \Phi(t, t) = I, \]
which for constant $A(t)\equiv A$ simplifies as
\[ \Phi(t, t_0) = e^{A(t-t_0)} .\]
For discrete-time linear system the state transition matrix is
\[ \Phi(t+1, t_0) = A(t)\Phi(t, t_0), \;\;\; \Phi(t, t) = I, \]
which for constant $A(t)\equiv A$ simplifies as
\[ \Phi(t, t_0) = A^{t-t_0} .\]

If the state transition matrix is invertible, $\Phi^{-1}(t, t_0) = \Phi(t_0, t)$.
The transition matrix is always invertible for continuous-time and for sampled discrete-time systems.
However, if for some $\tau$, $t_0\leq\tau<t$, $A(\tau)$
is degenerate (singular),  $\Phi(t, t_0)=\prod_{\tau=t_0}^{t-1}A(\tau)$, is also degenerate
and  cannot be inverted.

Following Cauchy's formula,
the reach set $\XX(t, t_0, \XX_0)$ for a linear system can be expressed as
\begin{equation}
\XX(t, t_0, \XX_0) =
\Phi(t, t_0)\XX_0 \oplus \int_{t_0}^t\Phi(t, \tau)B(\tau)\UU(\tau)d\tau
\label{ctlsrs}
\end{equation}
in continuous-time, and as
\begin{equation}
\XX(t, t_0, \XX_0) =
\Phi(t, t_0)\XX_0 \oplus \sum_{\tau=t_0}^{t-1}\Phi(t, \tau+1)B(\tau)\UU(\tau)
\tag*{(\ref{ctlsrs}d)}
\label{dtlsrs}
\end{equation}
in discrete-time case.

The operation `$\oplus$' is the \emph{geometric sum},
also known as \emph{Minkowski sum}.\footnote{Minkowski sum of sets
$\WW, \ZZ \subseteq {\bf R}^n$ is defined as
$\WW \oplus \ZZ = \{w+z ~|~ w\in\WW, ~ z\in\ZZ\}$.
Set $\WW\oplus\ZZ$ is nonempty if and only if both, $\WW$ and $\ZZ$ are
nonempty.
If $\WW$ and $\ZZ$ are convex, set $\WW\oplus\ZZ$ is convex.}
The geometric sum and linear (or affine)
transformations preserve compactness and convexity.
Hence, if the initial set $\XX_0$ and the control sets $\UU(\tau)$,
$t_0\leq\tau<t$, are compact and convex, so is the reach set
$\XX(t, t_0, \XX_0)$.

\bd[Backward reach set]
The backward reach set $\YY(t_1, t, y_1)$ for the target position
$(t_1, y_1)$ is the set of all states $y(t)$ for which there exists
some control $u(\tau, x(\tau))\in\UU(\tau)$, $t\leq\tau<t_1$,
that steers system (\ref{ctds1}), or \ref{dtds1}
to the state $y_1$ at time $t_1$.
For the target set $\YY_1$ at time $t_1$, the backward reach set $\YY(t_1, t, \YY_1)$ is
\[ \YY(t_1, t, \YY_1) = \bigcup_{y_1\in\YY_1}\YY(t_1, t, y_1). \]
\label{def_olbrs}
\ed
The backward reach set $\YY(t_1, t, \YY_1)$ is the largest \emph{weakly invariant}
set with respect to the target set $\YY_1$ and time values $t$ and
$t_1$.\footnote{$\MM$ is weakly invariant with respect to
the target set $\YY_1$ and times $t_0$ and $t$, if for every state $x_0\in\MM$
there exists a control $u(\tau, x(\tau))\in\UU(\tau)$, $t_0\leq\tau< t$, that
steers the system from $x_0$ at time $t_0$ to some state in $\YY_1$ at time $t$.
If \emph{all} controls in $\UU(\tau)$, $t_0\leq\tau<t$ steer the system
from every $x_0\in\MM$ at time $t_0$ to $\YY_1$ at time $t$,
set $\MM$ is said to be
\emph{strongly} invariant with respect to $\YY_1$, $t_0$ and $t$.}

{\bf Remark.}
Backward reach set can be computed for continuous-time system only if
the solution of (\ref{ctds1}) exists for $t<t_1$; and for discrete-time
system only if the right hand side of \ref{dtds1}
is invertible\footnote{There exists $f^{-1}(t,x,u)$ such that
$x(t)=f^{-1}(t, x(t+1), u, v)$.}.

These two facts about
the backward reach set $\YY$ are similar to those for forward reach sets.
\begin{enumerate}
\item $\YY(t_1, t, \YY_1)$ is the same for open-loop
and closed-loop control.

\item $\YY(t_1, t, \YY_1)$ satisfies the semigroup property,
\begin{equation}
\YY(t_1, t, \YY_1) = \YY(\tau, t, \YY(t_1, \tau, \YY_1)), \;\;\;
t\leq \tau< t_1.
\label{semigroup_b}
\end{equation}
\end{enumerate}
For the linear system (\ref{linearrhs}) the backward reach set
can be expressed as
\begin{equation}
\YY(t_1, t, \YY_1) =
\Phi(t, t_1)\YY_1 \oplus \int_{t_1}^t\Phi(t, \tau)B(\tau)\UU(\tau)d\tau
\label{ctlsbrs}
\end{equation}
in the continuous-time case, and as
\begin{equation}
\YY(t_1, t, \YY_1) =
\Phi(t, t_1)\YY_1 \oplus \sum_{\tau =t}^{t_1-1}-\Phi(t, \tau)B(\tau)\UU(\tau)
\tag*{(\ref{ctlsbrs}d)}
\label{dtlsbrs}
\end{equation}
in discrete-time case.
The last formula makes sense only for discrete-time linear systems with
invertible state transition matrix.
Degenerate discrete-time linear systems have unbounded backward reach sets and such sets
cannot be computed with available software tools.

Just as in the case of forward reach set, the backward reach set of
a linear system $\YY(t_1, t, \YY_1)$ is compact and convex if the target set $\YY_1$ and the control sets
$\UU(\tau)$, $t\leq\tau<t_1$, are compact and convex.

{\bf Remark.}
In the computer science literature the reach set is said to be the result
of operator \emph{post}, and the backward reach set is the result of
operator \emph{pre}.
In the control literature the backward reach set is also called the
\emph{solvability set}.


\subsection{Systems with disturbances}\label{subsec_sysdist}
\todo[inline]{This section was changed. I took the information from the previous (more complete) version}
Consider the continuous-time dynamical system with disturbance
\begin{equation}
\dot{x}(t) = f(t, x, u, v),
\label{ctds2}
\end{equation}
or the discrete-time dynamical system with disturbance
\begin{equation}
x(t+1) = f(t, x, u, v),
\tag*{(\ref{ctds2}d)}
\label{dtds2}
\end{equation}
in which
we also have the disturbance input $v\in{\bf R}^d$ with values $v(t)$  restricted
to a closed compact set $\VV(t)\subset{\bf R}^d$.

In the presence of disturbances the open-loop reach set (OLRS) is different from the
closed-loop reach set (CLRS).

Given the initial time $t_0$, the set of initial states $\XX_0$, and
terminal time $t$, there are two types of OLRS.

\bd[OLRS of maxmin type]
The maxmin open-loop reach set $\OXOL(t, t_0, \XX_0)$ is the set of all states $x$,
such that for any disturbance $v(\tau)\in\VV(\tau)$, there exist an initial state
$x_0\in\XX_0$ and a control $u(\tau)\in\UU(\tau)$, $t_0\leq\tau<t$, that
steers system (\ref{ctds2}) or \ref{dtds2} from $x(t_0)=x_0$ to $x(t)=x$.
\label{def_maxminolrs}
\ed
\bd[OLRS of minmax type]
The minmax open-loop reach set $\UXOL(t, t_0, \XX_0)$ is the set of all states $x$,
such that there exists a control $u(\tau)\in\UU(\tau)$ that for all disturbances
$v(\tau)\in\VV(\tau)$, $t_0\leq\tau<t$, assigns an initial state $x_0\in\XX_0$
and steers system (\ref{ctds2}), or \ref{dtds2}, from $x(t_0)=x_0$ to $x(t)=x$.
\label{def_minmaxolrs}
\ed
In the maxmin case the control is chosen \emph{after} knowing the disturbance
over the entire time interval $[t_0, t]$, whereas in the minmax case
the control is chosen \emph{before} any knowledge
of the disturbance.   Consequently, the OLRS do not satisfy the semigroup property.

The terms `maxmin' and `minmax' come from the fact that
$\OXOL(t, t_0, \XX_0)$ is the subzero level set of the value function
\begin{equation}
\underline{V}(t, x) =
\max_v\min_u\{{\bf dist}(x(t_0), \XX_0) ~|~ x(t)=x, \; u(\tau)\in\UU(\tau), \;
v(\tau)\in\VV(\tau), \; t_0\leq\tau<t\},
\label{maxminvf}
\end{equation}
i.e., $\OXOL(t, t_0, \XX_0) = \{ x~|~\underline{V}(t, x) \le 0\}$,
and $\UXOL(t, t_0, \XX_0)$ is the subzero level set of the value function
\begin{equation}
\overline{V}(t, x) =
\min_u\max_v\{{\bf dist}(x(t_0), \XX_0) ~|~ x(t)=x, \; u(\tau)\in\UU(\tau), \;
v(\tau)\in\VV(\tau), \; t_0\leq\tau<t\},
\label{minmaxvf}
\end{equation}
in which ${\bf dist}(\cdot, \cdot)$ denotes Hausdorff
semidistance.\footnote{Hausdorff semidistance between compact sets
$\WW, \ZZ \subseteq {\bf R}^n$ is defined as
\[ {\bf dist}(\WW, \ZZ) = \min\{\langle w-z, w-z\rangle^{1/2}
~|~ w\in\WW, \; z\in\ZZ\}, \]
where $\langle\cdot, \cdot\rangle$ denotes inner product.}
Since $\underline{V}(t, x)\leq\overline{V}(t, x)$,
$\UXOL(t, t_0, \XX_0)\subseteq\OXOL(t, t_0, \XX_0)$.

Note that maxmin and minmax OLRS imply \emph{guarantees}: these are states that can be reached
no matter what the disturbance is, whether it is known in advance
(maxmin case) or not (minmax case).
The OLRS may be empty.


Fixing time instant $\tau_1$, $t_0<\tau_1<t$, define the
\emph{piecewise maxmin open-loop reach set with one correction},
\begin{equation}
\OXOL^1(t, t_0, \XX_0) = \OXOL(t, \tau_1, \OXOL(\tau_1, t_0, \XX_0)),
\label{maxmin1}
\end{equation}
and the \emph{piecewise minmax open-loop reach set with one correction},
\begin{equation}
\UXOL^1(t, t_0, \XX_0) = \UXOL(t, \tau_1, \UXOL(\tau_1, t_0, \XX_0)).
\label{minmax1}
\end{equation}
The piecewise maxmin OLRS $\OXOL^1(t, t_0, \XX_0)$ is the subzero level set
of the value function
\begin{equation}
\underline{V}^1(t, x) =
\max_v\min_u\{\underline{V}(\tau_1, x(\tau_1)) ~|~ x(t)=x, \;
u(\tau)\in\UU(\tau), \; v(\tau)\in\VV(\tau), \; \tau_1\leq\tau<t\},
\label{maxminvf1}
\end{equation}
with $V(\tau_1, x(\tau_1))$ given by (\ref{maxminvf}), which yields
\[ \underline{V}^1(t, x) \geq \underline{V}(t, x), \]
and thus,
\[ \OXOL^1(t, t_0 \XX_0) \subseteq \OXOL(t, t_0, \XX_0) .\]
On the other hand, the piecewise minmax OLRS $\UXOL^1(t, t_0, \XX_0)$
is the subzero level set of the value function
\begin{equation}
\overline{V}^1(t, x) =
\min_u\max_v\{\overline{V}(\tau_1, x(\tau_1)) ~|~ x(t)=x, \;
u(\tau)\in\UU(\tau), \; v(\tau)\in\VV(\tau), \; \tau_1\leq\tau<t\},
\label{minmaxvf1}
\end{equation}
with $V(\tau_1, x(\tau_1))$ given by (\ref{minmaxvf}), which yields
\[ \overline{V}(t, x) \geq \overline{V}^1(t, x), \]
and thus,
\[ \UXOL(t, t_0 \XX_0) \subseteq \UXOL^1(t, t_0, \XX_0) .\]
We can now recursively define piecewise maxmin and minmax OLRS with
$k$ corrections for $t_0<\tau_1<\cdots<\tau_k<t$.
The maxmin piecewise OLRS with $k$ corrections is
\begin{equation}
\OXOL^k(t, t_0, \XX_0) =
\OXOL(t, \tau_k, \OXOL^{k-1}(\tau_k, t_0, \XX_0)),
\label{maxmink}
\end{equation}
which is the subzero level set of the corresponding value function
\begin{eqnarray}
&&\underline{V}^k(t, x) = \nonumber \\
&&\max_v\min_u\{\underline{V}^{k-1}(\tau_k, x(\tau_k)) ~|~ x(t)=x, \;
u(\tau)\in\UU(\tau), \; v(\tau)\in\VV(\tau), \; \tau_k\leq\tau<t\}.
\label{maxminvfk}
\end{eqnarray}
The minmax piecewise OLRS with $k$ corrections is
\begin{equation}
\UXOL^k(t, t_0, \XX_0) =
\UXOL(t, \tau_k, \UXOL^{k-1}(\tau_k, t_0, \XX_0)),
\label{minmaxk}
\end{equation}
which is the subzero level set of the corresponding value function
\begin{eqnarray}
&&\overline{V}^k(t, x) = \nonumber \\
&&\min_u\max_v\{\overline{V}^{k-1}(\tau_k, x(\tau_k)) ~|~ x(t)=x, \;
u(\tau)\in\UU(\tau), \; v(\tau)\in\VV(\tau), \; \tau_k\leq\tau<t\}.
\label{minmaxvfk}
\end{eqnarray}
From (\ref{maxminvf1}), (\ref{minmaxvf1}), (\ref{maxminvfk}) and
(\ref{minmaxvfk}) it follows that
\[ \underline{V}(t, x) \leq \underline{V}^1(t, x)\leq \cdots
\leq \underline{V}^k(t, x) \leq \overline{V}^k(t, x) \leq \cdots
\leq \overline{V}^1(t, x) \leq \overline{V}(t, x) .\]
Hence,
\begin{eqnarray}
&&\UXOL(t, t_0, \XX_0) \subseteq \UXOL^1(t, t_0, \XX_0) \subseteq \cdots
\subseteq \UXOL^k(t, t_0, \XX_0) \subseteq \nonumber \\
&&\OXOL^k(t, t_0, \XX_0) \subseteq \cdots \subseteq \OXOL^1(t, t_0, \XX_0)
\subseteq \OXOL(t, t_0, \XX_0) .
\label{olrsinclusion}
\end{eqnarray}
We call
\begin{equation}
\OXCL(t, t_0, \XX_0) = \OXOL^k(t, t_0, \XX_0), \;\;
k = \left\{\begin{array}{ll}
\infty & \mbox{ for continuous-time system}\\
t-t_0-1 & \mbox{ for discrete-time system}\end{array}\right.
\label{maxminclrs}
\end{equation}
the \emph{maxmin closed-loop reach set} of system (\ref{ctds2}) or \ref{dtds2}
at time $t$, and we call
\begin{equation}
\UXCL(t, t_0, \XX_0) = \UXOL^k(t, t_0, \XX_0), \;\;
k = \left\{\begin{array}{ll}
\infty & \mbox{ for continuous-time system}\\
t-t_0-1 & \mbox{ for discrete-time system}\end{array}\right.
\label{minmaxclrs}
\end{equation}
the \emph{minmax closed-loop reach set} of system (\ref{ctds2}) or \ref{dtds2}
at time $t$.
\bd[CLRS of maxmin type]
Given initial time $t_0$ and the set of initial states $\XX_0$, the maxmin
CLRS $\OXCL(t, t_0, \XX_0)$ of system (\ref{ctds2}) or \ref{dtds2} at time
$t>t_0$, is the set of all states $x$, for each of which and for every disturbance
$v(\tau)\in\VV(\tau)$, there exist an initial state $x_0\in\XX_0$ and a control
$u(\tau, x(\tau))\in\UU(\tau)$, such that the trajectory
$x(\tau | v(\tau), u(\tau, x(\tau)))$ satisfying $x(t_0) = x_0$ and
\[ \dot{x}(\tau | v(\tau), u(\tau, x(\tau))) \in
f(\tau, x(\tau), u(\tau, x(\tau)), v(\tau)) \]
in the continuous-time case, or
\[ x(\tau+1 | v(\tau), u(\tau, x(\tau))) \in
f(\tau, x(\tau), u(\tau, x(\tau)), v(\tau)) \]
in the discrete-time case, with $t_0\leq\tau<t$, is such that $x(t)=x$.
\label{def_maxminclrs}
\ed
\bd[CLRS of minmax type]
Given initial time $t_0$ and the set of initial states $\XX_0$, the maxmin
CLRS $\UXCL(t, t_0, \XX_0)$ of system (\ref{ctds2}) or \ref{dtds2}, at time
$t>t_0$, is the set of all states $x$, for each of which there exists a control
$u(\tau, x(\tau))\in\UU(\tau)$, and for every disturbance $v(\tau)\in\VV(\tau)$
there exists  an initial state $x_0\in\XX_0$, such that the trajectory
$x(\tau, v(\tau) | u(\tau, x(\tau)))$ satisfying $x(t_0) = x_0$ and
\[ \dot{x}(\tau, v(\tau) | u(\tau, x(\tau))) \in
f(\tau, x(\tau), u(\tau, x(\tau)), v(\tau)) \]
in the continuous-time case, or
\[ x(\tau+1, v(\tau) | u(\tau, x(\tau))) \in
f(\tau, x(\tau), u(\tau, x(\tau)), v(\tau)) \]
in the discrete-time case, with $t_0\leq\tau<t$, is such that
$x(t)=x$.
\label{def_minmaxclrs}
\ed
By construction, both maxmin and minmax CLRS satisfy the
semigroup property (\ref{semigroup}).

For some classes of dynamical systems and some types of constraints
on initial conditions, controls and disturbances, the
maxmin and minmax CLRS may coincide.
This is the case for continuous-time linear systems
with convex compact bounds on the initial set, controls and disturbances
under the condition that the initial set $\XX_0$ is large enough to ensure that
$\XX(t_0+\epsilon, t_0, \XX_0)$ is nonempty for some small $\epsilon>0$.

Consider the linear system case,
\begin{equation}
f(t, x, u) = A(t)x(t) + B(t)u + G(t)v,
\label{linearrhsdist}
\end{equation}
where $A(t)$ and $B(t)$ are as in (\ref{linearrhs}), and
$G(t)$ takes its values in ${\bf R}^d$.

The maxmin OLRS for the continuous-time linear system can be expressed through
set valued integrals,
\begin{equation}
\begin{array}{l}
\OXOL(t, t_0, \XX_0) = \\
\left(\Phi(t, t_0)\XX_0 \oplus
\int_{t_0}^t\Phi(t, \tau)B(\tau)\UU(\tau)d\tau\right) \dot{-} \\
\int_{t_0}^t\Phi(t, \tau)(-G(\tau))\VV(\tau)d\tau,
\end{array}
\label{ctlsmaxmin}
\end{equation}
and for discrete-time linear system through set-valued sums,
\begin{equation}
\begin{array}{l}
\OXOL(t, t_0, \XX_0) = \\
\left(\Phi(t, t_0)\XX_0 \oplus
\sum_{\tau=t_0}^{t-1}\Phi(t, \tau+1)B(\tau)\UU(\tau)\right) \dot{-} \\
\sum_{\tau=t_0}^{t-1}\Phi(t, \tau+1)(-G(\tau))\VV(\tau).
\end{array}
\tag*{(\ref{ctlsmaxmin}d)}
\label{dtlsmaxmin}
\end{equation}
Similarly, the minmax OLRS for the continuous-time linear system is
\begin{equation}
\begin{array}{l}
\UXOL(t, t_0, \XX_0) = \\
\left(\Phi(t, t_0)\XX_0 \dot{-}
\int_{t_0}^t\Phi(t, \tau)(-G(\tau))\VV(\tau)d\tau\right)
\oplus \\
\int_{t_0}^t\Phi(t, \tau)B(\tau)\UU(\tau)d\tau,
\end{array}
\label{ctlsminmax}
\end{equation}
and for the discrete-time linear system it is
\begin{equation}
\begin{array}{l}
\UXOL(t, t_0, \XX_0) = \\
\left(\Phi(t, t_0)\XX_0 \dot{-}
\sum_{\tau=t_0}^{t-1}\Phi(t, \tau+1)(-G(\tau))\VV(\tau)\right)
\oplus \\
\sum_{\tau=t_0}^{t-1}\Phi(t, \tau+1)B(\tau)\UU(\tau).
\end{array}
\tag*{(\ref{ctlsminmax}d)}
\label{dtlsminmax}
\end{equation}
The operation `$\dot{-}$' is \emph{geometric difference}, also known as
\emph{Minkowski difference}.\footnote{The Minkowski difference of sets
$\WW, \ZZ \in{\bf R}^n$ is defined as
$\WW\dot{-}\ZZ = \left\{\xi\in{\bf R}^n ~|~
\xi \oplus \ZZ \subseteq \WW\right\}$.
If  $\WW$ and $\ZZ$ are convex,  $\WW\dot{-}\ZZ$ is convex if it is nonempty.}

Now consider the piecewise OLRS with $k$ corrections.
Expression (\ref{maxmink}) translates into
\begin{equation}
\begin{array}{l}
\OXOL^k(t, t_0, \XX_0) = \\
\left(\Phi(t, \tau_k)\OXOL^{k-1}(\tau_k, t_0, \XX_0) \oplus
\int_{\tau_k}^t\Phi(t, \tau)B(\tau)\UU(\tau)d\tau\right) \dot{-} \\
\int_{\tau_k}^t\Phi(t, \tau)(-G(\tau))\VV(\tau)d\tau,
\end{array}
\label{ctlsmaxmink}
\end{equation}
in the continuous-time case, and for the discrete-time case into
\begin{equation}
\begin{array}{l}
\OXOL^k(t, t_0, \XX_0) = \\
\left(\Phi(t, \tau_k)\OXOL^{k-1}(\tau_k, t_0, \XX_0) \oplus
\sum_{\tau=\tau_k}^{t-1}\Phi(t, \tau+1)B(\tau)\UU(\tau)\right) \dot{-} \\
\sum_{\tau=\tau_k}^{t-1}\Phi(t, \tau+1)(-G(\tau))\VV(\tau).
\end{array}
\tag*{(\ref{ctlsmaxmink}d)}
\label{dtlsmaxmink}
\end{equation}
Expression (\ref{minmaxk}) translates into
\begin{equation}
\begin{array}{l}
\UXOL^k(t, t_0, \XX_0) = \\
\left(\Phi(t, \tau_k)\UXOL^{k-1}(t, t_0, \XX_0) \dot{-}
\int_{\tau_k}^t\Phi(t, \tau)(-G(\tau))\VV(\tau)d\tau\right)
\oplus \\
\int_{\tau_k}^t\Phi(t, \tau)B(\tau)\UU(\tau)d\tau,
\end{array}
\label{ctlsminmaxk}
\end{equation}
in the continuous-time case, and for the discrete-time case into
\begin{equation}
\begin{array}{l}
\UXOL^k(t, t_0, \XX_0) = \\
\left(\Phi(t, \tau_k)\UXOL^{k-1}(\tau_k, t_0, \XX_0) \dot{-}
\sum_{\tau=\tau_k}^{t-1}\Phi(t, \tau+1)(-G(\tau))\VV(\tau)\right)
\oplus \\
\sum_{\tau=\tau_k}^{t-1}\Phi(t, \tau+1)B(\tau)\UU(\tau).
\end{array}
\tag*{(\ref{ctlsminmaxk}d)}
\label{dtlsminmaxk}
\end{equation}
Since for any $\WW_1, \WW_2, \WW_3 \subseteq {\bf R}^n$ it is true that
\[ (\WW_1 \dot{-} \WW_2) \oplus \WW_3 =
(\WW_1 \oplus \WW_3) \dot{-} (\WW_2 \oplus \WW_3) \subseteq
(\WW_1 \oplus \WW_3) \dot{-} \WW_2, \]
from (\ref{ctlsmaxmink}), (\ref{ctlsminmaxk})  and from
\ref{dtlsmaxmink}, \ref{dtlsminmaxk}, it is clear that (\ref{olrsinclusion})
is true.

For linear systems,
if the initial set $\XX_0$, control bounds $\UU(\tau)$ and disturbance
bounds $\VV(\tau)$, $t_0\leq\tau<t$, are compact and convex, the
CLRS $\OXCL(t, t_0, \XX_0)$ and $\UXCL(t, t_0, \XX_0)$ are compact and convex, provided they are nonempty.
For continuous-time linear systems,
$\OXCL(t, t_0, \XX_0) = \UXCL(t, t_0, \XX_0) = \XX_{CL}(t, t_0, \XX_0)$.



% Backward reachability


Just as for forward reach sets, the backward reach sets can be open-loop (OLBRS)
or closed-loop (CLBRS).

\bd[OLBRS of maxmin type]
Given the terminal time $t_1$ and target set $\YY_1$, the
maxmin open-loop backward reach set $\OYOL(t_1, t, \YY_1)$
of system (\ref{ctds2}) or \ref{dtds2} at time $t<t_1$, is the set of all $y$,
such that for any disturbance $v(\tau)\in\VV(\tau)$
there exists a terminal state $y_1\in\YY_1$ and control $u(\tau)\in\UU(\tau)$,
$t\leq\tau<t_1$, which steers the system
from $y(t)=y$ to $y(t_1)=y_1$.
\label{def_maxminolbrs}
\ed
$\OYOL(t_1, t, \YY_1)$ is the subzero level set of the value function
\begin{eqnarray}
&&\underline{V}_b(t, y) = \nonumber \\
&&\max_v\min_u\{{\bf dist}(y(t_1), \YY_1) ~|~ y(t)=y, \; u(\tau)\in\UU(\tau), \;
v(\tau)\in\VV(\tau), \; t\leq\tau<t_1\},
\label{maxminvfb}
\end{eqnarray}
\bd[OLBRS of minmax type]
Given the terminal time $t_1$ and target set $\YY_1$, the
minmax open-loop backward reach set $\UYOL(t_1, t, \YY_1)$
of system (\ref{ctds2}) or \ref{dtds2} at time $t<t_1$, is the set of all $y$,
such that there exists a control $u(\tau)\in\UU(\tau)$ that for all
disturbances $v(\tau\in\VV(\tau)$, $t\leq\tau<t_1$, assigns a terminal
state $y_1\in\YY_1$ and steers the system
from $y(t)=y$ to $y(t_1)=y_1$.
\label{def_minmaxolbrs}
\ed
$\UYOL(t_1, t, \YY_1)$ is the subzero level set of the value function
\begin{eqnarray}
&&\overline{V}_b(t, y) = \nonumber \\
&&\min_u\max_v\{{\bf dist}(y(t_1), \YY_1) ~|~ y(t)=y, \; u(\tau)\in\UU(\tau), \;
v(\tau)\in\VV(\tau), \; t\leq\tau<t_1\},
\label{minmaxvfb}
\end{eqnarray}
{\bf Remark.}
The backward reach set can be computed for a continuous-time system only if
the solution of (\ref{ctds2}) exists for $t<t_1$, and for a discrete-time
system only if the right hand side of \ref{dtds2} is invertible.

Similarly to the forward reachability case, we construct piecewise
OLBRS with one correction at time $\tau_1$, $t<\tau_1<t_1$.
The piecewise maxmin OLBRS with one correction is
\begin{equation}
\OYOL^1(t_1, t, \YY_1) = \OYOL(\tau_1, t, \OYOL(t_1, \tau_1, \YY_1)),
\label{maxminb1}
\end{equation}
and it is the subzero level set of the function
\begin{eqnarray}
&&\underline{V}^1_b(t, y) = \nonumber \\
&&\max_v\min_u\{\underline{V}_b(\tau_1, y(\tau_1)) ~|~
y(t)=y, \; u(\tau)\in\UU(\tau), \;
v(\tau)\in\VV(\tau), \; t\leq\tau<\tau_1\}.
\label{maxminvfb1}
\end{eqnarray}
The piecewise minmax OLBRS with one correction is
\begin{equation}
\UYOL^1(t_1, t, \YY_1) = \UYOL(\tau_1, t, \UYOL(t_1, \tau_1, \YY_1)),
\label{minmaxb1}
\end{equation}
and it is the subzero level set of the function
\begin{eqnarray}
&&\overline{V}^1_b(t, y) = \nonumber \\
&&\min_u\max_v\{\overline{V}_b(\tau_1, y(\tau_1)) ~|~
y(t)=y, \; u(\tau)\in\UU(\tau), \;
v(\tau)\in\VV(\tau), \; t\leq\tau<\tau_1\},
\label{minmaxvfb1}
\end{eqnarray}
Recursively define maxmin and minmax OLBRS with $k$ corrections for
$t<\tau_k<\cdots<\tau_1<t_1$.
The maxmin OLBRS with $k$ corrections is
\begin{equation}
\OYOL^k(t_1, t, \YY_1) = \OYOL(\tau_k, t, \OYOL^{k-1}(t_1, \tau_k, \YY_1)),
\label{maxminbk}
\end{equation}
which is the subzero level set of function
\begin{eqnarray}
&&\underline{V}^k_b(t, y) = \nonumber \\
&&\max_v\min_u\{\underline{V}^{k-1}_b(\tau_k, y(\tau_k)) ~|~
y(t)=y, \; u(\tau)\in\UU(\tau), \;
v(\tau)\in\VV(\tau), \; t\leq\tau<\tau_k\}.
\label{maxminvfbk}
\end{eqnarray}
The minmax OLBRS with $k$ corrections is
\begin{equation}
\UYOL^k(t_1, t, \YY_1) = \UYOL(\tau_k, t, \UYOL^{k-1}(t_1, \tau_k, \YY_1)),
\label{minmaxbk}
\end{equation}
which is the subzero level set of the function
\begin{eqnarray}
&&\overline{V}^k_b(t, y) = \nonumber \\
&&\min_u\max_v\{\overline{V}^{k-1}_b(\tau_k, y(\tau_k)) ~|~
y(t)=y, \; u(\tau)\in\UU(\tau), \;
v(\tau)\in\VV(\tau), \; t\leq\tau<\tau_k\},
\label{minmaxvfbk}
\end{eqnarray}
From (\ref{maxminvfb1}), (\ref{minmaxvfb1}), (\ref{maxminvfbk}) and
(\ref{minmaxvfbk}) it follows that
\[ \underline{V}_b(t, y) \leq \underline{V}^1_b(t, y)\leq \cdots
\leq \underline{V}^k_b(t, y) \leq \overline{V}^k_b(t, y) \leq \cdots
\leq \overline{V}^1_b(t, y) \leq \overline{V}_b(t, y) .\]
Hence,
\begin{eqnarray}
&&\UYOL(t_1, t, \YY_1) \subseteq \UYOL^1(t_1, t, \YY_1) \subseteq \cdots
\subseteq \UYOL^k(t_1, t, \YY_1) \subseteq \nonumber \\
&&\OYOL^k(t_1, t, \YY_1) \subseteq \cdots \subseteq \OYOL^1(t_1, t, \YY_1)
\subseteq \OYOL(t_1, t, \YY_1) .
\label{olbrsinclusion}
\end{eqnarray}
We say that
\begin{equation}
\OYCL(t_1, t, \YY_1) = \OYOL^k(t_1, t, \YY_1), \;\;
k = \left\{\begin{array}{ll}
\infty & \mbox{ for continuous-time system}\\
t_1-t-1 & \mbox{ for discrete-time system}\end{array}\right.
\label{maxminclbrs}
\end{equation}
is the \emph{maxmin closed-loop backward reach set}
of system (\ref{ctds2}) or \ref{dtds2} at time $t$.

We say that
\begin{equation}
\UYCL(t_1, t, \YY_1) = \UYOL^k(t_1, t, \YY_1), \;\;
k = \left\{\begin{array}{ll}
\infty & \mbox{ for continuous-time system}\\
t_1-t-1 & \mbox{ for discrete-time system}\end{array}\right.
\label{minmaxclbrs}
\end{equation}
is the \emph{minmax closed-loop backward reach set}
of system (\ref{ctds2}) or \ref{dtds2} at time $t$.
\bd[CLBRS of maxmin type]
Given the terminal time $t_1$ and target set $\YY_1$, the
maxmin CLBRS $\OYCL(t_1, t, \YY_1)$ of
system (\ref{ctds2}) or \ref{dtds2} at time $t<t_1$,
is the set of all states $y$, for each of which for every disturbance
$v(\tau)\in\VV(\tau)$ there exists terminal state $y_1\in\YY_1$ and
control $u(\tau, y(\tau))\in\UU(\tau)$ that assigns
trajectory $y(\tau, | v(\tau), u(\tau, y(\tau)))$ satisfying
\[ \dot{y}(\tau | v(\tau), u(\tau, y(\tau))) \in
f(\tau, y(\tau), u(\tau, y(\tau)), v(\tau)) \]
in continuous-time case, or
\[ y(\tau+1 | v(\tau), u(\tau, y(\tau))) \in
f(\tau, y(\tau), u(\tau, y(\tau)), v(\tau)) \]
in discrete-time case, with $t\leq\tau<t_1$, such that
$y(t) = y$ and $y(t_1)=y_1$.
\label{def_maxminclbrs}
\ed
\bd[CLBRS of minmax type]
Given the terminal time $t_1$ and target set $\YY_1$, the
minmax CLBRS $\UYCL(t_1, t, \YY_1)$ of
system (\ref{ctds2}) or \ref{dtds2} at time $t<t_1$,
is the set of all states $y$, for each of which there exists control
$u(\tau, y(\tau))\in\UU(\tau)$ that for every disturbance
$v(\tau)\in\VV(\tau)$
assigns terminal state $y_1\in\YY_1$ and trajectory
$y(\tau, v(\tau) | u(\tau, y(\tau)))$ satisfying
\[ \dot{y}(\tau, v(\tau) | u(\tau, y(\tau))) \in
f(\tau, y(\tau), u(\tau, y(\tau)), v(\tau)) \]
in the continuous-time case, or
\[ y(\tau+1, v(\tau) | u(\tau, y(\tau))) \in
f(\tau, y(\tau), u(\tau, y(\tau)), v(\tau)) \]
in the discrete-time case, with $t\leq\tau<t_1$, such that
$y(t) = y$ and $y(t_1)=y_1$.
\label{def_minmaxclbrs}
\ed
Both maxmin and minmax CLBRS satisfy the semigroup property (\ref{semigroup_b}).

The maxmin OLBRS for the continuous-time linear system can be expressed through
set valued integrals,
\begin{equation}
\begin{array}{l}
\OYOL(t_1, t, \YY_1) = \\
\left(\Phi(t, t_1)\YY_1 \oplus
\int_{t_1}^t\Phi(t, \tau)B(\tau)\UU(\tau)d\tau\right) \dot{-} \\
\int_{t}^{t_1}\Phi(t, \tau)G(\tau)\VV(\tau)d\tau,
\end{array}
\label{ctlsmaxminb}
\end{equation}
and for the discrete-time linear system through set-valued sums,
\begin{equation}
\begin{array}{l}
\OYOL(t_1, t, \YY_1) = \\
\left(\Phi(t, t_1)\YY_1 \oplus
\sum_{\tau=t}^{t_1-1}-\Phi(t, \tau+1)B(\tau)\UU(\tau)\right) \dot{-} \\
\sum_{\tau=t}^{t_1-1}\Phi(t, \tau+1)G(\tau)\VV(\tau).
\end{array}
\tag*{(\ref{ctlsmaxminb}d)}
\label{dtlsmaxminb}
\end{equation}
Similarly, the minmax OLBRS for the continuous-time linear system is
\begin{equation}
\begin{array}{l}
\UYOL(t_1, t, \YY_1) = \\
\left(\Phi(t, t_1)\YY_1 \dot{-}
\int_{t}^{t_1}\Phi(t, \tau)G(\tau)\VV(\tau)d\tau\right)
\oplus \\
\int_{t_1}^{t}\Phi(t, \tau)B(\tau)\UU(\tau)d\tau,
\end{array}
\label{ctlsminmaxb}
\end{equation}
and for the discrete-time linear system it is
\begin{equation}
\begin{array}{l}
\UYOL(t_1, t, \YY_1) = \\
\left(\Phi(t, t_1)\YY_1 \dot{-}
\sum_{\tau=t}^{t_1-1}\Phi(t, \tau+1)G(\tau)\VV(\tau)\right)
\oplus \\
\sum_{\tau=t}^{t_1-1}-\Phi(t, \tau+1)B(\tau)\UU(\tau).
\end{array}
\tag*{(\ref{ctlsminmaxb}d)}
\label{dtlsminmaxb}
\end{equation}

Now consider piecewise OLBRS with $k$ corrections.
Expression (\ref{maxminbk}) translates into
\begin{equation}
\begin{array}{l}
\OYOL^k(t_1, t, \YY_1) = \\
\left(\Phi(t, \tau_k)\OYOL^{k-1}(t_1, \tau_k, \YY_1) \oplus
\int_{\tau_k}^t\Phi(t, \tau)B(\tau)\UU(\tau)d\tau\right) \dot{-} \\
\int^{\tau_k}_t\Phi(t, \tau)G(\tau)\VV(\tau)d\tau,
\end{array}
\label{ctlsmaxminbk}
\end{equation}
in the continuous-time case, and for the discrete-time case into
\begin{equation}
\begin{array}{l}
\OYOL^k(t_1, t, \YY_1) = \\
\left(\Phi(t, \tau_k)\OYOL^{k-1}(t_1, \tau_k, \YY_1) \oplus
\sum_{\tau=t}^{\tau_k-1}-\Phi(t, \tau+1)B(\tau)\UU(\tau)\right) \dot{-} \\
\sum_{\tau=t}^{\tau_k-1}\Phi(t, \tau+1)G(\tau)\VV(\tau).
\end{array}
\tag*{(\ref{ctlsmaxminbk}d)}
\label{dtlsmaxminbk}
\end{equation}
Expression (\ref{minmaxbk}) translates into
\begin{equation}
\begin{array}{l}
\UYOL^k(t_1, t, \YY_1) = \\
\left(\Phi(t, \tau_k)\OYOL^{k-1}(t_1, \tau_k, \YY_1) \dot{-}
\int^{\tau_k}_t\Phi(t, \tau)G(\tau)\VV(\tau)d\tau\right)
\oplus \\
\int_{\tau_k}^t\Phi(t, \tau)B(\tau)\UU(\tau)d\tau,
\end{array}
\label{ctlsminmaxbk}
\end{equation}
in the continuous-time case, and for the discrete-time case into
\begin{equation}
\begin{array}{l}
\UYOL^k(t_1, t, \YY_1) = \\
\left(\Phi(t, \tau_k)\OYOL^{k-1}(t_1, \tau_k, \YY_1) \dot{-}
\sum_{\tau=t}^{\tau_k-1}\Phi(t, \tau+1)G(\tau)\VV(\tau)\right)
\oplus \\
\sum_{\tau=t}^{\tau_k-1}-\Phi(t, \tau+1)B(\tau)\UU(\tau).
\end{array}
\tag*{(\ref{ctlsminmaxk}d)}
\label{dtlsminmaxbk}
\end{equation}

For continuous-time linear systems
$\OYCL(t_1, t, \YY_1) = \UYCL(t_1, t, \YY_1) = \YY_{CL}(t_1, t, \YY_1)$
under the condition that the target set $\YY_1$ is large enough to ensure
that $\UYCL(t_1, t_1-\epsilon, \YY_1)$ is nonempty for some small $\epsilon>0$.

Computation of backward reach sets for discrete-time linear systems
makes sense only if the state transition matrix $\Phi(t_1, t)$ is invertible.

If the target set $\YY_1$, control sets $\UU(\tau)$ and disturbance
sets $\VV(\tau)$, $t\leq\tau<t_1$, are compact and convex, then
CLBRS $\OYCL(t_1, t, \YY_1)$ and $\UYCL(t_1, t, \YY_1)$ are compact and convex, if they are nonempty.


























\subsection{Reachability problem}
Reachability analysis is concerned with the computation of the forward
$\XX(t, t_0, \XX_0)$ and backward  $\YY(t_1, t, \YY_1)$ reach sets (the
reach sets may  be maxmin or minmax)
in a way that can effectively meet requests like the following:
\begin{enumerate}
\item For the given time interval $[t_0, t]$,
determine whether the system can be steered into the given target set $\YY_1$.
In other words, is the set
$\YY_1\cap\bigcup_{t_0 \leq\tau\leq t}\XX(\tau, t_0, \XX_0)$ nonempty?
And if the answer is `yes', find a  control that steers the system to the target set (or avoids the target set).\footnote{So-called verification problems often consist in ensuring
 that the system is unable to reach an `unsafe' target set within a given
time interval.}

\item If the target set $\YY_1$ is reachable from the given initial
condition $\{t_0, \XX_0\}$ in the time interval $[t_0, t]$,
find the shortest time to reach $\YY_1$,
\[ \arg\min_{\tau}
\{\XX(\tau,t_0,\XX_0)\cap\YY_1\neq\emptyset ~|~ t_0\leq\tau\leq t\}. \]

\item Given the terminal time $t_1$, target set $\YY_1$ and time $t<t_1$
find the set of states starting at time $t$ from which  the system can reach
$\YY_1$ within time interval $[t, t_1]$.
In other words, find $\bigcup_{t\leq\tau<t_1}\YY(t_1, \tau, \YY_1)$.

\item Find a closed-loop control that steers a system with disturbances
to the given target set in given time.

\item Graphically display the projection of the reach set along any
specified two- or three-dimensional subspace.
\end{enumerate}
For linear systems, if the initial set $\XX_0$, target set $\YY_1$, control
bounds $\UU(\cdot)$ and disturbance bounds $\VV(\cdot)$ are compact and convex,
so are the forward $\XX(t, t_0, \XX_0)$ and backward $\YY(t_1, t, \YY_1)$
reach sets.
Hence reachability analysis requires the computationally effective manipulation
of convex sets, and performing the set-valued operations of unions,
intersections, geometric sums and differences.

Existing reach set computation tools can deal reliably only with linear
systems with convex constraints.
A claim that certain tool or method can be used \emph{effectively}
for nonlinear systems must be treated with caution,
and the first question to ask is for what class of nonlinear systems
and with what limit on the state space dimension does this tool work?
Some ``reachability methods for nonlinear systems'' reduce to
the local linearization of a system followed by the use of well-tested techniques
for linear system reach set computation.
Thus these approaches in fact use reachability methods for
linear systems.











