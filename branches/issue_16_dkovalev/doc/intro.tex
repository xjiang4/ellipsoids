Traditional control theory is concerned with the design
of linear feedback control with desirable asymptotic behavior,
such as stability and small steady state and tracking errors,
while properties of transient behavior are expressed in terms
of overshoot and speed of response.
External disturbances can be handled by modeling these as random processes,
leading to the Linear Quadratic Gaussian (LQG) problem formulation.
This theory has some limitations.  

Because the feedback law is specified to be linear,
it is not possible for design methods to explicitly incorporate hard bounds
on the control values, e.g.,
the requirement that the applied force should not exceed a specified limit.
Second, it is not possible to express finite time requirements, e.g.,
the requirement that the system state reach a pre-specified value
at a pre-specified time.
Third, it is not possible to demand guaranteed performance in the face
of disturbances, e.g, the requirement that a certain target state be reached,
no matter what the disturbance.
Thus control problems with hard bounds on the control values,
restrictions on the state trajectory over a finite time horizon,
and guaranteed behavior despite the disturbances,
are difficult to solve using frequency based design methods.

In order to address these problems one needs to study system evolution
in the time domain.
The central concept that emerges in  such studies is that of the \emph{reach}
set, which is the set of states that can be reached by using all possible
controls. 
This chapter is devoted to the formulation and computation
of the reach set of a linear system with disturbances.
The concept of reachability was introduced in \cite{leemarkus};
\cite{leitmann82} shows the reach set can be computed by solving
the forward Hamilton-Jacobi-Bellman-Isaacs (HJBI) partial differential
equation; and the notion of backward reachability with its application
to aiming at a specified target set is described in \cite{krasovski}.
Reachability of hybrid systems is addressed in \cite{purvar94, lygeros99}.
Over the last decade, significant advances were made in
the characterization of reach sets and their computation for linear systems.









{\it Ellipsoidal Toolbox} (ET) implements in MATLAB
the ellipsoidal calculus \cite{kurvalyi}
and its application to the reachability analysis of continuous-time
\cite{kurvar01}, discrete-time \cite{kurvar07}, possibly time-varying linear systems,
and linear systems with disturbances \cite{kurvar2},
for which ET calculates both open-loop and close-loop reach sets.
The ellipsoidal calculus provides the following benefits:
\begin{itemize}
\item The complexity of the
ellipsoidal representation is quadratic in the dimension of
the state space, and linear in the number of time steps.
\item It is possible to exactly represent the reach set of
linear system through both external and internal ellipsoids.
\item It is possible to single out individual external and internal
approximating ellipsoids that are optimal to some given criterion
(e.g. trace, volume, diameter), or combination of such criteria.
\item We obtain simple analytical expressions for the control
that steers the state to a desired target.
\end{itemize}
The report is organized as follows.
\newline
Chapter \ref{ch_ellcalc} describes the operations of the
ellipsoidal calculus: affine transformation, geometric sum,
geometric difference, intersections with
hyperplane, ellipsoid, halfspace and polytope.
\newline
Chapter \ref{ch_reachability} presents the reachability problem and
ellipsoidal methods for the reach set approximation.
\newline
Chapter \ref{ch_install} contains {\it Ellipsoidal Toolbox} installation
and quick start instructions, and lists the software packages
used by the toolbox.
\newline
Chapter \ref{ch_implementation} describes the implementation of methods
from Chapters \ref{ch_ellcalc} and \ref{ch_reachability}
and visualization routines.
\newline
Chapter \ref{ch_objects} describes structures and objects implemented and
used in the toolbox.
\newline
Chapter \ref{ch_examples} gives examples of how to use the toolbox.
\newline
Chapter \ref{ch_summary} collects some conclusions and plans for future
toolbox development.
\newline
The functions provided by the toolbox together with their descriptions
are listed in Appendix A.



