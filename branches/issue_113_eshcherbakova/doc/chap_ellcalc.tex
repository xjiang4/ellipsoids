\section{Basic Notions}
We start with basic definitions.
\bd
Ellipsoid $\EE(q,Q)$ in ${\bf R}^n$ with  center $q$
and  shape matrix $Q$ is the set
\begin{equation}
\EE(q,Q) = \{ x \in {\bf R}^n ~|~ \langle (x-q), Q^{-1}(x-q)\rangle\leq1 \},
\label{ellipsoid}
\end{equation}
wherein  $Q$ is positive definite ($Q=Q^T$ and $\langle x, Qx\rangle>0$
for all nonzero $x\in{\bf R}^n$).
\label{ellipsoiddef0}
\ed
Here $\langle\cdot,\cdot\rangle$ denotes inner product.
\bd
The support function of a set $\XX\subseteq{\bf R}^n$ is
\[ \rho(l~|~\XX) = \sup_{x\in\XX} \langle l,x\rangle. \]
\ed
In particular, the support function of the ellipsoid (\ref{ellipsoid}) is
\begin{equation}
\rho(l~|~\EE(q,Q)) = \langle l, q\rangle + \langle l, Ql\rangle^{1/2}.
\label{ellsupp}
\end{equation}
Although in (\ref{ellipsoid})    $Q$ is assumed to be
positive definite, in practice we may deal with situations when $Q$ is
singular, that is, with degenerate ellipsoids flat in those directions
for which the corresponding eigenvalues are zero. Therefore, it is
useful to give an alternative definition of an ellipsoid using the
expression (\ref{ellsupp}).
\bd
Ellipsoid $\EE(q,Q)$ in ${\bf R}^n$ with  center $q$
and  shape matrix $Q$ is the set
\begin{equation}
\EE(q,Q) = \{ x \in {\bf R}^n ~|~
\langle l,x\rangle\leq\langle l,q\rangle + \langle l,Ql\rangle^{1/2}
\mbox{ for all } l\in{\bf R}^n \},
\label{ellipsoid2}
\end{equation}
wherein matrix $Q$ is positive semidefinite
($Q=Q^T$ and $\langle x, Qx\rangle\geq0$ for all $x\in{\bf R}^n$).
\label{ellipsoiddef}
\ed
\todo[inline]{Added this definition}
The volume of ellipsoid $\EE(q,Q)$ is
\begin{equation}
{\bf Vol}(E(q,Q)) = {\bf Vol}_{\langle x,x\rangle\leq1}\sqrt{\det Q},
\label{ellvolume}
\end{equation}
where ${\bf Vol}_{\langle x,x\rangle\leq 1}$ is the volume of the unit ball
in ${\bf R}^n$:
\begin{equation}
{\bf Vol}_{\langle x,x\rangle\leq 1} = \left\{\begin{array}{ll}
\frac{\pi^{n/2}}{(n/2)!}, &
\mbox{ for even } n,\\
\frac{2^n\pi^{(n-1)/2}\left((n-1)/2\right)!}{n!}, &
\mbox{ for odd } n. \end{array}\right.
\label{ellunitball}
\end{equation}
The distance from $\EE(q,Q)$ to the fixed point $a$ is
\begin{equation}
{\bf dist}(\EE(q,Q),a) = \max_{\langle l,l\rangle=1}\left(\langle l,a\rangle -
\rho(l ~|~ \EE(q,Q)) \right) =
\max_{\langle l,l\rangle=1}\left(\langle l,a\rangle - \langle l,q\rangle -
\langle l,Ql\rangle^{1/2}\right). \label{dist_point}
\end{equation}
If ${\bf dist}(\EE(q,Q),a) > 0$, $a$ lies outside  $\EE(q,Q)$;
if ${\bf dist}(\EE(q,Q),a) = 0$, $a$ is a boundary point of $\EE(q,Q)$;
if ${\bf dist}(\EE(q,Q),a) < 0$, $a$ is an internal point of $\EE(q,Q)$.

\todo[inline]{Added this definition}
Given two ellipsoids, $\EE(q_1,Q_1)$ and $\EE(q_2,Q_2)$, the distance between
them is
\begin{eqnarray}
{\bf dist}(\EE(q_1,Q_1),\EE(q_2,Q_2)) & = & \max_{\langle l,l\rangle=1}
\left(-\rho(-l ~|~ \EE(q_1,Q_1)) - \rho(l ~|~ \EE(q_2,Q_2))\right) \\
& = & \max_{\langle l,l\rangle=1}\left(\langle l,q_1\rangle -
\langle l,Q_1l\rangle^{1/2} - \langle l,q_2\rangle -
\langle l,Q_2l\rangle^{1/2}\right). \label{dist_ell}
\end{eqnarray}
If ${\bf dist}(\EE(q_1,Q_1),\EE(q_2,Q_2)) > 0$,  the ellipsoids have no
common points;
if ${\bf dist}(\EE(q_1,Q_1),\EE(q_2,Q_2)) = 0$,  the ellipsoids have one
common point - they touch;
if ${\bf dist}(\EE(q_1,Q_1),\EE(q_2,Q_2)) < 0$,  the ellipsoids intersect.

Finding ${\bf dist}(\EE(q_1,Q_1),\EE(q_2,Q_2))$ using QCQP is
\[ d(\EE(q_1,Q_1),\EE(q_2,Q_2)) = \min \langle (x-y), (x-y)\rangle \]
subject to:
\begin{eqnarray*}
\langle (q_1-x), Q_1^{-1}(q_1-x)\rangle & \leq & 1,\\
\langle (q_2-x), Q_2^{-1}(q_2-y)\rangle & \leq & 1,
\end{eqnarray*}
where
\[ d(\EE(q_1,Q_1),\EE(q_2,Q_2))=\left\{\begin{array}{ll}
{\bf dist}^2(\EE(q_1,Q_1),\EE(q_2,Q_2)) &
\mbox{ if } {\bf dist}(\EE(q_1,Q_1),\EE(q_2,Q_2))>0, \\
0 & \mbox{ otherwise}. \end{array}\right. \]

Checking if $k$ nondegenerate ellipsoids $\EE(q_1,Q_1),\cdots,\EE(q_k,Q_k)$
have nonempty intersection, can be cast as a quadratically constrained
quadratic programming (QCQP) problem:
\[ \min 0 \]
subject to:
\[ \langle (x-q_i),Q_i^{-1}(x-q_i)\rangle - 1 \leq 0, ~~~ i=1,\cdots,k. \]
If this problem is feasible,  the intersection is nonempty.
\bd
Given compact convex set $\XX\subseteq{\bf R}^n$, its polar set, denoted
$\XX^\circ$, is
\[ \XX^\circ = \{x\in{\bf R}^n ~|~ \langle x,y\rangle\leq 1, ~ y\in\XX\}, \]
or, equivalently,
\[ \XX^\circ = \{l\in{\bf R}^n ~|~ \rho(l ~|~ \XX)\leq 1\}. \]
\ed
The properties of the polar set are
\begin{itemize}
\item If  $\XX$ contains the origin,  $(\XX^\circ)^\circ = \XX$;
\item If $\XX_1\subseteq\XX_2$,  $\XX_2^\circ\subseteq\XX_1^\circ$;
\item For any nonsingular matrix $A\in{\bf R}^{n\times n}$,
$(A\XX)^\circ = (A^T)^{-1}\XX^\circ$.
\end{itemize}
If a nondegenerate ellipsoid $\EE(q,Q)$ contains the origin,
 its polar set is also an ellipsoid:
\begin{eqnarray*}
\EE^\circ(q,Q) & = & \{l\in{\bf R}^n ~|~ \langle l,q\rangle +
\langle l,Ql\rangle^{1/2}\leq1 \}\\
& = & \{l\in{\bf R}^n ~|~ \langle l,(Q-qq^T)^{-1}l\rangle +
2\langle l,q\rangle\leq1 \}\\
& = & \{l\in{\bf R}^n ~|~ \langle(l+(Q-qq^T)^{-1}q),
(Q-qq^T)(l+(Q-qq^T)^{-1}q)\rangle\leq1+\langle q,(Q-qq^T)^{-1}q\rangle \}.
\end{eqnarray*}
The special case is
\[ \EE^\circ(0,Q) = \EE(0,Q^{-1}). \]
\bd
Given $k$ compact sets $\XX_1, \cdots, \XX_k\subseteq{\bf R}^n$,
their geometric (Minkowski) sum is
\begin{equation}
\XX_1\oplus\cdots\oplus\XX_k=\bigcup_{x_1\in\XX_1}\cdots\bigcup_{x_k\in\XX_k}
\{x_1 + \cdots + x_k\} .  \label{minksum}
\end{equation}
\ed
\bd
Given two compact sets $\XX_1, \XX_2 \subseteq{\bf R}^n$, their geometric
(Minkowski) difference is
\begin{equation}
\XX_1\dot{-}\XX_2 = \{x\in{\bf R}^n ~|~ x + \XX_2 \subseteq \XX_1 \}.
\label{minkdiff}
\end{equation}
\ed
Ellipsoidal calculus concerns the following set of operations:
\begin{itemize}
\item affine transformation of ellipsoid;
\item geometric sum of finite number of ellipsoids;
\item geometric difference of two ellipsoids;
\item intersection of finite number of ellipsoids.
\end{itemize}
These operations occur in reachability calculation and verification
of piecewise affine dynamical systems. The result of all of these operations,
except for the affine transformation, is \emph{not} generally an ellipsoid
but some convex set, for which we can compute external and internal ellipsoidal
approximations.

Additional operations implemented in the {\it Ellipsoidal Toolbox} include
external and internal approximations of intersections of ellipsoids with
hyperplanes, halfspaces and polytopes.
\bd
Hyperplane $H(c,\gamma)$ in ${\bf R}^n$ is the set
\begin{equation}
H = \{x\in{\bf R}^n ~|~ \langle c, x\rangle = \gamma\}
\label{hyperplane}
\end{equation}
with $c\in{\bf R}^n$ and $\gamma\in{\bf R}$ fixed.
\label{hyperplanedef}
\ed
The distance from ellipsoid $\EE(q,Q)$ to hyperplane $H(c,\gamma)$ is
\begin{equation}
{\bf dist}(\EE(q,Q),H(c,\gamma)) =
\frac{\left|\gamma-\langle c,q\rangle\right| -
\langle c,Qc\rangle^{1/2}}{\langle c,c\rangle^{1/2}}. \label{dist_hp}
\end{equation}
If ${\bf dist}(\EE(q,Q),H(c,\gamma))>0$, the ellipsoid and the hyperplane
do not intersect;
if ${\bf dist}(\EE(q,Q),H(c,\gamma))=0$, the hyperplane is a supporting
hyperplane for the ellipsoid;
if ${\bf dist}(\EE(q,Q),H(c,\gamma))<0$, the ellipsoid intersects the
hyperplane.
The intersection of an ellipsoid with a hyperplane is always an ellipsoid
and can be computed directly.

Checking if the intersection of $k$ nondegenerate ellipsoids
$E(q_1,Q_1),\cdots,\EE(q_k,Q_k)$ intersects  hyperplane $H(c,\gamma)$,
is equivalent to the feasibility check of the QCQP problem:
\[ \min 0 \]
subject to:
\begin{eqnarray*}
\langle (x-q_i),Q_i^{-1}(x-q_i)\rangle - 1 \leq 0, & & i=1,\cdots,k,\\
\langle c, x\rangle - \gamma = 0. & &
\end{eqnarray*}
A hyperplane defines two (closed) {\it halfspaces}:
\begin{equation}
{\bf S}_1 = \{x\in{\bf R}^n ~|~ \langle c, x\rangle \leq \gamma\}
\label{halfspace1}
\end{equation}
and
\begin{equation}
{\bf S}_2 = \{x\in{\bf R}^n ~|~ \langle c, x\rangle \geq \gamma\}.
\label{halfspace2}
\end{equation}
To avoid confusion, however, we shall further assume that a
 hyperplane $H(c,\gamma)$ specifies the halfspace in the sense
(\ref{halfspace1}). In order to refer to the
other halfspace, the same hyperplane should be defined as $H(-c,-\gamma)$.

The idea behind the calculation of intersection of an ellipsoid with a
halfspace is to treat the halfspace as an unbounded ellipsoid, that is, as the
ellipsoid with the shape matrix  all but one of whose eigenvalues are $\infty$.
\bd\label{polytope}
Polytope $P(C,g)$ is the  intersection of a finite number
of closed halfspaces:
\[ P = \{x\in{\bf R}^n ~|~ Cx\leq g\}, \]
wherein $C=[c_1 ~ \cdots ~ c_m]^T\in{\bf R}^{m\times n}$ and
$g=[\gamma_1 ~ \cdots ~ \gamma_m]^T\in{\bf R}^m$.
\ed
The distance from ellipsoid $\EE(q,Q)$ to the polytope $P(C,g)$ is
\begin{equation}
{\bf dist}(\EE(q,Q),P(C,g))=\min_{y\in P(C,g)}{\bf dist}(\EE(q,Q),y),
\label{dist_poly}
\end{equation}
where ${\bf dist}(\EE(q,Q),y)$ comes from (\ref{dist_point}).
If ${\bf dist}(\EE(q,Q),P(C,g))>0$, the ellipsoid and the polytope
do not intersect;
if ${\bf dist}(\EE(q,Q),P(C,g))=0$, the ellipsoid touches the polytope;
if ${\bf dist}(\EE(q,Q),P(C,g))<0$, the ellipsoid intersects the
polytope.

Checking if the intersection of $k$ nondegenerate ellipsoids
$E(q_1,Q_1),\cdots,\EE(q_k,Q_k)$ intersects  polytope $P(C,g)$
is equivalent to the feasibility check of the QCQP problem:
\[ \min 0 \]
subject to:
\begin{eqnarray*}
\langle (x-q_i),Q_i^{-1}(x-q_i)\rangle - 1 \leq 0, & & i=1,\cdots,k,\\
\langle c_j, x\rangle - \gamma_j \leq 0, & & j=1,\cdots,m.
\end{eqnarray*}



\section{Operations with Ellipsoids}
\subsection{Affine Transformation}
The simplest operation with ellipsoids is an affine transformation.
Let ellipsoid $\EE(q,Q)\subseteq{\bf R}^n$, matrix $A\in{\bf R}^{m\times n}$
and vector $b\in{\bf R}^m$. Then
\begin{equation}
A\EE(q,Q) + b = \EE(Aq+b, AQA^T) .\label{affinetrans}
\end{equation}
Thus, ellipsoids are preserved under affine transformation.
If the rows of $A$ are linearly independent (which implies  $m\leq n$), and
$b=0$, the affine transformation is called {\it projection}.


\subsection{Geometric Sum}
Consider the geometric sum (\ref{minksum}) in which $\XX_1,\cdots$,$\XX_k$
are  nondegenerate ellipsoids
$\EE(q_1,Q_1),\cdots$, $\EE(q_k,Q_k)\subseteq{\bf R}^n$.
The resulting set is not generally  an ellipsoid.
However, it can be tightly approximated by the parametrized families
of external and internal ellipsoids.

Let parameter $l$ be some nonzero vector in ${\bf R}^n$. Then the external
approximation $\EE(q,Q_l^+)$ and the internal approximation $\EE(q,Q_l^-)$
of the sum $\EE(q_1,Q_1)\oplus\cdots\oplus\EE(q_k,Q_k)$ are \emph{tight} along
direction $l$, i.e.,
\[ \EE(q,Q_l^-)\subseteq\EE(q_1,Q_1)\oplus\cdots\oplus\EE(q_k,Q_k)
\subseteq\EE(q,Q_l^+) \]
and
\[ \rho(\pm l ~|~ \EE(q,Q_l^-)) =
\rho(\pm l ~|~ \EE(q_1,Q_1)\oplus\cdots\oplus\EE(q_k,Q_k)) =
\rho(\pm l ~|~ \EE(q,Q_l^+)).\]
Here the center $q$ is
\begin{equation}
q = q_1 + \cdots + q_k , \label{minksum_c}
\end{equation}
the shape matrix of the external ellipsoid $Q_l^+$ is
\begin{equation}
Q_l^+ = \left(\langle l,Q_1l\rangle^{1/2} + \cdots
+ \langle l,Q_kl\rangle^{1/2}\right)
\left(\frac{1}{\langle l,Q_1l\rangle^{1/2}}Q_1 + \cdots +
\frac{1}{\langle l,Q_kl\rangle^{1/2}}Q_k\right), \label{minksum_ea}
\end{equation}
and the shape matrix of the internal ellipsoid $Q_l^-$ is
\begin{equation}
Q_l^- = \left(Q_1^{1/2} + S_2Q_2^{1/2} + \cdots + S_kQ_k^{1/2}\right)^T
\left(Q_1^{1/2} + S_2Q_2^{1/2} + \cdots + S_kQ_k^{1/2}\right),\label{minksum_ia}
\end{equation}
with matrices $S_i$, $i=2,\cdots,k$, being orthogonal ($S_iS_i^T=I$) and such
that vectors $Q_1^{1/2}l, S_2Q_2^{1/2}l, \cdots, S_kQ_k^{1/2}l$ are parallel.

Varying vector $l$ we get exact external and internal approximations,
\[ \bigcup_{\langle l,l\rangle=1} \EE(q,Q_l^-) =
\EE(q_1,Q_1)\oplus\cdots\oplus\EE(q_k,Q_k) =
\bigcap_{\langle l,l\rangle=1} \EE(q,Q_l^+) .\]
For proofs of formulas given in this section, see \cite{KURZHANSKI_VALYI_ELLIPSOIDAL_CALCULUS_FOR_ESTINATION_AND_CONTROL},
\cite{KURZHANSKI_VARAIYA_ON_ELLIPSOIDAL_TECHNIQUES_FOR_REACHABILITY_ANALYSIS}.

One last comment is about how to find orthogonal matrices $S_2,\cdots,S_k$
that align vectors $Q_2^{1/2}l, \cdots, Q_k^{1/2}l$ with $Q_1^{1/2}l$.
Let $v_1$ and $v_2$ be some unit vectors in ${\bf R}^n$.
\todo[inline]{Algorithms of S matrix calculation have been changed}
We have to find matrix $S$ such that $Sv_2\uparrow\uparrow v_1$.
We suggest explicit formulas for the calculation of this matrix ( \cite {DARYIN_KURZHANSKY_METHOD_CALC_INV_SETS_UNCERT_MIXED}):
\begin{eqnarray}
&&T = I + Q_1(S - I)Q_1^T,\\ \label{valign1}
&& S = \begin{pmatrix}
     c & s\\
     -s & c
    \end{pmatrix},\quad c = \langle\hat{v_1}, \hat{v_2}\rangle,\quad s = \sqrt{1 - c^2},\quad \hat{v_i} = \dfrac{v_i}{\|v_i\|}\\ \label{valign2}
&&  Q_1 = [q_1 \,q_2]\in \mathbb{R}^{n\times2},\quad q_1 = \hat{v_1}, \quad
q_2 = \begin{cases}
s^{-1}(\hat{v_2} - c\hat{v_1}),& s\ne 0\\
0,& s = 0.
\end{cases} \label{valign3}
\end{eqnarray}



\subsection{Geometric Difference}
Consider the geometric difference (\ref{minkdiff}) in which the sets $\XX_1$ and
$\XX_2$ are nondegenerate ellipsoids $\EE(q_1,Q_1)$ and $\EE(q_2,Q_2)$.
We say that ellipsoid $\EE(q_1,Q_1)$ is {\it bigger} than ellipsoid
$\EE(q_2,Q_2)$ if
\[ \EE(0,Q_2) \subseteq \EE(0,Q_1). \]
If this condition is not fulfilled,  the geometric difference
$\EE(q_1,Q_1)\dot{-}\EE(q_2,Q_2)$ is an empty set:
\[ \EE(0,Q_2) \not\subseteq \EE(0,Q_1) ~~~ \Rightarrow ~~~
\EE(q_1,Q_1) \dot{-}\EE(q_2,Q_2) = \emptyset. \]
If $\EE(q_1,Q_1)$ is bigger than $\EE(q_2,Q_2)$ and
$\EE(q_2,Q_2)$ is bigger than $\EE(q_1,Q_1)$, in other words, if $Q_1=Q_2$,
\[ \EE(q_1,Q_1) \dot{-}\EE(q_2,Q_2) = \{q_1-q_2\} ~~~ \mbox{and} ~~~
\EE(q_2,Q_2) \dot{-}\EE(q_1,Q_1) = \{q_2-q_1\}. \]
To check if ellipsoid $\EE(q_1,Q_1)$ is bigger than ellipsoid $\EE(q_2,Q_2)$,
we perform simultaneous diagonalization of matrices $Q_1$ and $Q_2$, that is,
we find matrix $T$ such that
\[ TQ_1T^T = I ~~~ \mbox{and} ~~~ TQ_2T^T=D, \]
where $D$ is some diagonal matrix.
Simultaneous diagonalization of $Q_1$ and $Q_2$ is possible
because both are symmetric positive definite (see \cite{GANTMACHER_MATRIX_THEORY}).
To find such matrix $T$, we first do the SVD of $Q_1$:
\begin{equation}
Q_1 = U_1\Sigma_1V_1^T . \label{simdiag1}
\end{equation}
Then the SVD of matrix $\Sigma_1^{-1/2}U_1^TQ_2U_1\Sigma_1^{-1/2}$:
\begin{equation}
\Sigma_1^{-1/2}U_1^TQ_2U_1\Sigma_1^{-1/2} = U_2\Sigma_2V_2^T. \label{simdiag2}
\end{equation}
Now, $T$ is defined as
\begin{equation}
T = U_2^T \Sigma_1^{-1/2}U_1^T.  \label{simdiag3}
\end{equation}
If the biggest diagonal element (eigenvalue) of matrix $D=TQ_2T^T$ is less than
or equal to $1$,  $\EE(0,Q_2)\subseteq\EE(0,Q_1)$.

Once it is established that ellipsoid $\EE(q_1,Q_1)$ is bigger than
ellipsoid $\EE(q_2,Q_2)$, we know that their geometric difference
$\EE(q_1,Q_1)\dot{-}\EE(q_2,Q_2)$ is a nonempty convex compact set.
Although  it is not generally an ellipsoid, we can find tight external
and internal approximations of this set parametrized by vector $l\in{\bf R}^n$.
Unlike geometric sum, however, ellipsoidal approximations for the geometric
difference do not exist  for every direction $l$.
Vectors for which the approximations do not exist are called
{\it bad directions}.

Given two ellipsoids $\EE(q_1,Q_1)$ and $\EE(q_2,Q_2)$ with
$\EE(0,Q_2)\subseteq\EE(0,Q_1)$, $l$ is a bad direction if
\[ \frac{\langle l,Q_1l\rangle^{1/2}}{\langle l,Q_2l\rangle^{1/2}}>r, \]
in which $r$ is a minimal root of the equation
\[ {\bf det}(Q_1-rQ_2) = 0. \]
To find $r$, compute matrix $T$ by (\ref{simdiag1}-\ref{simdiag3}) and define
\[ r = \frac{1}{\max({\bf diag}(TQ_2T^T))}. \]
If $l$ is {\it not} a bad direction, we can find tight external
and internal ellipsoidal approximations $\EE(q,Q^+_l)$ and
$\EE(q,Q^-_l)$ such that
\[ \EE(q,Q^-_l)\subseteq\EE(q_1,Q_1)\dot{-}\EE(q_2,Q_2)\subseteq\EE(q,Q^+_l) \]
and
\[ \rho(\pm l ~|~ \EE(q,Q_l^-)) =
\rho(\pm l ~|~ \EE(q_1,Q_1)\dot{-}\EE(q_2,Q_2)) =
\rho(\pm l ~|~ \EE(q,Q_l^+)).\]
The center $q$ is
\begin{equation}
q = q_1 - q_2;  \label{minkdiff_c}
\end{equation}
the shape matrix of the internal ellipsoid $Q^-_l$ is
\todo[inline]{Fixed misprints in 2.2.12, 2.2.13}
\begin{eqnarray}
&& P = \frac{\sqrt{\langle l, Q_1 l\rangle}}{\sqrt{\langle l, Q_2 \rangle}};\nonumber\\
&& Q^-_l = \left(1 - \dfrac{1}{P}\right)Q_1 + \left(1 - P\right)Q_2.
\label{minkdiff_ia}
\end{eqnarray}
and the shape matrix of the external ellipsoid $Q^+_l$ is
\begin{equation}
Q^+_l = \left(Q_1^{1/2} - SQ_2^{1/2}\right)^T
\left(Q_1^{1/2} - SQ_2^{1/2}\right).  \label{minkdiff_ea}
\end{equation}
Here $S$ is an orthogonal matrix such that vectors $Q_1^{1/2}l$
and $SQ_2^{1/2}l$ are parallel.
 $S$ is found from (\ref{valign1}-\ref{valign3}), with
$v_1=Q_2^{1/2}l$ and $v_2=Q_1^{1/2}l$.

Running $l$ over all unit directions that are not bad, we get
\[ \bigcup_{\langle l,l\rangle=1} \EE(q,Q_l^-) =
\EE(q_1,Q_1)\dot{-}\EE(q_2,Q_2) =
\bigcap_{\langle l,l\rangle=1} \EE(q,Q_l^+) .\]
For proofs of formulas given in this section, see \cite{KURZHANSKI_VALYI_ELLIPSOIDAL_CALCULUS_FOR_ESTINATION_AND_CONTROL}.


\subsection{Geometric Difference-Sum}\label{subsec_diffsum}
Given ellipsoids $\EE(q_1,Q_1)$, $\EE(q_2,Q_2)$ and $\EE(q_3,Q_3)$, it is
possible to compute families of external and internal approximating
ellipsoids for
\begin{equation}
\EE(q_1,Q_1) \dot{-} \EE(q_2,Q_2) \oplus \EE(q_3,Q_3) \label{minkmp}
\end{equation}
parametrized by direction $l$, if this set is nonempty
($\EE(0,Q_2)\subseteq\EE(0,Q_1)$).

First, using the result of the previous section, for any direction $l$ that
is not bad, we obtain tight external $\EE(q_1-q_2, Q_l^{0+})$ and internal
$\EE(q_1-q_2, Q_l^{0-})$ approximations of the set
$\EE(q_1,Q_1)\dot{-}\EE(q_2,Q_2)$.

The second and last step is, using the result of section 2.2.2, to find
tight external ellipsoidal approximation $\EE(q_1-q_2+q_3,Q_l^+)$ of the sum
$\EE(q_1-q_2,Q_l^{0+})\oplus\EE(q_3,Q_3)$, and tight internal ellipsoidal
approximation $\EE(q_1-q_2+q_3,Q_l^-)$ for the sum
$\EE(q_1-q_2,Q_l^{0-})\oplus\EE(q_3,Q_3)$.

As a result, we get
\[ \EE(q_1-q_2+q_3,Q_l^-) \subseteq
\EE(q_1,Q_1)\dot{-}\EE(q_2,Q_2)\oplus\EE(q_3,Q_3) \subseteq
\EE(q_1-q_2+q_3,Q_l^+) \]
and
\[ \rho(\pm l ~|~\EE(q_1-q_2+q_3,Q_l^-)) =
\rho(\pm l ~|~ \EE(q_1,Q_1)\dot{-}\EE(q_2,Q_2)\oplus\EE(q_3,Q_3)) =
\rho(\pm l ~|~ \EE(q_1-q_2+q_3,Q_l^+)). \]
Running $l$ over all unit vectors that are not bad, this translates to
\[ \bigcup_{\langle l,l\rangle=1} \EE(q_1-q_2+q_3,Q_l^-) =
\EE(q_1,Q_1)\dot{-}\EE(q_2,Q_2)\oplus\EE(q_3,Q_3) =
\bigcap_{\langle l,l\rangle=1} \EE(q_1-q_2+q_3,Q_l^+) .\]


\subsection{Geometric Sum-Difference}\label{subsec_sumdiff}
Given ellipsoids $\EE(q_1,Q1)$, $\EE(q_2,Q_2)$ and $\EE(q_3,Q_3)$, it is
possible to compute families of external and internal approximating
ellipsoids for
\begin{equation}
\EE(q_1,Q_1) \oplus \EE(q_2,Q_2) \dot{-} \EE(q_3,Q_3) \label{minkpm}
\end{equation}
parametrized by direction $l$, if this set is nonempty
($\EE(0,Q_3)\subseteq\EE(0,Q_1)\oplus\EE(0,Q_2)$).

First, using the result of section 2.2.2, we obtain tight external
$\EE(q_1+q_2,Q_l^{0+})$ and internal $\EE(q_1+q_2,Q_l^{0-})$ ellipsoidal
approximations of the set $\EE(q_1,Q_1)\oplus\EE(q_2,Q_2)$.
In order for the set (\ref{minkpm}) to be nonempty, inclusion
$\EE(0,Q_3)\subseteq\EE(0,Q_l^{0+})$ must be true for any $l$.
Note, however, that even if (\ref{minkpm}) is nonempty, it may be that
$\EE(0,Q_3)\not\subseteq\EE(0,Q_l^{0-})$, then internal approximation for this
direction does not exist.

Assuming that (\ref{minkpm}) is nonempty and
$\EE(0,Q_3)\subseteq\EE(0,Q_l^{0-})$, the second step would be, using the
results of section 2.2.3, to compute tight external ellipsoidal approximation
$\EE(q_1+q_2-q_3,Q_l^+)$ of the difference
$\EE(q_1+q_2,Q_l^{0+})\dot{-}\EE(q_3,Q_3)$, which exists only if $l$ is not
bad, and tight internal ellipsoidal approximation
$\EE(q_1+q_2-q_3,Q_l^-)$ of the difference
$\EE(q_1+q_2,Q_l^{0-})\dot{-}\EE(q_3,Q_3)$, which exists only if $l$ is not
bad for this difference.

If approximation $\EE(q_1+q_2-q_3,Q_l^+)$ exists, then
\[ \EE(q_1,Q_1)\oplus\EE(q_2,Q_2)\dot{-}\EE(q_3,Q_3) \subseteq
\EE(q_1+q_2-q_3,Q_l^+) \]
and
\[ \rho(\pm l ~|~ \EE(q_1,Q_1)\oplus\EE(q_2,Q_2)\dot{-}\EE(q_3,Q_3)) =
\rho(\pm l ~|~ \EE(q_1+q_2-q_3,Q_l^+)). \]
If approximation $\EE(q_1+q_2-q_3,Q_l^-)$ exists, then
\[ \EE(q_1+q_2-q_3,Q_l^-) \subseteq
\EE(q_1,Q_1)\oplus\EE(q_2,Q_2)\dot{-}\EE(q_3,Q_3) \]
and
\[ \rho(\pm l ~|~\EE(q_1+q_2-q_3,Q_l^-)) =
\rho(\pm l ~|~ \EE(q_1,Q_1)\oplus\EE(q_2,Q_2)\dot{-}\EE(q_3,Q_3)) . \]
For any fixed direction $l$ it may be the case that neither external nor
internal tight ellipsoidal approximations exist.


\subsection{Intersection of Ellipsoid and Hyperplane}
Let nondegenerate ellipsoid $\EE(q,Q)$ and hyperplane $H(c,\gamma)$ be such that
${\bf dist}(\EE(q,Q),H(c,\gamma))<0$. In other words,
\[ \EE_H(w,W) = \EE(q,Q)\cap H(c,\gamma) \neq \emptyset .\]
The intersection of ellipsoid with hyperplane, if nonempty, is always an
ellipsoid. Here we show how to find it.

First of all, we transform the hyperplane $H(c,\gamma)$
into $H([1~0~\cdots~0]^T, 0)$ by the affine transformation
\[ y = Sx - \frac{\gamma}{\langle c,c\rangle^{1/2}}Sc, \]
where $S$ is an orthogonal matrix found by (\ref{valign1}-\ref{valign3}) with
$v_1=c$ and $v_2=[1~0~\cdots~0]^T$.
The ellipsoid in the new coordinates becomes $\EE(q',Q')$ with
\begin{eqnarray*}
q' & = & q-\frac{\gamma}{\langle c,c\rangle^{1/2}}Sc, \\
Q' & = & SQS^T.
\end{eqnarray*}
Define matrix $M=Q'^{-1}$; $m_{11}$ is its element in position $(1,1)$,
$\bar{m}$ is the first column of  $M$ without the first element,
and $\bar{M}$ is the submatrix of $M$ obtained by stripping $M$ of its
first row and first column:
\[ M = \left[\begin{array}{c|cl}
m_{11} & & \bar{m}^T\\
 & \\
\hline
 & \\
\bar{m} & & \bar{M}\end{array}\right]. \]
The ellipsoid resulting from the intersection is $\EE_H(w',W')$ with
\begin{eqnarray*}
w' & = & q' + q_1'\left[\begin{array}{c}
-1\\
\bar{M}^{-1}\bar{m}\end{array}\right],\\
W' & = & \left(1-q_1'^2(m_{11}-
\langle\bar{m},\bar{M}^{-1}\bar{m}\rangle)\right)\left[\begin{array}{c|cl}
0 & & {\bf 0}\\
 & \\
\hline
 & \\
{\bf 0} & & \bar{M}^{-1}\end{array}\right],
\end{eqnarray*}
in which $q_1'$ represents the first element of vector $q'$.

Finally, it remains to do the inverse transform of the coordinates
to obtain ellipsoid $\EE_H(w,W)$:
\begin{eqnarray*}
w & = & S^Tw' + \frac{\gamma}{\langle c,c\rangle^{1/2}}c, \\
W & = & S^TW'S.
\end{eqnarray*}


\subsection{Intersection of Ellipsoid and Ellipsoid}
Given two nondegenerate ellipsoids $\EE(q_1,Q_1)$ and $\EE(q_2,Q_2)$,
${\bf dist}(\EE(q_1,Q_1),\EE(q_2,Q_2))<0$ implies that
\[ \EE(q_1,Q_1)\cap\EE(q_2,Q_2)\neq\emptyset .\]
This intersection can be approximated by ellipsoids from the outside and
from the inside.
Trivially, both  $\EE(q_1,Q_1)$ and $\EE(q_2,Q_2)$ are external approximations
of this intersection.
Here, however, we show how to find the external ellipsoidal approximation
of minimal volume.

Define matrices
\begin{equation}
W_1 = Q_1^{-1}, ~~~~ W_2 = Q_2^{-1} .\label{wmatrices}
\end{equation}
Minimal volume external ellipsoidal approximation $\EE(q+,Q^+)$ of
the intersection $\EE(q_1,Q_1)\cap\EE(q_2,Q_2)$ is determined from the set
of equations:
\begin{eqnarray}
Q^+ & = & \alpha X^{-1} \label{fusion1} \\
X & = & \pi W_1 + (1-\pi)W_2 \label{fusion2} \\
\alpha & = & 1-\pi(1-\pi)\langle(q_2-q_1), W_2X^{-1}W_1(q_2-q_1)\rangle
\label{fusion3} \\
q^+ & = & X^{-1}(\pi W_1q_1 + (1-\pi)W_2q_2) \label{fusion4} \\
0 & = & \alpha({\bf det}(X))^2{\bf trace}(X^{-1}(W_1-W_2)) \nonumber \\
& - & n({\bf det}(X))^2
\big{(}2\langle q^+,W_1q_1-W_2q_2\rangle +
\langle q^+,(W_2-W_1)q^+\rangle \nonumber\\
&  & - \langle q_1,W_1q_1\rangle +
\langle q_2,W_2q_2\rangle\big{)}, \label{fusion5}
\end{eqnarray}
with $0\leq\pi\leq1$. We substitute $X$, $\alpha$, $q^+$ defined in
(\ref{fusion2}-\ref{fusion4}) into (\ref{fusion5}) and get a polynomial of
degree $2n-1$ with respect to $\pi$, which has only one root in the
interval $[0,1]$, $\pi_0$. Then, substituting $\pi=\pi_0$ into
(\ref{fusion1}-\ref{fusion4}), we obtain $q^+$ and $Q^+$.
Special cases are $\pi_0=1$, whence $\EE(q^+,Q^+)=\EE(q_1,Q_1)$, and
$\pi_0=0$, whence $\EE(q^+,Q^+)=\EE(q_2,Q_2)$. These situations may occur
if, for example, one ellipsoid is contained in the other:
\begin{eqnarray*}
\EE(q_1,Q_1)\subseteq\EE(q_2,Q_2) & \Rightarrow & \pi_0 = 1,\\
\EE(q_2,Q_2)\subseteq\EE(q_1,Q_1) & \Rightarrow & \pi_0 = 0.\\
\end{eqnarray*}
The proof that the system of equations (\ref{fusion1}-\ref{fusion5})
correctly defines the minimal volume external ellipsoidal approximationi
of the intersection $\EE(q_1,Q_1)\cap\EE(q_2,Q_2)$ is given in \cite{ROS_SABATER_THOMAS_EN_ELLIPSOIDAL_CALCULUS_BASED_ON_PROPAGATION_AND_FUSION}.

To find the internal approximating ellipsoid
$\EE(q^-,Q^-)\subseteq\EE(q_1,Q_1)\cap\EE(q_2,Q_2)$, define
\begin{eqnarray}
\beta_1 & = &
\min_{\langle x,W_2x\rangle=1}\langle x,W_1x\rangle, \label{beta1}\\
\beta_2 & = & \min_{\langle x,W_1x\rangle=1}\langle x,W_2x\rangle, \label{beta2}
\end{eqnarray}
Notice that (\ref{beta1}) and (\ref{beta2}) are QCQP problems.
Parameters $\beta_1$ and $\beta_2$ are invariant with respect to affine
coordinate transformation and describe the position of ellipsoids
$\EE(q_1,Q_1)$, $\EE(q_2,Q_2)$ with respect to each other:
\begin{eqnarray*}
\beta_1\geq1,~\beta_2\geq1 & \Rightarrow &
{\bf int}(\EE(q_1,Q_1)\cap\EE(q_2,Q_2))=\emptyset, \\
\beta_1\geq1,~\beta_2\leq1 & \Rightarrow & \EE(q_1,Q_1)\subseteq\EE(q_2,Q_2), \\
\beta_1\leq1,~\beta_2\geq1 & \Rightarrow & \EE(q_2,Q_2)\subseteq\EE(q_1,Q_1), \\
\beta_1<1,~\beta_2<1 & \Rightarrow &
{\bf int}(\EE(q_1,Q_1)\cap\EE(q_2,Q_2))\neq\emptyset \\
& & \mbox{and} ~ \EE(q_1,Q_1)\not\subseteq\EE(q_2,Q_2) \\
& & \mbox{and} ~ \EE(q_2,Q_2)\not\subseteq\EE(q_1,Q_1).
\end{eqnarray*}
Define parametrized family of internal ellipsoids
$\EE(q^-_{\theta_1\theta_2},Q^-_{\theta_1\theta_2})$ with
\begin{eqnarray}
q^-_{\theta_1\theta_2} & = & (\theta_1W_1 +
\theta_2W_2)^{-1}(\theta_1W_1q_1 + \theta_2W_2q_2), \label{paramell1} \\
Q^-_{\theta_1\theta_2} & = & (1 - \theta_1\langle q_1,W_1q_1\rangle -
\theta_2\langle q_2,W_2q_2\rangle +
\langle q^-_{\theta_1\theta_2},(Q^-)^{-1}q^-_{\theta_1\theta_2}\rangle)
(\theta_1W_1 + \theta_2W_2)^{-1} .\label{paramell2}
\end{eqnarray}
The best internal ellipsoid
$\EE(q^-_{\hat{\theta}_1\hat{\theta}_2},Q^-_{\hat{\theta}_1\hat{\theta}_2})$
in the class (\ref{paramell1}-\ref{paramell2}), namely, such that
\[ \EE(q^-_{{\theta}_1{\theta}_2},Q^-_{{\theta}_1{\theta}_2})\subseteq
\EE(q^-_{\hat{\theta}_1\hat{\theta}_2},Q^-_{\hat{\theta}_1\hat{\theta}_2})
\subseteq \EE(q_1,Q_1)\cap\EE(q_2,Q_2) \]
for all $0\leq\theta_1,\theta_2\leq1$, is specified by the parameters
\begin{equation}
\hat{\theta}_1 = \frac{1-\hat{\beta}_2}{1-\hat{\beta}_1\hat{\beta}_2}, ~~~~
\hat{\theta}_2 = \frac{1-\hat{\beta}_1}{1-\hat{\beta}_1\hat{\beta}_2},
\label{thetapar}
\end{equation}
with
\[ \hat{\beta}_1=\min(1,\beta_1), ~~~~ \hat{\beta}_2=\min(1,\beta_2). \]
It is the ellipsoid that we look for:
$\EE(q^-,Q^-)=\EE(q^-_{\hat{\theta}_1\hat{\theta}_2},Q^-_{\hat{\theta}_1\hat{\theta}_2})$.
Two special cases are
\[ \hat{\theta}_1=1, ~ \hat{\theta}_2=0 ~~~ \Rightarrow ~~~
\EE(q_1,Q_1)\subseteq\EE(q_2,Q_2) ~~~ \Rightarrow ~~~
\EE(q^-,Q^-)=\EE(q_1,Q_1), \]
and
\[ \hat{\theta}_1=0, ~ \hat{\theta}_2=1 ~~~ \Rightarrow ~~~
\EE(q_2,Q_2)\subseteq\EE(q_1,Q_1) ~~~ \Rightarrow ~~~
\EE(q^-,Q^-)=\EE(q_2,Q_2). \]
The method of finding the internal ellipsoidal approximation of the
intersection of two ellipsoids is described in \cite{VASHENTSEV_ON_INTERNAL_ELLIPSOIDAL_APPROXIMATION_FOR_PROBLEN_OF_CONTROL_AND_SYNTHESIS}.


\subsection{Intersection of Ellipsoid and Halfspace}
Finding the intersection of ellipsoid and halfspace can be reduced to
finding the intersection of two ellipsoids, one of which is unbounded.
Let $\EE(q_1,Q_1)$ be a nondegenerate ellipsoid and let  $H(c,\gamma)$
define the halfspace
\[ {\bf S}(c,\gamma) = \{x\in{\bf R}^n ~|~ \langle c,x\rangle\leq\gamma\}. \]
We have to determine if the intersection $\EE(q_1,Q_1)\cap{\bf S}(c,\gamma)$
is empty, and if not, find its external and internal ellipsoidal approximations,
$\EE(q^+,Q^+)$ and $\EE(q^-,Q^-)$.
Two trivial situations are:
\begin{itemize}
\item ${\bf dist}(\EE(q_1,Q_1),H(c,\gamma))>0$ and $\langle c, q_1\rangle>0$,
which implies that $\EE(q_1,Q_1)\cap{\bf S}(c,\gamma)=\emptyset$;
\item ${\bf dist}(\EE(q_1,Q_1),H(c,\gamma))>0$ and $\langle c, q_1\rangle<0$,
so that $\EE(q_1,Q_1)\subseteq{\bf S}(c,\gamma)$, and then
$\EE(q^+,Q^+)=\EE(q^-,Q^-)=\EE(q_1,Q_1)$.
\end{itemize}
In case ${\bf dist}(\EE(q_1,Q_1),H(c,\gamma)<0$, i.e. the ellipsoid
intersects the hyperplane,
\[ \EE(q_1,Q_1)\cap{\bf S}(c,\gamma) =
\EE(q_1,Q_1)\cap\{x ~|~ \langle (x-q_2),W_2(x-q_2)\rangle\leq1\}, \]
with
\begin{eqnarray}
q_2 & = & (\gamma + 2\sqrt{\overline{\lambda}})c,\label{hsell1} \\
W_2 & = & \frac{1}{4\overline{\lambda}}cc^T,\label{hsell2}
\end{eqnarray}
 $\overline{\lambda}$ being the biggest eigenvalue of matrix $Q_1$.
After defining $W_1=Q_1^{-1}$, we obtain $\EE(q^+,Q^+)$ from  equations
(\ref{fusion1}-\ref{fusion5}), and $\EE(q^-,Q^-)$ from
(\ref{paramell1}-\ref{paramell2}), (\ref{thetapar}).

{\bf Remark.} Notice that matrix $W_2$ has rank $1$, which makes it singular
for $n>1$. Nevertheless, expressions (\ref{fusion1}-\ref{fusion2}),
(\ref{paramell1}-\ref{paramell2}) make sense because $W_1$ is nonsingular,
$\pi_0\neq0$ and $\hat{\theta}_1\neq0$.

To find the ellipsoidal approximations $\EE(q^+,Q^+)$ and $\EE(q^-,Q^-)$ of
the intersection of ellipsoid $\EE(q,Q)$ and polytope $P(C,g)$,
$C\in{\bf R}^{m\times n}$, $b\in{\bf R}^m$, such that
\[ \EE(q^-,Q^-)\subseteq\EE(q,Q)\cap P(C,g)\subseteq\EE(q^+,Q^+), \]
we first compute
\[ \EE(q^-_1,Q^-_1)\subseteq\EE(q,Q)\cap{\bf S}(c_1,\gamma_1)\subseteq
\EE(q^+_1,Q^+_1), \]
wherein ${\bf S}(c_1,\gamma_1)$ is the halfspace defined by the first row of matrix $C$,
$c_1$, and the first element of vector $g$, $\gamma_1$. Then, one by one, we get
\begin{eqnarray*}
& & \EE(q^-_2,Q^-_2)\subseteq\EE(q^-_1,Q^-_1)\cap{\bf S}(c_2,\gamma_2), ~~~
\EE(q^+_1,Q^+_1)\cap{\bf S}(c_2,\gamma_2)\subseteq\EE(q^+_2,Q^+_2), \\
& & \EE(q^-_3,Q^-_3)\subseteq\EE(q^-_2,Q^-_2)\cap{\bf S}(c_3,\gamma_3), ~~~
\EE(q^+_2,Q^+_2)\cap{\bf S}(c_3,\gamma_3)\subseteq\EE(q^+_3,Q^+_3), \\
& & \cdots \\
& & \EE(q^-_m,Q^-_m)\subseteq\EE(q^-_{m-1},Q^-_{m-1})\cap{\bf S}(c_m,\gamma_m), ~~~
\EE(q^+_{m-1},Q^+_{m-1})\cap{\bf S}(c_m,\gamma_m)\subseteq\EE(q^+_m,Q^+_m), \\
\end{eqnarray*}
The resulting ellipsoidal approximations are
\[ \EE(q^+,Q^+)=\EE(q^+_m,Q^+_m), ~~~~ \EE(q^-,Q^-)=\EE(q^-_m,Q^-_m) .\]

\subsection{Checking if \texorpdfstring{$\EE(q_1,Q_1)\subseteq\EE(q_2,Q_2)$}{enc}}\label{subsec_ellcontainment}
\todo[inline]{Added this section}
Theorem of alternatives, also known as \emph{$S$-procedure} \cite{BOYD_VANDERBERGHE_CONVEX_OPTIMIZATION},
states that the implication
\[
\langle x, A_1x\rangle + 2\langle b_1,x\rangle + c_1 \leq 0
~~ \Rightarrow ~~
\langle x, A_2x\rangle + 2\langle b_2,x\rangle + c_2 \leq 0,
\]
where $A_i\in{\bf R}^{n\times n}$ are symmetric matrices, $b_i\in{\bf R}^n$,
$c_i\in{\bf R}$, $i=1,2$, holds if and only if there exists $\lambda>0$
such that
\[
\left[\begin{array}{cc}
A_2 & b_2\\
b_2^T & c_2\end{array}\right]
\preceq
\lambda\left[\begin{array}{cc}
A_1 & b_1\\
b_1^T & c_1\end{array}\right].
\]

By $S$-procedure, $\EE(q_1,Q_1)\subseteq\EE(q_2,Q_2)$
(both ellipsoids are assumed to be nondegenerate)
if and only if the following SDP problem is feasible:
\[ \min 0 \]
subject to:
\begin{eqnarray*}
\lambda & > & 0, \\
\left[\begin{array}{cc}
Q_2^{-1} & -Q_2^{-1}q_2\\
(-Q_2^{-1}q_2)^T & q_2^TQ_2^{-1}q_2-1\end{array}\right]
& \preceq &
\lambda \left[\begin{array}{cc}
Q_1^{-1} & -Q_1^{-1}q_1\\
(-Q_1^{-1}q_1)^T & q_1^TQ_1^{-1}q_1-1\end{array}\right]
\end{eqnarray*}
where $\lambda\in{\bf R}$ is the variable.





\subsection{Minimum Volume Ellipsoids}
\todo[inline]{Added this section}
The minimum volume ellipsoid that contains set $S$ is called
\emph{L\"{o}wner-John ellipsoid} of the set $S$.
To characterize it we rewrite general ellipsoid $\EE(q,Q)$ as
\[ \EE(q,Q) = \{x ~|~ \langle (Ax + b), (Ax + b)\rangle \}, \]
where
\[ A = Q^{-1/2} ~~~ \mbox{ and } ~~~ b = -Aq .\]
For positive definite matrix $A$, the volume of $\EE(q,Q)$ is proportional
to $\det A^{-1}$.
So, finding the minimum volume ellipsoid containing $S$
can be expressed as semidefinite programming (SDP) problem
\[ \min \log \det A^{-1} \]
subject to:
\[ \sup_{v\in S} \langle (Av + b), (Av + b)\rangle \leq 1, \]
where the variables are $A\in{\bf R}^{n\times n}$ and $b\in{\bf R}^n$, and
there is an implicit constraint $A\succ 0$ ($A$ is positive definite).
The objective and constraint functions are both convex in $A$ and $b$, so
this problem is convex.
Evaluating the constraint function, however, requires solving a convex
maximization problem, and is tractable only in certain special cases.


For a finite set $S=\{x_1,\cdots,x_m\}\subset{\bf R}^n$, an ellipsoid covers
$S$ if and only if it covers its convex hull.
So, finding the minimum volume ellipsoid covering $S$ is the same as
finding the minimum volume ellipsoid containing the polytope
${\bf conv}\{x_1,\cdots,x_m\}$.
The SDP problem is
\[ \min \log \det A^{-1} \]
subject to:
\begin{eqnarray*}
A & \succ & 0, \\
\langle (Ax_i + b), (Ax_i + b)\rangle & \leq & 1, ~~~ i=1..m.
\end{eqnarray*}

We can find the minimum volume ellipsoid containing the union of ellipsoids
$\bigcup_{i=1}^m\EE(q_i,Q_i)$.
Using the fact that for $i=1..m$ $\EE(q_i,Q_i)\subseteq\EE(q,Q)$
if and only if there exists $\lambda_i>0$ such that
\[ \left[\begin{array}{cc}
A^2 - \lambda_i Q_i^{-1} & Ab + \lambda_i Q_i^{-1}q_i\\
(Ab + \lambda_i Q_i^{-1}q_i)^T & b^Tb-1 - \lambda_i (q_i^TQ_i^{-1}q_i-1) \end{array}
\right] \preceq 0 .\]
Changing variable $\tilde{b}=Ab$, we get convex SDP in the variables $A$,
$\tilde{b}$, and $\lambda_1,\cdots,\lambda_m$:
\[ \min \log \det A^{-1} \]
subject to:
\begin{eqnarray*}
\lambda_i & > & 0,\\
\left[\begin{array}{ccc}
A^2-\lambda_iQ_i^{-1} & \tilde{b}+\lambda_iQ_i^{-1}q_i & 0 \\
(\tilde{b}+\lambda_iQ_i^{-1}q_i)^T & -1-\lambda_i(q_i^TQ_i^{-1}q_i-1) & \tilde{b}^T \\
0 & \tilde{b} & -A^2\end{array}\right] & \preceq & 0, ~~~ i=1..m.
\end{eqnarray*}

After $A$ and $b$ are found,
\[ q=-A^{-1}b ~~~ \mbox{ and } ~~~ Q=(A^TA)^{-1}. \]

The results on the minimum volume ellipsoids are explained
and proven in \cite{BOYD_VANDERBERGHE_CONVEX_OPTIMIZATION}.

\subsection{Maximum Volume Ellipsoids}
\todo[inline]{Added this section}
Consider a problem of finding the maximum volume ellipsoid that lies inside
a bounded convex set $S$ with nonempty interior.
To formulate this problem we rewrite general ellipsoid $\EE(q,Q)$ as
\[ \EE(q,Q) = \{Bx + q ~|~ \langle x,x\rangle\leq 1\}, \]
where $B=Q^{1/2}$, so the volume of $\EE(q,Q)$ is proportional to $\det B$.

The maximum volume ellipsoid that lies inside $S$ can be found by solving
the following SDP problem:
\[ \max \log \det B \]
subject to:
\[ \sup_{\langle v,v\rangle\leq 1} I_S(Bv+q)\leq 0 ,\]
in the variables $B\in{\bf R}^{n\times n}$ - symmetric matrix, and
$q\in{\bf R}^n$, with implicit constraint $B\succ 0$, where $I_S$ is the
indicator function:
\[ I_S(x) = \left\{\begin{array}{ll}
0, & \mbox{ if } x\in S,\\
\infty, & \mbox{ otherwise.}\end{array}\right. \]

In case of polytope, $S=P(C,g)$ with $P(C,g)$ defined in (\ref{polytope}),
the SDP has the form
\[ \min \log \det B^{-1} \]
subject to:
\begin{eqnarray*}
B & \succ & 0,\\
\langle c_i, Bc_i\rangle + \langle c_i, q\rangle & \leq & \gamma_i,
~~~ i=1..m.
\end{eqnarray*}

We can find the maximum volume ellipsoid that lies inside the intersection
of given ellipsoids $\bigcap_{i=1}^m\EE(q_i,Q_i)$.
Using the fact that for $i=1..m$ $\EE(q,Q)\subseteq\EE(q_i,Q_i)$ if and only
if there exists $\lambda_i>0$ such that
\[
\left[\begin{array}{cc}
-\lambda_i - q^TQ_i^{-1}q + 2q_i^TQ_i^{-1}q - q_i^TQ_i^{-1}q_i + 1 & (Q_i^{-1}q-Q_i^{-1}q_i)^TB\\
B(Q_i^{-1}q-Q_i^{-1}q_i) & \lambda_iI-BQ_i^{-1}B\end{array}\right] \succeq 0.
\]

To find the maximum volume ellipsoid, we solve convex SDP in variables
$B$, $q$, and $\lambda_1,\cdots,\lambda_m$:
\[ \min \log \det B^{-1} \]
subject to:
\begin{eqnarray*}
\lambda_i & > & 0, \\
\left[\begin{array}{ccc}
1-\lambda_i & 0 & (q - q_i)^T\\
0 & \lambda_iI & B\\
q - q_i & B & Q_i\end{array}\right] & \succeq & 0, ~~~ i=1..m.
\end{eqnarray*}

After $B$ and $q$ are found,
\[ Q = B^TB. \]

The results on the maximum volume ellipsoids are explained
and proven in \cite{BOYD_VANDERBERGHE_CONVEX_OPTIMIZATION}.

