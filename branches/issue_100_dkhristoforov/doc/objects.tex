\section{Properties}
\todo[inline]{Added this section}
Functions of the {\it Ellipsoidal Toolbox} can be called with
user-specified values of certain global parameters. System of the parameters
are configured using xml files, which  available from a set of command-line
utilities:
\verbmcodef[Configuration download]
{mcodesnippets/s_chapter06_section01_snippet01.m}
Here we list system parameters available from the 'default' configuration:
\begin{enumerate}
\item {\tt version = '1.4'} - current version of {\it ET}.
\item {\tt isVerbose = 0} - makes all the calls to {\it ET}
routines silent, and no information except errors is displayed.
\item {\tt absTol = 1e-7} - absolute tolerance.
\item {\tt relTol = 1e-5} - relative tolerance.
\item {\tt nTimeGridPoints = 200} - density of the time grid for the
continuous time reach set computation.
This parameter directly affects the number of ellipsoids to
be stored in the {\tt ReachContinious $\backslash$ReachDiscrete} object.
\item {\tt ODESolverName = ode45} - specifies the ODE solver for continuous time
reach set computation.
\item {\tt isODENormControl = 'on'} - switches on and off the norm control
in the ODE solver. When turned on, it slows down the computation, but improves
the accuracy.
\item {\tt isEnabledOdeSolverOptions = 0} - when set to $0$, calls the ODE solver
without any additional options like norm control. It makes the computation
faster but less accurate. Otherwise, it is assumed to be $1$, and only in this
case the previous option makes a difference.
\item {\tt nPlot2dPoints = 200} - the number of points used to plot a
2D ellipsoid. This parameter also affects the quality of 2D reach tube
and reach set plots.
\item {\tt nPlot3dPoints = 200} - the number of points used to plot
a 3D ellipsoid. This parameter also affects the quality of 3D reach set plots.
\end{enumerate}
Once the configuration is loaded, the system parameters are available through
{\tt elltool.conf.Properties}.
{\tt elltool.conf.Properties} is a static class, providing emulation of static
properties for toolbox. It has two function types: setters and getters.
Using getters we obtain system parameters.
\verbmcodef[Getting parameters]
{mcodesnippets/s_chapter06_section01_snippet02.m}
 Some of the parameters can be changed
in run-time via setters.
\verbmcodef[Changing parameters]
{mcodesnippets/s_chapter06_section01_snippet03.m}




\section{Ellipsoid}
Each object represents a single ellipsoid with a center $q$ and a shape matrix $Q$.
For this object there are a large number of operations. User can compute distance
between ellipsoid and other objects in {\tt Ellipsoidal toolbox}, such as hyperplane
or polytope. We can found the ellipsoid which is a result of intersection between ellipsoid
and hyperplane or between two ellipsoids. Also an approximation of intersection can be found.
There are methods related to the parameters of the ellipsoid. User can return a center and
a shape matrix, can invert a shape matrix.
There are few comparison operations. It could be checked which of two ellipsoids is bigger
or less. Ellipsoid can be checked for emptiness. User can determine whether the
ellipsoids are equal. Ellipsoids could be plotted.
Arrays of ellipsoids also can be constructed.
For the full list of {\tt ellipsoid} methods, see appendix.

\section{Hyperplane}
Each object represents a single hyperplane with a normal $c$ and a shift $\gamma$.
User can determine whether the hyperplanes are equal. Hyperplane can be checked
for emptiness. It could be checked if the two hyperplanes are parallel.
We can return parameters of the hyperplane. Of course user can plot this object.
Arrays of hyperplanes also can be constructed.
For the full list of {\tt hyperplane} methods, see appendix.

\section{Linear system}
The object represent a linear system. {\it Ellipsoidal Toolbox} supports both
types of linear (affine) dynamical systems: continuous-time,
\begin{eqnarray*}
\dot{x}(t) & = & A(t)x(t) + B(t)u(t) + G(t)v(t),\\
y(t) & = & C(t)x(t) + w(t);
\end{eqnarray*}
and discrete-time,
\begin{eqnarray*}
x[k+1] & = & A[k]x[k] + B[k]u[k] + G[k]v[k], \\
y[k] & = & C[k]x[k] + w[k].
\end{eqnarray*}
Both can be time-invariant (have constant matrices $A$, $B$, $G$, $C$)
or time-variant.
For the full list of linear system methods, see appendix.



\section{Reachability set}
The object represents a reachability domain.
User can project a reach set onto a given orthogonal basis.
A further evolution could be computed in time for the given reachability
domain. We can extracts a piece of the reach set that corresponds to the specified time
value or the time interval. Also we can check an intersection of reachability domain with
ellipsoid, hyperplane or polytope. User can plot an approximation of the reach set.
For the full list of reachability set methods, see appendix.
