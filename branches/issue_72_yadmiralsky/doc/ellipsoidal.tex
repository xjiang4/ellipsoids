\subsection{Continuous-time systems}\label{subsec_ellappct}
Consider the system
\begin{equation}
\dot{x}(t) = A(t)x(t) + B(t)u + G(t)v, \label{ctsystem}
\end{equation}
in which $x\in{\bf R}^n$ is the state, $u\in{\bf R}^m$ is the control and
$v\in{\bf R}^d$ is the disturbance. $A(t)$, $B(t)$ and $G(t)$
are continuous and take their values in ${\bf R}^{n\times n}$,
${\bf R}^{n\times m}$ and ${\bf R}^{n\times d}$ respectively.
Control $u(t,x(t))$ and disturbance $v(t)$ are measurable functions
restricted by ellipsoidal constraints:
$u(t,x(t)) \in \EE(p(t), P(t))$ and $v(t) \in \EE(q(t), Q(t))$.
The set of initial states at initial time $t_0$ is assumed
to be the ellipsoid $\EE(x_0,X_0)$.

The reach sets for systems with disturbances computed by the Ellipsoidal
Toolbox are CLRS.
Henceforth, when describing backward reachability,
reach sets  refer to CLRS or CLBRS.
Recall that for continuous-time linear systems maxmin and minmax CLRS
coincide, and the same is true for maxmin and minmax CLBRS.

If the matrix $Q(\cdot)=0$, the system (\ref{ctsystem})
becomes an ordinary affine system with known $v(\cdot)=q(\cdot)$.
If $G(\cdot) = 0$, the system becomes linear.
For these two cases ($Q(\cdot)=0$ or $G(\cdot)=0$) the reach set is
as given in Definition \ref{def_olrs}, and so
the reach set will be denoted as
$\XX_{CL}(t, t_0, \EE(x_0, X_0)) = \XX(t, t_0, \EE(x_0,X_0))$.

The reach set $\XX(t,t_0,\EE(x_0,X_0))$ is a symmetric compact convex set,
whose center evolves in time according to
\begin{equation}
\dot{x}_c(t) = A(t)x_c(t) + B(t)p(t) + G(t)q(t), \;\;\;
x_c(t_0)=x_0. \label{fwdcenter}
\end{equation}
Fix a vector $l_0\in{\bf R}^n$, and consider the solution $l(t)$ of
the adjoint equation
\begin{equation}
\dot{l}(t) = -A^T(t)l(t), \;\;\; l(t_0) = l_0,
\label{adjointct}
\end{equation}
which is equivalent to
\[ l(t) = \Phi^T(t_0, t)l_0. \]
If the reach set $\XX(t, t_0, \EE(x_0,X_0))$ is nonempty,
there exist tight external and
tight internal approximating ellipsoids $\EE(x_c(t), X^+_l(t))$ and
$\EE(x_c(t), X^-_l(t))$, respectively, such that
\begin{equation}
\EE(x_c(t), X^-_l(t))\subseteq\XX(t,t_0,\EE(x_0,X_0))
\subseteq \EE(x_c(t), X^+_l(t)),
\label{fwdinclusion}
\end{equation}
and
\begin{equation}
\rho(l(t) ~|~ \EE(x_c(t), X^-_l(t))) =
\rho(l(t) ~|~ \XX(t, t_0, \EE(x_0,X_0))) =
\rho(l(t) ~|~ \EE(x_c(t), X^+_l(t))) .
\label{fwdtightness}
\end{equation}
The equation for the shape matrix of the external ellipsoid is
\begin{eqnarray}
\dot{X}^+_l(t) & = & A(t)X^+_l(t) + X^+_l(t)A^T(t) +\nonumber \\
& & \pi_l(t)X^+_l(t) + \frac{1}{\pi_l(t)}B(t)P(t)B^T(t) -\nonumber \\
& & (X_l^{+}(t))^{1/2}S_l(t)(G(t)Q(t)G^T(t))^{1/2} \nonumber -\\
& & (G(t)Q(t)G^T(t))^{1/2}S_l^T(t)(X_l^{+}(t))^{1/2}, \label{fwdext1} \\
X^+_l(t_0) & = & X_0, \label{fwdext2}
\end{eqnarray}
in which
\[ \pi_l(t) = \frac{\langle l(t),
B(t)P(t)B^T(t)l(t)\rangle^{1/2}}{\langle l(t), X^+_l(t)l(t)\rangle^{1/2}}, \]
and the orthogonal matrix $S_l(t)$  ($S_l(t)S_l^T(t) = I$) is determined
by the equation
\[ S_l(t)(G(t)Q(t)G^T(t))^{1/2}l(t) = \frac{\langle l(t),
G(t)Q(t)G^T(t)l(t)\rangle^{1/2}}{\langle l(t),
X_l^+(t)l(t)\rangle^{1/2}}(X_l^{+}(t))^{1/2}l(t). \]
In the presence of disturbance, if the reach set is empty, the
matrix $X^+_l(t)$ becomes sign indefinite.
For a system without disturbance, the terms containing $G(t)$ and $Q(t)$
vanish from the equation (\ref{fwdext1}).

The equation for the shape matrix of the internal ellipsoid is
\begin{eqnarray}
\dot{X}^-_l(t) & = & A(t)X^-_l(t) + X^-_l(t)A^T(t) +\nonumber \\
& & (X_l^{-}(t))^{1/2}T_l(t)(B(t)P(t)B^T(t))^{1/2} +\nonumber \\
& & (B(t)P(t)B^T(t))^{1/2}T_l^T(t)(X_l^{-}(t))^{1/2} -\nonumber \\
& & \eta_l(t)X^-_l(t) - \frac{1}{\eta_l(t)}G(t)Q(t)G^T(t), \label{fwdint1} \\
X^-_l(t_0) & = & X_0, \label{fwdint2}
\end{eqnarray}
in which
\[ \eta_l(t) = \frac{\langle l(t),
G(t)Q(t)G^T(t)l(t)\rangle^{1/2}}{\langle l(t), X^+_l(t)l(t)\rangle^{1/2}}, \]
and the orthogonal matrix $T_l(t)$ is determined by the equation
\[ T_l(t)(B(t)P(t)B^T(t))^{1/2}l(t) = \frac{\langle l(t),
B(t)P(t)B^T(t)l(t)\rangle^{1/2}}{\langle l(t),
X_l^-(t)l(t)\rangle^{1/2}}(X_l^{-}(t))^{1/2}l(t). \]
Similarly to the external case, the terms containing $G(t)$ and $Q(t)$
vanish from the equation (\ref{fwdint1}) for a system without disturbance.

The  point where the external and internal ellipsoids touch the
boundary of the reach set is given by
\[ x_l^*(t) = x_c(t) +
\frac{X^+_l(t)l(t)}{\langle l(t), X^+_l(t)l(t)\rangle^{1/2}} .\]
The boundary points $x^*_l(t)$ form trajectories, which we call
{\it extremal trajectories}.
Due to the nonsingular nature of the state transition matrix $\Phi(t,t_0)$,
every boundary point of the reach set belongs to an extremal trajectory.
To follow an extremal trajectory specified by parameter $l_0$, the
system has to start at time $t_0$ at initial state
\begin{equation}
x^0_l = x_0 + \frac{X_0l_0}{\langle l_0,X_0l_0\rangle^{1/2}}. \label{x0lct}
\end{equation}
In the absence of disturbances, the open-loop control
\begin{equation}
u_l(t) = p(t) + \frac{P(t)B^T(t)l(t)}{\langle l(t),
B(t)P(t)B^T(t)l(t)\rangle^{1/2}}. \label{uct}
\end{equation}
steers the system along the extremal trajectory defined by the vector $l_0$.
When a disturbance is present, this control keeps the system on an extremal
trajectory if and only if the disturbance plays against the control
always taking its extreme values.

Expressions (\ref{fwdinclusion}) and (\ref{fwdtightness}) lead
to the following fact,
\[ \bigcup_{\langle l_0,l_0\rangle=1}\EE(x_c(t),X^-_l(t)) =
\XX(t,t_0,\EE(x_0,X_0)) =
\bigcap_{\langle l_0,l_0\rangle=1}\EE(x_c(t),X^+_l(t)). \]
In practice this means that the more values of $l_0$
we use to compute $X^+_l(t)$ and
$X^-_l(t)$, the better will be our approximation.




% Continuous-time backward




Analogous results hold for the backward reach set.

Given the terminal time $t_1$ and ellipsoidal target set $\EE_(y_1,Y_1)$, the
CLBRS $\YY_{CL}(t_1, t, \YY_1)=\YY(t_1, t, \YY_1)$, $t<t_1$, if it is nonempty,
is a symmetric compact convex set whose center is governed by
\begin{equation}
y_c(t) = Ay_c(t) + B(t)p(t) + G(t)q(t), \;\;\; y_c(t_1) = y_1.\label{bckcenter}
\end{equation}
Fix a vector $l_1\in{\bf R}^n$, and consider
\begin{equation}
l(t) = \Phi(t_1, t)^Tl_1 .
\label{bckadjoint}
\end{equation}
If the backward reach set $\YY(t_1, t, \EE(y_1,Y_1))$ is nonempty,
there exist tight external and
tight internal approximating ellipsoids $\EE(y_c(t), Y^+_l(t))$ and
$\EE(y_c(t), Y^-_l(t))$ respectively, such that
\begin{equation}
\EE(y_c(t), Y^-_l(t))\subseteq\YY(t_1,t,\EE(y_1,Y_1))
\subseteq \EE(y_c(t), Y^+_l(t)),
\label{bckinclusion}
\end{equation}
and
\begin{equation}
\rho(l(t) ~|~ \EE(y_c(t), Y^-_l(t))) =
\rho(l(t) ~|~ \YY(t_1, t, \EE(y_0,Y_0))) =
\rho(l(t) ~|~ \EE(y_c(t), Y^+_l(t))) .
\label{bcktightness}
\end{equation}
The equation for the shape matrix of the external ellipsoid is
\begin{eqnarray}
\dot{Y}^+_l(t) & = & A(t)Y^+_l(t) + Y^+_l(t)A^T(t) -\nonumber \\
& & \pi_l(t)Y^+_l(t) - \frac{1}{\pi_l(t)}B(t)P(t)B^T(t) +\nonumber \\
& & (Y_l^{+}(t))^{1/2}S_l(t)(G(t)Q(t)G^T(t))^{1/2} +\nonumber \\
& & (G(t)Q(t)G^T(t))^{1/2}S_l^T(t)(Y_l^{+}(t))^{1/2}, \label{bckext1} \\
Y^+_l(t_1) & = & Y_1, \label{bckext2}
\end{eqnarray}
in which
\[ \pi_l(t) = \frac{\langle l(t),
B(t)P(t)B^T(t)l(t)\rangle^{1/2}}{\langle l(t),
Y^+_l(t)l(t)\rangle^{1/2}}, \]
and the orthogonal matrix $S_l(t)$ satisfies the equation
\[ S_l(t)(G(t)Q(t)G^T(t))^{1/2}l(t) = \frac{\langle l(t),
G(t)Q(t)G^T(t)l(t)\rangle^{1/2}}{\langle l(t),
Y_l^+(t)l(t)\rangle^{1/2}}(Y_l^{+}(t))^{1/2}l(t). \]
The equation for the shape matrix of the internal ellipsoid is
\begin{eqnarray}
\dot{Y}^-_l(t) & = & A(t)Y^-_l(t) + Y^-_l(t)A^T(t) -\nonumber \\
& & (Y_l^{-}(t))^{1/2}T_l(t)(B(t)P(t)B^T(t))^{1/2} -\nonumber \\
& & (B(t)P(t)B^T(t))^{1/2}T_l^T(t)(Y_l^{-}(t))^{1/2} +\nonumber \\
& & \eta_l(t)Y^-_l(t) + \frac{1}{\eta_l(t)}G(t)Q(t)G^T(t), \label{bckint1} \\
Y^-_l(t_1) & = & Y_1, \label{bckint2}
\end{eqnarray}
in which
\[ \eta_l(t) = \frac{\langle l(t),
G(t)Q(t)G^T(t)l(t)\rangle^{1/2}}{\langle l(t),
Y^+_l(t)l(t)\rangle^{1/2}}, \]
and the orthogonal matrix $T_l(t)$ is  determined by the equation
\[ T_l(t)(B(t)P(t)B^T(t))^{1/2}l(t) = \frac{\langle l(t),
B(t)P(t)B^T(t)l(t)\rangle^{1/2}}{\langle l(t),
Y_l^-(t)l(t)\rangle^{1/2}}(Y_l^{-}(t))^{1/2}l(t). \]
Just as in the forward reachability case, the terms containing $G(t)$ and
$Q(t)$ vanish from  equations (\ref{bckext1}) and
(\ref{bckint1}) in the absence of disturbances.
The boundary value problems (\ref{bckcenter}), (\ref{bckext1})
and (\ref{bckint1}) are converted to the initial value
problems by the change of variables $s = -t$.

Due to (\ref{bckinclusion}) and (\ref{bcktightness}),
\[ \bigcup_{\langle l_1,l_1\rangle=1}\EE(y_c(t),Y^-_l(t)) =
\YY(t_1,t,\EE(y_1,Y_1)) =
\bigcap_{\langle l_1,l_1\rangle=1}\EE(y_c(t),Y^+_l(t)). \]

{\bf Remark.} In expressions (\ref{fwdext1}), (\ref{fwdint1}),
(\ref{bckext1}) and (\ref{bckint1}) the terms $\frac{1}{\pi_l(t)}$
and $\frac{1}{\eta_l(t)}$ may not be well defined for some vectors $l$,
because matrices $B(t)P(t)B^T(t)$ and $G(t)Q(t)G^T(t)$ may be singular.
In such cases, we set these entire expressions to  zero.















\newcommand{\OXP}{\overline{X}^+_l}
\newcommand{\OXM}{\overline{X}^-_l}
\newcommand{\UXP}{\underline{X}^+_l}
\newcommand{\UXM}{\underline{X}^-_l}
\newcommand{\OOXP}{\hat{X}^+_l}
\newcommand{\OOXM}{\hat{X}^-_l}
\newcommand{\UUXP}{\breve{X}^+_l}
\newcommand{\UUXM}{\breve{X}^-_l}
\newcommand{\OYP}{\overline{Y}^+_l}
\newcommand{\OYM}{\overline{Y}^-_l}
\newcommand{\UYP}{\underline{Y}^+_l}
\newcommand{\UYM}{\underline{Y}^-_l}









\subsection{Discrete-time systems}\label{ellappdt}
Consider the discrete-time linear system,
\begin{equation}
x(t+1) = A(t)x(t) + B(t)u(t,x(t)) + G(t)v(t),
\tag*{(\ref{ctsystem})}
\label{dtsystem}
\end{equation}
in which $x(t)\in{\bf R}^n$ is the state, $u(t, x(t))\in{\bf R}^m$ is the control
bounded by the ellipsoid $\EE(p(t),P(t))$,
$v(t)\in{\bf R}^d$ is disturbance bounded by ellipsoid $\EE(q(t),Q(t))$,
and matrices $A(t)$, $B(t)$, $G(t)$ are in ${\bf R}^{n\times n}$,
${\bf R}^{n\times m}$, ${\bf R}^{n\times d}$ respectively.
Here we shall assume $A(t)$ to be nonsingular.\footnote{The case
when $A(t)$ is singular is described in \cite{kurvar07}.
The idea is to substitute $A(t)$ with the nonsingular
$A_\delta(t) = A(t) + \delta U(t)W(t)$,
in which $U(t)$ and $W(t)$ are obtained from the singular value decomposition
\[ A(t) = U(t)\Sigma(t)V(t) .\]
The parameter $\delta$ can be chosen based on the number of time steps for which
the reach set must be computed and the required accuracy.
The issue of inverting ill-conditioned matrices is also addressed
in \cite{kurvar07}.}
The set of initial conditions at initial time $t_0$
is ellipsoid $\EE(x_0,X_0)$.

Ellipsoidal Toolbox computes maxmin and minmax CLRS
$\OXCL(t, t_0, \EE(x_0, X_0)$ and $\UXCL(t, t_0, \EE(x_0, X_0)$ for
discrete-time systems.

If matrix $Q(\cdot)=0$,  the system (\ref{dtsystem}) becomes an
ordinary affine system with known $v(\cdot)=q(\cdot)$.
If matrix $G(\cdot)=0$,  the system reduces to a linear controlled
system.
In the absence of disturbance ($Q(\cdot)=0$ or $G(\cdot)=0$),
$\OXCL(t,t_0,\EE(x_0,X_0))=\UXCL(t,t_0,\EE(x_0,X_0))=\XX(t,t_0,\EE(x_0,X_0))$,
the reach set is as in Definition \ref{def_olrs}.

Maxmin and minmax CLRS $\OXCL(t, t_0, \EE(x_0, X_0)$ and
$\UXCL(t, t_0, \EE(x_0, X_0)$, if nonempty, are symmetric convex and compact,
with the center evolving in time according to
\begin{equation}
x_c(t+1) = A(t)x_c(t) + B(t)p(t) + G(t)v(t), \;\;\; x_c(t_0)=x_0.
\label{fwdcenterd}
\end{equation}
Fix some vector $l_0\in{\bf R}^n$ and consider $l(t)$ that satisfies the
discrete-time adjoint equation,\footnote{Note that for (\ref{adjointdt})
$A(t)$ must be invertible.}
\begin{equation}
l(t+1) = \left(A^T\right)^{-1}(t)l(t), \;\;\; l(t_0) = l_0,
\label{adjointdt}
\end{equation}
or, equivalently
\[ l(t) = \Phi^T(t_0, t)l_0 .\]
There exist tight external ellipsoids
$\EE(x_c(t), \OXP(t))$, $\EE(x_c(t), \UXP(t))$ and tight internal
ellipsoids
$\EE(x_c(t), \OXM(t))$, $\EE(x_c(t), \UXM(t))$ such that
\begin{equation}
\EE(x_c(t), \OXM(t))\subseteq\OXCL(t,t_0,\EE(x_0,X_0))
\subseteq \EE(x_c(t), \OXP(t)),
\label{maxmininclusion}
\end{equation}
\begin{equation}
\rho(l(t) ~|~ \EE(x_c(t), \OXM(t))) =
\rho(l(t) ~|~ \OXCL(t, t_0, \EE(x_0,X_0))) =
\rho(l(t) ~|~ \EE(x_c(t), \OXP(t))) .
\label{maxmintightness}
\end{equation}
and
\begin{equation}
\EE(x_c(t), \UXM(t))\subseteq\UXCL(t,t_0,\EE(x_0,X_0))
\subseteq \EE(x_c(t), \UXP(t)),
\label{minmaxinclusion}
\end{equation}
\begin{equation}
\rho(l(t) ~|~ \EE(x_c(t), \UXM(t))) =
\rho(l(t) ~|~ \UXCL(t, t_0, \EE(x_0,X_0))) =
\rho(l(t) ~|~ \EE(x_c(t), \UXP(t))) .
\label{minmaxtightness}
\end{equation}
The shape matrix of the external ellipsoid for maxmin reach set
is determined from
\begin{eqnarray}
\OOXP(t) & = & (1+\overline{\pi}_l(t))A(t)\OXP(t)A^T(t) +
\left(1+\frac{1}{\overline{\pi}_l(t)}\right)
B(t)P(t)B^T(t), \label{fwdextmaxmin1} \\
\OXP(t+1) & = & \left((\OOXP(t))^{1/2} +
\overline{S}_l(t)(G(t)Q(t)G^T(t))^{1/2}\right)^T
\times \nonumber \\
& &\left((\OOXP(t))^{1/2} + \overline{S}_l(t)(G(t)Q(t)G^T(t))^{1/2}\right),
\label{fwdextmaxmin2}\\
\OXP(t_0) & = & X_0, \label{fwdextmaxmin3}
\end{eqnarray}
wherein
\[ \overline{\pi}_l(t) = \frac{\langle l(t+1),
B(t)P(t)B^T(t)l(t+1)\rangle^{1/2}}{\langle l(t),
\OXP(t)l(t)\rangle^{1/2}}, \]
and the orthogonal matrix $\overline{S}_l(t)$ is determined by the equation
\begin{eqnarray*}
& & \overline{S}_l(t)(G(t)Q(t)G^T(t))^{1/2}l(t+1) = \\
& & \frac{\langle l(t+1),
G(t)Q(t)G^T(t)l(t+1)\rangle^{1/2}}{\langle l(t+1),
\OOXP(t)l(t+1)\rangle^{1/2}}(\OOXP(t))^{1/2}l(t+1).
\end{eqnarray*}
Equation (\ref{fwdextmaxmin2}) is valid only if
$\EE(0,G(t)Q(t)G^T(t))\subseteq\EE(0,\OOXP(t))$, otherwise the
maxmin CLRS $\OXCL(t,t_0,\EE(x_0,X_0))$ is empty.

The shape matrix of the external ellipsoid for minmax reach set
is determined from
\begin{eqnarray}
\UUXP(t) & = &
\left((A(t)\UXP(t)A^T(t))^{1/2} +
\underline{S}_l(t)(G(t)Q(t)G^T(t))^{1/2}\right)^T
\times \nonumber \\
& &\left((A(t)\UXP(t)A^T(t))^{1/2} +
\underline{S}_l(t)(G(t)Q(t)G^T(t))^{1/2}\right)
\label{fwdextminmax1}\\
\UXP(t+1) & = &
(1+\underline{\pi}_l(t))\UUXP(t) +
\left(1+\frac{1}{\underline{\pi}_l(t)}\right)
B(t)P(t)B^T(t), \label{fwdextminmax2} \\
\UXP(t_0) & = & X_0, \label{fwdextminmax3}
\end{eqnarray}
where
\[ \underline{\pi}_l(t) = \frac{\langle l(t+1),
B(t)P(t)B^T(t)l(t+1)\rangle^{1/2}}{\langle l(t+1),
\UUXP(t)l(t+1)\rangle^{1/2}}, \]
and $\underline{S}_l(t)$ is orthogonal matrix determined from the equation
\begin{eqnarray*}
& & \underline{S}_l(t)(G(t)Q(t)G^T(t))^{1/2}l(t+1) = \\
& & \frac{\langle l(t+1),
G(t)Q(t)G^T(t)l(t+1)\rangle^{1/2}}{\langle l(t),
\UXP(t)l(t)\rangle^{1/2}}(A(t)\UXP(t)A^T(t))^{1/2}l(t+1).
\end{eqnarray*}
Equations (\ref{fwdextminmax1}), (\ref{fwdextminmax2}) are valid only if
$\EE(0,G(t)Q(t)G^T(t)\subseteq\EE(0,A(t)\UXP(t)A^T(t))$, otherwise
minmax CLRS $\UXCL(t,t_0,\EE(x_0,X_0))$ is empty.

The shape matrix of the internal ellipsoid for maxmin reach set
is determined from
\begin{eqnarray}
\OOXM(t) & = &
\left((A(t)\OXM(t)A^T(t))^{1/2} +
\overline{T}_l(t)(B(t)P(t)B^T(t))^{1/2}\right)^T
\times \nonumber \\
& &\left((A(t)\OXM(t)A^T(t))^{1/2} +
\overline{T}_l(t)(B(t)P(t)B^T(t))^{1/2}\right)
\label{fwdintmaxmin1}\\
\OXM(t+1) & = &
(1+\overline{\eta}_l(t))\OOXM(t) +
\left(1+\frac{1}{\underline{\eta}_l(t)}\right)
G(t)Q(t)G^T(t), \label{fwdintmaxmin2} \\
\OXM(t_0) & = & X_0, \label{fwdintmaxmin3}
\end{eqnarray}
where
\[ \overline{\eta}_l(t) = \frac{\langle l(t+1),
G(t)Q(t)G^T(t)l(t+1)\rangle^{1/2}}{\langle l(t+1),
\OOXM(t)l(t+1)\rangle^{1/2}}, \]
and $\overline{T}_l(t)$ is orthogonal matrix determined from the equation
\begin{eqnarray*}
& & \overline{T}_l(t)(B(t)P(t)B^T(t))^{1/2}l(t+1) = \\
& & \frac{\langle l(t+1),
B(t)P(t)B^T(t)l(t+1)\rangle^{1/2}}{\langle l(t),
\OXM(t)l(t)\rangle^{1/2}}(A(t)\OXM(t)A^T(t))^{1/2}l(t+1).
\end{eqnarray*}
Equation (\ref{fwdintmaxmin2}) is valid only if
$\EE(0,G(t)Q(t)G^T(t)\subseteq\EE(0,\OOXM(t))$.

The shape matrix of the internal ellipsoid for the minmax reach set
is determined by
\begin{eqnarray}
\UUXM(t) & = & (1+\underline{\eta}_l(t))A(t)\UXM(t)A^T(t) +
\left(1+\frac{1}{\underline{\eta}_l(t)}\right)
G(t)Q(t)G^T(t), \label{fwdintminmax1} \\
\UXM(t+1) & = & \left((\UUXM(t))^{1/2} +
\underline{T}_l(t)(B(t)P(t)B^T(t))^{1/2}\right)^T
\times \nonumber \\
& &\left((\UUXM(t))^{1/2} + \underline{T}_l(t)(B(t)P(t)B^T(t))^{1/2}\right),
\label{fwdintminmax2}\\
\UXM(t_0) & = & X_0, \label{fwdintminmax3}
\end{eqnarray}
wherein
\[ \underline{\eta}_l(t) = \frac{\langle l(t+1),
G(t)Q(t)G^T(t)l(t+1)\rangle^{1/2}}{\langle l(t),
\UXM(t)l(t)\rangle^{1/2}}, \]
and the orthogonal matrix $\underline{T}_l(t)$ is determined by the equation
\begin{eqnarray*}
& & \underline{T}_l(t)(B(t)P(t)B^T(t))^{1/2}l(t+1) = \\
& & \frac{\langle l(t+1),
B(t)P(t)B^T(t)l(t+1)\rangle^{1/2}}{\langle l(t+1),
\UUXM(t)l(t+1)\rangle^{1/2}}(\UUXM(t))^{1/2}l(t+1).
\end{eqnarray*}
Equations (\ref{fwdintminmax1}), (\ref{fwdintminmax2}) are valid only if
$\EE(0,G(t)Q(t)G^T(t)\subseteq\EE(0,A(t)\UXM(t)A^T(t))$.

The point where the external and the internal ellipsoids both touch
the boundary of the maxmin CLRS is
\[ x_l^+(t) = x_c(t) + \frac{\OXP(t)l(t)}{\langle l(t),
\OXP(t)l(t)\rangle^{1/2}} ,\]
and the bounday point of minmax CLRS is
\[ x_l^-(t) = x_c(t) + \frac{\OXM(t)l(t)}{\langle l(t),
\OXM(t)l(t)\rangle^{1/2}} .\]
Points $x^{\pm}_l(t)$, $t\geq t_0$, form extremal trajectories.
In order for the system to follow the extremal trajectory specified by
some vector $l_0$, the initial state must be
\begin{equation}
x_l^0 = x_0 + \frac{X_0l_0}{\langle l_0, X_0l_0\rangle^{1/2}}. \label{dx0l}
\end{equation}
When there is no disturbance ($G(t)=0$ or $Q(t)=0$),
$\OXP(t)=\UXP(t)$ and $\OXM(t)=\UXM(t)$, and
the open-loop control that steers the system along the extremal trajectory
defined by $l_0$ is
\begin{equation}
u_l(t) = p(t) + \frac{P(t)B^T(t)l(t+1)}{\langle l(t+1),
B(t)P(t)B^T(t)l(t+1)\rangle^{1/2}}. \label{udt}
\end{equation}
Each choice of $l_0$ defines an external and internal approximation.
If $\OXCL(t,t_0,\EE(x_0,X_0))$ is nonempty, 
\[ \bigcup_{\langle l_0,l_0\rangle=1}\EE(x_c(t),\OXM(t)) =
\OXCL(t,t_0,\EE(x_0,X_0)) =
\bigcap_{\langle l_0,l_0\rangle=1}\EE(x_c(t),\OXP(t)). \]
Similarly for $\UXCL(t,t_0,\EE(x_0,X_0))$,
\[ \bigcup_{\langle l_0,l_0\rangle=1}\EE(x_c(t),\UXM(t)) =
\UXCL(t,t_0,\EE(x_0,X_0)) =
\bigcap_{\langle l_0,l_0\rangle=1}\EE(x_c(t),\UXP(t)). \]






% Discrete-time backwards reachability




Similarly, tight ellipsoidal approximations of maxmin and minmax CLBRS
with terminating conditions $(t_1, \EE(y_1,Y_1))$ can be obtained
for those directions $l(t)$ satisfying
\begin{equation}
l(t) = \Phi^T(t_1,t)l_1,
\tag*{(\ref{bckadjoint})}
\label{bckadjointd}
\end{equation}
with some fixed $l_1$, for which they exist.

With boundary conditions
\begin{equation}
y_c(t_1)=y_1, ~~~ \OYP(t_1)=\OYM(t_1)=\UYP(t_1)=\UYM(t_1)=Y_1,
\label{bndconds}
\end{equation}
external and internal ellipsoids for maxmin CLBRS $\OYCL(t_1,t,\EE(y_1,Y_1))$
at time $t$, $\EE(y_c(t),\OYP(t))$ and $\EE(y_c(t),\OYM(t))$, are computed
as external and internal ellipsoidal approximations of
the geometric sum-difference
\[ A^{-1}(t)\left(
\EE(y_c(t+1),\OYP(t+1)) \oplus B(t)\EE(-p(t),P(t))
\dot{-}G(t)\EE(-q(t),Q(t))
\right)\]
and
\[ A^{-1}(t)\left(
\EE(y_c(t+1),\OYM(t+1)) \oplus B(t)\EE(-p(t),P(t))
\dot{-}G(t)\EE(-q(t),Q(t))
\right)\]
in direction $l(t)$ from \ref{bckadjointd}.
Section \ref{subsec_sumdiff} describes the operation of
geometric sum-difference for ellipsoids.

External and internal ellipsoids for minmax CLBRS $\UYCL(t_1,t,\EE(y_1,Y_1))$
at time $t$, $\EE(y_c(t),\UYP(t))$ and $\EE(y_c(t),\UYM(t))$, are computed
as external and internal ellipsoidal approximations of
the geometric difference-sum
\[ A^{-1}(t)\left(
\EE(y_c(t+1),\UYP(t+1))
\dot{-}G(t)\EE(-q(t),Q(t))
\oplus B(t)\EE(-p(t),P(t))
\right)\]
and
\[ A^{-1}(t)\left(
\EE(y_c(t+1),\UYM(t+1))
\dot{-}G(t)\EE(-q(t),Q(t))
\oplus B(t)\EE(-p(t),P(t))
\right)\]
in direction $l(t)$ from \ref{bckadjointd}.
Section \ref{subsec_diffsum} describes the operation of
geometric difference-sum for ellipsoids.






