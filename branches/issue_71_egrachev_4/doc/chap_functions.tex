\section{ellipsoid}\label{secClassDescr:ellipsoid}
\subsection{\texorpdfstring{ellipsoid.calcGrid}{calcGrid}}\label{method:ellipsoid.calcGrid}
\begin{verbatim}
CALCGRID - computes grid of 2d or 3d sphere and vertices for each face
           in the grid with number of points taken from ellObj
           nPlot2dPoints or nPlot3dPoints parameters
\end{verbatim}
\subsection{\texorpdfstring{ellipsoid.checkIsMe}{checkIsMe}}\label{method:ellipsoid.checkIsMe}
\begin{verbatim}
CHECKISME - determine whether input object is ellipsoid. And display
            message and abort function if input object
            is not ellipsoid

Input:
  regular:
      someObjArr: any[] - any type array of objects.

Example:
  ellObj = ellipsoid([1; 2], eye(2));
  ellipsoid.checkIsMe(ellObj)
\end{verbatim}
\subsection{\texorpdfstring{ellipsoid.contents}{contents}}\label{method:ellipsoid.contents}
\begin{verbatim}
Ellipsoid library of the Ellipsoidal Toolbox.


Constructor and data accessing functions:
-----------------------------------------
 ellipsoid    - Constructor of ellipsoid object.
 double       - Returns parameters of ellipsoid, i.e. center and shape
                matrix.
 parameters   - Same function as 'double'(legacy matter).
 dimension    - Returns dimension of ellipsoid and its rank.
 isdegenerate - Checks if ellipsoid is degenerate.
 isempty      - Checks if ellipsoid is empty.
 maxeig       - Returns the biggest eigenvalue of the ellipsoid.
 mineig       - Returns the smallest eigenvalue of the ellipsoid.
 trace        - Returns the trace of the ellipsoid.
 volume       - Returns the volume of the ellipsoid.


Overloaded operators and functions:
-----------------------------------
 eq      - Checks if two ellipsoids are equal.
 ne      - The opposite of 'eq'.
 gt, ge  - E1 > E2 (E1 >= E2) checks if, given the same center ellipsoid
           E1 contains E2.
 lt, le  - E1 < E2 (E1 <= E2) checks if, given the same center ellipsoid
           E2 contains E1.
 mtimes  - Given matrix A in R^(mxn) and ellipsoid E in R^n, returns
           (A * E).
 plus    - Given vector b in R^n and ellipsoid E in R^n, returns (E + b).
 minus   - Given vector b in R^n and ellipsoid E in R^n, returns (E - b).
 uminus  - Changes the sign of the center of ellipsoid.
 display - Displays the details about given ellipsoid object.
 inv     - inverts the shape matrix of the ellipsoid.
 plot    - Plots ellipsoid in 1D, 2D and 3D.


Geometry functions:
-------------------
 move2origin        - Moves the center of ellipsoid to the origin.
 shape              - Same as 'mtimes', but modifies only shape matrix of
                      the ellipsoid leaving its center as is.
 rho                - Computes the value of support function and
                      corresponding boundary point of the ellipsoid in
                      the given direction.
 polar              - Computes the polar ellipsoid to an ellipsoid that
                      contains the origin.
 projection         - Projects the ellipsoid onto a subspace specified
                      by  orthogonal basis vectors.
 minksum            - Computes and plots the geometric (Minkowski) sum of
                      given ellipsoids in 1D, 2D and 3D.
 minksum_ea         - Computes the external ellipsoidal approximation of
                      geometric sum of given ellipsoids in given
                      direction.
 minksum_ia         - Computes the internal ellipsoidal approximation of
                      geometric sum of given ellipsoids in given
                      direction.
 minkdiff           - Computes and plots the geometric (Minkowski)
                      difference of given ellipsoids in 1D, 2D and 3D.
 minkdiff_ea        - Computes the external ellipsoidal approximation of
                      geometric difference of two ellipsoids in given
                      direction.
 minkdiff_ia        - Computes the internal ellipsoidal approximation of
                      geometric difference of two ellipsoids in given
                      direction
 minkpm             - Computes and plots the geometric (Minkowski)
                      difference of a geometric sum of ellipsoids and a
                      single ellipsoid in 1D, 2D and 3D.
 minkpm_ea          - Computes the external ellipsoidal approximation of
                      the geometric difference of a geometric sum of
                      ellipsoids and a single ellipsoid in given
                      direction.
 minkpm_ia          - Computes the internal ellipsoidal approximation of
                      the geometric difference of a geometric sum of
                      ellipsoids and a single ellipsoid in given
                      direction.
 minkmp             - Computes and plots the geometric (Minkowski) sum of
                      a geometric difference of two single ellipsoids and
                      a geometric sum of ellipsoids in 1D, 2D and 3D.
 minkmp_ea          - Computes the external ellipsoidal approximation of
                      the geometric sum of a geometric difference of two
                      single ellipsoids and a geometric sum of ellipsoids
                      in given direction.
 minkmp_ia          -  Computes the internal ellipsoidal approximation of
                      the geometric sum of a geometric difference of
                      two single ellipsoids and a geometric sum of ellipsoids
                      in given direction.
 isbaddirection     - Checks if ellipsoidal approximation of geometric difference
                      of two ellipsoids in the given direction can be computed.
 doesIntersectionContain           - Checks if the union or intersection of
                      ellipsoids or polytopes lies inside the intersection
                      of given ellipsoids.
 isinternal         - Checks if given vector belongs to the union or intersection
                      of given ellipsoids.
 distance           - Computes the distance from ellipsoid to given point,
                      ellipsoid, hyperplane or polytope.
 intersect          - Checks if the union or intersection of ellipsoids intersects
                      with given ellipsoid, hyperplane, or polytope.
 intersection_ea    - Computes the minimal volume ellipsoid containing intersection
                      of two ellipsoids, ellipsoid and halfspace, or ellipsoid
                      and polytope.
 intersection_ia    - Computes the maximal ellipsoid contained inside the
                      intersection of two ellipsoids, ellipsoid and halfspace
                      or ellipsoid and polytope.
 ellintersection_ia - Computes maximum volume ellipsoid that is contained
                      in the intersection of given ellipsoids (can be more than 2).
 ellunion_ea        - Computes minimum volume ellipsoid that contains
                      the union of given ellipsoids.
 hpintersection     - Computes the intersection of ellipsoid with hyperplane.
\end{verbatim}
\subsection{\texorpdfstring{ellipsoid.dimension}{dimension}}\label{method:ellipsoid.dimension}
\begin{verbatim}
DIMENSION - returns the dimension of the space in which the ellipsoid is
            defined and the actual dimension of the ellipsoid.

Input:
  regular:
    myEllArr: ellipsoid[nDims1,nDims2,...,nDimsN] - array of ellipsoids.

Output:
  regular:
    dimArr: double[nDims1,nDims2,...,nDimsN] - space dimensions.

  optional:
    rankArr: double[nDims1,nDims2,...,nDimsN] - dimensions of the
           ellipsoids in myEllArr.

Example:
  firstEllObj = ellipsoid();
  tempMatObj = [3 1; 0 1; -2 1];
  secEllObj = ellipsoid([1; -1; 1], tempMatObj*tempMatObj');
  thirdEllObj = ellipsoid(eye(2));
  fourthEllObj = ellipsoid(0);
  ellMat = [firstEllObj secEllObj; thirdEllObj fourthEllObj];
  [dimMat, rankMat] = ellMat.dimension()

  dimMat =

     0     3
     2     1

  rankMat =

     0     2
     2     0
\end{verbatim}
\subsection{\texorpdfstring{ellipsoid.disp}{disp}}\label{method:ellipsoid.disp}
\begin{verbatim}
DISP - Displays ellipsoid object.

Input:
  regular:
    myEllMat: ellipsoid [mRows, nCols] - matrix of ellipsoids.

Example:
  ellObj = ellipsoid([-2; -1], [2 -1; -1 1]);
  disp(ellObj)

  Ellipsoid with parameters
  Center:
      -2
      -1

  Shape Matrix:
       2    -1
      -1     1
\end{verbatim}
\subsection{\texorpdfstring{ellipsoid.display}{display}}\label{method:ellipsoid.display}
\begin{verbatim}
DISPLAY - Displays the details of the ellipsoid object.

Input:
  regular:
      myEllMat: ellipsoid [mRows, nCols] - matrix of ellipsoids.

Example:
  ellObj = ellipsoid([-2; -1], [2 -1; -1 1]);
  display(ellObj)

  ellObj =

  Center:
      -2
      -1

  Shape Matrix:
       2    -1
      -1     1

  Nondegenerate ellipsoid in R^2.
\end{verbatim}
\subsection{\texorpdfstring{ellipsoid.distance}{distance}}\label{method:ellipsoid.distance}
\begin{verbatim}
DISTANCE - computes distance between the given ellipsoid (or array of
           ellipsoids) to the specified object (or arrays of objects):
           vector, ellipsoid, hyperplane or polytope.

Input:
  regular:
      ellObjArr: ellipsoid [nDims1, nDims2,..., nDimsN] -  array of
         ellipsoids of the same dimension.
      objArray: double / ellipsoid / hyperplane / polytope [nDims1,
          nDims2,..., nDimsN] - array of vectors or ellipsoids or
          hyperplanes or polytopes. If number of elements in objArray
          is more than 1, then it must be equal to the number of elements
          in ellObjArr.

  optional:
      isFlagOn: logical[1,1] - if true then distance is  computed in
          ellipsoidal metric, if false - in Euclidean metric (by default
          isFlagOn=false).

Output:
  regular:
    distValArray: double [nDims1, nDims2,..., nDimsN] - array of pairwise
          calculated distances.
          Negative distance value means
              for ellipsoid and vector: vector belongs to the ellipsoid,
              for ellipsoid and hyperplane: ellipsoid intersects the
                  hyperplane.
              Zero distance value means for ellipsoid and vector: vector
                  is aboundary point of the ellipsoid,
              for ellipsoid and hyperplane: ellipsoid  touches the
                  hyperplane.
  optional:
      statusArray: double [nDims1, nDims2,..., nDimsN] - array of time of
          computation of ellipsoids-vectors or ellipsoids-ellipsoids
          distances, or status of cvx solver for ellipsoids-polytopes
          distances.

Literature:
 1. Lin, A. and Han, S. On the Distance between Two Ellipsoids.
    SIAM Journal on Optimization, 2002, Vol. 13, No. 1 : pp. 298-308
 2. Stanley Chan, "Numerical method for Finding Minimum Distance to an
    Ellipsoid".
    http://videoprocessing.ucsd.edu/~stanleychan/publication/...
    unpublished/Ellipse.pdf

Example:
  ellObj = ellipsoid([-2; -1], [4 -1; -1 1]);
  tempMat = [1 1; 1 -1; -1 1; -1 -1]';
  distVec = ellObj.distance(tempMat)

  distVec =

       2.3428    1.0855    1.3799    -1.0000
\end{verbatim}
\subsection{\texorpdfstring{ellipsoid.doesContain}{doesContain}}\label{method:ellipsoid.doesContain}
\begin{verbatim}
DOESCONTAIN - checks if one ellipsoid contains the other ellipsoid or
              polytope. The condition for E1 = firstEllArr to contain
              E2 = secondEllArr is
              min(rho(l | E1) - rho(l | E2)) > 0, subject to <l, l> = 1.
              How checked if ellipsoid contains polytope is explained in
              doesContainPoly.
Input:
  regular:
      firstEllArr: ellipsoid [nDims1,nDims2,...,nDimsN]/[1,1] - first
          array of ellipsoids.
      secondObjArr: ellipsoid [nDims1,nDims2,...,nDimsN]/
          polytope[nDims1,nDims2,...,nDimsN]/[1,1] - array of the same
          size as firstEllArr or single ellipsoid or polytope.

   properties:
      mode: char[1, 1] - 'u' or 'i', go to description.
      computeMode: char[1,] - 'highDimFast' or 'lowDimFast'. Determines,
          which way function is computed, when secObjArr is polytope. If
          secObjArr is ellipsoid computeMode is ignored. 'highDimFast'
          works  faster for  high dimensions, 'lowDimFast' for low. If
          this property is omitted if dimension of ellipsoids is greater
          then 10, then 'hightDimFast' is choosen, otherwise -
          'lowDimFast'

Output:
  isPosArr: logical[nDims1,nDims2,...,nDimsN],
      resArr(iCount) = true - firstEllArr(iCount)
      contains secondObjArr(iCount), false - otherwise.

Example:
  firstEllObj = ellipsoid([-2; -1], [2 -1; -1 1]);
  secEllObj = ellipsoid([-1;0], eye(2));
  doesContain(firstEllObj,secEllObj)

  ans =

       0
\end{verbatim}
\subsection{\texorpdfstring{ellipsoid.doesIntersectionContain}{doesIntersectionContain}}\label{method:ellipsoid.doesIntersectionContain}
\begin{verbatim}
DOESINTERSECTIONCONTAIN - checks if the intersection of ellipsoids
                          contains the union or intersection of given
                          ellipsoids or polytopes.

  res = DOESINTERSECTIONCONTAIN(fstEllArr, secEllArr, mode)
      Checks if the union
      (mode = 'u') or intersection (mode = 'i') of ellipsoids in
      secEllArr lies inside the intersection of ellipsoids in
      fstEllArr. Ellipsoids in fstEllArr and secEllArr must be
      of the same dimension. mode = 'u' (default) - union of
      ellipsoids in secEllArr. mode = 'i' - intersection.
  res = DOESINTERSECTIONCONTAIN(fstEllArr, secPolyArr, mode)
       Checks if the union
      (mode = 'u') or intersection (mode = 'i')  of polytopes in
      secPolyArr lies inside the intersection of ellipsoids in
      fstEllArr. Ellipsoids in fstEllArr and polytopes in secPolyArr
      must be of the same dimension. mode = 'u' (default) - union of
      polytopes in secPolyMat. mode = 'i' - intersection.

  To check if the union of ellipsoids secEllArr belongs to the
  intersection of ellipsoids fstEllArr, it is enough to check that
  every ellipsoid of secEllMat is contained in every
  ellipsoid of fstEllArr.
  Checking if the intersection of ellipsoids in secEllMat is inside
  intersection fstEllMat can be formulated as quadratically
  constrained quadratic programming (QCQP) problem.

  Let fstEllArr(iEll) = E(q, Q) be an ellipsoid with center q and shape
  matrix Q. To check if this ellipsoid contains the intersection of
  ellipsoids in secObjArr:
  E(q1, Q1), E(q2, Q2), ..., E(qn, Qn), we define the QCQP problem:
                    J(x) = <(x - q), Q^(-1)(x - q)> --> max
  with constraints:
                    <(x - q1), Q1^(-1)(x - q1)> <= 1   (1)
                    <(x - q2), Q2^(-1)(x - q2)> <= 1   (2)
                    ................................
                    <(x - qn), Qn^(-1)(x - qn)> <= 1   (n)

  If this problem is feasible, i.e. inequalities (1)-(n) do not
  contradict, or, in other words, intersection of ellipsoids
  E(q1, Q1), E(q2, Q2), ..., E(qn, Qn) is nonempty, then we can find
  vector y such that it satisfies inequalities (1)-(n)
  and maximizes function J. If J(y) <= 1, then ellipsoid E(q, Q)
  contains the given intersection, otherwise, it does not.

  The intersection of polytopes is a polytope, which is computed
  by the standard routine of MPT. How checked if intersection of
  ellipsoids contains polytope is explained in doesContainPoly.

  Checking if the union of polytopes belongs to the intersection
  of ellipsoids is the same as checking if its convex hull belongs
  to this intersection.

Input:
  regular:
      fstEllArr: ellipsoid [nDims1,nDims2,...,nDimsN] - array of ellipsoids
          of the same size.
      secEllArr: ellipsoid /
          polytope [nDims1,nDims2,...,nDimsN] - array of ellipsoids or
          polytopes of the same sizes.

          note: if mode == 'i', then fstEllArr, secEllVec should be
              array.

  properties:
      mode: char[1, 1] - 'u' or 'i', go to description.
      computeMode: char[1,] - 'highDimFast' or 'lowDimFast'. Determines,
          which way function is computed, when secObjArr is polytope. If
          secObjArr is ellipsoid computeMode is ignored. 'highDimFast'
          works  faster for  high dimensions, 'lowDimFast' for low. If
          this property is omitted if dimension of ellipsoids is greater
          then 10, then 'hightDimFast' is choosen, otherwise -
          'lowDimFast'


Output:
  res: double[1, 1] - result:
      -1 - problem is infeasible, for example, if s = 'i',
          but the intersection of ellipsoids in E2 is an empty set;
      0 - intersection is empty;
      1 - if intersection is nonempty.
  status: double[0, 0]/double[1, 1] - status variable. status is empty
      if mode == 'u' or mSecRows == nSecCols == 1.

Example:
  firstEllObj = [0 ; 0] + ellipsoid(eye(2, 2));
  secEllObj = [0 ; 0] + ellipsoid(2*eye(2, 2));
  thirdEllObj = [1; 0] + ellipsoid(0.5 * eye(2, 2));
  secEllObj.doesIntersectionContain([firstEllObj secEllObj], 'i')

  ans =

       1
\end{verbatim}
\subsection{\texorpdfstring{ellipsoid.double}{double}}\label{method:ellipsoid.double}
\begin{verbatim}
DOUBLE - returns parameters of the ellipsoid.

Input:
  regular:
      myEll: ellipsoid [1, 1] - single ellipsoid of dimention nDims.


Output:
  myEllCentVec: double[nDims, 1] - center of the ellipsoid myEll.

  myEllShMat: double[nDims, nDims] - shape matrix of the ellipsoid myEll.

Example:
  ellObj = ellipsoid([-2; -1], [2 -1; -1 1]);
  [centVec, shapeMat] = double(ellObj)
  centVec =

      -2
      -1


  shapeMat =

       2    -1
      -1     1
\end{verbatim}
\subsection{\texorpdfstring{ellipsoid.ellbndr\_2d}{ellbndr\_2d}}\label{method:ellipsoid.ellbndr2d}
\begin{verbatim}
ELLBNDR_2D - compute the boundary of 2D ellipsoid. Private method.

Input:
  regular:
      myEll: ellipsoid [1, 1]- ellipsoid of the dimention 2.
  optional:
      nPoints: number of boundary points

Output:
  regular:
      bpMat: double[nPoints,2] - boundary points of ellipsoid
  optional:
      fVec: double[1,nFaces] - indices of points in each face of
          bpMat graph
\end{verbatim}
\subsection{\texorpdfstring{ellipsoid.ellbndr\_3d}{ellbndr\_3d}}\label{method:ellipsoid.ellbndr3d}
\begin{verbatim}
ELLBNDR_3D - compute the boundary of 3D ellipsoid.

Input:
  regular:
      myEll: ellipsoid [1, 1]- ellipsoid of the dimention 3.

  optional:
      nPoints: number of boundary points

Output:
  regular:
      bpMat: double[nPoints,3] - boundary points of ellipsoid
  optional:
      fMat: double[nFaces,3] - indices of face verties in bpMat
\end{verbatim}
\subsection{\texorpdfstring{ellipsoid.ellintersection\_ia}{ellintersection\_ia}}\label{method:ellipsoid.ellintersectionia}
\begin{verbatim}
ELLINTERSECTION_IA - computes maximum volume ellipsoid that is contained
                     in the intersection of given ellipsoids.


Input:
  regular:
      inpEllArr: ellipsoid [nDims1,nDims2,...,nDimsN] - array of
          ellipsoids of the same dimentions.

Output:
  outEll: ellipsoid [1, 1] - resulting maximum volume ellipsoid.

Example:
  firstEllObj = ellipsoid([-1; 1], [2 0; 0 3]);
  secEllObj = ellipsoid([1 2], eye(2);
  ellVec = [firstEllObj secEllObj];
  resEllObj = ellintersection_ia(ellVec)

  resEllObj =

  Center:
      0.1847
      1.6914

  Shape Matrix:
      0.0340   -0.0607
     -0.0607    0.1713

  Nondegenerate ellipsoid in R^2.
\end{verbatim}
\subsection{\texorpdfstring{ellipsoid.ellipsoid}{ellipsoid}}\label{method:ellipsoid.ellipsoid}
\begin{verbatim}
ELLIPSOID - constructor of the ellipsoid object.

  Ellipsoid E = { x in R^n : <(x - q), Q^(-1)(x - q)> <= 1 }, with current
      "Properties". Here q is a vector in R^n, and Q in R^(nxn) is positive
      semi-definite matrix

  ell = ELLIPSOID - Creates an empty ellipsoid

  ell = ELLIPSOID(shMat) - creates an ellipsoid with shape matrix shMat,
      centered at 0

   ell = ELLIPSOID(centVec, shMat) - creates an ellipsoid with shape matrix
      shMat and center centVec

  ell = ELLIPSOID(centVec, shMat, 'propName1', propVal1,...,
      'propNameN',propValN) - creates an ellipsoid with shape
      matrix shMat, center centVec and propName1 = propVal1,...,
      propNameN = propValN. In other cases "Properties"
      are taken from current values stored in
      elltool.conf.Properties.
  ellMat = Ellipsoid(centVecArray, shMatArray,
      ['propName1', propVal1,...,'propNameN',propValN]) -
      creates an array (possibly multidimensional) of
      ellipsoids with centers centVecArray(:,dim1,...,dimn)
      and matrices shMatArray(:,:,dim1,...dimn) with
      properties if given.

  These parameters can be accessed by DOUBLE(E) function call.
  Also, DIMENSION(E) function call returns the dimension of
  the space in which ellipsoid E is defined and the actual
  dimension of the ellipsoid; function ISEMPTY(E) checks if
  ellipsoid E is empty; function ISDEGENERATE(E) checks if
  ellipsoid E is degenerate.

Input:
  Case1:
    regular:
      shMatArray: double [nDim, nDim] /
          double [nDim, nDim, nDim1,...,nDimn] -
          shape matrices array

  Case2:
    regular:
      centVecArray: double [nDim,1] /
          double [nDim, 1, nDim1,...,nDimn] -
          centers array
      shMatArray: double [nDim, nDim] /
          double [nDim, nDim, nDim1,...,nDimn] -
          shape matrices array


  properties:
      absTol: double [1,1] - absolute tolerance with default value 10^(-7)
      relTol: double [1,1] - relative tolerance with default value 10^(-5)
      nPlot2dPoints: double [1,1] - number of points for 2D plot with
          default value 200
      nPlot3dPoints: double [1,1] - number of points for 3D plot with
           default value 200.

Output:
  ellMat: ellipsoid [1,1] / ellipsoid [nDim1,...nDimn] -
      ellipsoid with specified properties
      or multidimensional array of ellipsoids.

Example:
  ellObj = ellipsoid([1 0 -1 6]', 9*eye(4));
\end{verbatim}
\subsection{\texorpdfstring{ellipsoid.ellunion\_ea}{ellunion\_ea}}\label{method:ellipsoid.ellunionea}
\begin{verbatim}
ELLUNION_EA - computes minimum volume ellipsoid that contains union
              of given ellipsoids.

Input:
  regular:
      inpEllMat: ellipsoid [nDims1,nDims2,...,nDimsN] - array of
          ellipsoids of the same dimentions.

Output:
  outEll: ellipsoid [1, 1] - resulting minimum volume ellipsoid.

Example:
  firstEllObj = ellipsoid([-1; 1], [2 0; 0 3]);
  secEllObj = ellipsoid([1 2], eye(2));
  ellVec = [firstEllObj secEllObj];
  resEllObj = ellunion_ea(ellVec)
  resEllObj =

  Center:
     -0.3188
      1.2936

  Shape Matrix:
      5.4573    1.3386
      1.3386    4.1037

  Nondegenerate ellipsoid in R^2.
\end{verbatim}
\subsection{\texorpdfstring{ellipsoid.fromRepMat}{fromRepMat}}\label{method:ellipsoid.fromRepMat}
\begin{verbatim}
FROMREPMAT - returns array of equal ellipsoids the same
             size as stated in sizeVec argument

  ellArr = fromRepMat(sizeVec) - creates an array  size
           sizeVec of empty ellipsoids.

  ellArr = fromRepMat(shMat,sizeVec) - creates an array
           size sizeVec of ellipsoids with shape matrix
           shMat.

  ellArr = fromRepMat(cVec,shMat,sizeVec) - creates an
           array size sizeVec of ellipsoids with shape
           matrix shMat and center cVec.

Input:
  Case1:
      regular:
          sizeVec: double[1,n] - vector of size, have
          integer values.

  Case2:
      regular:
          shMat: double[nDim, nDim] - shape matrix of
          ellipsoids.
          sizeVec: double[1,n] - vector of size, have
          integer values.

  Case3:
      regular:
          cVec: double[nDim,1] - center vector of
          ellipsoids
          shMat: double[nDim, nDim] - shape matrix of
          ellipsoids.
          sizeVec: double[1,n] - vector of size, have
          integer values.

  properties:
      absTol: double [1,1] - absolute tolerance with default
          value 10^(-7)
      relTol: double [1,1] - relative tolerance with default
          value 10^(-5)
      nPlot2dPoints: double [1,1] - number of points for 2D plot
          with default value 200
      nPlot3dPoints: double [1,1] - number of points for 3D plot
          with default value 200.
\end{verbatim}
\subsection{\texorpdfstring{ellipsoid.fromStruct}{fromStruct}}\label{method:ellipsoid.fromStruct}
\begin{verbatim}
fromStruct -- converts structure array into ellipsoid array.

Input:
  regular:
      SEllArr: struct [nDim1, nDim2, ...] - array
          of structures with the following fields:

      q: double[1, nEllDim] - the center of ellipsoid
      Q: double[nEllDim, nEllDim] - the shape matrix of ellipsoid
Output:
      ellArr: ellipsoid [nDim1, nDim2, ...] - ellipsoid array with size of
          SEllArr.

Example:
s = struct('Q', eye(2), 'q', [0 0]);
ellipsoid.fromStruct(s)

-------ellipsoid object-------
Properties:
   |
   |-- actualClass : 'ellipsoid'
   |--------- size : [1, 1]

Fields (name, type, description):
    'Q'    'double'    'Configuration matrix'
    'q'    'double'    'Center'

Data:
   |
   |-- q : [0 0]
   |       -----
   |-- Q : |1|0|
   |       |0|1|
   |       -----
\end{verbatim}
\subsection{\texorpdfstring{ellipsoid.getAbsTol}{getAbsTol}}\label{method:ellipsoid.getAbsTol}
\begin{verbatim}
GETABSTOL - gives the array of absTol for all elements in ellArr

Input:
  regular:
      ellArr: ellipsoid[nDim1, nDim2, ...] - multidimension array
          of ellipsoids
  optional
      fAbsTolFun: function_handle[1,1] - function that apply
          to the absTolArr. The default is @min.

Output:
  regular:
      absTolArr: double [absTol1, absTol2, ...] - return absTol for
          each element in ellArr
  optional:
      absTol: double[1,1] - return result of work fAbsTolFun with
          the absTolArr

Usage:
  use [~,absTol] = ellArr.getAbsTol() if you want get only
      absTol,
  use [absTolArr,absTol] = ellArr.getAbsTol() if you want get
      absTolArr and absTol,
  use absTolArr = ellArr.getAbsTol() if you want get only absTolArr

Example:
  firstEllObj = ellipsoid([-1; 1], [2 0; 0 3]);
  secEllObj = ellipsoid([1 2], eye(2));
  ellVec = [firstEllObj secEllObj];
  absTolVec = ellVec.getAbsTol()

  absTolVec =

     1.0e-07 *

      1.0000    1.0000
\end{verbatim}
\subsection{\texorpdfstring{ellipsoid.getBoundary}{getBoundary}}\label{method:ellipsoid.getBoundary}
\begin{verbatim}
GETBOUNDARY - computes the boundary of an ellipsoid.

Input:
  regular:
      myEll: ellipsoid [1, 1]- ellipsoid of the dimention 2 or 3.
  optional:
      nPoints: number of boundary points

Output:
  regular:
      bpMat: double[nPoints,nDim] - boundary points of ellipsoid
  optional:
      fVec: double[1,nFaces]/double[nFacex,nDim] - indices of points in
          each face of bpMat graph
\end{verbatim}
\subsection{\texorpdfstring{ellipsoid.getBoundaryByFactor}{getBoundaryByFactor}}\label{method:ellipsoid.getBoundaryByFactor}
\begin{verbatim}
  GETBOUNDARYBYFACTOR - computes grid of 2d or 3d ellipsoid and vertices
                        for each face in the grid
\end{verbatim}
\subsection{\texorpdfstring{ellipsoid.getCenterVec}{getCenterVec}}\label{method:ellipsoid.getCenterVec}
\begin{verbatim}
GETCENTERVEC - returns centerVec vector of given ellipsoid

Input:
  regular:
     self: ellipsoid[1,1]

Output:
  centerVecVec: double[nDims,1] - centerVec of ellipsoid

Example:
  ellObj = ellipsoid([1; 2], eye(2));
  getCenterVec(ellObj)

  ans =

       1
       2
\end{verbatim}
\subsection{\texorpdfstring{ellipsoid.getCopy}{getCopy}}\label{method:ellipsoid.getCopy}
\begin{verbatim}
GETCOPY - gives array the same size as ellArr with copies of elements of
          ellArr.

Input:
  regular:
      ellArr: ellipsoid[nDim1, nDim2,...] - multidimensional array of
          ellipsoids.

Output:
  copyEllArr: ellipsoid[nDim1, nDim2,...] - multidimension array of
      copies of elements of ellArr.

Example:
  firstEllObj = ellipsoid([-1; 1], [2 0; 0 3]);
  secEllObj = ellipsoid([1; 2], eye(2));
  ellVec = [firstEllObj secEllObj];
  copyEllVec = getCopy(ellVec)

  copyEllVec =
  1x2 array of ellipsoids.
\end{verbatim}
\subsection{\texorpdfstring{ellipsoid.getInv}{getInv}}\label{method:ellipsoid.getInv}
\begin{verbatim}
GETINV - do the same as INV method: inverts shape matrices of ellipsoids
      in the given array, with only difference, that it doesn't modify
      input array of ellipsoids.

Input:
  regular:
    myEllArr: ellipsoid [nDims1,nDims2,...,nDimsN] - array of ellipsoids.

Output:
   invEllArr: ellipsoid [nDims1,nDims2,...,nDimsN] - array of ellipsoids
      with inverted shape matrices.

Example:
  ellObj = ellipsoid([1; 1], [4 -1; -1 5]);
  invEllObj = ellObj.getInv()

  invEllObj =

  Center:
       1
       1

  Shape Matrix:
      0.2632    0.0526
      0.0526    0.2105

  Nondegenerate ellipsoid in R^2.
\end{verbatim}
\subsection{\texorpdfstring{ellipsoid.getMove2Origin}{getMove2Origin}}\label{method:ellipsoid.getMove2Origin}
\begin{verbatim}
GETMOVE2ORIGIN - do the same as MOVE2ORIGIN method: moves ellipsoids in
      the given array to the origin, with only difference, that it doesn't
      modify input array of ellipsoids.

Input:
  regular:
      inpEllArr: ellipsoid [nDims1,nDims2,...,nDimsN] - array of
          ellipsoids.

Output:
  outEllArr: ellipsoid [nDims1,nDims2,...,nDimsN] - array of ellipsoids
      with the same shapes as in inpEllArr centered at the origin.

Example:
  ellObj = ellipsoid([-2; -1], [4 -1; -1 1]);
  outEllObj = ellObj.getMove2Origin()

  outEllObj =

  Center:
       0
       0

  Shape:
       4    -1
      -1     1

  Nondegenerate ellipsoid in R^2.
\end{verbatim}
\subsection{\texorpdfstring{ellipsoid.getNPlot2dPoints}{getNPlot2dPoints}}\label{method:ellipsoid.getNPlot2dPoints}
\begin{verbatim}
GETNPLOT2DPOINTS - gives value of nPlot2dPoints property of ellipsoids
                   in ellArr

Input:
  regular:
      ellArr: ellipsoid[nDim1, nDim2,...] - mltidimensional array of
          ellipsoids

Output:
      nPlot2dPointsArr: double[nDim1, nDim2,...] - multidimension array
          of nPlot2dPoints property for ellipsoids in ellArr
Example:
  firstEllObj = ellipsoid([-1; 1], [2 0; 0 3]);
  secEllObj = ellipsoid([1 ;2], eye(2));
  ellVec = [firstEllObj secEllObj];
  ellVec.getNPlot2dPoints()

  ans =

     200   200
\end{verbatim}
\subsection{\texorpdfstring{ellipsoid.getNPlot3dPoints}{getNPlot3dPoints}}\label{method:ellipsoid.getNPlot3dPoints}
\begin{verbatim}
GETNPLOT3DPOINTS - gives value of nPlot3dPoints property of ellipsoids
                   in ellArr

Input:
  regular:
      ellArr: ellipsoid[nDim1, nDim2,...] - mltidimensional array  of
         ellipsoids

Output:
      nPlot2dPointsArr: double[nDim1, nDim2,...] - multidimension array
          of nPlot3dPoints property for ellipsoids in ellArr

Example:
  firstEllObj = ellipsoid([-1; 1], [2 0; 0 3]);
  secEllObj = ellipsoid([1 ;2], eye(2));
  ellVec = [firstEllObj secEllObj];
  ellVec.getNPlot3dPoints()

  ans =

     200   200
\end{verbatim}
\subsection{\texorpdfstring{ellipsoid.getProjection}{getProjection}}\label{method:ellipsoid.getProjection}
\begin{verbatim}
GETPROJECTION - do the same as PROJECTION method: computes projection of
      the ellipsoid onto the given subspace, with only difference, that
      it doesn't modify input array of ellipsoids.

Input:
  regular:
      ellArr: ellipsoid [nDims1,nDims2,...,nDimsN] - array
          of ellipsoids.
      basisMat: double[nDim, nSubSpDim] - matrix of orthogonal basis
          vectors

Output:
  projEllArr: ellipsoid [nDims1,nDims2,...,nDimsN] - array of
      projected ellipsoids, generally, of lower dimension.

Example:
  ellObj = ellipsoid([-2; -1; 4], [4 -1 0; -1 1 0; 0 0 9]);
  basisMat = [0 1 0; 0 0 1]';
  outEllObj = ellObj.getProjection(basisMat)

  outEllObj =

  Center:
      -1
       4

  Shape:
      1     0
      0     9

  Nondegenerate ellipsoid in R^2.
\end{verbatim}
\subsection{\texorpdfstring{ellipsoid.getRelTol}{getRelTol}}\label{method:ellipsoid.getRelTol}
\begin{verbatim}
GETRELTOL - gives the array of relTol for all elements in ellArr

Input:
  regular:
      ellArr: ellipsoid[nDim1, nDim2, ...] - multidimension array
          of ellipsoids
  optional:
      fRelTolFun: function_handle[1,1] - function that apply
          to the relTolArr. The default is @min.
Output:
  regular:
      relTolArr: double [relTol1, relTol2, ...] - return relTol for
          each element in ellArr
  optional:
      relTol: double[1,1] - return result of work fRelTolFun with
          the relTolArr

Usage:
  use [~,relTol] = ellArr.getRelTol() if you want get only
      relTol,
  use [relTolArr,relTol] = ellArr.getRelTol() if you want get
      relTolArr and relTol,
  use relTolArr = ellArr.getRelTol() if you want get only relTolArr

Example:
  firstEllObj = ellipsoid([-1; 1], [2 0; 0 3]);
  secEllObj = ellipsoid([1 ;2], eye(2));
  ellVec = [firstEllObj secEllObj];
  ellVec.getRelTol()

  ans =

     1.0e-05 *

      1.0000    1.0000
\end{verbatim}
\subsection{\texorpdfstring{ellipsoid.getShape}{getShape}}\label{method:ellipsoid.getShape}
\begin{verbatim}
GETSHAPE -  do the same as SHAPE method: modifies the shape matrix of the
   ellipsoid without changing its center, with only difference, that
   it doesn't modify input array of ellipsoids.

Input:
  regular:
      ellArr: ellipsoid [nDims1,nDims2,...,nDimsN] - array
          of ellipsoids.
      modMat: double[nDim, nDim]/[1,1] - square matrix or scalar

Output:
   outEllArr: ellipsoid [nDims1,nDims2,...,nDimsN] - array of modified
      ellipsoids.

Example:
  ellObj = ellipsoid([-2; -1], [4 -1; -1 1]);
  tempMat = [0 1; -1 0];
  outEllObj = ellObj.getShape(tempMat)

  outEllObj =

  Center:
      -2
      -1

  Shape:
      1     1
      1     4

  Nondegenerate ellipsoid in R^2.
\end{verbatim}
\subsection{\texorpdfstring{ellipsoid.getShapeMat}{getShapeMat}}\label{method:ellipsoid.getShapeMat}
\begin{verbatim}
GETSHAPEMAT - returns shapeMat matrix of given ellipsoid

Input:
  regular:
     self: ellipsoid[1,1]

Output:
  shMat: double[nDims,nDims] - shapeMat matrix of ellipsoid

Example:
  ellObj = ellipsoid([1; 2], eye(2));
  getShapeMat(ellObj)

  ans =

       1     0
       0     1
\end{verbatim}
\subsection{\texorpdfstring{ellipsoid.hpintersection}{hpintersection}}\label{method:ellipsoid.hpintersection}
\begin{verbatim}
HPINTERSECTION - computes the intersection of ellipsoid with hyperplane.

Input:
  regular:
      myEllArr: ellipsoid [nDims1,nDims2,...,nDimsN]/[1,1] - array
          of ellipsoids.
      myHypArr: hyperplane [nDims1,nDims2,...,nDimsN]/[1,1] - array
          of hyperplanes of the same size.

Output:
  intEllArr: ellipsoid [nDims1,nDims2,...,nDimsN] - array of ellipsoids
      resulting from intersections.

  isnIntersectedArr: logical [nDims1,nDims2,...,nDimsN].
      isnIntersectedArr(iCount) = true, if myEllArr(iCount)
      doesn't intersect myHipArr(iCount),
      isnIntersectedArr(iCount) = false, otherwise.

Example:
  ellObj = ellipsoid([-2; -1], [4 -1; -1 1]);
  hypMat = [hyperplane([0 -1; -1 0]', 1); hyperplane([0 -2; -1 0]', 1)];
  ellMat = ellObj.hpintersection(hypMat)

  ellMat =
  2x2 array of ellipsoids.
\end{verbatim}
\subsection{\texorpdfstring{ellipsoid.intersect}{intersect}}\label{method:ellipsoid.intersect}
\begin{verbatim}
INTERSECT - checks if the union or intersection of ellipsoids intersects
            given ellipsoid, hyperplane or polytope.

  resArr = INTERSECT(myEllArr, objArr, mode) - Checks if the union
      (mode = 'u') or intersection (mode = 'i') of ellipsoids
      in myEllArr intersects with objects in objArr.
      objArr can be array of ellipsoids, array of hyperplanes,
      or array of polytopes.
      Ellipsoids, hyperplanes or polytopes in objMat must have
      the same dimension as ellipsoids in myEllArr.
      mode = 'u' (default) - union of ellipsoids in myEllArr.
      mode = 'i' - intersection.

  If we need to check the intersection of union of ellipsoids in
  myEllArr (mode = 'u'), or if myEllMat is a single ellipsoid,
  it can be done by calling distance function for each of the
  ellipsoids in myEllArr and objMat, and if it returns negative value,
  the intersection is nonempty. Checking if the intersection of
  ellipsoids in myEllArr (with size of myEllMat greater than 1)
  intersects with ellipsoids or hyperplanes in objArr is more
  difficult. This problem can be formulated as quadratically
  constrained quadratic programming (QCQP) problem.

  Let objArr(iObj) = E(q, Q) be an ellipsoid with center q and shape
  matrix Q. To check if this ellipsoid intersects (or touches) the
  intersection of ellipsoids in meEllArr: E(q1, Q1), E(q2, Q2), ...,
  E(qn, Qn), we define the QCQP problem:
                    J(x) = <(x - q), Q^(-1)(x - q)> --> min
  with constraints:
                     <(x - q1), Q1^(-1)(x - q1)> <= 1   (1)
                     <(x - q2), Q2^(-1)(x - q2)> <= 1   (2)
                     ................................
                     <(x - qn), Qn^(-1)(x - qn)> <= 1   (n)

  If this problem is feasible, i.e. inequalities (1)-(n) do not
  contradict, or, in other words, intersection of ellipsoids
  E(q1, Q1), E(q2, Q2), ..., E(qn, Qn) is nonempty, then we can find
  vector y such that it satisfies inequalities (1)-(n) and minimizes
  function J. If J(y) <= 1, then ellipsoid E(q, Q) intersects or touches
  the given intersection, otherwise, it does not. To check if E(q, Q)
  intersects the union of E(q1, Q1), E(q2, Q2), ..., E(qn, Qn),
  we compute the distances from this ellipsoids to those in the union.
  If at least one such distance is negative,
  then E(q, Q) does intersect the union.

  If we check the intersection of ellipsoids with hyperplane
  objArr = H(v, c), it is enough to check the feasibility
  of the problem
                      1'x --> min
  with constraints (1)-(n), plus
                    <v, x> - c = 0.

  Checking the intersection of ellipsoids with polytope
  objArr = P(A, b) reduces to checking if there any x, satisfying
  constraints (1)-(n) and
                       Ax <= b.

Input:
  regular:
      myEllArr: ellipsoid [nDims1,nDims2,...,nDimsN] - array of
           ellipsoids.
      objArr: ellipsoid / hyperplane /
          / polytope [nDims1,nDims2,...,nDimsN] - array of ellipsoids or
          hyperplanes or polytopes of the same sizes.

  optional:
      mode: char[1, 1] - 'u' or 'i', go to description.

          note: If mode == 'u', then mRows, nCols should be equal to 1.

Output:
  resArr: double[nDims1,nDims2,...,nDimsN] - return:
      resArr(iCount) = -1 in case parameter mode is set
          to 'i' and the intersection of ellipsoids in myEllArr
          is empty.
      resArr(iCount) = 0 if the union or intersection of
          ellipsoids in myEllArr does not intersect the object
          in objArr(iCount).
      resArr(iCount) = 1 if the union or intersection of
          ellipsoids in myEllArr and the object in objArr(iCount)
          have nonempty intersection.
  statusArr: double[0, 0]/double[nDims1,nDims2,...,nDimsN] - status
      variable. statusArr is empty if mode = 'u'.

Example:
  firstEllObj = ellipsoid([-2; -1], [4 -1; -1 1]);
  secEllObj = firstEllObj + [5; 5];
  hypObj  = hyperplane([1; -1]);
  ellVec = [firstEllObj secEllObj];
  ellVec.intersect(hypObj)

  ans =

       1

  ellVec.intersect(hypObj, 'i')

  ans =

      -1
\end{verbatim}
\subsection{\texorpdfstring{ellipsoid.intersection\_ea}{intersection\_ea}}\label{method:ellipsoid.intersectionea}
\begin{verbatim}
INTERSECTION_EA - external ellipsoidal approximation of the
                  intersection of two ellipsoids, or ellipsoid and
                  halfspace, or ellipsoid and polytope.

  outEllArr = INTERSECTION_EA(myEllArr, objArr) Given two ellipsoidal
      matrixes of equal sizes, myEllArr and objArr = ellArr, or,
      alternatively, myEllArr or ellMat must be a single ellipsoid,
      computes the ellipsoid that contains the intersection of two
      corresponding ellipsoids from myEllArr and from ellArr.
  outEllArr = INTERSECTION_EA(myEllArr, objArr) Given matrix of
      ellipsoids myEllArr and matrix of hyperplanes objArr = hypArr
      whose sizes match, computes the external ellipsoidal
      approximations of intersections of ellipsoids
      and halfspaces defined by hyperplanes in hypArr.
      If v is normal vector of hyperplane and c - shift,
      then this hyperplane defines halfspace
              <v, x> <= c.
  outEllArr = INTERSECTION_EA(myEllArr, objArr) Given matrix of
      ellipsoids myEllArr and matrix of polytopes objArr = polyArr
      whose sizes match, computes the external ellipsoidal
      approximations of intersections of ellipsoids myEllMat and
      polytopes polyArr.

  The method used to compute the minimal volume overapproximating
  ellipsoid is described in "Ellipsoidal Calculus Based on
  Propagation and Fusion" by Lluis Ros, Assumpta Sabater and
  Federico Thomas; IEEE Transactions on Systems, Man and Cybernetics,
  Vol.32, No.4, pp.430-442, 2002. For more information, visit
  http://www-iri.upc.es/people/ros/ellipsoids.html

  For polytopes this method won't give the minimal volume
  overapproximating ellipsoid, but just some overapproximating ellipsoid.

Input:
  regular:
      myEllArr: ellipsoid [nDims1,nDims2,...,nDimsN]/[1,1] - array
          of ellipsoids.
      objArr: ellipsoid / hyperplane /
          / polytope [nDims1,nDims2,...,nDimsN]/[1,1]  - array of
          ellipsoids or hyperplanes or polytopes of the same sizes.

Example:
  firstEllObj = ellipsoid([-2; -1], [4 -1; -1 1]);
  secEllObj = firstEllObj + [5; 5];
  ellVec = [firstEllObj secEllObj];
  thirdEllObj  = ell_unitball(2);
  externalEllVec = ellVec.intersection_ea(thirdEllObj)

  externalEllVec =
  1x2 array of ellipsoids.
\end{verbatim}
\subsection{\texorpdfstring{ellipsoid.intersection\_ia}{intersection\_ia}}\label{method:ellipsoid.intersectionia}
\begin{verbatim}
INTERSECTION_IA - internal ellipsoidal approximation of the
                  intersection of ellipsoid and ellipsoid,
                  or ellipsoid and halfspace, or ellipsoid
                  and polytope.

  outEllArr = INTERSECTION_IA(myEllArr, objArr) - Given two
      ellipsoidal matrixes of equal sizes, myEllArr and
      objArr = ellArr, or, alternatively, myEllMat or ellMat must be
      a single ellipsoid, comuptes the internal ellipsoidal
      approximations of intersections of two corresponding ellipsoids
      from myEllMat and from ellMat.
  outEllArr = INTERSECTION_IA(myEllArr, objArr) - Given matrix of
      ellipsoids myEllArr and matrix of hyperplanes objArr = hypArr
      whose sizes match, computes the internal ellipsoidal
      approximations of intersections of ellipsoids and halfspaces
      defined by hyperplanes in hypMat.
      If v is normal vector of hyperplane and c - shift,
      then this hyperplane defines halfspace
                 <v, x> <= c.
  outEllArr = INTERSECTION_IA(myEllArr, objArr) - Given matrix of
      ellipsoids  myEllArr and matrix of polytopes objArr = polyArr
      whose sizes match, computes the internal ellipsoidal
      approximations of intersections of ellipsoids myEllArr
      and polytopes polyArr.

  The method used to compute the minimal volume overapproximating
  ellipsoid is described in "Ellipsoidal Calculus Based on
  Propagation and Fusion" by Lluis Ros, Assumpta Sabater and
  Federico Thomas; IEEE Transactions on Systems, Man and Cybernetics,
  Vol.32, No.4, pp.430-442, 2002. For more information, visit
  http://www-iri.upc.es/people/ros/ellipsoids.html

  The method used to compute maximum volume ellipsoid inscribed in
  intersection of ellipsoid and polytope, is modified version of
  algorithm of finding maximum volume ellipsoid inscribed in intersection
  of ellipsoids discribed in Stephen Boyd and Lieven Vandenberghe "Convex
  Optimization". It works properly for nondegenerate ellipsoid, but for
  degenerate ellipsoid result would not lie in this ellipsoid. The result
  considered as empty ellipsoid, when maximum absolute velue of element
  in its matrix is less than myEllipsoid.getAbsTol().

Input:
  regular:
      myEllArr: ellipsoid [nDims1,nDims2,...,nDimsN]/[1,1] - array
          of ellipsoids.
      objArr: ellipsoid / hyperplane /
          / polytope [nDims1,nDims2,...,nDimsN]/[1,1]  - array of
          ellipsoids or hyperplanes or polytopes of the same sizes.

Output:
   outEllArr: ellipsoid [nDims1,nDims2,...,nDimsN] - array of internal
      approximating ellipsoids; entries can be empty ellipsoids
      if the corresponding intersection is empty.

Example:
  firstEllObj = ellipsoid([-2; -1], [4 -1; -1 1]);
  secEllObj = firstEllObj + [5; 5];
  ellVec = [firstEllObj secEllObj];
  thirdEllObj  = ell_unitball(2);
  internalEllVec = ellVec.intersection_ia(thirdEllObj)

  internalEllVec =
  1x2 array of ellipsoids.
\end{verbatim}
\subsection{\texorpdfstring{ellipsoid.inv}{inv}}\label{method:ellipsoid.inv}
\begin{verbatim}
INV - inverts shape matrices of ellipsoids in the given array,
      modified given array is on output (not its copy).


  invEllArr = INV(myEllArr)  Inverts shape matrices of ellipsoids
      in the array myEllMat. In case shape matrix is sigular, it is
      regularized before inversion.

Input:
  regular:
    myEllArr: ellipsoid [nDims1,nDims2,...,nDimsN] - array of ellipsoids.

Output:
   myEllArr: ellipsoid [nDims1,nDims2,...,nDimsN] - array of ellipsoids
      with inverted shape matrices.

Example:
  ellObj = ellipsoid([1; 1], [4 -1; -1 5]);
  ellObj.inv()

  ans =

  Center:
       1
       1

  Shape Matrix:
      0.2632    0.0526
      0.0526    0.2105

  Nondegenerate ellipsoid in R^2.
\end{verbatim}
\subsection{\texorpdfstring{ellipsoid.isEmpty}{isEmpty}}\label{method:ellipsoid.isEmpty}
\begin{verbatim}
ISEMPTY - checks if the ellipsoid object is empty.

Input:
  regular:
      myEllArr: ellipsoid [nDims1,nDims2,...,nDimsN] - array of
           ellipsoids.

Output:
  isPositiveArr: logical[nDims1,nDims2,...,nDimsN],
      isPositiveArr(iCount) = true - if ellipsoid
      myEllMat(iCount) is empty, false - otherwise.

Example:
  ellObj = ellipsoid();
  isempty(ellObj)

  ans =

       1
\end{verbatim}
\subsection{\texorpdfstring{ellipsoid.isEqual}{isEqual}}\label{method:ellipsoid.isEqual}
\begin{verbatim}
ISEQUAL - produces logical array the same size as
          ellFirstArr/ellFirstArr (if they have the same).
          isEqualArr[iDim1, iDim2,...] is true if corresponding
          ellipsoids are equal and false otherwise.

Input:
  regular:
      ellFirstArr: ellipsoid[nDim1, nDim2,...] - multidimensional array
          of ellipsoids.
      ellSecArr: ellipsoid[nDim1, nDim2,...] - multidimensional array
          of ellipsoids.
  properties:
      'isPropIncluded': makes to compare second value properties, such as
      absTol etc.
Output:
  isEqualArr: logical[nDim1, nDim2,...] - multidimension array of
      logical values. isEqualArr[iDim1, iDim2,...] is true if
      corresponding ellipsoids are equal and false otherwise.

  reportStr: char[1,] - comparison report.
\end{verbatim}
\subsection{\texorpdfstring{ellipsoid.isInside}{isInside}}\label{method:ellipsoid.isInside}
\begin{verbatim}
ISINSIDE - checks if given ellipsoid(or array of
           ellipsoids) lies inside given object(or array
           of objects): ellipsoid or polytope.

Input:
  regular:
      ellArr: ellipsoid[nDims1,nDims2,...,nDimsN] - array
              of ellipsoids of the same dimension.
      objArr: ellipsoid/
              polytope[nDims1,nDims2,...,nDimsN] of
              objects of the same dimension. If
              ellArr and objArr both non-scalar, than
              size of ellArr must be the same as size of
              objArr. Note that polytopes could be
              combined only in vector of size [1,N].
Output:
  regular:
      resArr: logical[nDims1,nDims2,...,nDimsN] array of
              results. resArr[iDim1,...,iDimN] = true, if
              ellArr[iDim1,...,iDimN] lies inside
              objArr[iDim1,...,iDimN].

Example:
  firstEllObj = [0 ; 0] + ellipsoid(eye(2, 2));
  secEllObj = [0 ; 0] + ellipsoid(2*eye(2, 2));
  firstEllObj.isInside(secEllObj)

  ans =

       1
\end{verbatim}
\subsection{\texorpdfstring{ellipsoid.isbaddirection}{isbaddirection}}\label{method:ellipsoid.isbaddirection}
\begin{verbatim}
ISBADDIRECTION - checks if ellipsoidal approximations of geometric
                 difference of two ellipsoids can be computed for
                 given directions.
  isBadDirVec = ISBADDIRECTION(fstEll, secEll, dirsMat) - Checks if
      it is possible to build ellipsoidal approximation of the
      geometric difference of two ellipsoids fstEll - secEll in
      directions specified by matrix dirsMat (columns of dirsMat
      are direction vectors). Type 'help minkdiff_ea' or
      'help minkdiff_ia' for more information.

Input:
  regular:
      fstEll: ellipsoid [1, 1] - first ellipsoid. Suppose nDim - space
          dimension.
      secEll: ellipsoid [1, 1] - second ellipsoid of the same dimention.
      dirsMat: numeric[nDims, nCols] - matrix whose columns are
          direction vectors that need to be checked.
      absTol: double [1,1] - absolute tolerance

Output:
   isBadDirVec: logical[1, nCols] - array of true or false with length
      being equal to the number of columns in matrix dirsMat.
      ture marks direction vector as bad - ellipsoidal approximation
      true marks direction vector as bad - ellipsoidal approximation
      cannot be computed for this direction. false means the opposite.
\end{verbatim}
\subsection{\texorpdfstring{ellipsoid.isbigger}{isbigger}}\label{method:ellipsoid.isbigger}
\begin{verbatim}
ISBIGGER - checks if one ellipsoid would contain the other if their
           centers would coincide.

  isPositive = ISBIGGER(fstEll, secEll) - Given two single ellipsoids
      of the same dimension, fstEll and secEll, check if fstEll
      would contain secEll inside if they were both
      centered at origin.

Input:
  regular:
      fstEll: ellipsoid [1, 1] - first ellipsoid.
      secEll: ellipsoid [1, 1] - second ellipsoid
          of the same dimention.

Output:
  isPositive: logical[1, 1], true - if ellipsoid fstEll
      would contain secEll inside, false - otherwise.

Example:
  firstEllObj = ellipsoid([1; 1], eye(2));
  secEllObj = ellipsoid([1; 1], [4 -1; -1 5]);
  isbigger(firstEllObj, secEllObj)

  ans =

       0
\end{verbatim}
\subsection{\texorpdfstring{ellipsoid.isdegenerate}{isdegenerate}}\label{method:ellipsoid.isdegenerate}
\begin{verbatim}
ISDEGENERATE - checks if the ellipsoid is degenerate.

Input:
  regular:
      myEllArr: ellipsoid[nDims1,nDims2,...,nDimsN] - array of ellipsoids.

Output:
  isPositiveArr: logical[nDims1,nDims2,...,nDimsN],
      isPositiveArr(iCount) = true if ellipsoid myEllMat(iCount)
      is degenerate, false - otherwise.

Example:
  ellObj = ellipsoid([1; 1], eye(2));
  isdegenerate(ellObj)

  ans =

       0
\end{verbatim}
\subsection{\texorpdfstring{ellipsoid.isinternal}{isinternal}}\label{method:ellipsoid.isinternal}
\begin{verbatim}
ISINTERNAL - checks if given points belong to the union or intersection
             of ellipsoids in the given array.

  isPositiveVec = ISINTERNAL(myEllArr,  matrixOfVecMat, mode) - Checks
      if vectors specified as columns of matrix matrixOfVecMat
      belong to the union (mode = 'u'), or intersection (mode = 'i')
      of the ellipsoids in myEllArr. If myEllArr is a single
      ellipsoid, then this function checks if points in matrixOfVecMat
      belong to myEllArr or not. Ellipsoids in myEllArr must be
      of the same dimension. Column size of matrix  matrixOfVecMat
      should match the dimension of ellipsoids.

   Let myEllArr(iEll) = E(q, Q) be an ellipsoid with center q and shape
   matrix Q. Checking if given vector matrixOfVecMat = x belongs
   to E(q, Q) is equivalent to checking if inequality
                   <(x - q), Q^(-1)(x - q)> <= 1
   holds.
   If x belongs to at least one of the ellipsoids in the array, then it
   belongs to the union of these ellipsoids. If x belongs to all
   ellipsoids in the array,
   then it belongs to the intersection of these ellipsoids.
   The default value of the specifier s = 'u'.

   WARNING: be careful with degenerate ellipsoids.

Input:
  regular:
      myEllArr: ellipsoid [nDims1,nDims2,...,nDimsN] - array
          of ellipsoids.
      matrixOfVecMat: double [mRows, nColsOfVec] - matrix which
          specifiy points.

  optional:
      mode: char[1, 1] - 'u' or 'i', go to description.

Output:
   isPositiveVec: logical[1, nColsOfVec] -
      true - if vector belongs to the union or intersection
      of ellipsoids, false - otherwise.

Example:
  firstEllObj = ellipsoid([-2; -1], [4 -1; -1 1]);
  secEllObj = firstEllObj + [5; 5];
  ellVec = [firstEllObj secEllObj];
  ellVec.isinternal([-2 3; -1 4], 'i')

  ans =

       0     0

  ellVec.isinternal([-2 3; -1 4])

  ans =

       1     1
\end{verbatim}
\subsection{\texorpdfstring{ellipsoid.maxeig}{maxeig}}\label{method:ellipsoid.maxeig}
\begin{verbatim}
MAXEIG - return the maximal eigenvalue of the ellipsoid.

Input:
  regular:
      inpEllArr: ellipsoid [nDims1,nDims2,...,nDimsN] - array of
           ellipsoids.

Output:
  maxEigArr: double[nDims1,nDims2,...,nDimsN] - array of maximal
      eigenvalues of ellipsoids in the input matrix inpEllMat.

Example:
  ellObj = ellipsoid([-2; 4], [4 -1; -1 5]);
  maxEig = maxeig(ellObj)

  maxEig =

      5.6180
\end{verbatim}
\subsection{\texorpdfstring{ellipsoid.mineig}{mineig}}\label{method:ellipsoid.mineig}
\begin{verbatim}
MINEIG - return the minimal eigenvalue of the ellipsoid.

Input:
   regular:
      inpEllArr: ellipsoid [nDims1,nDims2,...,nDimsN] - array of
        ellipsoids.

Output:
   minEigArr: double[nDims1,nDims2,...,nDimsN] - array of minimal
      eigenvalues of ellipsoids in the input array inpEllMat.

Example:
  ellObj = ellipsoid([-2; 4], [4 -1; -1 5]);
  minEig = mineig(ellObj)

  minEig =

      3.3820
\end{verbatim}
\subsection{\texorpdfstring{ellipsoid.minkCommonAction}{minkCommonAction}}\label{method:ellipsoid.minkCommonAction}
\begin{verbatim}
MINKCOMMONACTION - plot Minkowski operation  of ellipsoids in 2D or 3D.
Usage:
minkCommonAction(getEllArr,fCalcBodyTriArr,...
   fCalcCenterTriArr,varargin) -  plot Minkowski operation  of
           ellipsoids in 2D or 3D, using triangulation  of output object

Input:
  regular:
      getEllArr:  Ellipsoid: [dim11Size,dim12Size,...,dim1kSize] -
               array of 2D or 3D Ellipsoids objects. All ellipsoids in
               ellArr must be either 2D or 3D simutaneously.
fCalcBodyTriArr - function, calculeted triangulation of output object
   fCalcCenterTriArr - function, calculeted center  of output object
           properties:
      'shawAll': logical[1,1] - if 1, plot all ellArr.
                   Default value is 0.
      'fill': logical[1,1]/logical[dim11Size,dim12Size,...,dim1kSize]  -
              if 1, ellipsoids in 2D will be filled with color.
              Default value is 0.
      'lineWidth': double[1,1]/double[dim11Size,dim12Size,...,dim1kSize]  -
                   line width for 1D and 2D plots. Default value is 1.
      'color': double[1,3]/double[dim11Size,dim12Size,...,dim1kSize,3] -
               sets default colors in the form [x y z].
              Default value is [1 0 0].
      'shade': double[1,1]/double[dim11Size,dim12Size,...,dim1kSize]  -
               level of transparency between 0 and 1
                  (0 - transparent, 1 - opaque).
               Default value is 0.4.
      'relDataPlotter' - relation data plotter object.

Output:
  centVec: double[nDim, 1] - center of the resulting set.
  boundPointMat: double[nDim, nBoundPoints] - set of boundary
      points (vertices) of resulting set.
\end{verbatim}
\subsection{\texorpdfstring{ellipsoid.minkdiff}{minkdiff}}\label{method:ellipsoid.minkdiff}
\begin{verbatim}
MINKDIFF - computes geometric (Minkowski) difference of two
            ellipsoids in 2D or 3D.
 Usage:
MINKDIFF(inpEllMat,'Property',PropValue,...) - Computes
geometric difference of two ellipsoids in the array inpEllMat, if
1 <= min(dimension(inpEllMat)) = max(dimension(inpEllMat)) <= 3,
       and plots it if no output arguments are specified.

   [centVec, boundPointMat] = MINKDIFF(inpEllMat) - Computes
       geometric difference of two ellipsoids in inpEllMat.
       Here centVec is
       the center, and boundPointMat - array of boundary points.
   MINKDIFF(inpEllMat) - Plots geometric differencr of two
   ellipsoids in inpEllMat in default (red) color.
   MINKDIFF(inpEllMat, 'Property',PropValue,...) -
    Plots geometric sum of inpEllMat
       with setting properties.

   In order for the geometric difference to be nonempty set,
   ellipsoid fstEll must be bigger than secEll in the sense that
   if fstEll and secEll had the same centerVec, secEll would be
   contained inside fstEll.
 Input:
   regular:
       ellArr:  Ellipsoid: [dim11Size,dim12Size,...,dim1kSize] -
                array of 2D or 3D Ellipsoids objects. All ellipsoids in ellArr
                must be either 2D or 3D simutaneously.

   properties:
       'shawAll': logical[1,1] - if 1, plot all ellArr.
                    Default value is 0.
       'fill': logical[1,1]/logical[dim11Size,dim12Size,...,dim1kSize]  -
               if 1, ellipsoids in 2D will be filled with color.
               Default value is 0.
       'lineWidth': double[1,1]/double[dim11Size,dim12Size,...,dim1kSize]  -
                    line width for 1D and 2D plots. Default value is 1.
       'color': double[1,3]/double[dim11Size,dim12Size,...,dim1kSize,3] -
                sets default colors in the form [x y z].
               Default value is [1 0 0].
       'shade': double[1,1]/double[dim11Size,dim12Size,...,dim1kSize]  -
                level of transparency between 0 and 1
                   (0 - transparent, 1 - opaque).
                Default value is 0.4.
       'relDataPlotter' - relation data plotter object.
       Notice that property vector could have different dimensions, only
       total number of elements must be the same.

 Output:
   centVec: double[nDim, 1] - center of the resulting set.
   boundPointMat: double[nDim, nBoundPoints] - set of boundary
       points (vertices) of resulting set.

 Example:
   firstEllObj = ellipsoid([-1; 1], [2 0; 0 3]);
   secEllObj = ellipsoid([1 2], eye(2));
   [centVec, boundPointMat] = minkdiff(firstEllObj, secEllObj);
\end{verbatim}
\subsection{\texorpdfstring{ellipsoid.minkdiff\_ea}{minkdiff\_ea}}\label{method:ellipsoid.minkdiffea}
\begin{verbatim}
MINKDIFF_EA - computation of external approximating ellipsoids
              of the geometric difference of two ellipsoids along
              given directions.

  extApprEllVec = MINKDIFF_EA(fstEll, secEll, directionsMat) -
      Computes external approximating ellipsoids of the
      geometric difference of two ellipsoids fstEll - secEll
      along directions specified by columns of matrix directionsMat

  First condition for the approximations to be computed, is that
  ellipsoid fstEll = E1 must be bigger than ellipsoid secEll = E2
  in the sense that if they had the same center, E2 would be contained
  inside E1. Otherwise, the geometric difference E1 - E2
  is an empty set.
  Second condition for the approximation in the given direction l
  to exist, is the following. Given
      P = sqrt(<l, Q1 l>)/sqrt(<l, Q2 l>)
  where Q1 is the shape matrix of ellipsoid E1, and
  Q2 - shape matrix of E2, and R being minimal root of the equation
      det(Q1 - R Q2) = 0,
  parameter P should be less than R.
  If both of these conditions are satisfied, then external
  approximating ellipsoid is defined by its shape matrix
      Q = (Q1^(1/2) + S Q2^(1/2))' (Q1^(1/2) + S Q2^(1/2)),
  where S is orthogonal matrix such that vectors
      Q1^(1/2)l and SQ2^(1/2)l
  are parallel, and its center
      q = q1 - q2,
  where q1 is center of ellipsoid E1 and q2 - center of E2.

Input:
  regular:
      fstEll: ellipsoid [1, 1] - first ellipsoid. Suppose
          nDim - space dimension.
      secEll: ellipsoid [1, 1] - second ellipsoid
          of the same dimention.
      directionsMat: double[nDim, nCols] - matrix whose columns
          specify the directions for which the approximations
          should be computed.

Output:
  extApprEllVec: ellipsoid [1, nCols] - array of external
      approximating ellipsoids (empty, if for all specified
      directions approximations cannot be computed).

Example:
  firstEllObj= ellipsoid([-2; -1], [4 -1; -1 1]);
  secEllObj = 3*ell_unitball(2);
  dirsMat = [1 0; 1 1; 0 1; -1 1]';
  externalEllVec = secEllObj.minkdiff_ea(firstEllObj, dirsMat)

  externalEllVec =
  1x2 array of ellipsoids.
\end{verbatim}
\subsection{\texorpdfstring{ellipsoid.minkdiff\_ia}{minkdiff\_ia}}\label{method:ellipsoid.minkdiffia}
\begin{verbatim}
MINKDIFF_IA - computation of internal approximating ellipsoids
              of the geometric difference of two ellipsoids along
              given directions.

  intApprEllVec = MINKDIFF_IA(fstEll, secEll, directionsMat) -
      Computes internal approximating ellipsoids of the geometric
      difference of two ellipsoids fstEll - secEll along directions
      specified by columns of matrix directionsMat.

  First condition for the approximations to be computed, is that
  ellipsoid fstEll = E1 must be bigger than ellipsoid secEll = E2
  in the sense that if they had the same center, E2 would be contained
  inside E1. Otherwise, the geometric difference E1 - E2 is an
  empty set. Second condition for the approximation in the given
  direction l to exist, is the following. Given
      P = sqrt(<l, Q1 l>)/sqrt(<l, Q2 l>)
  where Q1 is the shape matrix of ellipsoid E1,
  and Q2 - shape matrix of E2, and R being minimal root of the equation
      det(Q1 - R Q2) = 0,
  parameter P should be less than R.
  If these two conditions are satisfied, then internal approximating
  ellipsoid for the geometric difference E1 - E2 along the
  direction l is defined by its shape matrix
      Q = (1 - (1/P)) Q1 + (1 - P) Q2
  and its center
      q = q1 - q2,
  where q1 is center of E1 and q2 - center of E2.

Input:
  regular:
      fstEll: ellipsoid [1, 1] - first ellipsoid. Suppose
          nDim - space dimension.
      secEll: ellipsoid [1, 1] - second ellipsoid
          of the same dimention.
      directionsMat: double[nDim, nCols] - matrix whose columns
          specify the directions for which the approximations
          should be computed.

Output:
  intApprEllVec: ellipsoid [1, nCols] - array of internal
      approximating ellipsoids (empty, if for all specified directions
      approximations cannot be computed).

Example:
  firstEllObj = ellipsoid([-2; -1], [4 -1; -1 1]);
  secEllObj = 3*ell_unitball(2);
  dirsMat = [1 0; 1 1; 0 1; -1 1]';
  internalEllVec = secEllObj.minkdiff_ia(firstEllObj, dirsMat)

  internalEllVec =
  1x2 array of ellipsoids.
\end{verbatim}
\subsection{\texorpdfstring{ellipsoid.minkmp}{minkmp}}\label{method:ellipsoid.minkmp}
\begin{verbatim}
MINKMP - computes and plots geometric (Minkowski) sum of the
         geometric difference of two ellipsoids and the geometric
         sum of n ellipsoids in 2D or 3D:
         (E - Em) + (E1 + E2 + ... + En),
         where E = firstEll, Em = secondEll,
         E1, E2, ..., En - are ellipsoids in sumEllArr

Usage:
  MINKMP(firEll,secEll,ellMat,'Property',PropValue,...) -
          Computes (E1 - E2) + (E3 + E4+ ... + En), if
      1 <= min(dimension(inpEllMat)) = max(dimension(inpEllMat)) <= 3,
      and plots it if no output arguments are specified.

  [centVec, boundPointMat] = MINKMP(firEll,secEll,ellMat) - Computes
     (E1 - E2) + (E3 + E4+ ... + En). Here centVec is
      the center, and boundPointMat - array of boundary points.
Input:
  regular:
      ellArr:  Ellipsoid: [dim11Size,dim12Size,...,dim1kSize] -
          array of 2D or 3D Ellipsoids objects. All ellipsoids in ellArr
               must be either 2D or 3D simutaneously.

  properties:
      'showAll': logical[1,1] - if 1, plot all ellArr.
                   Default value is 0.
      'fill': logical[1,1]/logical[dim11Size,dim12Size,...,dim1kSize]  -
              if 1, ellipsoids in 2D will be filled with color.
              Default value is 0.
      'lineWidth': double[1,1]/double[dim11Size,dim12Size,...,dim1kSize]-
                   line width for 1D and 2D plots. Default value is 1.
      'color': double[1,3]/double[dim11Size,dim12Size,...,dim1kSize,3] -
               sets default colors in the form [x y z].
                  Default value is [1 0 0].
      'shade': double[1,1]/double[dim11Size,dim12Size,...,dim1kSize]  -
               level of transparency between 0 and 1
              (0 - transparent, 1 - opaque).
               Default value is 0.4.
      'relDataPlotter' - relation data plotter object.
      Notice that property vector could have different dimensions, only
      total number of elements must be the same.

Output:
  centVec: double[nDim, 1] - center of the resulting set.
  boundPointMat: double[nDim, nBoundPoints] - set of boundary
      points (vertices) of resulting set.

Example:
  firstEllObj = ellipsoid([-2; -1], [2 -1; -1 1]);
  secEllObj = ell_unitball(2);
  ellVec = [firstEllObj secEllObj ellipsoid([-3; 1], eye(2))];
  minkmp(firstEllObj, secEllObj, ellVec);
\end{verbatim}
\subsection{\texorpdfstring{ellipsoid.minkmp\_ea}{minkmp\_ea}}\label{method:ellipsoid.minkmpea}
\begin{verbatim}
MINKMP_EA - computation of external approximating ellipsoids
            of (E - Em) + (E1 + ... + En) along given directions.
            where E = fstEll, Em = secEll,
            E1, E2, ..., En - are ellipsoids in sumEllArr

  extApprEllVec = MINKMP_EA(fstEll, secEll, sumEllArr, dirMat) -
      Computes external approximating
      ellipsoids of (E - Em) + (E1 + E2 + ... + En),
      where E1, E2, ..., En are ellipsoids in array sumEllArr,
      E = fstEll, Em = secEll,
      along directions specified by columns of matrix dirMat.

Input:
  regular:
      fstEll: ellipsoid [1, 1] - first ellipsoid. Suppose
          nDims - space dimension.
      secEll: ellipsoid [1, 1] - second ellipsoid
          of the same dimention.
      sumEllArr: ellipsoid [nDims1, nDims2,...,nDimsN] - array of
          ellipsoids of the same dimentions nDims.
      dirMat: double[nDims, nCols] - matrix whose columns specify the
          directions for which the approximations should be computed.

Output:
  extApprEllVec: ellipsoid [1, nCols] - array of external
      approximating ellipsoids (empty, if for all specified
      directions approximations cannot be computed).

Example:
  firstEllObj = ellipsoid([-2; -1], [4 -1; -1 1]);
  secEllObj = 3*ell_unitball(2);
  dirsMat = [1 0; 1 1; 0 1; -1 1]';
  bufEllVec = [secEllObj firstEllObj];
  externalEllVec = secEllObj.minkmp_ea(firstEllObj, bufEllVec, dirsMat)

  externalEllVec =
  1x2 array of ellipsoids.
\end{verbatim}
\subsection{\texorpdfstring{ellipsoid.minkmp\_ia}{minkmp\_ia}}\label{method:ellipsoid.minkmpia}
\begin{verbatim}
MINKMP_IA - computation of internal approximating ellipsoids
            of (E - Em) + (E1 + ... + En) along given directions.
            where E = fstEll, Em = secEll,
            E1, E2, ..., En - are ellipsoids in sumEllArr

  intApprEllVec = MINKMP_IA(fstEll, secEll, sumEllArr, dirMat) -
      Computes internal approximating
      ellipsoids of (E - Em) + (E1 + E2 + ... + En),
      where E1, E2, ..., En are ellipsoids in array sumEllArr,
      E = fstEll, Em = secEll,
      along directions specified by columns of matrix dirMat.

Input:
  regular:
      fstEll: ellipsoid [1, 1] - first ellipsoid. Suppose
          nDim - space dimension.
      secEll: ellipsoid [1, 1] - second ellipsoid
          of the same dimention.
      sumEllArr: ellipsoid [nDims1, nDims2,...,nDimsN] - array of
          ellipsoids of the same dimentions.
      dirMat: double[nDim, nCols] - matrix whose columns specify the
          directions for which the approximations should be computed.

Output:
  intApprEllVec: ellipsoid [1, nCols] - array of internal
      approximating ellipsoids (empty, if for all specified
      directions approximations cannot be computed).

Example:
  firstEllObj = ellipsoid([-2; -1], [4 -1; -1 1]);
  secEllObj = 3*ell_unitball(2);
  dirsMat = [1 0; 1 1; 0 1; -1 1]';
  bufEllVec = [secEllObj firstEllObj];
  internalEllVec = secEllObj.minkmp_ia(firstEllObj, bufEllVec, dirsMat)

  internalEllVec =
  1x2 array of ellipsoids.
\end{verbatim}
\subsection{\texorpdfstring{ellipsoid.minkpm}{minkpm}}\label{method:ellipsoid.minkpm}
\begin{verbatim}
MINKPM - computes and plots geometric (Minkowski) difference
         of the geometric sum of ellipsoids and a single ellipsoid
         in 2D or 3D: (E1 + E2 + ... + En) - E,
         where E = inpEll,
         E1, E2, ... En - are ellipsoids in inpEllArr.

  MINKPM(inpEllArr, inpEll, OPTIONS)  Computes geometric difference
      of the geometric sum of ellipsoids in inpEllMat and
      ellipsoid inpEll, if
      1 <= dimension(inpEllArr) = dimension(inpArr) <= 3,
      and plots it if no output arguments are specified.

  [centVec, boundPointMat] = MINKPM(inpEllArr, inpEll) - pomputes
      (geometric sum of ellipsoids in inpEllArr) - inpEll.
      Here centVec is the center, and boundPointMat - array
      of boundary points.
  MINKPM(inpEllArr, inpEll) - plots (geometric sum of ellipsoids
      in inpEllArr) - inpEll in default (red) color.
  MINKPM(inpEllArr, inpEll, Options) - plots
      (geometric sum of ellipsoids in inpEllArr) - inpEll using
      options given in the Options structure.

Input:
  regular:
      inpEllArr: ellipsoid [nDims1, nDims2,...,nDimsN] - array of
          ellipsoids of the same dimentions 2D or 3D.
      inpEll: ellipsoid [1, 1] - ellipsoid of the same
          dimention 2D or 3D.

  optional:
      Options: structure[1, 1] - fields:
          show_all: double[1, 1] - if 1, displays
              also ellipsoids fstEll and secEll.
          newfigure: double[1, 1] - if 1, each plot
              command will open a new figure window.
          fill: double[1, 1] - if 1, the resulting
              set in 2D will be filled with color.
          color: double[1, 3] - sets default colors
              in the form [x y z].
          shade: double[1, 1] = 0-1 - level of transparency
              (0 - transparent, 1 - opaque).

Output:
   centVec: double[nDim, 1]/double[0, 0] - center of the resulting set.
      centerVec may be empty.
   boundPointMat: double[nDim, ]/double[0, 0] - set of boundary
      points (vertices) of resulting set. boundPointMat may be empty.
\end{verbatim}
\subsection{\texorpdfstring{ellipsoid.minkpm\_ea}{minkpm\_ea}}\label{method:ellipsoid.minkpmea}
\begin{verbatim}
MINKPM_EA - computation of external approximating ellipsoids
            of (E1 + E2 + ... + En) - E along given directions.
            where E = inpEll,
            E1, E2, ... En - are ellipsoids in inpEllArr.

  ExtApprEllVec = MINKPM_EA(inpEllArr, inpEll, dirMat) - Computes
      external approximating ellipsoids of
      (E1 + E2 + ... + En) - E, where E1, E2, ..., En are ellipsoids
      in array inpEllArr, E = inpEll,
      along directions specified by columns of matrix dirMat.

Input:
  regular:
      inpEllArr: ellipsoid [nDims1, nDims2,...,nDimsN] -
          array of ellipsoids of the same dimentions.
      inpEll: ellipsoid [1, 1] - ellipsoid of the same dimention.
      dirMat: double[nDim, nCols] - matrix whose columns specify
          the directions for which the approximations
          should be computed.

Output:
  extApprEllVec: ellipsoid [1, nCols]/[0, 0] - array of external
      approximating ellipsoids. Empty, if for all specified
      directions approximations cannot be computed.

Example:
  firstEllObj = ellipsoid([2; -1], [9 -5; -5 4]);
  secEllObj = ellipsoid([-2; -1], [4 -1; -1 1]);
  thirdEllObj = ell_unitball(2);
  dirsMat = [1 0; 1 1; 0 1; -1 1]';
  ellVec = [thirdEllObj firstEllObj];
  externalEllVec = ellVec.minkpm_ea(secEllObj, dirsMat)

  externalEllVec =
  1x4 array of ellipsoids.
\end{verbatim}
\subsection{\texorpdfstring{ellipsoid.minkpm\_ia}{minkpm\_ia}}\label{method:ellipsoid.minkpmia}
\begin{verbatim}
MINKPM_IA - computation of internal approximating ellipsoids
            of (E1 + E2 + ... + En) - E along given directions.
            where E = inpEll,
            E1, E2, ... En - are ellipsoids in inpEllArr.

  intApprEllVec = MINKPM_IA(inpEllArr, inpEll, dirMat) - Computes
      internal approximating ellipsoids of
      (E1 + E2 + ... + En) - E, where E1, E2, ..., En are ellipsoids
      in array inpEllArr, E = inpEll,
      along directions specified by columns of matrix dirArr.

Input:
  regular:
      inpEllArr: ellipsoid [nDims1, nDims2,...,nDimsN] -
          array of ellipsoids of the same dimentions.
      inpEll: ellipsoid [1, 1] - ellipsoid of the same dimention.
      dirMat: double[nDim, nCols] - matrix whose columns specify
          the directions for which the approximations
          should be computed.

Output:
  intApprEllVec: ellipsoid [1, nCols]/[0, 0] - array of internal
      approximating ellipsoids. Empty, if for all specified
      directions approximations cannot be computed.

Example:
  firstEllObj = ellipsoid([2; -1], [9 -5; -5 4]);
  secEllObj = ellipsoid([-2; -1], [4 -1; -1 1]);
  thirdEllObj = ell_unitball(2);
  ellVec = [thirdEllObj firstEllObj];
  dirsMat = [1 0; 1 1; 0 1; -1 1]';
  internalEllVec = ellVec.minkpm_ia(secEllObj, dirsMat)

  internalEllVec =
  1x3 array of ellipsoids.
\end{verbatim}
\subsection{\texorpdfstring{ellipsoid.minksum}{minksum}}\label{method:ellipsoid.minksum}
\begin{verbatim}
MINKSUM - computes geometric (Minkowski) sum of ellipsoids in 2D or 3D.

Usage:
  MINKSUM(inpEllMat,'Property',PropValue,...) - Computes geometric sum of
      ellipsoids in the array inpEllMat, if
      1 <= min(dimension(inpEllMat)) = max(dimension(inpEllMat)) <= 3,
      and plots it if no output arguments are specified.

  [centVec, boundPointMat] = MINKSUM(inpEllMat) - Computes
      geometric sum of ellipsoids in inpEllMat. Here centVec is
      the center, and boundPointMat - array of boundary points.
  MINKSUM(inpEllMat) - Plots geometric sum of ellipsoids in
      inpEllMat in default (red) color.
  MINKSUM(inpEllMat, 'Property',PropValue,...) - Plots geometric sum of
  inpEllMat with setting properties.

Input:
  regular:
      ellArr:  Ellipsoid: [dim11Size,dim12Size,...,dim1kSize] -
               array of 2D or 3D Ellipsoids objects. All ellipsoids
               in ellArr must be either 2D or 3D simutaneously.

  properties:
   'showAll': logical[1,1] - if 1, plot all ellArr.
                   Default value is 0.
   'fill': logical[1,1]/logical[dim11Size,dim12Size,...,dim1kSize]  -
              if 1, ellipsoids in 2D will be filled with color. Default
              value is 0.
   'lineWidth': double[1,1]/double[dim11Size,dim12Size,...,dim1kSize]-
                   line width for 1D and 2D plots. Default value is 1.
   'color': double[1,3]/double[dim11Size,dim12Size,...,dim1kSize,3] -
       sets default colors in the form [x y z]. Default value is [1 0 0].
   'shade': double[1,1]/double[dim11Size,dim12Size,...,dim1kSize]  -
     level of transparency between 0 and 1 (0 - transparent, 1 - opaque).
               Default value is 0.4.
      'relDataPlotter' - relation data plotter object.
      Notice that property vector could have different dimensions, only
      total number of elements must be the same.

Output:
  centVec: double[nDim, 1] - center of the resulting set.
  boundPointMat: double[nDim, nBoundPoints] - set of boundary
      points (vertices) of resulting set.

Example:
  firstEllObj = ellipsoid([-2; -1], [2 -1; -1 1]);
  secEllObj = ell_unitball(2);
  ellVec = [firstEllObj, secellObj]
  sumVec = minksum(ellVec);
\end{verbatim}
\subsection{\texorpdfstring{ellipsoid.minksum\_ea}{minksum\_ea}}\label{method:ellipsoid.minksumea}
\begin{verbatim}
MINKSUM_EA - computation of external approximating ellipsoids
             of the geometric sum of ellipsoids along given directions.

  extApprEllVec = MINKSUM_EA(inpEllArr, dirMat) - Computes
      tight external approximating ellipsoids for the geometric
      sum of the ellipsoids in the array inpEllArr along directions
      specified by columns of dirMat.
      If ellipsoids in inpEllArr are n-dimensional, matrix
      dirMat must have dimension (n x k) where k can be
      arbitrarily chosen.
      In this case, the output of the function will contain k
      ellipsoids computed for k directions specified in dirMat.

  Let inpEllArr consists of E(q1, Q1), E(q2, Q2), ..., E(qm, Qm) -
  ellipsoids in R^n, and dirMat(:, iCol) = l - some vector in R^n.
  Then tight external approximating ellipsoid E(q, Q) for the
  geometric sum E(q1, Q1) + E(q2, Q2) + ... + E(qm, Qm)
  along direction l, is such that
      rho(l | E(q, Q)) = rho(l | (E(q1, Q1) + ... + E(qm, Qm)))
  and is defined as follows:
      q = q1 + q2 + ... + qm,
      Q = (p1 + ... + pm)((1/p1)Q1 + ... + (1/pm)Qm),
  where
      p1 = sqrt(<l, Q1l>), ..., pm = sqrt(<l, Qml>).

Input:
  regular:
      inpEllArr: ellipsoid [nDims1, nDims2,...,nDimsN] - array
          of ellipsoids of the same dimentions.
      dirMat: double[nDims, nCols] - matrix whose columns specify
          the directions for which the approximations
          should be computed.

Output:
  extApprEllVec: ellipsoid [1, nCols] - array of external
      approximating ellipsoids.

Example:
  firstEllObj = ellipsoid([-2; -1], [4 -1; -1 1]);
  secEllObj = ell_unitball(2);
  ellVec = [firstEllObj secEllObj firstEllObj.inv()];
  dirsMat = [1 0; 1 1; 0 1; -1 1]';
  externalEllVec = ellVec.minksum_ea(dirsMat)

  externalEllVec =
  1x4 array of ellipsoids.
\end{verbatim}
\subsection{\texorpdfstring{ellipsoid.minksum\_ia}{minksum\_ia}}\label{method:ellipsoid.minksumia}
\begin{verbatim}
MINKSUM_IA - computation of internal approximating ellipsoids
             of the geometric sum of ellipsoids along given directions.

  intApprEllVec = MINKSUM_IA(inpEllArr, dirMat) - Computes
      tight internal approximating ellipsoids for the geometric
      sum of the ellipsoids in the array inpEllArr along directions
      specified by columns of dirMat. If ellipsoids in
      inpEllArr are n-dimensional, matrix dirMat must have
      dimension (n x k) where k can be arbitrarily chosen.
      In this case, the output of the function will contain k
      ellipsoids computed for k directions specified in dirMat.

  Let inpEllArr consist of E(q1, Q1), E(q2, Q2), ..., E(qm, Qm) -
  ellipsoids in R^n, and dirMat(:, iCol) = l - some vector in R^n.
  Then tight internal approximating ellipsoid E(q, Q) for the
  geometric sum E(q1, Q1) + E(q2, Q2) + ... + E(qm, Qm) along
  direction l, is such that
      rho(l | E(q, Q)) = rho(l | (E(q1, Q1) + ... + E(qm, Qm)))
  and is defined as follows:
      q = q1 + q2 + ... + qm,
      Q = (S1 Q1^(1/2) + ... + Sm Qm^(1/2))' *
          * (S1 Q1^(1/2) + ... + Sm Qm^(1/2)),
  where S1 = I (identity), and S2, ..., Sm are orthogonal
  matrices such that vectors
  (S1 Q1^(1/2) l), ..., (Sm Qm^(1/2) l) are parallel.

Input:
  regular:
      inpEllArr: ellipsoid [nDims1, nDims2,...,nDimsN] - array
          of ellipsoids of the same dimentions.
      dirMat: double[nDim, nCols] - matrix whose columns specify the
          directions for which the approximations should be computed.

Output:
  intApprEllVec: ellipsoid [1, nCols] - array of internal
      approximating ellipsoids.

Example:
  firstEllObj = ellipsoid([-2; -1], [4 -1; -1 1]);
  secEllObj = ell_unitball(2);
  ellVec = [firstEllObj secEllObj firstEllObj.inv()];
  dirsMat = [1 0; 1 1; 0 1; -1 1]';
  internalEllVec = ellVec.minksum_ia(dirsMat)

  internalEllVec =
  1x4 array of ellipsoids.
\end{verbatim}
\subsection{\texorpdfstring{ellipsoid.minus}{minus}}\label{method:ellipsoid.minus}
\begin{verbatim}
MINUS - overloaded operator '-'

  outEllArr = MINUS(inpEllArr, inpVec) implements E(q, Q) - b
      for each ellipsoid E(q, Q) in inpEllArr.
  outEllArr = MINUS(inpVec, inpEllArr) implements b - E(q, Q)
      for each ellipsoid E(q, Q) in inpEllArr.

  Operation E - b where E = inpEll is an ellipsoid in R^n,
  and b = inpVec - vector in R^n. If E(q, Q) is an ellipsoid
  with center q and shape matrix Q, then
  E(q, Q) - b = E(q - b, Q).

Input:
  regular:
      inpEllArr: ellipsoid [nDims1,nDims2,...,nDimsN] - array of
          ellipsoids of the same dimentions nDims.
      inpVec: double[nDims, 1] - vector.

Output:
   outEllVec: ellipsoid [nDims1,nDims2,...,nDimsN] - array of ellipsoids
      with same shapes as inpEllVec, but with centers shifted by vectors
      in -inpVec.

Example:
  ellVec  = [ellipsoid([-2; -1], [4 -1; -1 1]) ell_unitball(2)];
  outEllVec = ellVec - [1; 1];
  outEllVec(1)

  ans =

  Center:
      -3
      -2

  Shape:
       4    -1
      -1     1

  Nondegenerate ellipsoid in R^2.

  outEllVec(2)

  ans =

  Center:
      -1
      -1

  Shape:
       1     0
       0     1

  Nondegenerate ellipsoid in R^2.
\end{verbatim}
\subsection{\texorpdfstring{ellipsoid.move2origin}{move2origin}}\label{method:ellipsoid.move2origin}
\begin{verbatim}
MOVE2ORIGIN - moves ellipsoids in the given array to the origin. Modified
              given array is on output (not its copy).

  outEllArr = MOVE2ORIGIN(inpEll) - Replaces the centers of
      ellipsoids in inpEllArr with zero vectors.

Input:
  regular:
      inpEllArr: ellipsoid [nDims1,nDims2,...,nDimsN] - array of
          ellipsoids.

Output:
  inpEllArr: ellipsoid [nDims1,nDims2,...,nDimsN] - array of ellipsoids
      with the same shapes as in inpEllArr centered at the origin.

Example:
  ellObj = ellipsoid([-2; -1], [4 -1; -1 1]);
  outEllObj = ellObj.move2origin()

  outEllObj =

  Center:
       0
       0

  Shape:
       4    -1
      -1     1

  Nondegenerate ellipsoid in R^2.
\end{verbatim}
\subsection{\texorpdfstring{ellipsoid.mtimes}{mtimes}}\label{method:ellipsoid.mtimes}
\begin{verbatim}
MTIMES - overloaded operator '*'.

  Multiplication of the ellipsoid by a matrix or a scalar.
  If inpEllVec(iEll) = E(q, Q) is an ellipsoid, and
  multMat = A - matrix of suitable dimensions,
  then A E(q, Q) = E(Aq, AQA').

Input:
  regular:
      multMat: double[mRows, nDims]/[1, 1] - scalar or
          matrix in R^{mRows x nDim}
      inpEllVec: ellipsoid [1, nCols] - array of ellipsoids.

Output:
  outEllVec: ellipsoid [1, nCols] - resulting ellipsoids.

Example:
  ellObj = ellipsoid([-2; -1], [4 -1; -1 1]);
  tempMat = [0 1; -1 0];
  outEllObj = tempMat*ellObj

  outEllObj =

  Center:
      -1
       2

  Shape:
       1     1
       1     4

  Nondegenerate ellipsoid in R^2.
\end{verbatim}
\subsection{\texorpdfstring{ellipsoid.parameters}{parameters}}\label{method:ellipsoid.parameters}
\begin{verbatim}
PARAMETERS - returns parameters of the ellipsoid.

Input:
  regular:
      myEll: ellipsoid [1, 1] - single ellipsoid of dimention nDims.

Output:
  myEllCenterVec: double[nDims, 1] - center of the ellipsoid myEll.
  myEllShapeMat: double[nDims, nDims] - shape matrix
      of the ellipsoid myEll.

Example:
  ellObj = ellipsoid([-2; 4], [4 -1; -1 5]);
  [centVec shapeMat] = parameters(ellObj)
  centVec =

      -2
       4

  shapeMat =

      4    -1
     -1     5
\end{verbatim}
\subsection{\texorpdfstring{ellipsoid.plot}{plot}}\label{method:ellipsoid.plot}
\begin{verbatim}
PLOT - plots ellipsoids in 2D or 3D.


Usage:
      plot(ell) - plots ellipsoid ell in default (red) color.
      plot(ellArr) - plots an array of ellipsoids.
      plot(ellArr, 'Property',PropValue,...) - plots ellArr with setting
                                               properties.

Input:
  regular:
      ellArr:  Ellipsoid: [dim11Size,dim12Size,...,dim1kSize] -
               array of 2D or 3D Ellipsoids objects. All ellipsoids in ellArr
               must be either 2D or 3D simutaneously.
  optional:
      color1Spec: char[1,1] - color specification code, can be 'r','g',
                              etc (any code supported by built-in Matlab function).
      ell2Arr: Ellipsoid: [dim21Size,dim22Size,...,dim2kSize] -
                                          second ellipsoid array...
      color2Spec: char[1,1] - same as color1Spec but for ell2Arr
      ....
      ellNArr: Ellipsoid: [dimN1Size,dim22Size,...,dimNkSize] -
                                           N-th ellipsoid array
      colorNSpec - same as color1Spec but for ellNArr.
  properties:
      'newFigure': logical[1,1] - if 1, each plot command will open a new figure window.
                   Default value is 0.
      'fill': logical[1,1]/logical[dim11Size,dim12Size,...,dim1kSize]  -
              if 1, ellipsoids in 2D will be filled with color. Default value is 0.
      'lineWidth': double[1,1]/double[dim11Size,dim12Size,...,dim1kSize]  -
                   line width for 1D and 2D plots. Default value is 1.
      'color': double[1,3]/double[dim11Size,dim12Size,...,dim1kSize,3] -
               sets default colors in the form [x y z]. Default value is [1 0 0].
      'shade': double[1,1]/double[dim11Size,dim12Size,...,dim1kSize]  -
               level of transparency between 0 and 1 (0 - transparent, 1 - opaque).
               Default value is 0.4.
      'relDataPlotter' - relation data plotter object.
      Notice that property vector could have different dimensions, only
      total number of elements must be the same.
Output:
  regular:
      plObj: smartdb.disp.RelationDataPlotter[1,1] - returns the relation
      data plotter object.

Examples:
      plot([ell1, ell2, ell3], 'color', [1, 0, 1; 0, 0, 1; 1, 0, 0]);
      plot([ell1, ell2, ell3], 'color', [1; 0; 1; 0; 0; 1; 1; 0; 0]);
      plot([ell1, ell2, ell3; ell1, ell2, ell3], 'shade', [1, 1, 1; 1, 1,
      1]);
      plot([ell1, ell2, ell3; ell1, ell2, ell3], 'shade', [1; 1; 1; 1; 1;
      1]);
      plot([ell1, ell2, ell3], 'shade', 0.5);
      plot([ell1, ell2, ell3], 'lineWidth', 1.5);
      plot([ell1, ell2, ell3], 'lineWidth', [1.5, 0.5, 3]);
\end{verbatim}
\subsection{\texorpdfstring{ellipsoid.plus}{plus}}\label{method:ellipsoid.plus}
\begin{verbatim}
PLUS - overloaded operator '+'

  outEllArr = PLUS(inpEllArr, inpVec) implements E(q, Q) + b
      for each ellipsoid E(q, Q) in inpEllArr.
  outEllArr = PLUS(inpVec, inpEllArr) implements b + E(q, Q)
      for each ellipsoid E(q, Q) in inpEllArr.

   Operation E + b (or b+E) where E = inpEll is an ellipsoid in R^n,
  and b=inpVec - vector in R^n. If E(q, Q) is an ellipsoid
  with center q and shape matrix Q, then
  E(q, Q) + b = b + E(q,Q) = E(q + b, Q).

Input:
  regular:
      ellArr: ellipsoid [nDims1,nDims2,...,nDimsN] - array of ellipsoids
          of the same dimentions nDims.
      bVec: double[nDims, 1] - vector.

Output:
  outEllArr: ellipsoid [nDims1,nDims2,...,nDimsN] - array of ellipsoids
      with same shapes as ellVec, but with centers shifted by vectors
      in inpVec.

Example:
  ellVec  = [ellipsoid([-2; -1], [4 -1; -1 1]) ell_unitball(2)];
  outEllVec = ellVec + [1; 1];
  outEllVec(1)

  ans =

  Center:
      -1
       0

  Shape:
      4    -1
     -1     1

  Nondegenerate ellipsoid in R^2.

  outEllVec(2)

  ans =

  Center:
       1
       1

  Shape:
      1     0
      0     1

  Nondegenerate ellipsoid in R^2.
\end{verbatim}
\subsection{\texorpdfstring{ellipsoid.polar}{polar}}\label{method:ellipsoid.polar}
\begin{verbatim}
POLAR - computes the polar ellipsoids.

  polEllArr = POLAR(ellArr)  Computes the polar ellipsoids for those
      ellipsoids in ellArr, for which the origin is an interior point.
      For those ellipsoids in E, for which this condition does not hold,
      an empty ellipsoid is returned.

  Given ellipsoid E(q, Q) where q is its center, and Q - its shape matrix,
  the polar set to E(q, Q) is defined as follows:
  P = { l in R^n  | <l, q> + sqrt(<l, Q l>) <= 1 }
  If the origin is an interior point of ellipsoid E(q, Q),
  then its polar set P is an ellipsoid.

Input:
  regular:
      ellArr: ellipsoid [nDims1,nDims2,...,nDimsN] - array
          of ellipsoids.

Output:
  polEllArr: ellipsoid [nDims1,nDims2,...,nDimsN] - array of
       polar ellipsoids.

Example:
  ellObj = ellipsoid([4 -1; -1 1]);
  ellObj.polar() == ellObj.inv()

  ans =

      1
\end{verbatim}
\subsection{\texorpdfstring{ellipsoid.projection}{projection}}\label{method:ellipsoid.projection}
\begin{verbatim}
PROJECTION - computes projection of the ellipsoid onto the given subspace.
             modified given array is on output (not its copy).

  projEllArr = projection(ellArr, basisMat)  Computes projection of the
      ellipsoid ellArr onto a subspace, specified by orthogonal
      basis vectors basisMat. ellArr can be an array of ellipsoids of
      the same dimension. Columns of B must be orthogonal vectors.

Input:
  regular:
      ellArr: ellipsoid [nDims1,nDims2,...,nDimsN] - array
          of ellipsoids.
      basisMat: double[nDim, nSubSpDim] - matrix of orthogonal basis
          vectors

Output:
  ellArr: ellipsoid [nDims1,nDims2,...,nDimsN] - array of
      projected ellipsoids, generally, of lower dimension.

Example:
  ellObj = ellipsoid([-2; -1; 4], [4 -1 0; -1 1 0; 0 0 9]);
  basisMat = [0 1 0; 0 0 1]';
  outEllObj = ellObj.projection(basisMat)

  outEllObj =

  Center:
      -1
       4

  Shape:
      1     0
      0     9

  Nondegenerate ellipsoid in R^2.
\end{verbatim}
\subsection{\texorpdfstring{ellipsoid.repMat}{repMat}}\label{method:ellipsoid.repMat}
\begin{verbatim}
REPMAT - is analogous to built-in repmat function with one exception - it
         copies the objects, not just the handles

Example:
  firstEllObj = ellipsoid([1; 2], eye(2));
  secEllObj = ellipsoid([1; 1], 2*eye(2));
  ellVec = [firstEllObj secEllObj];
  repMat(ellVec)

  ans =
  1x2 array of ellipsoids.
\end{verbatim}
\subsection{\texorpdfstring{ellipsoid.rho}{rho}}\label{method:ellipsoid.rho}
\begin{verbatim}
RHO - computes the values of the support function for given ellipsoid
      and given direction.

      supArr = RHO(ellArr, dirsMat)  Computes the support function of the
      ellipsoid ellArr in directions specified by the columns of matrix
      dirsMat. Or, if ellArr is array of ellipsoids, dirsMat is expected
      to be a single vector.

      [supArr, bpArr] = RHO(ellArr, dirstMat)  Computes the support function
      of the ellipsoid ellArr in directions specified by the columns of
      matrix dirsMat, and boundary points bpArr of this ellipsoid that
      correspond to directions in dirsMat. Or, if ellArr is array of
      ellipsoids, and dirsMat - single vector, then support functions and
      corresponding boundary points are computed for all the given
      ellipsoids in the array in the specified direction dirsMat.

      The support function is defined as
  (1)  rho(l | E) = sup { <l, x> : x belongs to E }.
      For ellipsoid E(q,Q), where q is its center and Q - shape matrix,
  it is simplified to
  (2)  rho(l | E) = <q, l> + sqrt(<l, Ql>)
  Vector x, at which the maximum at (1) is achieved is defined by
  (3)  q + Ql/sqrt(<l, Ql>)

Input:
  regular:
      ellArr: ellipsoid [nDims1,nDims2,...,nDimsN]/[1,1] - array
          of ellipsoids.
      dirsMat: double[nDim,nDims1,nDims2,...,nDimsN]/
          double[nDim,nDirs]/[nDim,1] - array or matrix of directions.

Output:
      supArr: double [nDims1,nDims2,...,nDimsN]/[1,nDirs] - support function
      of the ellArr in directions specified by the columns of matrix
      dirsMat. Or, if ellArr is array of ellipsoids, support function of
      each ellipsoid in ellArr specified by dirsMat direction.

  bpArr: double[nDim,nDims1,nDims2,...,nDimsN]/
          double[nDim,nDirs]/[nDim,1] - array or matrix of boundary points

Example:
  ellObj = ellipsoid([-2; 4], [4 -1; -1 1]);
  dirsMat = [-2 5; 5 1];
  suppFuncVec = rho(ellObj, dirsMat)

  suppFuncVec =

      31.8102    3.5394
\end{verbatim}
\subsection{\texorpdfstring{ellipsoid.shape}{shape}}\label{method:ellipsoid.shape}
\begin{verbatim}
SHAPE - modifies the shape matrix of the ellipsoid without
  changing its center. Modified given array is on output (not its copy).

   modEllArr = SHAPE(ellArr, modMat)  Modifies the shape matrices of
      the ellipsoids in the ellipsoidal array ellArr. The centers
      remain untouched - that is the difference of the function SHAPE and
      linear transformation modMat*ellArr. modMat is expected to be a
      scalar or a square matrix of suitable dimension.

Input:
  regular:
      ellArr: ellipsoid [nDims1,nDims2,...,nDimsN] - array
          of ellipsoids.
      modMat: double[nDim, nDim]/[1,1] - square matrix or scalar

Output:
   ellArr: ellipsoid [nDims1,nDims2,...,nDimsN] - array of modified
      ellipsoids.

Example:
  ellObj = ellipsoid([-2; -1], [4 -1; -1 1]);
  tempMat = [0 1; -1 0];
  outEllObj = shape(ellObj, tempMat)

  outEllObj =

  Center:
      -2
      -1

  Shape:
      1     1
      1     4

  Nondegenerate ellipsoid in R^2.
\end{verbatim}
\subsection{\texorpdfstring{ellipsoid.toPolytope}{toPolytope}}\label{method:ellipsoid.toPolytope}
\begin{verbatim}
TOPOLYTOPE - for ellipsoid ell makes polytope object represanting the
             boundary of ell

Input:
  regular:
      ell: ellipsoid[1,1] - ellipsoid in 3D or 2D.
  optional:
      nPoints: double[1,1] - number of boundary points.
               Actually number of points in resulting
               polytope will be ecual to lowest
               number of points of icosaeder, that greater
               than nPoints.

Output:
  regular:
      poly: polytope[1,1] - polytop in 3D or 2D.
\end{verbatim}
\subsection{\texorpdfstring{ellipsoid.toStruct}{toStruct}}\label{method:ellipsoid.toStruct}
\begin{verbatim}
toStruct -- converts ellipsoid array into structural array.

Input:
  regular:
      ellArr: ellipsoid [nDim1, nDim2, ...] - array
          of ellipsoids.
Output:
  SDataArr: struct[nDims1,...,nDimsk] - structure array same size, as
      ellArr, contain all data.
  SFieldNiceNames: struct[1,1] - structure with the same fields as SDataArr. Field values
      contain the nice names.
  SFieldDescr: struct[1,1] - structure with same fields as SDataArr,
      values contain field descriptions.

      q: double[1, nEllDim] - the center of ellipsoid
      Q: double[nEllDim, nEllDim] - the shape matrix of ellipsoid

Example:
  ellObj = ellipsoid([1 1]', eye(2));
  ellObj.toStruct()

  ans =

  Q: [2x2 double]
  q: [1 1]
\end{verbatim}
\subsection{\texorpdfstring{ellipsoid.trace}{trace}}\label{method:ellipsoid.trace}
\begin{verbatim}
TRACE - returns the trace of the ellipsoid.

   trArr = TRACE(ellArr)  Computes the trace of ellipsoids in
      ellipsoidal array ellArr.

Input:
  regular:
      ellArr: ellipsoid [nDims1,nDims2,...,nDimsN] - array
          of ellipsoids.

Output:
   trArr: double [nDims1,nDims2,...,nDimsN] - array of trace values,
      same size as ellArr.

Example:
  firstEllObj = ellipsoid([4 -1; -1 1]);
  secEllObj = ell_unitball(2);
  ellVec = [firstEllObj secEllObj];
  trVec = ellVec.trace()

  trVec =

      5     2
\end{verbatim}
\subsection{\texorpdfstring{ellipsoid.uminus}{uminus}}\label{method:ellipsoid.uminus}
\begin{verbatim}
UMINUS - changes the sign of the centerVec of ellipsoid.

Input:
   regular:
      ellArr: ellipsoid [nDims1,nDims2,...,nDimsN] - array of ellipsoids.


Output:
   outEllArr: ellipsoid [nDims1,nDims2,...,nDimsN] - array of ellipsoids,
       same size as ellArr.

Example:
  ellObj = -ellipsoid([-2; -1], [4 -1; -1 1])

  ellObj =

  Center:
       2
       1

  Shape:
       4    -1
      -1     1

  Nondegenerate ellipsoid in R^2.
\end{verbatim}
\subsection{\texorpdfstring{ellipsoid.volume}{volume}}\label{method:ellipsoid.volume}
\begin{verbatim}
VOLUME - returns the volume of the ellipsoid.

   volArr = VOLUME(ellArr)  Computes the volume of ellipsoids in
      ellipsoidal array ellArr.

   The volume of ellipsoid E(q, Q) with center q and shape matrix Q
   is given by V = S sqrt(det(Q)) where S is the volume of unit ball.

Input:
  regular:
      ellArr: ellipsoid [nDims1,nDims2,...,nDimsN] - array
          of ellipsoids.

Output:
   volArr: double [nDims1,nDims2,...,nDimsN] - array of
      volume values, same size as ellArr.

Example:
  firstEllObj = ellipsoid([4 -1; -1 1]);
  secEllObj = ell_unitball(2);
  ellVec = [firstEllObj secEllObj]
  volVec = ellVec.volume()

  volVec =

      5.4414     3.1416
\end{verbatim}
\section{hyperplane}\label{secClassDescr:hyperplane}
\subsection{\texorpdfstring{hyperplane.checkIsMe}{checkIsMe}}\label{method:hyperplane.checkIsMe}
\begin{verbatim}
CHECKISME - determine whether input object is hyperplane. And display
            message and abort function if input object
            is not hyperplane

Input:
  regular:
      someObjArr: any[] - any type array of objects.

Example:
  hypObj = hyperplane([-2, 0]);
  hyperplane.checkIsMe(hypObj)
\end{verbatim}
\subsection{\texorpdfstring{hyperplane.contains}{contains}}\label{method:hyperplane.contains}
\begin{verbatim}
CONTAINS - checks if given vectors belong to the hyperplanes.

  isPosArr = CONTAINS(myHypArr, xArr) - Checks if vectors specified
      by columns xArr(:, hpDim1, hpDim2, ...) belong
      to hyperplanes in myHypArr.

Input:
  regular:
      myHypArr: hyperplane [nCols, 1]/[1, nCols]/
          /[hpDim1, hpDim2, ...]/[1, 1] - array of hyperplanes
          of the same dimentions nDims.
      xArr: double[nDims, nCols]/[nDims, hpDim1, hpDim2, ...]/
          /[nDims, 1]/[nDims, nVecArrDim1, nVecArrDim2, ...] - array
          whose columns represent the vectors needed to be checked.

          note: if size of myHypArr is [hpDim1, hpDim2, ...], then
              size of xArr is [nDims, hpDim1, hpDim2, ...]
              or [nDims, 1], if size of myHypArr [1, 1], then xArr
              can be any size [nDims, nVecArrDim1, nVecArrDim2, ...],
              in this case output variable will has
              size [1, nVecArrDim1, nVecArrDim2, ...]. If size of
              xArr is [nDims, nCols], then size of myHypArr may be
              [nCols, 1] or [1, nCols] or [1, 1], output variable
              will has size respectively
              [nCols, 1] or [1, nCols] or [nCols, 1].

Output:
  isPosArr: logical[hpDim1, hpDim2,...] /
      / logical[1, nVecArrDim1, nVecArrDim2, ...],
      isPosArr(iDim1, iDim2, ...) = true - myHypArr(iDim1, iDim2, ...)
      contains xArr(:, iDim1, iDim2, ...), false - otherwise.

Example:
  hypObj = hyperplane([-1; 1]);
  tempMat = [100 -1 2; 100 1 2];
  hypObj.contains(tempMat)

  ans =

       1
       0
       1
\end{verbatim}
\subsection{\texorpdfstring{hyperplane.contents}{contents}}\label{method:hyperplane.contents}
\begin{verbatim}
Hyperplane object of the Ellipsoidal Toolbox.


Functions:
----------
 hyperplane - Constructor of hyperplane object.
 double     - Returns parameters of hyperplane, i.e. normal vector and
              shift.
 parameters - Same function as 'double' (legacy matter).
 dimension  - Returns dimension of hyperplane.
 isempty    - Checks if hyperplane is empty.
 isparallel - Checks if one hyperplane is parallel to the other one.
 contains   - Check if hyperplane contains given point.


Overloaded operators and functions:
-----------------------------------
 eq      - Checks if two hyperplanes are equal.
 ne      - The opposite of 'eq'.
 uminus  - Switches signs of normal and shift parameters to the opposite.
 display - Displays the details about given hyperplane object.
 plot    - Plots hyperplane in 2D and 3D.
\end{verbatim}
\subsection{\texorpdfstring{hyperplane.dimension}{dimension}}\label{method:hyperplane.dimension}
\begin{verbatim}
DIMENSION - returns dimensions of hyperplanes in the array.

  dimsArr = DIMENSION(hypArr) - returns dimensions of hyperplanes
      described by hyperplane structures in the array hypArr.

Input:
  regular:
      hypArr: hyperplane [nDims1, nDims2, ...] - array
          of hyperplanes.

Output:
      dimsArr: double[nDims1, nDims2, ...] - dimensions
          of hyperplanes.

Example:
  firstHypObj = hyperplane([-1; 1]);
  secHypObj = hyperplane([-1; 1; 8; -2; 3], 7);
  thirdHypObj = hyperplane([1; 2; 0], -1);
  hypVec = [firstHypObj secHypObj thirdHypObj];
  dimsVec  = hypVec.dimension()

  dimsVec =

     2     5     3
\end{verbatim}
\subsection{\texorpdfstring{hyperplane.display}{display}}\label{method:hyperplane.display}
\begin{verbatim}
DISPLAY - Displays hyperplane object.

Input:
  regular:
      myHypArr: hyperplane [hpDim1, hpDim2, ...] - array
          of hyperplanes.

Example:
  hypObj = hyperplane([-1; 1]);
  display(hypObj)

  hypObj =
  size: [1 1]

  Element: [1 1]
  Normal:
      -1
       1

  Shift:
       0

  Hyperplane in R^2.
\end{verbatim}
\subsection{\texorpdfstring{hyperplane.double}{double}}\label{method:hyperplane.double}
\begin{verbatim}
DOUBLE - return parameters of hyperplane - normal vector and shift.

  [normVec, hypScal] = DOUBLE(myHyp) - returns normal vector
      and scalar value of the hyperplane.

Input:
  regular:
      myHyp: hyperplane [1, 1] - single hyperplane of dimention nDims.

Output:
  normVec: double[nDims, 1] - normal vector of the hyperplane myHyp.
  hypScal: double[1, 1] - scalar of the hyperplane myHyp.

Example:
  hypObj = hyperplane([-1; 1]);
  [normVec, hypScal] = double(hypObj)

  normVec =

      -1
       1

  hypScal =

       0
\end{verbatim}
\subsection{\texorpdfstring{hyperplane.fromRepMat}{fromRepMat}}\label{method:hyperplane.fromRepMat}
\begin{verbatim}
FROMREPMAT - returns array of equal hyperplanes the same
             size as stated in sizeVec argument

  hpArr = fromRepMat(sizeVec) - creates an array  size
           sizeVec of empty hyperplanes.

  hpArr = fromRepMat(normalVec,sizeVec) - creates an array
           size sizeVec of hyperplanes with normal
           normalVec.

  hpArr = fromRepMat(normalVec,shift,sizeVec) - creates an
           array size sizeVec of hyperplanes with normal normalVec
           and hyperplane shift shift.

Input:
  Case1:
      regular:
          sizeVec: double[1,n] - vector of size, have
          integer values.

  Case2:
      regular:
          normalVec: double[nDim, 1] - normal of
          hyperplanes.
          sizeVec: double[1, n] - vector of size, have
          integer values.

  Case3:
      regular:
          normalVec: double[nDim, 1] - normal of
          hyperplanes.
          shift: double[1, 1] - shift of hyperplane.
          sizeVec: double[1,n] - vector of size, have
          integer values.

  properties:
      absTol: double [1,1] - absolute tolerance with default
          value 10^(-7)
\end{verbatim}
\subsection{\texorpdfstring{hyperplane.fromStruct}{fromStruct}}\label{method:hyperplane.fromStruct}
\begin{verbatim}
fromStruct -- converts structural array into hyperplanes array.

Input:
  regular:
  SHpArr: struct [hpDim1, hpDim2, ...] -  structural array with following fields:

       normal: double[nHpDim, 1] - the normal of hyperplane
       shift: double[1, 1] - the shift of hyperplane

Output:
  hpArr : hyperplane [nDim1, nDim2, ...] - hyperplane array with size of
      SHpArr.


Example:
  hpObj = hyperplane([1 1]', 1);
  hpObj.toStruct()

  ans =

  normal: [2x1 double]
  shift: 0.7071
\end{verbatim}
\subsection{\texorpdfstring{hyperplane.getAbsTol}{getAbsTol}}\label{method:hyperplane.getAbsTol}
\begin{verbatim}
GETABSTOL - gives the array of absTol for all elements in hplaneArr

Input:
  regular:
      ellArr: hyperplane[nDim1, nDim2, ...] - multidimension array
          of hyperplane
  optional
      fAbsTolFun: function_handle[1,1] - function that apply
          to the absTolArr. The default is @min.

Output:
  regular:
      absTolArr: double [absTol1, absTol2, ...] - return absTol for
          each element in hplaneArr
  optional:
      absTol: double[1, 1] - return result of work fAbsTolFun with
          the absTolArr

Usage:
  use [~,absTol] = hplaneArr.getAbsTol() if you want get only
      absTol,
  use [absTolArr,absTol] = hplaneArr.getAbsTol() if you want get
      absTolArr and absTol,
  use absTolArr = hplaneArr.getAbsTol() if you want get only absTolArr

Example:
  firstHypObj = hyperplane([-1; 1]);
  secHypObj = hyperplane([-2; 5]);
  hypVec = [firstHypObj secHypObj];
  hypVec.getAbsTol()

  ans =

     1.0e-07 *

      1.0000    1.0000
\end{verbatim}
\subsection{\texorpdfstring{hyperplane.getCopy}{getCopy}}\label{method:hyperplane.getCopy}
\begin{verbatim}
GETCOPY - gives array the same size as hpArr with copies of elements of
          hpArr.

Input:
  regular:
      hpArr: hyperplane[nDim1, nDim2,...] - multidimensional array of
          hyperplanes.

Output:
  copyHpArr: hyperplane[nDim1, nDim2,...] - multidimension array of
      copies of elements of hpArr.

Example:
  firstHpObj = hyperplane([-1; 1], [2 0; 0 3]);
  secHpObj = hyperplane([1; 2], eye(2));
  hpVec = [firstHpObj secHpObj];
  copyHpVec = getCopy(hpVec)

  copyHpVec =
  1x2 array of hyperplanes.
\end{verbatim}
\subsection{\texorpdfstring{hyperplane.getProperty}{getProperty}}\label{method:hyperplane.getProperty}
\begin{verbatim}
GETPROPERTY - gives array the same size as hpArr with values of
              propName properties for each hyperplane in hpArr.
              Private method, used in every public property getter.

Input:
  regular:
      hpArr: hyperplane[nDim1, nDim2,...] - mltidimensional array
          of hyperplanes
      propName: char[1,N] - name property
  optional:
      fPropFun: function_handle[1,1] - function that apply
          to the propArr. The default is @min.

Output:
  regular:
      propArr: double[nDim1, nDim2,...] - multidimension array of
          propName properties for hyperplanes in rsArr
  optional:
      propVal: double[1, 1] - return result of work fPropFun with
          the propArr
\end{verbatim}
\subsection{\texorpdfstring{hyperplane.getRelTol}{getRelTol}}\label{method:hyperplane.getRelTol}
\begin{verbatim}
GETRELTOL - gives the array of relTol for all elements in hpArr

Input:
  regular:
      hpArr: hyperplane[nDim1, nDim2, ...] - multidimension array
          of hyperplanes
  optional:
      fRelTolFun: function_handle[1,1] - function that apply
          to the relTolArr. The default is @min.
Output:
  regular:
      relTolArr: double [relTol1, relTol2, ...] - return relTol for
          each element in hpArr
  optional:
      relTol: double[1,1] - return result of work fRelTolFun with
          the relTolArr

Usage:
  use [~,relTol] = hpArr.getRelTol() if you want get only
      relTol,
  use [relTolArr,relTol] = hpArr.getRelTol() if you want get
      relTolArr and relTol,
  use relTolArr = hpArr.getRelTol() if you want get only relTolArr

Example:
  firsthpObj = hyperplane([-1; 1], 1);
  sechpObj = hyperplane([1 ;2], 2);
  hpVec = [firsthpObj sechpObj];
  hpVec.getRelTol()

  ans =

     1.0e-05 *

      1.0000    1.0000
\end{verbatim}
\subsection{\texorpdfstring{hyperplane.hyperplane}{hyperplane}}\label{method:hyperplane.hyperplane}
\begin{verbatim}
HYPERPLANE - creates hyperplane structure
             (or array of hyperplane structures).

  Hyperplane H = { x in R^n : <v, x> = c },
  with current "Properties"..
  Here v must be vector in R^n, and c - scalar.

  hypH = HYPERPLANE - create empty hyperplane.

  hypH = HYPERPLANE(hypNormVec) - create
      hyperplane object hypH with properties:
          hypH.normal = hypNormVec,
          hypH.shift = 0.

  hypH = HYPERPLANE(hypNormVec, hypConst) - create
      hyperplane object hypH with properties:
          hypH.normal = hypNormVec,
          hypH.shift = hypConst.

  hypH = HYPERPLANE(hypNormVec, hypConst, ...
      'absTol', absTolVal) - create
      hyperplane object hypH with properties:
          hypH.normal = hypNormVec,
          hypH.shift = hypConst.
          hypH.absTol = absTolVal

  hypObjArr = HYPERPLANE(hypNormArr, hypConstArr) - create
      array of hyperplanes object just as
      hyperplane(hypNormVec, hypConst).

  hypObjArr = HYPERPLANE(hypNormArr, hypConstArr, ...
      'absTol', absTolValArr) - create
      array of hyperplanes object just as
      hyperplane(hypNormVec, hypConst, 'absTol', absTolVal).

Input:
  Case1:
    regular:
      hypNormArr: double[hpDims, nDims1, nDims2,...] -
          array of vectors in R^hpDims. There hpDims -
          hyperplane dimension.

  Case2:
    regular:
      hypNormArr: double[hpDims, nCols] /
          / [hpDims, nDims1, nDims2,...] /
          / [hpDims, 1] - array of vectors
          in R^hpDims. There hpDims - hyperplane dimension.
      hypConstArr: double[1, nCols] / [nCols, 1] /
          / [nDims1, nDims2,...] /
          / [nVecArrDim1, nVecArrDim2,...] -
          array of scalar.

  Case3:
    regular:
      hypNormArr: double[hpDims, nCols] /
          / [hpDims, nDims1, nDims2,...] /
          / [hpDims, 1] - array of vectors
          in R^hpDims. There hpDims - hyperplane dimension.
      hypConstArr: double[1, nCols] / [nCols, 1] /
          / [nDims1, nDims2,...] /
          / [nVecArrDim1, nVecArrDim2,...] -
          array of scalar.
      absTolValArr: double[1, 1] - value of
          absTol propeties.

    properties:
      propMode: char[1,] - property mode, the following
          modes are supported:
          'absTol' - name of absTol properties.

          note: if size of hypNormArr is
              [hpDims, nDims1, nDims2,...], then size of
              hypConstArr is [nDims1, nDims2, ...] or
              [1, 1], if size of hypNormArr [hpDims, 1],
              then hypConstArr can be any size
              [nVecArrDim1, nVecArrDim2, ...],
              in this case output variable will has
              size [nVecArrDim1, nVecArrDim2, ...].
              If size of hypNormArr is [hpDims, nCols],
              then size of hypConstArr may be
              [1, nCols] or [nCols, 1],
              output variable will has size
              respectively [1, nCols] or [nCols, 1].

Output:
  hypObjArr: hyperplane [nDims1, nDims2...] /
      / hyperplane [nVecArrDim1, nVecArrDim2, ...] -
      array of hyperplane structure hypH:
          hypH.normal - vector in R^hpDims,
          hypH.shift  - scalar.

Example:
  hypNormMat = [1 1 1; 1 1 1];
  hypConstVec = [1 -5 0];
  hypObj = hyperplane(hypNormMat, hypConstVec);
\end{verbatim}
\subsection{\texorpdfstring{hyperplane.isEmpty}{isEmpty}}\label{method:hyperplane.isEmpty}
\begin{verbatim}
ISEMPTY - checks if hyperplanes in H are empty.

Input:
  regular:
      myHypArr: hyperplane [nDims1, nDims2, ...] - array
          of hyperplanes.

Output:
  isPositiveArr: logical[nDims1, nDims2, ...],
      isPositiveArr(iDim1, iDim2, ...) = true - if ellipsoid
      myHypArr(iDim1, iDim2, ...) is empty, false - otherwise.

Example:
  hypObj = hyperplane();
  isempty(hypObj)

  ans =

       1
\end{verbatim}
\subsection{\texorpdfstring{hyperplane.isEqual}{isEqual}}\label{method:hyperplane.isEqual}
\begin{verbatim}
ISEQUAL - produces logical array the same size as
          ellFirstArr/ellFirstArr (if they have the same).
          isEqualArr[iDim1, iDim2,...] is true if corresponding
          ellipsoids are equal and false otherwise.

Input:
  regular:
      ellFirstArr: ellipsoid[nDim1, nDim2,...] - multidimensional array
          of ellipsoids.
      ellSecArr: ellipsoid[nDim1, nDim2,...] - multidimensional array
          of ellipsoids.
  properties:
      'isPropIncluded': makes to compare second value properties, such as
      absTol etc.
Output:
  isEqualArr: logical[nDim1, nDim2,...] - multidimension array of
      logical values. isEqualArr[iDim1, iDim2,...] is true if
      corresponding ellipsoids are equal and false otherwise.

  reportStr: char[1,] - comparison report.
\end{verbatim}
\subsection{\texorpdfstring{hyperplane.isparallel}{isparallel}}\label{method:hyperplane.isparallel}
\begin{verbatim}
ISPARALLEL - check if two hyperplanes are parallel.

  isResArr = ISPARALLEL(fstHypArr, secHypArr) - Checks if hyperplanes
      in fstHypArr are parallel to hyperplanes in secHypArr and
      returns array of true and false of the size corresponding
      to the sizes of fstHypArr and secHypArr.

Input:
  regular:
      fstHypArr: hyperplane [nDims1, nDims2, ...] - first array
          of hyperplanes
      secHypArr: hyperplane [nDims1, nDims2, ...] - second array
          of hyperplanes

Output:
  isPosArr: logical[nDims1, nDims2, ...] -
      isPosArr(iFstDim, iSecDim, ...) = true -
      if fstHypArr(iFstDim, iSecDim, ...) is parallel
      secHypArr(iFstDim, iSecDim, ...), false - otherwise.

Example:
  hypObj = hyperplane([-1 1 1; 1 1 1; 1 1 1], [2 1 0]);
  hypObj.isparallel(hypObj(2))

  ans =

       0     1     1
\end{verbatim}
\subsection{\texorpdfstring{hyperplane.parameters}{parameters}}\label{method:hyperplane.parameters}
\begin{verbatim}
PARAMETERS - return parameters of hyperplane - normal vector and shift.

  [normVec, hypScal] = PARAMETERS(myHyp) - returns normal vector
      and scalar value of the hyperplane.

Input:
  regular:
      myHyp: hyperplane [1, 1] - single hyperplane of dimention nDims.

Output:
  normVec: double[nDims, 1] - normal vector of the hyperplane myHyp.
  hypScal: double[1, 1] - scalar of the hyperplane myHyp.

Example:
  hypObj = hyperplane([-1; 1]);
  [normVec, hypScal] = parameters(hypObj)

  normVec =

      -1
       1


  hypScal =

       0
\end{verbatim}
\subsection{\texorpdfstring{hyperplane.plot}{plot}}\label{method:hyperplane.plot}
\begin{verbatim}
PLOT - plots hyperplaces in 2D or 3D.


Usage:
      plot(hyp) - plots hyperplace hyp in default (red) color.
      plot(hypArr) - plots an array of hyperplaces.
      plot(hypArr, 'Property',PropValue,...) - plots hypArr with setting
                                               properties.

Input:
  regular:
      hypArr:  Hyperplace: [dim11Size,dim12Size,...,dim1kSize] -
               array of 2D or 3D hyperplace objects. All hyperplaces in hypArr
               must be either 2D or 3D simutaneously.
  optional:
      color1Spec: char[1,1] - color specification code, can be 'r','g',
                              etc (any code supported by built-in Matlab function).
      hyp2Arr: Hyperplane: [dim21Size,dim22Size,...,dim2kSize] -
                                          second Hyperplane array...
      color2Spec: char[1,1] - same as color1Spec but for hyp2Arr
      ....
      hypNArr: Hyperplane: [dimN1Size,dim22Size,...,dimNkSize] -
                                           N-th Hyperplane array
      colorNSpec - same as color1Spec but for hypNArr.
  properties:
      'newFigure': logical[1,1] - if 1, each plot command will open a new figure window.
                   Default value is 0.
      'fill': logical[1,1]/logical[dim11Size,dim12Size,...,dim1kSize]  -
              if 1, ellipsoids in 2D will be filled with color. Default value is 0.
      'lineWidth': double[1,1]/double[dim11Size,dim12Size,...,dim1kSize]  -
                   line width for 1D and 2D plots. Default value is 1.
      'color': double[1,3]/double[dim11Size,dim12Size,...,dim1kSize,3] -
               sets default colors in the form [x y z]. Default value is [1 0 0].
      'shade': double[1,1]/double[dim11Size,dim12Size,...,dim1kSize]  -
               level of transparency between 0 and 1 (0 - transparent, 1 - opaque).
               Default value is 0.4.
      'size': double[1,1] - length of the line segment in 2D, or square diagonal in 3D.
      'center': double[1,dimHyp] - center of the line segment in 2D, of the square in 3D
      'relDataPlotter' - relation data plotter object.
      Notice that property vector could have different dimensions, only
      total number of elements must be the same.
Output:
  regular:
      plObj: smartdb.disp.RelationDataPlotter[1,1] - returns the relation
      data plotter object.
\end{verbatim}
\subsection{\texorpdfstring{hyperplane.toStruct}{toStruct}}\label{method:hyperplane.toStruct}
\begin{verbatim}
toStruct -- converts hyperplanes array into structural array.

Input:
  regular:
      hpArr: hyperplane [hpDim1, hpDim2, ...] - array
          of hyperplanes.

Output:
  ShpArr : struct[nDim1, nDim2, ...] - structural array with size of
      hpArr with the following fields:

      normal: double[nHpDim, 1] - the normal of hyperplane
      shift: double[1, 1] - the shift of hyperplane
\end{verbatim}
\subsection{\texorpdfstring{hyperplane.uminus}{uminus}}\label{method:hyperplane.uminus}
\begin{verbatim}
UMINUS - switch signs of normal vector and the shift scalar
         to the opposite.

Input:
  regular:
      inpHypArr: hyperplane [nDims1, nDims2, ...] - array
          of hyperplanes.

Output:
  outHypArr: hyperplane [nDims1, nDims2, ...] - array
      of the same hyperplanes as in inpHypArr whose
      normals and scalars are multiplied by -1.

Example:
  hypObj = -hyperplane([-1; 1], 1)

  hypObj =
  size: [1 1]

  Element: [1 1]
  Normal:
       1
      -1

  Shift:
      -1

  Hyperplane in R^2.
\end{verbatim}
\section{elltool.conf.Properties}\label{secClassDescr:elltool.conf.Properties}
\subsection{\texorpdfstring{elltool.conf.Properties.Properties}{Properties}}\label{method:elltool.conf.Properties.Properties}
\begin{verbatim}
PROPERTIES - a static class, providing emulation of static properties for
             toolbox.
\end{verbatim}
\subsection{\texorpdfstring{elltool.conf.Properties.checkSettings}{checkSettings}}\label{method:elltool.conf.Properties.checkSettings}
\begin{verbatim}
Example:
  elltool.conf.Properties.checkSettings()
\end{verbatim}
\subsection{\texorpdfstring{elltool.conf.Properties.getAbsTol}{getAbsTol}}\label{method:elltool.conf.Properties.getAbsTol}
\begin{verbatim}
Example:
  elltool.conf.Properties.getAbsTol();
\end{verbatim}
\subsection{\texorpdfstring{elltool.conf.Properties.getConfRepoMgr}{getConfRepoMgr}}\label{method:elltool.conf.Properties.getConfRepoMgr}
\begin{verbatim}
Example:
  elltool.conf.Properties.getConfRepoMgr()

  ans =

    elltool.conf.ConfRepoMgr handle
    Package: elltool.conf

    Properties:
      DEFAULT_STORAGE_BRANCH_KEY: '_default'
\end{verbatim}
\subsection{\texorpdfstring{elltool.conf.Properties.getIsEnabledOdeSolverOptions}{getIsEnabledOdeSolverOptions}}\label{method:elltool.conf.Properties.getIsEnabledOdeSolverOptions}
\begin{verbatim}
Example:
  elltool.conf.Properties.getIsEnabledOdeSolverOptions();
\end{verbatim}
\subsection{\texorpdfstring{elltool.conf.Properties.getIsODENormControl}{getIsODENormControl}}\label{method:elltool.conf.Properties.getIsODENormControl}
\begin{verbatim}
Example:
  elltool.conf.Properties.getIsODENormControl();
\end{verbatim}
\subsection{\texorpdfstring{elltool.conf.Properties.getIsVerbose}{getIsVerbose}}\label{method:elltool.conf.Properties.getIsVerbose}
\begin{verbatim}
Example:
  elltool.conf.Properties.getIsVerbose();
\end{verbatim}
\subsection{\texorpdfstring{elltool.conf.Properties.getNPlot2dPoints}{getNPlot2dPoints}}\label{method:elltool.conf.Properties.getNPlot2dPoints}
\begin{verbatim}
Example:
  elltool.conf.Properties.getNPlot2dPoints();
\end{verbatim}
\subsection{\texorpdfstring{elltool.conf.Properties.getNPlot3dPoints}{getNPlot3dPoints}}\label{method:elltool.conf.Properties.getNPlot3dPoints}
\begin{verbatim}
Example:
  elltool.conf.Properties.getNPlot3dPoints();
\end{verbatim}
\subsection{\texorpdfstring{elltool.conf.Properties.getNTimeGridPoints}{getNTimeGridPoints}}\label{method:elltool.conf.Properties.getNTimeGridPoints}
\begin{verbatim}
Example:
  elltool.conf.Properties.getNTimeGridPoints();
\end{verbatim}
\subsection{\texorpdfstring{elltool.conf.Properties.getODESolverName}{getODESolverName}}\label{method:elltool.conf.Properties.getODESolverName}
\begin{verbatim}
Example:
  elltool.conf.Properties.getODESolverName();
\end{verbatim}
\subsection{\texorpdfstring{elltool.conf.Properties.getPropStruct}{getPropStruct}}\label{method:elltool.conf.Properties.getPropStruct}
\begin{verbatim}
Example:
  elltool.conf.Properties.getConfRepoMgr.getCurConf()

  ans =

                    version: '1.4dev'
                  isVerbose: 0
                     absTol: 1.0000e-07
                     relTol: 1.0000e-05
            nTimeGridPoints: 200
              ODESolverName: 'ode45'
           isODENormControl: 'on'
  isEnabledOdeSolverOptions: 0
              nPlot2dPoints: 200
              nPlot3dPoints: 200
                    logging: [1x1 struct]
\end{verbatim}
\subsection{\texorpdfstring{elltool.conf.Properties.getRegTol}{getRegTol}}\label{method:elltool.conf.Properties.getRegTol}
\begin{verbatim}

\end{verbatim}
\subsection{\texorpdfstring{elltool.conf.Properties.getRelTol}{getRelTol}}\label{method:elltool.conf.Properties.getRelTol}
\begin{verbatim}

\end{verbatim}
\subsection{\texorpdfstring{elltool.conf.Properties.getVersion}{getVersion}}\label{method:elltool.conf.Properties.getVersion}
\begin{verbatim}
Example:
  elltool.conf.Properties.getVersion();
\end{verbatim}
\subsection{\texorpdfstring{elltool.conf.Properties.init}{init}}\label{method:elltool.conf.Properties.init}
\begin{verbatim}
Example:
  elltool.conf.Properties.init()
\end{verbatim}
\subsection{\texorpdfstring{elltool.conf.Properties.parseProp}{parseProp}}\label{method:elltool.conf.Properties.parseProp}
\begin{verbatim}
PARSEPROP - parses input into cell array with values of properties listed
           in neededPropNameList.
           Values are  taken from args or, if there no value for some
           property in args, in current Properties.


Input:
  regular:
      args: cell[1,] of any[] - cell array of arguments that
          should be parsed.
  optional
      neededPropNameList: cell[1,nProp] of char[1,] - cell array of strings
          containing names of parameters, that output should consist of.
          The following properties are supported:
              version
              isVerbose
              absTol
              relTol
              regTol
              ODESolverName
              isODENormControl
              isEnabledOdeSolverOptions
              nPlot2dPoints
              nPlot3dPoints
              nTimeGridPoints
          trying to specify other properties would be result in error
          If neededPropNameList is not specified, the list of all
          supported properties is assumed.

Output:
  propVal1:  - value of the first property specified
                             in neededPropNameList in the same order as
                             they listed in neededPropNameList
      ....
  propValN:  - value of the last property from neededPropNameList
  restList: cell[1,nRest] - list of the input arguments that were not
      recognized as properties

Example:
    testAbsTol = 1;
    testRelTol = 2;
    nPlot2dPoints = 3;
    someArg = 4;
    args = {'absTol',testAbsTol, 'relTol',testRelTol,'nPlot2dPoints',...
        nPlot2dPoints, 'someOtherArg', someArg};
    neededPropList = {'absTol','relTol'};
    [absTol, relTol,resList]=elltool.conf.Properties.parseProp(args,...
        neededPropList)

    absTol =

         1


    relTol =

         2


    resList =

        'nPlot2dPoints'    [3]    'someOtherArg'    [4]
\end{verbatim}
\subsection{\texorpdfstring{elltool.conf.Properties.setConfRepoMgr}{setConfRepoMgr}}\label{method:elltool.conf.Properties.setConfRepoMgr}
\begin{verbatim}
Example:
  prevConfRepo = Properties.getConfRepoMgr();
  prevAbsTol = prevConfRepo.getParam('absTol');
  elltool.conf.Properties.setConfRepoMgr(prevConfRepo);
\end{verbatim}
\subsection{\texorpdfstring{elltool.conf.Properties.setIsVerbose}{setIsVerbose}}\label{method:elltool.conf.Properties.setIsVerbose}
\begin{verbatim}
Example:
  elltool.conf.Properties.setIsVerbose(true);
\end{verbatim}
\subsection{\texorpdfstring{elltool.conf.Properties.setNPlot2dPoints}{setNPlot2dPoints}}\label{method:elltool.conf.Properties.setNPlot2dPoints}
\begin{verbatim}
Example:
  elltool.conf.Properties.setNPlot2dPoints(300);
\end{verbatim}
\subsection{\texorpdfstring{elltool.conf.Properties.setNTimeGridPoints}{setNTimeGridPoints}}\label{method:elltool.conf.Properties.setNTimeGridPoints}
\begin{verbatim}
Example:
  elltool.conf.Properties.setNTimeGridPoints(300);
\end{verbatim}
\subsection{\texorpdfstring{elltool.conf.Properties.setRelTol}{setRelTol}}\label{method:elltool.conf.Properties.setRelTol}
\begin{verbatim}
SETRELTOL - set global relative tolerance

Input
relTol: double[1,1]
\end{verbatim}
\section{elltool.core.GenEllipsoid}\label{secClassDescr:elltool.core.GenEllipsoid}
\subsection{\texorpdfstring{elltool.core.GenEllipsoid.GenEllipsoid}{GenEllipsoid}}\label{method:elltool.core.GenEllipsoid.GenEllipsoid}
\begin{verbatim}
GENELLIPSOID - class of generalized ellipsoids

Input:
  Case1:
    regular:
      qVec: double[nDim,1] - ellipsoid center
      qMat: double[nDim,nDim] / qVec: double[nDim,1] - ellipsoid matrix
          or diagonal vector of eigenvalues, that may contain infinite
          or zero elements

  Case2:
    regular:
      qMat: double[nDim,nDim] / qVec: double[nDim,1] - diagonal matrix or
          vector, may contain infinite or zero elements

  Case3:
    regular:
      qVec: double[nDim,1] - ellipsoid center
      dMat: double[nDim,nDim] / dVec: double[nDim,1] - diagonal matrix or
          vector, may contain infinite or zero elements
      wMat: double[nDim,nDim] - any square matrix


Output:
  self: GenEllipsoid[1,1] - created generalized ellipsoid

Example:
  ellObj = elltool.core.GenEllipsoid([5;2], eye(2));
  ellObj = elltool.core.GenEllipsoid([5;2], eye(2), [1 3; 4 5]);
\end{verbatim}
\subsection{\texorpdfstring{elltool.core.GenEllipsoid.dimension}{dimension}}\label{method:elltool.core.GenEllipsoid.dimension}
\begin{verbatim}
Example:
  firstEllObj = elltool.core.GenEllipsoid([1; 1], eye(2));
  secEllObj = elltool.core.GenEllipsoid([0; 5], 2*eye(2));
  ellVec = [firstEllObj secEllObj];
  ellVec.dimension()

  ans =

       2     2
\end{verbatim}
\subsection{\texorpdfstring{elltool.core.GenEllipsoid.display}{display}}\label{method:elltool.core.GenEllipsoid.display}
\begin{verbatim}
Example:
  ellObj = elltool.core.GenEllipsoid([5;2], eye(2), [1 3; 4 5]);
  ellObj.display()
     |
     |----- q : [5 2]
     |          -------
     |----- Q : |10|19|
     |          |19|41|
     |          -------
     |          -----
     |-- QInf : |0|0|
     |          |0|0|
     |          -----
\end{verbatim}
\subsection{\texorpdfstring{elltool.core.GenEllipsoid.getCenter}{getCenter}}\label{method:elltool.core.GenEllipsoid.getCenter}
\begin{verbatim}
Example:
  ellObj = elltool.core.GenEllipsoid([5;2], eye(2), [1 3; 4 5]);
  ellObj.getCenter()

  ans =

       5
       2
\end{verbatim}
\subsection{\texorpdfstring{elltool.core.GenEllipsoid.getCheckTol}{getCheckTol}}\label{method:elltool.core.GenEllipsoid.getCheckTol}
\begin{verbatim}
Example:
  ellObj = elltool.core.GenEllipsoid([5;2], eye(2), [1 3; 4 5]);
  ellObj.getCheckTol()

  ans =

     1.0000e-09
\end{verbatim}
\subsection{\texorpdfstring{elltool.core.GenEllipsoid.getDiagMat}{getDiagMat}}\label{method:elltool.core.GenEllipsoid.getDiagMat}
\begin{verbatim}
Example:
  ellObj = elltool.core.GenEllipsoid([5;2], eye(2), [1 3; 4 5]);
  ellObj.getDiagMat()

  ans =

      0.9796         0
           0   50.0204
\end{verbatim}
\subsection{\texorpdfstring{elltool.core.GenEllipsoid.getEigvMat}{getEigvMat}}\label{method:elltool.core.GenEllipsoid.getEigvMat}
\begin{verbatim}
Example:
  ellObj = elltool.core.GenEllipsoid([5;2], eye(2), [1 3; 4 5]);
  ellObj.getEigvMat()

  ans =

      0.9034   -0.4289
     -0.4289   -0.9034
\end{verbatim}
\subsection{\texorpdfstring{elltool.core.GenEllipsoid.getIsGoodDir}{getIsGoodDir}}\label{method:elltool.core.GenEllipsoid.getIsGoodDir}
\begin{verbatim}
Example:
  firstEllObj = elltool.core.GenEllipsoid([10;0], 2*eye(2));
  secEllObj = elltool.core.GenEllipsoid([0;0], [1 0; 0 0.1]);
  curDirMat = [1; 0];
  isOk=getIsGoodDir(firstEllObj,secEllObj,dirsMat)

  isOk =

       1
\end{verbatim}
\subsection{\texorpdfstring{elltool.core.GenEllipsoid.inv}{inv}}\label{method:elltool.core.GenEllipsoid.inv}
\begin{verbatim}
INV - create generalized ellipsoid whose matrix in pseudoinverse
      to the matrix of input generalized ellipsoid

Input:
  regular:
      ellObj: GenEllipsoid: [1,1] - generalized ellipsoid

Output:
  ellInvObj: GenEllipsoid: [1,1] - inverse generalized ellipsoid

Example:
  ellObj = elltool.core.GenEllipsoid([5;2], [1 0; 0 0.7]);
  ellObj.inv()
     |
     |----- q : [5 2]
     |          -----------------
     |----- Q : |1      |0      |
     |          |0      |1.42857|
     |          -----------------
     |          -----
     |-- QInf : |0|0|
     |          |0|0|
     |          -----
\end{verbatim}
\subsection{\texorpdfstring{elltool.core.GenEllipsoid.minkDiffEa}{minkDiffEa}}\label{method:elltool.core.GenEllipsoid.minkDiffEa}
\begin{verbatim}
MINKDIFFEA - computes tight external ellipsoidal approximation for
             Minkowsky difference of two generalized ellipsoids

Input:
  regular:
      ellObj1: GenEllipsoid: [1,1] - first generalized ellipsoid
      ellObj2: GenEllipsoid: [1,1] - second generalized ellipsoid
      dirMat: double[nDim,nDir] - matrix whose columns specify
          directions for which approximations should be computed
Output:
  resEllVec: GenEllipsoid[1,nDir] - vector of generalized ellipsoids of
      external approximation of the dirrence of first and second
      generalized ellipsoids (may contain empty ellipsoids if in specified
      directions approximation cannot be computed)

Example:
  firstEllObj = elltool.core.GenEllipsoid([10;0], 2*eye(2));
  secEllObj = elltool.core.GenEllipsoid([0;0], [1 0; 0 0.1]);
  dirsMat = [1,0].';
  resEllVec  = minkDiffEa( firstEllObj, secEllObj, dirsMat)
     |
     |----- q : [10 0]
     |          -------------------
     |----- Q : |0.171573|0       |
     |          |0       |1.20557 |
     |          -------------------
     |          -----
     |-- QInf : |0|0|
     |          |0|0|
     |          -----
\end{verbatim}
\subsection{\texorpdfstring{elltool.core.GenEllipsoid.minkDiffIa}{minkDiffIa}}\label{method:elltool.core.GenEllipsoid.minkDiffIa}
\begin{verbatim}
MINKDIFFIA - computes tight internal ellipsoidal approximation for
             Minkowsky difference of two generalized ellipsoids

Input:
  regular:
      ellObj1: GenEllipsoid: [1,1] - first generalized ellipsoid
      ellObj2: GenEllipsoid: [1,1] - second generalized ellipsoid
      dirMat: double[nDim,nDir] - matrix whose columns specify
          directions for which approximations should be computed
Output:
  resEllVec: GenEllipsoid[1,nDir] - vector of generalized ellipsoids of
      internal approximation of the dirrence of first and second
      generalized ellipsoids

Example:
  firstEllObj = elltool.core.GenEllipsoid([10;0], 2*eye(2));
  secEllObj = elltool.core.GenEllipsoid([0;0], [1 0; 0 0.1]);
  dirsMat = [1,0].';
  resEllVec  = minkDiffIa( firstEllObj, secEllObj, dirsMat)
     |
     |----- q : [10 0]
     |          -------------------
     |----- Q : |0.171573|0       |
     |          |0       |0.544365|
     |          -------------------
     |          -----
     |-- QInf : |0|0|
     |          |0|0|
     |          -----
\end{verbatim}
\subsection{\texorpdfstring{elltool.core.GenEllipsoid.minkSumEa}{minkSumEa}}\label{method:elltool.core.GenEllipsoid.minkSumEa}
\begin{verbatim}
MINKSUMEA - computes tight external ellipsoidal approximation for
            Minkowsky sum of the set of generalized ellipsoids

Input:
  regular:
      ellObjVec: GenEllipsoid: [kSize,mSize] - vector of  generalized
                                          ellipsoid
      dirMat: double[nDim,nDir] - matrix whose columns specify
          directions for which approximations should be computed
Output:
  ellResVec: GenEllipsoid[1,nDir] - vector of generalized ellipsoids of
      external approximation of the dirrence of first and second
      generalized ellipsoids

Example:
  firstEllObj = elltool.core.GenEllipsoid([1;1],eye(2));
  secEllObj = elltool.core.GenEllipsoid([5;0],[3 0; 0 2]);
  ellVec = [firstEllObj secEllObj];
  dirsMat = [1 3; 2 4];
  ellResVec  = minkSumEa(ellVec, dirsMat )

  Structure(1)
     |
     |----- q : [6 1]
     |          -----------------
     |----- Q : |7.50584|0      |
     |          |0      |5.83164|
     |          -----------------
     |          -----
     |-- QInf : |0|0|
     |          |0|0|
     |          -----
     O

  Structure(2)
     |
     |----- q : [6 1]
     |          -----------------
     |----- Q : |7.48906|0      |
     |          |0      |5.83812|
     |          -----------------
     |          -----
     |-- QInf : |0|0|
     |          |0|0|
     |          -----
     O
\end{verbatim}
\subsection{\texorpdfstring{elltool.core.GenEllipsoid.minkSumIa}{minkSumIa}}\label{method:elltool.core.GenEllipsoid.minkSumIa}
\begin{verbatim}
MINKSUMIA - computes tight internal ellipsoidal approximation for
            Minkowsky sum of the set of generalized ellipsoids

Input:
  regular:
      ellObjVec: GenEllipsoid: [kSize,mSize] - vector of  generalized
                                          ellipsoid
      dirMat: double[nDim,nDir] - matrix whose columns specify
          directions for which approximations should be computed
Output:
  ellResVec: GenEllipsoid[1,nDir] - vector of generalized ellipsoids of
      internal approximation of the dirrence of first and second
      generalized ellipsoids

Example:
  firstEllObj = elltool.core.GenEllipsoid([1;1],eye(2));
  secEllObj = elltool.core.GenEllipsoid([5;0],[3 0; 0 2]);
  ellVec = [firstEllObj secEllObj];
  dirsMat = [1 3; 2 4];
  ellResVec  = minkSumIa(ellVec, dirsMat )

  Structure(1)
     |
     |----- q : [6 1]
     |          ---------------------
     |----- Q : |7.45135  |0.0272432|
     |          |0.0272432|5.81802  |
     |          ---------------------
     |          -----
     |-- QInf : |0|0|
     |          |0|0|
     |          -----
     O

  Structure(2)
     |
     |----- q : [6 1]
     |          ---------------------
     |----- Q : |7.44698  |0.0315642|
     |          |0.0315642|5.81445  |
     |          ---------------------
     |          -----
     |-- QInf : |0|0|
     |          |0|0|
     |          -----
     O
\end{verbatim}
\subsection{\texorpdfstring{elltool.core.GenEllipsoid.plot}{plot}}\label{method:elltool.core.GenEllipsoid.plot}
\begin{verbatim}
PLOT - plots ellipsoids in 2D or 3D.


Usage:
      plot(ell) - plots generic ellipsoid ell in default (red) color.
      plot(ellArr) - plots an array of generic ellipsoids.
      plot(ellArr, 'Property',PropValue,...) - plots ellArr with setting
                                               properties.

Input:
  regular:
      ellArr:  elltool.core.GenEllipsoid: [dim11Size,dim12Size,...,
               dim1kSize] - array of 2D or 3D GenEllipsoids objects.
               All ellipsoids in ellArr  must be either 2D or 3D
               simutaneously.
  optional:
      color1Spec: char[1,1] - color specification code, can be 'r','g',
                              etc (any code supported by built-in Matlab
                              function).
      ell2Arr: elltool.core.GenEllipsoid: [dim21Size,dim22Size,...,
                              dim2kSize] - second ellipsoid array...
      color2Spec: char[1,1] - same as color1Spec but for ell2Arr
      ....
      ellNArr: elltool.core.GenEllipsoid: [dimN1Size,dim22Size,...,
                               dimNkSize] - N-th ellipsoid array
      colorNSpec - same as color1Spec but for ellNArr.
  properties:
      'newFigure': logical[1,1] - if 1, each plot command will open a new .
                   figure window Default value is 0.
      'fill': logical[1,1]/logical[dim11Size,dim12Size,...,dim1kSize]  -
              if 1, ellipsoids in 2D will be filled with color.
              Default value is 0.
      'lineWidth': double[1,1]/double[dim11Size,dim12Size,...,dim1kSize]  -
               line width for 1D and 2D plots.
               Default value is 1.
      'color': double[1,3]/double[dim11Size,dim12Size,...,dim1kSize,3] -
               sets default colors in the form [x y z].
               Default value is [1 0 0].
      'shade': double[1,1]/double[dim11Size,dim12Size,...,dim1kSize]  -
               level of transparency between 0 and 1 (0 - transparent,
               1 - opaque).
               Default value is 0.4.
      'relDataPlotter' - relation data plotter object.
      Notice that property vector could have different dimensions, only
      total number of elements must be the same.
Output:
  regular:
      plObj: smartdb.disp.RelationDataPlotter[1,1] - returns the relation
      data plotter object.

Examples:
  plot([ell1, ell2, ell3], 'color', [1, 0, 1; 0, 0, 1; 1, 0, 0]);
  plot([ell1, ell2, ell3], 'color', [1; 0; 1; 0; 0; 1; 1; 0; 0]);
  plot([ell1, ell2, ell3; ell1, ell2, ell3], 'shade', [1, 1, 1; 1, 1,
    1]);
  plot([ell1, ell2, ell3; ell1, ell2, ell3], 'shade', [1; 1; 1; 1; 1;
      1]);
  plot([ell1, ell2, ell3], 'shade', 0.5);
  plot([ell1, ell2, ell3], 'lineWidth', 1.5);
  plot([ell1, ell2, ell3], 'lineWidth', [1.5, 0.5, 3]);
\end{verbatim}
\subsection{\texorpdfstring{elltool.core.GenEllipsoid.rho}{rho}}\label{method:elltool.core.GenEllipsoid.rho}
\begin{verbatim}
Example:
  ellObj = elltool.core.GenEllipsoid([1;1],eye(2));
  dirsVec = [1; 0];
  [resRho, bndPVec] = rho(ellObj, dirsVec)

  resRho =

       2

 bndPVec =

       2
       1
\end{verbatim}
\section{smartdb.relations.ATypifiedStaticRelation}\label{secClassDescr:smartdb.relations.ATypifiedStaticRelation}
\subsection{\texorpdfstring{smartdb.relations.ATypifiedStaticRelation.ATypifiedStaticRelation}{ATypifiedStaticRelation}}\label{method:smartdb.relations.ATypifiedStaticRelation.ATypifiedStaticRelation}
\begin{verbatim}
ATYPIFIEDSTATICRELATION is a constructor of static relation class
object

Usage: self=AStaticRelation(obj) or
       self=AStaticRelation(varargin)

Input:
  optional
    inpObj: ARelation[1,1]/SData: struct[1,1]
        structure with values of all fields
        for all tuples

    SIsNull: struct [1,1] - structure of fields with is-null
       information for the field content, it can be logical for
       plain real numbers of cell of logicals for cell strs or
       cell of cell of str for more complex types

    SIsValueNull: struct [1,1] - structure with logicals
        determining whether value corresponding to each field
        and each tuple is null or not

  properties:
      fillMissingFieldsWithNulls: logical[1,1] - if true,
          the relation fields absent in the input data
          structures are filled with null values

Output:
  regular:
    self: ATYPIFIEDSTATICRELATION [1,1] - constructed class object

Note: In the case the first interface is used, SData and
      SIsNull are taken from class object obj
\end{verbatim}
\subsection{\texorpdfstring{smartdb.relations.ATypifiedStaticRelation.addData}{addData}}\label{method:smartdb.relations.ATypifiedStaticRelation.addData}
\begin{verbatim}
ADDDATA - adds a set of field values to existing data in a form of new
          tuples

Input:
  regular:
     self:ARelation [1,1] - class object
\end{verbatim}
\subsection{\texorpdfstring{smartdb.relations.ATypifiedStaticRelation.addDataAlongDim}{addDataAlongDim}}\label{method:smartdb.relations.ATypifiedStaticRelation.addDataAlongDim}
\begin{verbatim}
ADDDATAALONGDIM - adds a set of field values to existing data using
                  a concatenation along a specified dimension

Input:
  regular:
      self: CubeStruct [1,1] - the object
\end{verbatim}
\subsection{\texorpdfstring{smartdb.relations.ATypifiedStaticRelation.addTuples}{addTuples}}\label{method:smartdb.relations.ATypifiedStaticRelation.addTuples}
\begin{verbatim}
ADDTUPLES - adds a set of new tuples to the relation

Usage: addTuplesInternal(self,varargin)

input:
  regular:
      self: ARelation [1,1] - class object
      SData: struct [1,1] - structure with values of all fields  for all
       tuples
  optional:
      SIsNull: struct [1,1] - structure of fields with is-null
        information for the field content, it can be logical for plain
        real numbers of cell of logicals for cell strs or cell of cell of
        str for more complex types

      SIsValueNull: struct [1,1] - structure with logicals determining
        whether value corresponding to each field and each tuple is null
        or not

  properties:
      checkConsistency: logical[1,1], if true, a consistency between the
         input structures is not checked, true by default
\end{verbatim}
\subsection{\texorpdfstring{smartdb.relations.ATypifiedStaticRelation.applyGetFunc}{applyGetFunc}}\label{method:smartdb.relations.ATypifiedStaticRelation.applyGetFunc}
\begin{verbatim}
APPLYGETFUNC - applies a function to the specified fields as columns, i.e.
               the function is applied to each field as whole, not to
               each cell separately

Input:
  regular:
      hFunc: function_handle[1,1] - function to apply to each of the
         field values
  optional:
      toFieldNameList: char/cell[1,] of char - a list of fields to which
         the function specified by hFunc is to be applied

    Note: hFunc can optionally be specified after toFieldNameList
          parameter

Notes: this function currently has a lots of limitations:
  1) it assumes that the output is uniform
  2) the function is applies to SData part of field value
  3) no additional arguments can be passed
  All this limitations will eventually go away though so stay tuned...
\end{verbatim}
\subsection{\texorpdfstring{smartdb.relations.ATypifiedStaticRelation.applySetFunc}{applySetFunc}}\label{method:smartdb.relations.ATypifiedStaticRelation.applySetFunc}
\begin{verbatim}
APPLYSETFUNC - applies some function to each cell of the specified fields
               of a given CubeStruct object

Usage: applySetFunc(self,toFieldNameList,hFunc)
       applySetFunc(self,hFunc,toFieldNameList)

Input:
  regular:
      self: CubeStruct [1,1] - class object

      hFunc: function handle [1,1] - handle of function to be
        applied to fields, the function is assumed to
          1) have the same number of input/output arguments
          2) the number of input arguments should be
             length(structNameList)*length(fieldNameList)
          3) the input arguments should be ordered according to the
          following rule
              (x_struct_1_field_1,x_struct_1_field_2,...,struct_n_field1,
              ...,struct_n_field_m)

  optional:

      toFieldNameList: char or char cell [1,nFields] - list of
        field names to which given function should be applied

        Note1: field lists of length>1 are not currently supported !
        Note2: it is possible to specify toFieldNameList before hFunc in
           which case the parameters will be recognized automatically

  properties:
      uniformOutput: logical[1,1] - specifies if the result
         of the function is uniform to be stored in non-cell
         field, by default it is false for cell fileds and
         true for non-cell fields

      structNameList: char[1,]/cell[1,], name of data structure/list of
        data structure names to which the function is to
             be applied, can be composed from the following values

           SData - data itself

           SIsNull - contains is-null indicator information for data
             values

           SIsValueNull - contains is-null indicators for CubeStruct
              cells (not for cell values)

        structNameList={'SData'} by default

      inferIsNull: logical[1,2] - if the first(second) element is true,
          SIsNull(SIsValueNull) indicators are inferred from SData,
          i.e. with this indicator set to true it is sufficient to apply
          the function only to SData while the rest of the structures
          will be adjusted automatically.

      inputType: char[1,] - specifies a way in which the field value is
         partitioned into individual cells before being passed as an
         input parameter to hFunc. This parameter directly corresponds to
         outputType parameter of toArray method, see its documentation
         for a list of supported input types.
\end{verbatim}
\subsection{\texorpdfstring{smartdb.relations.ATypifiedStaticRelation.applyTupleGetFunc}{applyTupleGetFunc}}\label{method:smartdb.relations.ATypifiedStaticRelation.applyTupleGetFunc}
\begin{verbatim}
APPLYTUPLEGETFUNC - applies a function to the specified fields
                    separately to each tuple

Input:
  regular:
      hFunc: function_handle[1,1] - function to apply to the specified
         fields
  optional:
      toFieldNameList: char/cell[1,] of char - a list of fields to which
         the function specified by hFunc is to be applied

  properties:
      uniformOutput: logical[1,1] - if true, output is expected to be
          uniform as in cellfun with 'UniformOutput'=true, default
           value is true

Output:
  funcOut1Arr: <type1>[] - array corresponding to the first output of the
      applied function
          ....
  funcOutNArr: <typeN>[] - array corresponding to the last output of the
      applied function


Notes: this function currently has a lots of limitations:
  1) the function is applies to SData part of field value
  2) no additional arguments can be passed
  All this limitations will eventually go away though so stay tuned...
\end{verbatim}
\subsection{\texorpdfstring{smartdb.relations.ATypifiedStaticRelation.clearData}{clearData}}\label{method:smartdb.relations.ATypifiedStaticRelation.clearData}
\begin{verbatim}
CLEARDATA - deletes all the data from the object

Usage: self.clearData(self)

Input:
  regular:
    self: CubeStruct [1,1] - class object
\end{verbatim}
\subsection{\texorpdfstring{smartdb.relations.ATypifiedStaticRelation.clone}{clone}}\label{method:smartdb.relations.ATypifiedStaticRelation.clone}
\begin{verbatim}
CLONE - creates a copy of a specified object via calling
        a copy constructor for the object class

Input:
  regular:
    self: any [] - current object
  optional
    any parameters applicable for relation constructor

Ouput:
  self: any [] - constructed object
\end{verbatim}
\subsection{\texorpdfstring{smartdb.relations.ATypifiedStaticRelation.copyFrom}{copyFrom}}\label{method:smartdb.relations.ATypifiedStaticRelation.copyFrom}
\begin{verbatim}
COPYFROM - reconstruct CubeStruct object within a current object using the
           input CubeStruct object as a prototype

Input:
  regular:
    self: CubeStruct [n_1,...,n_k]
    obj: any [] - internal representation of the object

  optional:
    fieldNameList: cell[1,nFields] - list of fields to copy
\end{verbatim}
\subsection{\texorpdfstring{smartdb.relations.ATypifiedStaticRelation.createInstance}{createInstance}}\label{method:smartdb.relations.ATypifiedStaticRelation.createInstance}
\begin{verbatim}
CREATEINSTANCE - returns an object of the same class by calling a default
                 constructor (with no parameters)

Usage: resObj=getInstance(self)

input:
  regular:
    self: any [] - current object
  optional
    any parameters applicable for relation constructor

Ouput:
  self: any [] - constructed object
\end{verbatim}
\subsection{\texorpdfstring{smartdb.relations.ATypifiedStaticRelation.dispOnUI}{dispOnUI}}\label{method:smartdb.relations.ATypifiedStaticRelation.dispOnUI}
\begin{verbatim}
DISPONUI - displays a content of the given relation as a data grid UI
           component.

Input:
  regular:
      self:
  properties:
      tableType: char[1,] - type of table used for displaying the data,
          the following types are supported:
          'sciJavaGrid' - proprietary Java-based data grid component
              is used
          'uitable'  - Matlab built-in uitable component is used.
              if not specified, the method tries to use sciJavaGrid
              if it is available, if not - uitable is used.

Output:
  hFigure: double[1,1] - figure handle containing the component
  gridObj: smartdb.relations.disp.UIDataGrid[1,1] - data grid component
      instance used for displaying a content of the relation object
\end{verbatim}
\subsection{\texorpdfstring{smartdb.relations.ATypifiedStaticRelation.display}{display}}\label{method:smartdb.relations.ATypifiedStaticRelation.display}
\begin{verbatim}
DISPLAY - puts some textual information about CubeStruct object in screen

Input:
 regular:
     self.
\end{verbatim}
\subsection{\texorpdfstring{smartdb.relations.ATypifiedStaticRelation.fromStructList}{fromStructList}}\label{method:smartdb.relations.ATypifiedStaticRelation.fromStructList}
\begin{verbatim}
FROMSTRUCTLIST - creates a dynamic relation from a list of structures
                 interpreting each structure as the data for
                 several tuples.

Input:
  regular:
      className: name of object class which will be created,
          the class constructor should accept 2 properties:
          'fieldNameList' and 'fieldTypeSpecList'

      structList: cell[] of struct[1,1] - list of structures

Output:
  relDataObj: smartdb.relations.DynamicRelation[1,1] -
     constructed relation
\end{verbatim}
\subsection{\texorpdfstring{smartdb.relations.ATypifiedStaticRelation.getCopy}{getCopy}}\label{method:smartdb.relations.ATypifiedStaticRelation.getCopy}
\begin{verbatim}
GETCOPY - returns an object copy

Usage: resObj=getCopy(self)

Input:
  regular:
    self: CubeStruct [1,1] - current CubeStruct object
  optional:
    same as for getData
\end{verbatim}
\subsection{\texorpdfstring{smartdb.relations.ATypifiedStaticRelation.getData}{getData}}\label{method:smartdb.relations.ATypifiedStaticRelation.getData}
\begin{verbatim}
GETDATA - returns an indexed projection of CubeStruct object's content

Input:
  regular:
      self: CubeStruct [1,1] - the object

  optional:

      subIndCVec:
        Case#1: numeric[1,]/numeric[,1]

        Case#2: cell[1,nDims]/cell[nDims,1] of double [nSubElem_i,1]
              for i=1,...,nDims

          -array of indices of field value slices that are selected
          to be returned; if not given (default),
          no indexation is performed

        Note!: numeric components of subIndVec are allowed to contain
           zeros which are be treated as they were references to null
           data slices

      dimVec: numeric[1,nDims]/numeric[nDims,1] - vector of dimension
          numbers corresponding to subIndCVec

  properties:

      fieldNameList: char[1,]/cell[1,nFields] of char[1,]
          list of field names to return

      structNameList: char[1,]/cell[1,nStructs] of char[1,]
          list of internal structures to return (by default it
          is {SData, SIsNull, SIsValueNull}

      replaceNull: logical[1,1] if true, null values are replaced with
          certain default values uniformly across all the cells,
              default value is false

      nullReplacements: cell[1,nReplacedFields]  - list of null
          replacements for each of the fields

      nullReplacementFields: cell[1,nReplacedFields] - list of fields in
         which the nulls are to be replaced with the specified values,
         if not specified it is assumed that all fields are to be
         replaced

         NOTE!: all fields not listed in this parameter are replaced with
         the default values

      checkInputs: logical[1,1] - true by default (input arguments are
         checked for correctness

Output:
  regular:
    SData: struct [1,1] - structure containing values of
        fields at the selected slices, each field is an array
        containing values of the corresponding type

    SIsNull: struct [1,1] - structure containing a nested
        array with is-null indicators for each CubeStruct cell content

    SIsValueNull: struct [1,1] - structure containing a
       logical array [] for each of the fields (true
       means that a corresponding cell doesn't not contain
          any value
\end{verbatim}
\subsection{\texorpdfstring{smartdb.relations.ATypifiedStaticRelation.getFieldDescrList}{getFieldDescrList}}\label{method:smartdb.relations.ATypifiedStaticRelation.getFieldDescrList}
\begin{verbatim}
GETFIELDDESCRLIST - returns the list of CubeStruct field descriptions

Usage: value=getFieldDescrList(self)

Input:
  regular:
      self: CubeStruct [1,1]
  optional:
      fieldNameList: cell[1,nSpecFields] of char[1,] - field names for
         which descriptions should be returned

Output:
  regular:
    value: char cell [1,nFields] - list of CubeStruct object field
        descriptions
\end{verbatim}
\subsection{\texorpdfstring{smartdb.relations.ATypifiedStaticRelation.getFieldIsNull}{getFieldIsNull}}\label{method:smartdb.relations.ATypifiedStaticRelation.getFieldIsNull}
\begin{verbatim}
GETFIELDISNULL - returns for given field a nested logical/cell array
                 containing is-null indicators for cell content

Usage: fieldIsNullCVec=getFieldIsNull(self,fieldName)

Input:
  regular:
    self: CubeStruct [1,1]
    fieldName: char - field name
Output:
  regular:
    fieldIsCVec: logical/cell[] - nested cell/logical array containing
       is-null indicators for content of the field
\end{verbatim}
\subsection{\texorpdfstring{smartdb.relations.ATypifiedStaticRelation.getFieldIsValueNull}{getFieldIsValueNull}}\label{method:smartdb.relations.ATypifiedStaticRelation.getFieldIsValueNull}
\begin{verbatim}
GETFIELDISVALUENULL - returns for given field logical vector determining
                      whether value of this field in each cell is null
                      or not.

BEWARE OF confusing this with getFieldIsNull method which returns is-null
   indicators for a field content

Usage: isNullVec=getFieldValueIsNull(self,fieldName)

Input:
  regular:
    self: CubeStruct [1,1]
    fieldName: char - field name

Output:
  regular:
    isValueNullVec: logical[] - array of isValueNull indicators for the
       specified field
\end{verbatim}
\subsection{\texorpdfstring{smartdb.relations.ATypifiedStaticRelation.getFieldNameList}{getFieldNameList}}\label{method:smartdb.relations.ATypifiedStaticRelation.getFieldNameList}
\begin{verbatim}
GETFIELDNAMELIST - returns the list of CubeStruct object field names

Usage: value=getFieldNameList(self)

Input:
  regular:
    self: CubeStruct [1,1]
Iutput:
  regular:
    value: char cell [1,nFields] - list of CubeStruct object field
        names
\end{verbatim}
\subsection{\texorpdfstring{smartdb.relations.ATypifiedStaticRelation.getFieldProjection}{getFieldProjection}}\label{method:smartdb.relations.ATypifiedStaticRelation.getFieldProjection}
\begin{verbatim}
GETFIELDPROJECTION - project object with specified fields.

Input:
  regular:
      self: ARelation[1,1] - original object
      fieldNameList: cell[1,nFields] of char[1,] - field name list

Output:
  obj: DynamicRelation[1,1] - projected object
\end{verbatim}
\subsection{\texorpdfstring{smartdb.relations.ATypifiedStaticRelation.getFieldTypeList}{getFieldTypeList}}\label{method:smartdb.relations.ATypifiedStaticRelation.getFieldTypeList}
\begin{verbatim}
GETFIELDTYPELIST - returns list of field types in given CubeStruct object

Usage: fieldTypeList=getFieldTypeList(self)

Input:
  regular:
      self: CubeStruct [1,1]

  optional:
      fieldNameList: cell[1,nFields] - list of field names

Output:
 regular:
  fieldTypeList: cell [1,nFields] of smartdb.cubes.ACubeStructFieldType[1,1]
      - list of field types
\end{verbatim}
\subsection{\texorpdfstring{smartdb.relations.ATypifiedStaticRelation.getFieldTypeSpecList}{getFieldTypeSpecList}}\label{method:smartdb.relations.ATypifiedStaticRelation.getFieldTypeSpecList}
\begin{verbatim}
GETFIELDTYPESPECLIST - returns a list of field type specifications. Field
                       type specification is a sequence of type names
                       corresponding to field value types starting with
                       the top level and going down into the nested
                       content of a field (for a field having a complex
                       type).

Input:
  regular:
      self:
  optional:
      fieldNameList: cell [1,nFields] of char[1,] - list of field names
  properties:
      uniformOutput: logical[1,1] - if true, the result is concatenated
         across all the specified fields

Output:
  typeSpecList:
       Case#1: uniformOutput=false
          cell[1,nFields] of cell[1,nNestedLevels_i] of char[1,.]
       Case#2: uniformOutput=true
          cell[1,nFields*prod(nNestedLevelsVec)] of char[1,.]
       - list of field type specifications
\end{verbatim}
\subsection{\texorpdfstring{smartdb.relations.ATypifiedStaticRelation.getFieldValueSizeMat}{getFieldValueSizeMat}}\label{method:smartdb.relations.ATypifiedStaticRelation.getFieldValueSizeMat}
\begin{verbatim}
GETFIELDVALUESIZEMAT - returns a matrix composed from the size vectors
                       for the specified fields

Input:
  regular:
      self:

  optional:
      fieldNameList: cell[1,nFields] - a list of fileds for which the size
         matrix is to be generated

  properties:
      skipMinDimensions: logical[1,1] - if true, the dimensions from 1 up
          to minDimensionality are skipped

      minDimension: numeric[1,1] - minimum dimension which definies a
         minimum number of columns in the resulting matrix

Output:
  sizeMat: double[nFields,nMaxDims]
\end{verbatim}
\subsection{\texorpdfstring{smartdb.relations.ATypifiedStaticRelation.getIsFieldValueNull}{getIsFieldValueNull}}\label{method:smartdb.relations.ATypifiedStaticRelation.getIsFieldValueNull}
\begin{verbatim}
GETISFIELDVALUENULL - returns a vector indicating whether a particular
                      field is composed of null values completely

Usage: isValueNullVec=getIsFieldValueNull(self,fieldNameList)

Input:
  regular:
    self: CubeStruct [1,1]

  optional:
    fieldNameList: cell[1,nFields] of char[1,] - list of field names

Output:
  regular:
    isValueNullVec: logical[1,nFields]
\end{verbatim}
\subsection{\texorpdfstring{smartdb.relations.ATypifiedStaticRelation.getJoinWith}{getJoinWith}}\label{method:smartdb.relations.ATypifiedStaticRelation.getJoinWith}
\begin{verbatim}
GETJOINWITH - returns a result of INNER join of given relation with
              another relation by the specified key fields

LIMITATION: key fields by which the join is peformed are required to form
a unique key in the given relation

Input:
  regular:
      self:
      otherRel: smartdb.relations.ARelation[1,1]
      keyFieldNameList: char[1,]/cell[1,nFields] of char[1,]

  properties:
      joinType: char[1,] - type of join, can be
          'inner' (DEFAULT)
          'leftOuter'

Output:
  resRel: smartdb.relations.ARelation[1,1] - join result
\end{verbatim}
\subsection{\texorpdfstring{smartdb.relations.ATypifiedStaticRelation.getMinDimensionSize}{getMinDimensionSize}}\label{method:smartdb.relations.ATypifiedStaticRelation.getMinDimensionSize}
\begin{verbatim}
GETMINDIMENSIONSIZE - returns a size vector for the specified
                      dimensions. If no dimensions are specified, a size
                      vector for all dimensions up to minimum CubeStruct
                      dimension is returned

Input:
  regular:
      self:
  optional:
      dimNumVec: numeric[1,nDims] - a vector of dimension
          numbers

Output:
  minDimensionSizeVec: double [1,nDims] - a size vector for
     the requested dimensions
\end{verbatim}
\subsection{\texorpdfstring{smartdb.relations.ATypifiedStaticRelation.getMinDimensionality}{getMinDimensionality}}\label{method:smartdb.relations.ATypifiedStaticRelation.getMinDimensionality}
\begin{verbatim}
GETMINDIMENSIONALITY - returns a minimum dimensionality for a given
                       object

Input:
  regular:
      self

Output:
  minDimensionality: double[1,1] - minimum dimensionality of
     self object
\end{verbatim}
\subsection{\texorpdfstring{smartdb.relations.ATypifiedStaticRelation.getNElems}{getNElems}}\label{method:smartdb.relations.ATypifiedStaticRelation.getNElems}
\begin{verbatim}
GETNELEMS - returns a number of elements in a given object
Input:
  regular:
     self:

Output:
  nElems:double[1, 1] - number of elements in a given object
\end{verbatim}
\subsection{\texorpdfstring{smartdb.relations.ATypifiedStaticRelation.getNFields}{getNFields}}\label{method:smartdb.relations.ATypifiedStaticRelation.getNFields}
\begin{verbatim}
GETNFIELDS - returns number of fields in given object

Usage: nFields=getNFields(self)

Input:
  regular:
    self: CubeStruct [1,1]
Output:
  regular:
    nFields: double [1,1] - number of fields in given object
\end{verbatim}
\subsection{\texorpdfstring{smartdb.relations.ATypifiedStaticRelation.getNTuples}{getNTuples}}\label{method:smartdb.relations.ATypifiedStaticRelation.getNTuples}
\begin{verbatim}
GETNTUPLES - returns number of tuples in given relation

Usage: nTuples=getNTuples(self)

input:
  regular:
    self: ARelation [1,1] - class object
output:
  regular:
    nTuples: double [1,1] - number of tuples in given  relation
\end{verbatim}
\subsection{\texorpdfstring{smartdb.relations.ATypifiedStaticRelation.getSortIndex}{getSortIndex}}\label{method:smartdb.relations.ATypifiedStaticRelation.getSortIndex}
\begin{verbatim}
GETSORTINDEX - gets sort index for all tuples of given relation with
               respect to some of its fields

Usage: sortInd=getSortIndex(self,sortFieldNameList,varargin)

input:
  regular:
    self: ARelation [1,1] - class object
    sortFieldNameList: char or char cell [1,nFields] - list of field
       names with respect to which tuples are sorted

  properties:
    Direction: char or char cell [1,nFields] - direction of sorting for
        all fields (if one value is given) or for each field separately;
        each value may be 'asc' or 'desc'
output:
  regular:
   sortIndex: double [nTuples,1] - sort index for all tuples such that if
       fieldValueVec is a vector of values for some field of given
       relation, then fieldValueVec(sortIndex) is a vector of values for
       this field when tuples of the relation are sorted
\end{verbatim}
\subsection{\texorpdfstring{smartdb.relations.ATypifiedStaticRelation.getTuples}{getTuples}}\label{method:smartdb.relations.ATypifiedStaticRelation.getTuples}
\begin{verbatim}
GETTUPLES - selects tuples with given indices from given relation and
            returns the result as new relation

Usage: obj=getTuples(self,subIndVec)

input:
  regular:
    self: ARelation [1,1] - class object
    subIndVec: double [nSubTuples,1]/logical[nTuples,1] - array of
        indices for tuples that are selected
output:
  regular:
    obj: ARelation [1,1] - new class object containing only selected
        tuples
\end{verbatim}
\subsection{\texorpdfstring{smartdb.relations.ATypifiedStaticRelation.getTuplesFilteredBy}{getTuplesFilteredBy}}\label{method:smartdb.relations.ATypifiedStaticRelation.getTuplesFilteredBy}
\begin{verbatim}
GETTUPLESFILTEREDBY - selects tuples from given relation such that a
                      fixed index field contains values from a given set
                      of value and returns the result as new relation

Input:
  regular:
    self: ARelation [1,1] - class object
    filterFieldName: char - name of index field
    filterValueVec: numeric/ cell of char [nValues,1] - vector of index
        values

  properties:
    keepNulls: logical[1,1] - if true, null values are not filteed out,
       and removed otherwise,
          default: false

Output:
  regular:
    obj: ARelation [1,1] - new class object containing only selected
        tuples
    isThereVec: logical[nTuples,1] - contains true for the kept tuples
\end{verbatim}
\subsection{\texorpdfstring{smartdb.relations.ATypifiedStaticRelation.getTuplesIndexedBy}{getTuplesIndexedBy}}\label{method:smartdb.relations.ATypifiedStaticRelation.getTuplesIndexedBy}
\begin{verbatim}
 GETTUPLESINDEXEDBY - selects tuples from given relation such that fixed
                      index field contains given in a specified order
                      values and returns the result as new relation.
                      It is required that the original relation
                      contains only one record for each field value

 input:
   regular:
     self: ARelation [1,1] - class object
     indexFieldName: char - name of index field
     indexValueVec: numeric or char cell [nValues,1] - vector of index
         values
 output:
   regular:
     obj: ARelation [1,1] - new class object containing only selected
         tuples

TODO add type check
\end{verbatim}
\subsection{\texorpdfstring{smartdb.relations.ATypifiedStaticRelation.getTuplesJoinedWith}{getTuplesJoinedWith}}\label{method:smartdb.relations.ATypifiedStaticRelation.getTuplesJoinedWith}
\begin{verbatim}
GETTUPLESJOINEDWITH - returns the tuples of the given relation
                      INNER-joined with other relation by the specified
                      key fields

Input:
  regular:
      self:
      otherRel: smartdb.relations.ARelation[1,1]
      keyFieldNameList: char[1,]/cell[1,nFields] of char[1,]

  properties:
      joinType: char[1,] - type of join, can be
          'inner' (DEFAULT) - inner join
          'leftOuter' - left outer join
          'rightOuter' - right outer join
          'fullOuter' - full outer join

      fieldDescrSource: char[1,] - defines where the field descriptions
         are taken from, can be
          'useOriginal' - field descriptions are taken from the left hand
              side argument of the join operation
          'useOther' - field descriptions are taken from the right hand
              side of the join operation

Output:
  resRel: smartdb.relations.ARelation[1,1] - join result
\end{verbatim}
\subsection{\texorpdfstring{smartdb.relations.ATypifiedStaticRelation.getUniqueData}{getUniqueData}}\label{method:smartdb.relations.ATypifiedStaticRelation.getUniqueData}
\begin{verbatim}
GETUNIQUEDATA - returns internal representation for a set of unique
                tuples for given relation

Usage: [SData,SIsNull,SIsValueNull]=getUniqueData(self,varargin)

Input:
  regular:
    self: ARelation [1,1] - class object
  properties
      fieldNameList: list of field names used for finding the unique
          elements; only the specified fields are returned in SData,
          SIsNull,SIsValueNull structures
      structNameList: list of internal structures to return (by default it
          is {SData, SIsNull, SIsValueNull}
      replaceNull: logical[1,1] if true, null values are replaced with
          certain default values uniformly across all the tuples
              default value is false

Output:
  regular:

    SData: struct [1,1] - structure containing values of fields in
        selected tuples, each field is an array containing values of the
        corresponding type

    SIsNull: struct [1,1] - structure containing info whether each value
        in selected tuples is null or not, each field is either logical
        array or cell array containing logical arrays

    SIsValueNull: struct [1,1] - structure containing a
       logical array [nTuples,1] for each of the fields (true
       means that a corresponding cell doesn't not contain
          any value

    indForward: double[1,nUniqueTuples] - indices of unique entries in
       the original tuple set

    indBackward: double[1,nTuples] - indices that map the unique tuple
       set back to the original tuple set
\end{verbatim}
\subsection{\texorpdfstring{smartdb.relations.ATypifiedStaticRelation.getUniqueDataAlongDim}{getUniqueDataAlongDim}}\label{method:smartdb.relations.ATypifiedStaticRelation.getUniqueDataAlongDim}
\begin{verbatim}
GETUNIQUEDATAALONGDIM - returns internal representation of CubeStruct

Input:
  regular:
    self:
    catDim: double[1,1] - dimension number along which uniqueness is
       checked

  properties
      fieldNameList: list of field names used for finding the unique
          elements; only the specified fields are returned in SData,
          SIsNull,SIsValueNull structures
      structNameList: list of internal structures to return (by default
          it is {SData, SIsNull, SIsValueNull}
      replaceNull: logical[1,1] if true, null values are replaced with
          certain default values uniformly across all CubeStruct cells
              default value is false
      checkInputs: logical[1,1] - if true, the input parameters are
         checked for consistency

Output:
  regular:
    SData: struct [1,1] - structure containing values of fields

    SIsNull: struct [1,1] - structure containing info whether each value
        in selected cells is null or not, each field is either logical
        array or cell array containing logical arrays

    SIsValueNull: struct [1,1] - structure containing a
       logical array [nSlices,1] for each of the fields (true
       means that a corresponding cell doesn't not contain
          any value

    indForwardVec: double[nUniqueSlices,1] - indices of unique entries in
       the original CubeStruct data set

    indBackwardVec: double[nSlices,1] - indices that map the unique data
       set back to the original data setdata set unique along a specified
       dimension
\end{verbatim}
\subsection{\texorpdfstring{smartdb.relations.ATypifiedStaticRelation.getUniqueTuples}{getUniqueTuples}}\label{method:smartdb.relations.ATypifiedStaticRelation.getUniqueTuples}
\begin{verbatim}
GETUNIQUETUPLES - returns a relation containing the unique tuples from
                  the original relation

Usage: [resRel,indForwardVec,indBackwardVec]=getUniqueTuples(self,varargin)

Input:
  regular:
    self: ARelation [1,1] - class object
  properties
      fieldNameList: list of field names used for finding the unique
         tuples
      structNameList: list of internal structures to return (by default it
          is {SData, SIsNull, SIsValueNull}
      replaceNull: logical[1,1] if true, null values are replaced with
          certain default values uniformly across all the tuples
              default value is false

Output:
  regular:

    resRel: ARelation[1,1] - resulting relation

    indForward: double[1,nUniqueTuples] - indices of unique entries in
       the original tuple set

    indBackward: double[1,nTuples] - indices that map the unique tuple
       set back to the original tuple set
\end{verbatim}
\subsection{\texorpdfstring{smartdb.relations.ATypifiedStaticRelation.initByEmptyDataSet}{initByEmptyDataSet}}\label{method:smartdb.relations.ATypifiedStaticRelation.initByEmptyDataSet}
\begin{verbatim}
INITBYEMPTYDATASET - initializes cube struct object with null value arrays
                     of specified size based on minDimVec specified.

For instance, if minDimVec=[2,3,4,5,6] and minDimensionality of cube
struct object cb is 2, then cb.initByEmptyDataSet(minDimVec) will create
a cube struct object with element array of [2,3] size where each element
has size of [4,5,6,0]

Input:
  regular:
      self:
  optional
      minDimVec: double[1,nDims] - size vector of null value arrays
\end{verbatim}
\subsection{\texorpdfstring{smartdb.relations.ATypifiedStaticRelation.initByNullDataSet}{initByNullDataSet}}\label{method:smartdb.relations.ATypifiedStaticRelation.initByNullDataSet}
\begin{verbatim}
INITBYDEFAULTDATASET - initializes cube struct object with null value
                       arrays of specified size based on minDimVec
                       specified.

For instance, if minDimVec=[2,3,4,5,6] and minDimensionality of cube
struct object cb is 2, then cb.initByEmptyDataSet(minDimVec) will create
a cube struct object with element array of [2,3] size where each element
has size of [4,5,6]

Input:
  regular:
      self:
  optional
      minDimVec: double[1,nDims] - size vector of null value arrays
\end{verbatim}
\subsection{\texorpdfstring{smartdb.relations.ATypifiedStaticRelation.isEqual}{isEqual}}\label{method:smartdb.relations.ATypifiedStaticRelation.isEqual}
\begin{verbatim}
ISEQUAL - compares current relation object with other relation object and
          returns true if they are equal, otherwise it returns false


Usage: isEq=isEqual(self,otherObj)

Input:
  regular:
    self: ARelation [1,1] - current relation object
    otherObj: ARelation [1,1] - other relation object

  properties:
    checkFieldOrder/isFieldOrderCheck: logical [1,1] - if true, then fields
        in compared relations must be in the same order, otherwise the
        order is not  important (false by default)
    checkTupleOrder: logical[1,1] -  if true, then the tuples in the
        compared relations are expected to be in the same order,
        otherwise the order is not important (false by default)

    maxTolerance: double [1,1] - maximum allowed tolerance

    compareMetaDataBackwardRef: logical[1,1] if true, the CubeStruct's
        referenced from the meta data objects are also compared

    maxRelativeTolerance: double [1,1] - maximum allowed
    relative tolerance

Output:
  isEq: logical[1,1] - result of comparison
  reportStr: char[1,] - report of comparsion
\end{verbatim}
\subsection{\texorpdfstring{smartdb.relations.ATypifiedStaticRelation.isFields}{isFields}}\label{method:smartdb.relations.ATypifiedStaticRelation.isFields}
\begin{verbatim}
 ISFIELDS - returns whether all fields whose names are given in the input
            list are in the field list of given object or not

 Usage: isPositive=isFields(self,fieldList)

 Input:
   regular:
     self: CubeStruct [1,1]
     fieldList: char or char cell [1,nFields]/[nFields,1] - input list of
         given field names
 Output:
   isPositive: logical [1,1] - true if all gields whose
       names are given in the input list are in the field
       list of given object, false otherwise

   isUniqueNames: logical[1,1] - true if the specified names contain
      unique field values

   isThereVec: logical[1,nFields] - each element indicate whether the
       corresponding field is present in the cube

TODO allow for varargins
\end{verbatim}
\subsection{\texorpdfstring{smartdb.relations.ATypifiedStaticRelation.isMemberAlongDim}{isMemberAlongDim}}\label{method:smartdb.relations.ATypifiedStaticRelation.isMemberAlongDim}
\begin{verbatim}
ISMEMBERALONGDIM - performs ismember operation of CubeStruct data slices
                   along the specified dimension
Input:
  regular:
    self: ARelation [1,1] - class object
    other: ARelation [1,1] - other class object
    dim: double[1,1] - dimension number for ismember operation

  properties:
    keyFieldNameList/fieldNameList: char or char cell [1,nKeyFields] -
        list  of fields to which ismember is applied; by default all
        fields of first (self) object are used


Output:
  regular:
    isThere: logical [nSlices,1] - determines for each data slice of the
        first (self) object whether combination of values for key fields
        is in the second (other) object or not
    indTheres: double [nSlices,1] - zero if the corresponding coordinate
        of isThere is false, otherwise the highest index of the
        corresponding data slice in the second (other) object
\end{verbatim}
\subsection{\texorpdfstring{smartdb.relations.ATypifiedStaticRelation.isMemberTuples}{isMemberTuples}}\label{method:smartdb.relations.ATypifiedStaticRelation.isMemberTuples}
\begin{verbatim}
ISMEMBER - performs ismember operation for tuples of two relations by key
           fields given by special list

Usage: isTuple=isMemberTuples(self,otherRel,keyFieldNameList) or
       [isTuple indTuples]=isMemberTuples(self,otherRel,keyFieldNameList)

Input:
  regular:
    self: ARelation [1,1] - class object
    other: ARelation [1,1] - other class object
  optional:
    keyFieldNameList: char or char cell [1,nKeyFields] - list of fields
        to which ismember is applied; by default all fields of first
        (self) object are used
Output:
  regular:
    isTuple: logical [nTuples,1] - determines for each tuple of first
        (self) object whether combination of values for key fields is in
        the second (other) relation or not
    indTuples: double [nTuples,1] - zero if the corresponding coordinate
        of isTuple is false, otherwise the highest index of the
        corresponding tuple in the second (other) relation
\end{verbatim}
\subsection{\texorpdfstring{smartdb.relations.ATypifiedStaticRelation.isUniqueKey}{isUniqueKey}}\label{method:smartdb.relations.ATypifiedStaticRelation.isUniqueKey}
\begin{verbatim}
ISUNIQUEKEY - checks if a specified set of fields forms a unique key

Usage: isPositive=self.isUniqueKey(fieldNameList)

Input:
  regular:
      self: ARelation [1,1] - class object
      fieldNameList: cell[1,nFields] - list of field names for a unique
          key candidate
Output:
  isPositive: logical[1,1] - true means that a specified set of fields is
     a unique key
\end{verbatim}
\subsection{\texorpdfstring{smartdb.relations.ATypifiedStaticRelation.isequal}{isequal}}\label{method:smartdb.relations.ATypifiedStaticRelation.isequal}
\begin{verbatim}
ISEQUAL - compares current relation object with other relation object and
          returns true if they are equal, otherwise it returns false


Usage: isEq=isEqual(self,otherObj)

Input:
  regular:
    self: ARelation [1,1] - current relation object
    otherObj: ARelation [1,1] - other relation object

  properties:
    checkFieldOrder/isFieldOrderCheck: logical [1,1] - if true, then fields
        in compared relations must be in the same order, otherwise the
        order is not  important (false by default)
    checkTupleOrder: logical[1,1] -  if true, then the tuples in the
        compared relations are expected to be in the same order,
        otherwise the order is not important (false by default)

    maxTolerance: double [1,1] - maximum allowed tolerance

    compareMetaDataBackwardRef: logical[1,1] if true, the CubeStruct's
        referenced from the meta data objects are also compared

    maxRelativeTolerance: double [1,1] - maximum allowed
    relative tolerance

Output:
  isEq: logical[1,1] - result of comparison
  reportStr: char[1,] - report of comparsion
\end{verbatim}
\subsection{\texorpdfstring{smartdb.relations.ATypifiedStaticRelation.removeDuplicateTuples}{removeDuplicateTuples}}\label{method:smartdb.relations.ATypifiedStaticRelation.removeDuplicateTuples}
\begin{verbatim}
REMOVEDUPLICATETUPLES - removes all duplicate tuples from the relation

Usage: [indForwardVec,indBackwardVec]=...
           removeDuplicateTuples(self,varargin)

Input:
  regular:
    self: ARelation [1,1] - class object

  properties:
      replaceNull: logical[1,1] if true, null values are replaced with
          certain default values for all fields uniformly across all
          relation tuples
              default value is false

Output:
  optional:
    indForwardVec: double[nUniqueSlices,1] - indices of unique tuples in
       the original relation

    indBackwardVec: double[nSlices,1] - indices that map the unique
       tuples back to the original tuples
\end{verbatim}
\subsection{\texorpdfstring{smartdb.relations.ATypifiedStaticRelation.removeTuples}{removeTuples}}\label{method:smartdb.relations.ATypifiedStaticRelation.removeTuples}
\begin{verbatim}
REMOVETUPLES - removes tuples with given indices from given relation

Usage: self.removeTuples(subIndVec)

Input:
  regular:
    self: ARelation [1,1] - class object
    subIndVec: double [nSubTuples,1]/logical[nTuples,1] - array of
       indices for tuples that are selected to be removed
\end{verbatim}
\subsection{\texorpdfstring{smartdb.relations.ATypifiedStaticRelation.reorderData}{reorderData}}\label{method:smartdb.relations.ATypifiedStaticRelation.reorderData}
\begin{verbatim}
REORDERDATA - reorders cells of CubeStruct object along the specified
              dimensions according to the specified index vectors

Input:
  regular:
      self: CubeStruct [1,1] - the object
      subIndCVec: numeric[1,]/cell[1,nDims] of double [nSubElem_i,1]
          for i=1,...,nDims array of indices of field value slices that
          are selected to be returned;
          if not given (default), no indexation is performed

  optional:
      dimVec: numeric[1,nDims] - vector of dimension numbers
          corresponding to subIndCVec
\end{verbatim}
\subsection{\texorpdfstring{smartdb.relations.ATypifiedStaticRelation.saveObj}{saveObj}}\label{method:smartdb.relations.ATypifiedStaticRelation.saveObj}
\begin{verbatim}
SAVEOBJ- transforms given CubeStruct object into structure containing
         internal representation of object properties

Input:
  regular:
    self: CubeStruct [nDim1,...,nDim2]


Output:
  regular:
    SObjectData: struct [n1,...,n_k] - structure containing an internal
       representation of the specified object
\end{verbatim}
\subsection{\texorpdfstring{smartdb.relations.ATypifiedStaticRelation.setData}{setData}}\label{method:smartdb.relations.ATypifiedStaticRelation.setData}
\begin{verbatim}
SETDATA - sets values of all cells for all fields

Input:
  regular:
    self: CubeStruct[1,1]

  optional:
    SData: struct [1,1] - structure with values of all cells for
        all fields

    SIsNull: struct [1,1] - structure of fields with is-null
       information for the field content, it can be logical for
       plain real numbers of cell of logicals for cell strs or
       cell of cell of str for more complex types

    SIsValueNull: struct [1,1] - structure with logicals
        determining whether value corresponding to each field
        and field cell is null or not

  properties:
      fieldNameList: cell[1,] of char[1,] - list of fields for which data
          should be generated, if not specified, all fields from the
          relation are taken

      isConsistencyCheckedVec: logical [1,1]/[1,2]/[1,3] -
          the first element defines if a consistency between the value
              elements (data, isNull and isValueNull) is checked;
          the second element (if specified) defines if
              value's type is checked.
          the third element defines if consistency between of sizes
              between different fields is checked
            If isConsistencyCheckedVec
              if scalar, it is automatically replicated to form a
                  3-element vector
              if the third element is not specified it is assumed
                  to be true

      transactionSafe: logical[1,1], if true, the operation is performed
         in a transaction-safe manner

      checkStruct: logical[1,nStruct] - an array of indicators which when
         all true force checking of structure content (including presence
         of required fields). The first element correspod to SData, the
         second and the third (if specified) to SIsNull and SIsValueNull
         correspondingly

      structNameList: char[1,]/cell[1,], name of data structure/list of
        data structure names to which the function is to
             be applied, can be composed from the following values

           SData - data itself

           SIsNull - contains is-null indicator information for data
                values

           SIsValueNull - contains is-null indicators for CubeStruct cells
               (not for cell values)
        structNameList={'SData'} by default

      fieldMetaData: smartdb.cubes.CubeStructFieldInfo[1,] - field meta
         data array which is used for data validity checking and for
         replacing the existing meta-data

      mdFieldNameList: cell[1,] of char - list of names of fields for
         which meta data is specified

      dataChangeIsComplete: logical[1,1] - indicates whether a change
          performed by the function is complete

Note: call of setData with an empty list of arguments clears
   the data
\end{verbatim}
\subsection{\texorpdfstring{smartdb.relations.ATypifiedStaticRelation.setField}{setField}}\label{method:smartdb.relations.ATypifiedStaticRelation.setField}
\begin{verbatim}
SETFIELDINTERNAL - sets values of all cells for given field

Usage: setFieldInternal(self,fieldName,value)

Input:
  regular:
    self: CubeStruct [1,1]
    fieldName: char - name of field
    value: array [] of some type - field values

  optional:
    isNull: logical/cell[]
    isValueNull: logical[]

  properties:
    structNameList: list of internal structures to return (by default it
      is {SData, SIsNull, SIsValueNull}

    inferIsNull: logical[1,2] - the first (second) element = false
      means that IsNull (IsValueNull) indicator for a field in question
          is kept intact (default = [true,true])

      Note: if structNameList contains 'SIsValueNull' entry,
       inferIsValueNull parameter is overwritten by false
\end{verbatim}
\subsection{\texorpdfstring{smartdb.relations.ATypifiedStaticRelation.sortBy}{sortBy}}\label{method:smartdb.relations.ATypifiedStaticRelation.sortBy}
\begin{verbatim}
SORTBY - sorts all tuples of given relation with respect to some of its
         fields

Usage: sortBy(self,sortFieldNameList,varargin)

input:
  regular:
    self: ARelation [1,1] - class object
    sortFieldNameList: char or char cell [1,nFields] - list of field
        names with respect to which tuples are sorted
  properties:
    direction: char or char cell [1,nFields] - direction of sorting for
        all fields (if one value is given) or for each field separately;
        each value may be 'asc' or 'desc'
\end{verbatim}
\subsection{\texorpdfstring{smartdb.relations.ATypifiedStaticRelation.sortByAlongDim}{sortByAlongDim}}\label{method:smartdb.relations.ATypifiedStaticRelation.sortByAlongDim}
\begin{verbatim}
SORTBYALONGDIM -  sorts data of given CubeStruct object along the
                  specified dimension using the specified fields

Usage: sortByInternal(self,sortFieldNameList,varargin)

input:
  regular:
    self: CubeStruct [1,1] - class object
    sortFieldNameList: char or char cell [1,nFields] - list of field
        names with respect to which field content is sorted
    sortDim: numeric[1,1] - dimension number along which the sorting is
       to be performed
    properties:
    direction: char or char cell [1,nFields] - direction of sorting for
        all fields (if one value is given) or for each field separately;
        each value may be 'asc' or 'desc'
\end{verbatim}
\subsection{\texorpdfstring{smartdb.relations.ATypifiedStaticRelation.toArray}{toArray}}\label{method:smartdb.relations.ATypifiedStaticRelation.toArray}
\begin{verbatim}
TOARRAY - transforms values of all CubeStruct cells into a multi-
          dimentional array

Usage: resCArray=toArray(self,varargin)

Input:
  regular:
    self: CubeStruct [1,1]

  properties:
    checkInputs: logical[1,1] - if false, the method skips checking the
       input parameters for consistency

    fieldNameList: cell[1,] - list of filed names to return

    structNameList: cell[1,]/char[1,], data structure list
       for which the data is to be taken from, can consist of the
       following values

      SData - data itself
      SIsNull - contains is-null indicator information for data values
      SIsValueNull - contains is-null indicators for CubeStruct cells
         (not for cell values)

    groupByColumns: logical[1,1], if true, each column is returned in a
       separate cell

    outputType: char[1,] - method of formign an output array, the
       following methods are supported:
           'uniformMat' - the field values are concatenated without any
                   type/size transformations. As a result, this method
                   will fail if the specified fields have different types
                   or/and sizes along any dimension apart from catDim

           'uniformCell' - not-cell fields are converted to cells
                   element-wise but no size-transformations is performed.
                   This method will fail if the specified fields have
                   different sizes along any dimension apart from catDim

           'notUniform' - this method doesn't make any assumptions about
                   size or type of the fields. Each field value is wrapped
                   into cell in a such way that a size of resulting cell
                   is minDimensionSizeVec for each field. Thus if for
                   instance is size of cube object is [2,3,4] and a field
                   size is [2,4,5,10,30] its value is splitted into 2*4*5
                   pieces with each piece of size [1,1,1,10,30] put it
                   its separate cell
           'adaptiveCell' - functions similarly to 'nonUniform' except for
                   the cases when a field value size equals
                   minDimensionSizeVec exactly i.e. the field takes only
                   scalar values. In such cases no wrapping into cell is
                   performed which allows to get a more transparent
                   output.

    catDim: double[1,1] - dimension number for
       concatenating outputs when groupByColumns is false


    replaceNull: logical[1,1], if true, null values from SData are
       replaced by null replacement, = true by default

    nullTopReplacement: - can be of any type and currently only applicable
      when  UniformOutput=false and of
      the corresponding column type if UniformOutput=true.

      Note!: this parameter is disregarded for any dataStructure different
         from 'SData'.

      Note!: the main difference between this parameter and the following
         parameters is that nullTopReplacement can violate field type
         constraints thus allowing to replace doubles with strings for
         instance (for non-uniform output types only of course)


    nullReplacements: cell[1,nReplacedFields]  - list of null
       replacements for each of the fields

    nullReplacementFields: cell[1,nReplacedFields] - list of fields in
       which the nulls are to be replaced with the specified values,
       if not specified it is assumed that all fields are to be replaced

       NOTE!: all fields not listed in this parameter are replaced with
       the default values


Output:
  Case1 (one output is requested and length(structNameList)==1):

      resCMat: matrix/cell[]  with values of all fields (or
        fields selected by optional arguments) for all CubeStruct
        data cells

  Case2 (multiple outputs are requested and their number =
    length(structNameList) each output is assigned resCMat for the
    corresponding struct

  Case3 (2 outputs is requested or length(structNameList)+1 outputs is
  requested). In this case the last output argument is

       isConvertedToCell: logical[nFields,nStructs] -  matrix with true
          values on the positions which correspond to fields converted to
          cells
\end{verbatim}
\subsection{\texorpdfstring{smartdb.relations.ATypifiedStaticRelation.toCell}{toCell}}\label{method:smartdb.relations.ATypifiedStaticRelation.toCell}
\begin{verbatim}
TOCELL - transforms values of all fields for all tuples into two
         dimensional cell array

Usage: resCMat=toCell(self,varargin)

input:
  regular:
    self: ARelation [1,1] - class object
  optional:
    fieldName1: char - name of first field
    ...
    fieldNameN: char - name of N-th field
output:
  resCMat: cell [nTuples,nFields(N)] - cell with values of all fields (or
      fields selected by optional arguments) for all tuples

FIXME - order fields in setData method
\end{verbatim}
\subsection{\texorpdfstring{smartdb.relations.ATypifiedStaticRelation.toCellIsNull}{toCellIsNull}}\label{method:smartdb.relations.ATypifiedStaticRelation.toCellIsNull}
\begin{verbatim}
TOCELLISNULL - transforms is-null indicators of all fields for all tuples
               into two dimensional cell array

Usage: resCMat=toCell(self,varargin)

input:
  regular:
    self: ARelation [1,1] - class object
  optional:
    fieldName1: char - name of first field
    ...
    fieldNameN: char - name of N-th field
output:
  resCMat: cell [nTuples,nFields(N)] - cell with values of all fields (or
      fields selected by optional arguments) for all tuples

FIXME - order fields in setData method
\end{verbatim}
\subsection{\texorpdfstring{smartdb.relations.ATypifiedStaticRelation.toDispCell}{toDispCell}}\label{method:smartdb.relations.ATypifiedStaticRelation.toDispCell}
\begin{verbatim}
TODISPCELL - transforms values of all fields into their character
             representation

Usage: resCMat=toDispCell(self)

Input:
  regular:
    self: ARelation [1,1] - class object

  properties:
      nullTopReplacement: any[1,1] - value used to replace null values
      fieldNameList: cell[1,] of char[1,] - field name list

Output:
  dataCell: cell[nRows,nCols] of char[1,] - cell array containing the
      character representation of field values
\end{verbatim}
\subsection{\texorpdfstring{smartdb.relations.ATypifiedStaticRelation.toMat}{toMat}}\label{method:smartdb.relations.ATypifiedStaticRelation.toMat}
\begin{verbatim}
TOMAT - transforms values of all fields for all tuples into two
        dimensional array

Usage: resCMat=toMat(self,varargin)

input:
  regular:
    self: ARelation [1,1] - class object

  optional:
    fieldNameList: cell[1,] - list of filed names to return

    uniformOutput: logical[1,1], true - cell is returned, false - the
       functions tries to return a result as a matrix

    groupByColumns: logical[1,1], if true, each column is returned in a
       separate cell

    structNameList/dataStructure: char[1,], data structure for which the
       data is to be taken from, can have one of the following values

      SData - data itself
      SIsNull - contains is-null indicator information for data values
      SIsValueNull - contains is-null indicators for relation cells (not
         for cell values

    replaceNull: logical[1,1], if true, null values from SData are
       replaced by null replacement, = true by default

    nullTopReplacement: - can be of any type and currently only applicable
      when  UniformOutput=false and of
      the corresponding column type if UniformOutput=true.

      Note!: this parameter is disregarded for any dataStructure different
         from 'SData'.

      Note!: the main difference between this parameter and the following
         parameters is that nullTopReplacement can violate field type
         constraints thus allowing to replace doubles with strings for
         instance (for non-uniform output types only of course)


    nullReplacements: cell[1,nReplacedFields]  - list of null
       replacements for each of the fields

    nullReplacementFields: cell[1,nReplacedFields] - list of fields in
       which the nulls are to be replaced with the specified values,
       if not specified it is assumed that all fields are to be replaced

       NOTE!: all fields not listed in this parameter are replaced with
       the default values

output:
  resCMat:  [nTuples,nFields(N)] - matrix/cell with values of all fields
      (or fields selected by optional arguments) for all tuples
\end{verbatim}
\subsection{\texorpdfstring{smartdb.relations.ATypifiedStaticRelation.toStruct}{toStruct}}\label{method:smartdb.relations.ATypifiedStaticRelation.toStruct}
\begin{verbatim}
TOSTRUCT - transforms given CubeStruct object into structure

Input:
  regular:
    self: CubeStruct [nDim1,...,nDim2]


Output:
  regular:
    SObjectData: struct [n1,...,n_k] - structure containing an internal
       representation of the specified object
\end{verbatim}
\subsection{\texorpdfstring{smartdb.relations.ATypifiedStaticRelation.unionWith}{unionWith}}\label{method:smartdb.relations.ATypifiedStaticRelation.unionWith}
\begin{verbatim}
UNIONWITH - adds tuples of the input relation to the set of tuples of the
            original relation
Usage: self.unionWith(inpRel)

Input:
  regular:
    self: ARelation [1,1] - class object
    inpRel1: ARelation [1,1] - object to get the additional tuples from
      ...
    inpRelN: ARelation [1,1] - object to get the additional tuples from

  properties:
      checkType: logical[1,1] - if true, union is only performed when the
          types of relations is the same. Default value is false

      checkStruct: logical[1,nStruct] - an array of indicators which when
         true force checking of structure content (including presence
         of all required fields). The first element correspod to SData,
         the second and the third (if specified) to SIsNull and
         SIsValueNull correspondingly

      checkConsistency: logical [1,1]/[1,2] - the
          first element defines if a consistency between the value
          elements (data, isNull and isValueNull) is checked;
          the second element (if specified) defines if
          value's type is checked. If isConsistencyChecked
          is scalar, it is automatically replicated to form a
          two-element vector.
          Note: default value is true
\end{verbatim}
\subsection{\texorpdfstring{smartdb.relations.ATypifiedStaticRelation.unionWithAlongDim}{unionWithAlongDim}}\label{method:smartdb.relations.ATypifiedStaticRelation.unionWithAlongDim}
\begin{verbatim}
UNIONWITHALONGDIM - adds data from the input CubeStructs

Usage: self.unionWithAlongDim(unionDim,inpCube)

Input:
  regular:
  self:
      inpCube1: CubeStruct [1,1] - object to get the additional data from
          ...
      inpCubeN: CubeStruct [1,1] - object to get the additional data from

  properties:
      checkType: logical[1,1] - if true, union is only performed when the
          types of relations is the same. Default value is false

      checkStruct: logical[1,nStruct] - an array of indicators which when
         true force checking of structure content (including presence of
all required fields). The first element correspod to SData, the
         second and the third (if specified) to SIsNull and SIsValueNull
         correspondingly

      checkConsistency: logical [1,1]/[1,2] - the
          first element defines if a consistency between the value
          elements (data, isNull and isValueNull) is checked;
          the second element (if specified) defines if
          value's type is checked. If isConsistencyChecked
          is scalar, it is automatically replicated to form a
          two-element vector.
          Note: default value is true
\end{verbatim}
\subsection{\texorpdfstring{smartdb.relations.ATypifiedStaticRelation.writeToCSV}{writeToCSV}}\label{method:smartdb.relations.ATypifiedStaticRelation.writeToCSV}
\begin{verbatim}
WRITETOCSV - writes a content of relation into Excel spreadsheet file
Input:
  regular:
      self:
      filePath: char[1,] - file path

Output:
  none
\end{verbatim}
\subsection{\texorpdfstring{smartdb.relations.ATypifiedStaticRelation.writeToXLS}{writeToXLS}}\label{method:smartdb.relations.ATypifiedStaticRelation.writeToXLS}
\begin{verbatim}
WRITETOXLS - writes a content of relation into Excel spreadsheet file
Input:
  regular:
      self:
      filePath: char[1,] - file path

Output:
  fileName: char[1,] - resulting file name, may not match with filePath
      when Excel is not available and csv format is used instead
\end{verbatim}
\section{gras.ellapx.smartdb.rels.EllTube}\label{secClassDescr:gras.ellapx.smartdb.rels.EllTube}
\subsection{\texorpdfstring{gras.ellapx.smartdb.rels.EllTube.EllTube}{EllTube}}\label{method:gras.ellapx.smartdb.rels.EllTube.EllTube}
\begin{verbatim}
EllTube - class which keeps ellipsoidal tubes

Fields:
  QArray:cell[1, nElem] - Array of ellipsoid matrices
  aMat:cell[1, nElem] - Array of ellipsoid centers
  scaleFactor:double[1, 1] - Tube scale factor
  MArray:cell[1, nElem] - Array of regularization ellipsoid matrices
  dim :double[1, 1] - Dimensionality
  sTime:double[1, 1] - Time s
  approxSchemaName:cell[1,] - Name
  approxSchemaDescr:cell[1,] - Description
  approxType:gras.ellapx.enums.EApproxType - Type of approximation
                (external, internal, not defined)
  timeVec:cell[1, m] - Time vector
  calcPrecision:double[1, 1] - Calculation precision
  indSTime:double[1, 1]  - index of sTime within timeVec
  ltGoodDirMat:cell[1, nElem] - Good direction curve
  lsGoodDirVec:cell[1, nElem] - Good direction at time s
  ltGoodDirNormVec:cell[1, nElem] - Norm of good direction curve
  lsGoodDirNorm:double[1, 1] - Norm of good direction at time s
  xTouchCurveMat:cell[1, nElem] - Touch point curve for good
                                  direction
  xTouchOpCurveMat:cell[1, nElem] - Touch point curve for direction
                                    opposite to good direction
  xsTouchVec:cell[1, nElem]  - Touch point at time s
  xsTouchOpVec :cell[1, nElem] - Touch point at time s

  TODO: correct description of the fields in gras.ellapx.smartdb.rels.EllTube
\end{verbatim}


See the description of the following methods in section \ref{secClassDescr:smartdb.relations.ATypifiedStaticRelation}
 for smartdb.relations.ATypifiedStaticRelation:

\begin{list}{}{}
 \item \hyperref[method:smartdb.relations.ATypifiedStaticRelation.addData]{addData}
 \item \hyperref[method:smartdb.relations.ATypifiedStaticRelation.addDataAlongDim]{addDataAlongDim}
 \item \hyperref[method:smartdb.relations.ATypifiedStaticRelation.addTuples]{addTuples}
 \item \hyperref[method:smartdb.relations.ATypifiedStaticRelation.applyGetFunc]{applyGetFunc}
 \item \hyperref[method:smartdb.relations.ATypifiedStaticRelation.applySetFunc]{applySetFunc}
 \item \hyperref[method:smartdb.relations.ATypifiedStaticRelation.applyTupleGetFunc]{applyTupleGetFunc}
 \item \hyperref[method:smartdb.relations.ATypifiedStaticRelation.clearData]{clearData}
 \item \hyperref[method:smartdb.relations.ATypifiedStaticRelation.clone]{clone}
 \item \hyperref[method:smartdb.relations.ATypifiedStaticRelation.copyFrom]{copyFrom}
 \item \hyperref[method:smartdb.relations.ATypifiedStaticRelation.createInstance]{createInstance}
 \item \hyperref[method:smartdb.relations.ATypifiedStaticRelation.dispOnUI]{dispOnUI}
 \item \hyperref[method:smartdb.relations.ATypifiedStaticRelation.display]{display}
 \item \hyperref[method:smartdb.relations.ATypifiedStaticRelation.fromStructList]{fromStructList}
 \item \hyperref[method:smartdb.relations.ATypifiedStaticRelation.getCopy]{getCopy}
 \item \hyperref[method:smartdb.relations.ATypifiedStaticRelation.getFieldDescrList]{getFieldDescrList}
 \item \hyperref[method:smartdb.relations.ATypifiedStaticRelation.getFieldIsNull]{getFieldIsNull}
 \item \hyperref[method:smartdb.relations.ATypifiedStaticRelation.getFieldIsValueNull]{getFieldIsValueNull}
 \item \hyperref[method:smartdb.relations.ATypifiedStaticRelation.getFieldNameList]{getFieldNameList}
 \item \hyperref[method:smartdb.relations.ATypifiedStaticRelation.getFieldProjection]{getFieldProjection}
 \item \hyperref[method:smartdb.relations.ATypifiedStaticRelation.getFieldTypeList]{getFieldTypeList}
 \item \hyperref[method:smartdb.relations.ATypifiedStaticRelation.getFieldTypeSpecList]{getFieldTypeSpecList}
 \item \hyperref[method:smartdb.relations.ATypifiedStaticRelation.getFieldValueSizeMat]{getFieldValueSizeMat}
 \item \hyperref[method:smartdb.relations.ATypifiedStaticRelation.getIsFieldValueNull]{getIsFieldValueNull}
 \item \hyperref[method:smartdb.relations.ATypifiedStaticRelation.getMinDimensionSize]{getMinDimensionSize}
 \item \hyperref[method:smartdb.relations.ATypifiedStaticRelation.getMinDimensionality]{getMinDimensionality}
 \item \hyperref[method:smartdb.relations.ATypifiedStaticRelation.getNElems]{getNElems}
 \item \hyperref[method:smartdb.relations.ATypifiedStaticRelation.getNFields]{getNFields}
 \item \hyperref[method:smartdb.relations.ATypifiedStaticRelation.getNTuples]{getNTuples}
 \item \hyperref[method:smartdb.relations.ATypifiedStaticRelation.getSortIndex]{getSortIndex}
 \item \hyperref[method:smartdb.relations.ATypifiedStaticRelation.getTuples]{getTuples}
 \item \hyperref[method:smartdb.relations.ATypifiedStaticRelation.getTuplesFilteredBy]{getTuplesFilteredBy}
 \item \hyperref[method:smartdb.relations.ATypifiedStaticRelation.getTuplesIndexedBy]{getTuplesIndexedBy}
 \item \hyperref[method:smartdb.relations.ATypifiedStaticRelation.getTuplesJoinedWith]{getTuplesJoinedWith}
 \item \hyperref[method:smartdb.relations.ATypifiedStaticRelation.getUniqueData]{getUniqueData}
 \item \hyperref[method:smartdb.relations.ATypifiedStaticRelation.getUniqueDataAlongDim]{getUniqueDataAlongDim}
 \item \hyperref[method:smartdb.relations.ATypifiedStaticRelation.getUniqueTuples]{getUniqueTuples}
 \item \hyperref[method:smartdb.relations.ATypifiedStaticRelation.initByEmptyDataSet]{initByEmptyDataSet}
 \item \hyperref[method:smartdb.relations.ATypifiedStaticRelation.initByNullDataSet]{initByNullDataSet}
 \item \hyperref[method:smartdb.relations.ATypifiedStaticRelation.isFields]{isFields}
 \item \hyperref[method:smartdb.relations.ATypifiedStaticRelation.isMemberAlongDim]{isMemberAlongDim}
 \item \hyperref[method:smartdb.relations.ATypifiedStaticRelation.isMemberTuples]{isMemberTuples}
 \item \hyperref[method:smartdb.relations.ATypifiedStaticRelation.isUniqueKey]{isUniqueKey}
 \item \hyperref[method:smartdb.relations.ATypifiedStaticRelation.isequal]{isequal}
 \item \hyperref[method:smartdb.relations.ATypifiedStaticRelation.removeDuplicateTuples]{removeDuplicateTuples}
 \item \hyperref[method:smartdb.relations.ATypifiedStaticRelation.removeTuples]{removeTuples}
 \item \hyperref[method:smartdb.relations.ATypifiedStaticRelation.reorderData]{reorderData}
 \item \hyperref[method:smartdb.relations.ATypifiedStaticRelation.saveObj]{saveObj}
 \item \hyperref[method:smartdb.relations.ATypifiedStaticRelation.setData]{setData}
 \item \hyperref[method:smartdb.relations.ATypifiedStaticRelation.setField]{setField}
 \item \hyperref[method:smartdb.relations.ATypifiedStaticRelation.sortBy]{sortBy}
 \item \hyperref[method:smartdb.relations.ATypifiedStaticRelation.sortByAlongDim]{sortByAlongDim}
 \item \hyperref[method:smartdb.relations.ATypifiedStaticRelation.toArray]{toArray}
 \item \hyperref[method:smartdb.relations.ATypifiedStaticRelation.toCell]{toCell}
 \item \hyperref[method:smartdb.relations.ATypifiedStaticRelation.toCellIsNull]{toCellIsNull}
 \item \hyperref[method:smartdb.relations.ATypifiedStaticRelation.toDispCell]{toDispCell}
 \item \hyperref[method:smartdb.relations.ATypifiedStaticRelation.toMat]{toMat}
 \item \hyperref[method:smartdb.relations.ATypifiedStaticRelation.toStruct]{toStruct}
 \item \hyperref[method:smartdb.relations.ATypifiedStaticRelation.unionWith]{unionWith}
 \item \hyperref[method:smartdb.relations.ATypifiedStaticRelation.unionWithAlongDim]{unionWithAlongDim}
 \item \hyperref[method:smartdb.relations.ATypifiedStaticRelation.writeToCSV]{writeToCSV}
 \item \hyperref[method:smartdb.relations.ATypifiedStaticRelation.writeToXLS]{writeToXLS}
\end{list}
\subsection{\texorpdfstring{gras.ellapx.smartdb.rels.EllTube.cat}{cat}}\label{method:gras.ellapx.smartdb.rels.EllTube.cat}
\begin{verbatim}
CAT  - concatenates data from relation objects.

Input:
  regular:
      self.
      newEllTubeRel: smartdb.relation.StaticRelation[1, 1]/
          smartdb.relation.DynamicRelation[1, 1] - relation object
  properties:
      isReplacedByNew: logical[1,1] - if true, sTime and
          values of properties corresponding to sTime are taken
          from newEllTubeRel. Common times in self and
          newEllTubeRel are allowed, however the values for
          those times are taken either from self or from
          newEllTubeRel depending on value of isReplacedByNew
          property

      isCommonValuesChecked: logical[1,1] - if true, values
          at common times (if such are found) are checked for
          strong equality (with zero precision). If not equal
          - an exception is thrown. True by default.

      commonTimeAbsTol: double[1,1] - absolute tolerance used
          for comparing values at common times, =0 by default

      commonTimeRelTol: double[1,1] - absolute tolerance used
          for comparing values at common times, =0 by default

Output:
  catEllTubeRel:smartdb.relation.StaticRelation[1, 1]/
      smartdb.relation.DynamicRelation[1, 1] - relation object
      resulting from CAT operation
\end{verbatim}
\subsection{\texorpdfstring{gras.ellapx.smartdb.rels.EllTube.cut}{cut}}\label{method:gras.ellapx.smartdb.rels.EllTube.cut}
\begin{verbatim}

\end{verbatim}
\subsection{\texorpdfstring{gras.ellapx.smartdb.rels.EllTube.fromEllArray}{fromEllArray}}\label{method:gras.ellapx.smartdb.rels.EllTube.fromEllArray}
\begin{verbatim}
FROMELLARRAY  - creates a relation object using an array of ellipsoids

Input:
  regular:
    qEllArray: ellipsoid[nDim1, nDim2, ..., nDimN] - array of ellipsoids

  optional:
   timeVec:cell[1, m] - time vector
   ltGoodDirArray:cell[1, nElem] - good direction at time s
   sTime:double[1, 1] - time s
   approxType:gras.ellapx.enums.EApproxType - type of approximation
                (external, internal, not defined)
   approxSchemaName:cell[1,] - name of the schema
   approxSchemaDescr:cell[1,] - description of the schema
   calcPrecision:double[1, 1] - calculation precision

Output:
   ellTubeRel: smartdb.relation.StaticRelation[1, 1] - constructed relation
       object
\end{verbatim}
\subsection{\texorpdfstring{gras.ellapx.smartdb.rels.EllTube.fromEllMArray}{fromEllMArray}}\label{method:gras.ellapx.smartdb.rels.EllTube.fromEllMArray}
\begin{verbatim}
FROMELLMARRAY  - creates a relation object using an array of ellipsoids.
                 This method uses regularizer in the form of a matrix
                 function.

Input:
  regular:
    qEllArray: ellipsoid[nDim1, nDim2, ..., nDimN] - array of ellipsoids
    ellMArr: double[nDim1, nDim2, ..., nDimN] - regularization ellipsoid
        matrices

  optional:
   timeVec:cell[1, m] - time vector
   ltGoodDirArray:cell[1, nElem] - good direction at time s
   sTime:double[1, 1] - time s
   approxType:gras.ellapx.enums.EApproxType - type of approximation
                (external, internal, not defined)
   approxSchemaName:cell[1,] - name of the schema
   approxSchemaDescr:cell[1,] - description of the schema
   calcPrecision:double[1, 1] - calculation precision

Output:
   ellTubeRel: smartdb.relation.StaticRelation[1, 1] - constructed relation
         object
\end{verbatim}
\subsection{\texorpdfstring{gras.ellapx.smartdb.rels.EllTube.fromQArrays}{fromQArrays}}\label{method:gras.ellapx.smartdb.rels.EllTube.fromQArrays}
\begin{verbatim}
FROMQARRAYS  - creates a relation object using an array of ellipsoids,
               described by the array of ellipsoid matrices and
               array of ellipsoid centers.This method used default
               scale factor.

Input:
  regular:
    QArrayList: double[nDim1, nDim2, ..., nDimN] - array of ellipsoid
        matrices
    aMat: double[nDim1, nDim2, ..., nDimN] - array of ellipsoid centers

Optional:
   MArrayList:cell[1, nElem] - array of regularization ellipsoid matrices
   timeVec:cell[1, m] - time vector
   ltGoodDirArray:cell[1, nElem] - good direction at time s
   sTime:double[1, 1] - time s
   approxType:gras.ellapx.enums.EApproxType - type of approximation
                (external, internal, not defined)
   approxSchemaName:cell[1,] - name of the schema
   approxSchemaDescr:cell[1,] - description of the schema
   calcPrecision:double[1, 1] - calculation precision

Output:
   ellTubeRel: smartdb.relation.StaticRelation[1, 1] - constructed relation
       object
\end{verbatim}
\subsection{\texorpdfstring{gras.ellapx.smartdb.rels.EllTube.fromQMArrays}{fromQMArrays}}\label{method:gras.ellapx.smartdb.rels.EllTube.fromQMArrays}
\begin{verbatim}
FROMQMARRAYS  - creates a relation object using an array of ellipsoids,
                described by the array of ellipsoid matrices and
                array of ellipsoid centers. Also this method uses
                regularizer in the form of a matrix function. This method
                used default scale factor.

Input:
  regular:
  QArrayList: double[nDim1, nDim2, ..., nDimN] - array of ellipsoid
        matrices
  aMat: double[nDim1, nDim2, ..., nDimN] - array of ellipsoid centers
  MArrayList: double[nDim1, nDim2, ..., nDimN] - ellipsoid  matrices of
        regularization

 optional:
   timeVec:cell[1, m] - time vector
   ltGoodDirArray:cell[1, nElem] - good direction at time s
   sTime:double[1, 1] - time s
   approxType:gras.ellapx.enums.EApproxType - type of approximation
                (external, internal, not defined)
   approxSchemaName:cell[1,] - name of the schema
   approxSchemaDescr:cell[1,] - description of the schema
   calcPrecision:double[1, 1] - calculation precision

Output:
   ellTubeRel: smartdb.relation.StaticRelation[1, 1] - constructed relation
         object
\end{verbatim}
\subsection{\texorpdfstring{gras.ellapx.smartdb.rels.EllTube.fromQMScaledArrays}{fromQMScaledArrays}}\label{method:gras.ellapx.smartdb.rels.EllTube.fromQMScaledArrays}
\begin{verbatim}
FROMQMSCALEDARRAYS  - creates a relation object using an array of ellipsoids,
                      described by the array of ellipsoid matrices and
                      array of ellipsoid centers. Also this method uses
                      regularizer in the form of a matrix function.


Input:
  regular:
    QArrayList: double[nDim1, nDim2, ..., nDimN] - array of ellipsoid
        matrices
    aMat: double[nDim1, nDim2, ..., nDimN] - array of ellipsoid centers
    MArrayList: double[nDim1, nDim2, ..., nDimN] - ellipsoid matrices
              of regularization
    scaleFactor:double[1, 1] - tube scale factor

 optional:
   timeVec:cell[1, m] - time vector
   ltGoodDirArray:cell[1, nElem] - good direction at time s
   sTime:double[1, 1] - time s
   approxType:gras.ellapx.enums.EApproxType - type of approximation
                (external, internal, not defined)
   approxSchemaName:cell[1,] - name of the schema
   approxSchemaDescr:cell[1,] - description of the schema
   calcPrecision:double[1, 1] - calculation precision

Output:
   ellTubeRel: smartdb.relation.StaticRelation[1, 1] - constructed relation
         object
\end{verbatim}
\subsection{\texorpdfstring{gras.ellapx.smartdb.rels.EllTube.getData}{getData}}\label{method:gras.ellapx.smartdb.rels.EllTube.getData}
\begin{verbatim}
GETDATA - returns an indexed projection of CubeStruct object's content

Input:
  regular:
      self: CubeStruct [1,1] - the object

  optional:

      subIndCVec:
        Case#1: numeric[1,]/numeric[,1]

        Case#2: cell[1,nDims]/cell[nDims,1] of double [nSubElem_i,1]
              for i=1,...,nDims

          -array of indices of field value slices that are selected
          to be returned; if not given (default),
          no indexation is performed

        Note!: numeric components of subIndVec are allowed to contain
           zeros which are be treated as they were references to null
           data slices

      dimVec: numeric[1,nDims]/numeric[nDims,1] - vector of dimension
          numbers corresponding to subIndCVec

  properties:

      fieldNameList: char[1,]/cell[1,nFields] of char[1,]
          list of field names to return

      structNameList: char[1,]/cell[1,nStructs] of char[1,]
          list of internal structures to return (by default it
          is {SData, SIsNull, SIsValueNull}

      replaceNull: logical[1,1] if true, null values are replaced with
          certain default values uniformly across all the cells,
              default value is false

      nullReplacements: cell[1,nReplacedFields]  - list of null
          replacements for each of the fields

      nullReplacementFields: cell[1,nReplacedFields] - list of fields in
         which the nulls are to be replaced with the specified values,
         if not specified it is assumed that all fields are to be
         replaced

         NOTE!: all fields not listed in this parameter are replaced with
         the default values

      checkInputs: logical[1,1] - true by default (input arguments are
         checked for correctness

Output:
  regular:
    SData: struct [1,1] - structure containing values of
        fields at the selected slices, each field is an array
        containing values of the corresponding type

    SIsNull: struct [1,1] - structure containing a nested
        array with is-null indicators for each CubeStruct cell content

    SIsValueNull: struct [1,1] - structure containing a
       logical array [] for each of the fields (true
       means that a corresponding cell doesn't not contain
          any value
\end{verbatim}
\subsection{\texorpdfstring{gras.ellapx.smartdb.rels.EllTube.getEllArray}{getEllArray}}\label{method:gras.ellapx.smartdb.rels.EllTube.getEllArray}
\begin{verbatim}
GETELLARRAY - returns array of matrix's ellipsoid according to
              approxType

Input:
 regular:
    self.
    approxType:char[1,] - type of approximation(internal/external)

Output:
  apprEllMat:double[nDim1,..., nDimN] - array of array of ellipsoid's
           matrices
\end{verbatim}
\subsection{\texorpdfstring{gras.ellapx.smartdb.rels.EllTube.getJoinWith}{getJoinWith}}\label{method:gras.ellapx.smartdb.rels.EllTube.getJoinWith}
\begin{verbatim}
GETJOINWITH - returns a result of INNER join of given relation with
              another relation by the specified key fields

LIMITATION: key fields by which the join is peformed are required to form
a unique key in the given relation

Input:
  regular:
      self:
      otherRel: smartdb.relations.ARelation[1,1]
      keyFieldNameList: char[1,]/cell[1,nFields] of char[1,]

  properties:
      joinType: char[1,] - type of join, can be
          'inner' (DEFAULT)
          'leftOuter'

Output:
  resRel: smartdb.relations.ARelation[1,1] - join result
\end{verbatim}
\subsection{\texorpdfstring{gras.ellapx.smartdb.rels.EllTube.getNoCatOrCutFieldsList}{getNoCatOrCutFieldsList}}\label{method:gras.ellapx.smartdb.rels.EllTube.getNoCatOrCutFieldsList}
\begin{verbatim}

\end{verbatim}
\subsection{\texorpdfstring{gras.ellapx.smartdb.rels.EllTube.interp}{interp}}\label{method:gras.ellapx.smartdb.rels.EllTube.interp}
\begin{verbatim}

\end{verbatim}
\subsection{\texorpdfstring{gras.ellapx.smartdb.rels.EllTube.isEqual}{isEqual}}\label{method:gras.ellapx.smartdb.rels.EllTube.isEqual}
\begin{verbatim}
ISEQUAL - compares current relation object with other relation object and
          returns true if they are equal, otherwise it returns false


Usage: isEq=isEqual(self,otherObj)

Input:
  regular:
    self: ARelation [1,1] - current relation object
    otherObj: ARelation [1,1] - other relation object

  properties:
    checkFieldOrder/isFieldOrderCheck: logical [1,1] - if true, then fields
        in compared relations must be in the same order, otherwise the
        order is not  important (false by default)
    checkTupleOrder: logical[1,1] -  if true, then the tuples in the
        compared relations are expected to be in the same order,
        otherwise the order is not important (false by default)

    maxTolerance: double [1,1] - maximum allowed tolerance

    compareMetaDataBackwardRef: logical[1,1] if true, the CubeStruct's
        referenced from the meta data objects are also compared

    maxRelativeTolerance: double [1,1] - maximum allowed
    relative tolerance

Output:
  isEq: logical[1,1] - result of comparison
  reportStr: char[1,] - report of comparsion
\end{verbatim}
\subsection{\texorpdfstring{gras.ellapx.smartdb.rels.EllTube.plot}{plot}}\label{method:gras.ellapx.smartdb.rels.EllTube.plot}
\begin{verbatim}
PLOT - displays ellipsoidal tubes using the specified RelationDataPlotter


Input:
  regular:
      self:
      plObj: smartdb.disp.RelationDataPlotter[1,1] - plotter
          object used for displaying ellipsoidal tubes
\end{verbatim}
\subsection{\texorpdfstring{gras.ellapx.smartdb.rels.EllTube.project}{project}}\label{method:gras.ellapx.smartdb.rels.EllTube.project}
\begin{verbatim}
PROJECT - computes projection of the relation object onto given time
          dependent subspase
Input:
  regular:
      self.
      projType: gras.ellapx.enums.EProjType[1,1] -
          type of the projection, can be
          'Static' and 'DynamicAlongGoodCurve'
      projMatList: cell[1,nProj] of double[nSpDim,nDim] - list of
          projection matrices, not necessarily orthogonal
   fGetProjMat: function_handle[1,1] - function which creates
      vector of the projection
            matrices
       Input:
        regular:
          projMat:double[nDim, mDim] - matrix of the projection at the
            instant of time
          timeVec:double[1, nDim] - time interval
        optional:
           sTime:double[1,1] - instant of time
       Output:
          projOrthMatArray:double[1, nSpDim] - vector of the projection
            matrices
          projOrthMatTransArray:double[nSpDim, 1] - transposed vector of
            the projection matrices
Output:
   ellTubeProjRel: gras.ellapx.smartdb.rels.EllTubeProj[1, 1]/
       gras.ellapx.smartdb.rels.EllTubeUnionProj[1, 1] -
          projected ellipsoidal tube

   indProj2OrigVec:cell[nDim, 1] - index of the line number from
            which is obtained the projection

Example:
  function example
   aMat = [0 1; 0 0]; bMat = eye(2);
   SUBounds = struct();
   SUBounds.center = {'sin(t)'; 'cos(t)'};
   SUBounds.shape = [9 0; 0 2];
   sys = elltool.linsys.LinSysContinuous(aMat, bMat, SUBounds);
   x0EllObj = ell_unitball(2);
   timeVec = [0 10];
   dirsMat = [1 0; 0 1]';
   rsObj = elltool.reach.ReachContinuous(sys, x0EllObj, dirsMat, timeVec);
   ellTubeObj = rsObj.getEllTubeRel();
   unionEllTube = ...
    gras.ellapx.smartdb.rels.EllUnionTube.fromEllTubes(ellTubeObj);
   projMatList = {[1 0;0 1]};
   projType = gras.ellapx.enums.EProjType.Static;
   statEllTubeProj = unionEllTube.project(projType,projMatList,...
      @fGetProjMat);
   plObj=smartdb.disp.RelationDataPlotter();
   statEllTubeProj.plot(plObj);
end

function [projOrthMatArray,projOrthMatTransArray]=fGetProjMat(projMat,...
    timeVec,varargin)
  nTimePoints=length(timeVec);
  projOrthMatArray=repmat(projMat,[1,1,nTimePoints]);
  projOrthMatTransArray=repmat(projMat.',[1,1,nTimePoints]);
 end
\end{verbatim}
\subsection{\texorpdfstring{gras.ellapx.smartdb.rels.EllTube.projectStatic}{projectStatic}}\label{method:gras.ellapx.smartdb.rels.EllTube.projectStatic}
\begin{verbatim}

\end{verbatim}
\subsection{\texorpdfstring{gras.ellapx.smartdb.rels.EllTube.projectToOrths}{projectToOrths}}\label{method:gras.ellapx.smartdb.rels.EllTube.projectToOrths}
\begin{verbatim}
PROJECTTOORTHS - project elltube onto subspace defined by
vectors of standart basis with indices specified in indVec

Input:
  regular:
      self: gras.ellapx.smartdb.rels.EllTube[1, 1] - elltube
          object
      indVec: double[1, nProjDims] - indices specifying a subset of
          standart basis
  optional:
      projType: gras.ellapx.enums.EProjType[1, 1] -  type of
          projection

Output:
  regular:
      ellTubeProjRel: gras.ellapx.smartdb.rels.EllTubeProj[1, 1] -
          elltube projection

Example:
  ellTubeProjRel = ellTubeRel.projectToOrths([1,2])
  projType = gras.ellapx.enums.EProjType.DynamicAlongGoodCurve
  ellTubeProjRel = ellTubeRel.projectToOrths([3,4,5], projType)
\end{verbatim}
\subsection{\texorpdfstring{gras.ellapx.smartdb.rels.EllTube.scale}{scale}}\label{method:gras.ellapx.smartdb.rels.EllTube.scale}
\begin{verbatim}
SCALE - scales relation object

 Input:
  regular:
     self.
     fCalcFactor - function which calculates factor for
                    fields in fieldNameList
       Input:
         regular:
           fieldNameList: char/cell[1,] of char - a list of fields
                  for which factor will be calculated
        Output:
            factor:double[1, 1] - calculated factor

      fieldNameList:cell[1,nElem]/char[1,] - names of the fields

 Output:
      none

Example:
  nPoints=5;
  calcPrecision=0.001;
  approxSchemaDescr=char.empty(1,0);
  approxSchemaName=char.empty(1,0);
  nDims=3;
  nTubes=1;
  lsGoodDirVec=[1;0;1];
  aMat=zeros(nDims,nPoints);
  timeVec=1:nPoints;
  sTime=nPoints;
  approxType=gras.ellapx.enums.EApproxType.Internal;
  qArrayList=repmat({repmat(diag([1 2 3]),[1,1,nPoints])},1,nTubes);
  ltGoodDirArray=repmat(lsGoodDirVec,[1,nTubes,nPoints]);
  fromMatEllTube=...
        gras.ellapx.smartdb.rels.EllTube.fromQArrays(qArrayList,...
        aMat, timeVec,ltGoodDirArray, sTime, approxType,...
        approxSchemaName, approxSchemaDescr, calcPrecision);
  fromMatEllTube.scale(@(varargin)2,{});
\end{verbatim}
\subsection{\texorpdfstring{gras.ellapx.smartdb.rels.EllTube.sortDetermenistically}{sortDetermenistically}}\label{method:gras.ellapx.smartdb.rels.EllTube.sortDetermenistically}
\begin{verbatim}

\end{verbatim}
\subsection{\texorpdfstring{gras.ellapx.smartdb.rels.EllTube.thinOutTuples}{thinOutTuples}}\label{method:gras.ellapx.smartdb.rels.EllTube.thinOutTuples}
\begin{verbatim}

\end{verbatim}
\section{gras.ellapx.smartdb.rels.EllTubeProj}\label{secClassDescr:gras.ellapx.smartdb.rels.EllTubeProj}
\subsection{\texorpdfstring{gras.ellapx.smartdb.rels.EllTubeProj.EllTubeProj}{EllTubeProj}}\label{method:gras.ellapx.smartdb.rels.EllTubeProj.EllTubeProj}
\begin{verbatim}
EllTubeProj - class which keeps ellipsoidal tube's projection

Fields:
  QArray:cell[1, nElem] - Array of ellipsoid matrices
  aMat:cell[1, nElem] - Array of ellipsoid centers
  scaleFactor:double[1, 1] - Tube scale factor
  MArray:cell[1, nElem] - Array of regularization ellipsoid matrices
  dim :double[1, 1] - Dimensionality
  sTime:double[1, 1] - Time s
  approxSchemaName:cell[1,] - Name
  approxSchemaDescr:cell[1,] - Description
  approxType:gras.ellapx.enums.EApproxType - Type of approximation
                (external, internal, not defined)
  timeVec:cell[1, m] - Time vector
  calcPrecision:double[1, 1] - Calculation precision
  indSTime:double[1, 1]  - index of sTime within timeVec
  ltGoodDirMat:cell[1, nElem] - Good direction curve
  lsGoodDirVec:cell[1, nElem] - Good direction at time s
  ltGoodDirNormVec:cell[1, nElem] - Norm of good direction curve
  lsGoodDirNorm:double[1, 1] - Norm of good direction at time s
  xTouchCurveMat:cell[1, nElem] - Touch point curve for good
                                  direction
  xTouchOpCurveMat:cell[1, nElem] - Touch point curve for direction
                                    opposite to good direction
  xsTouchVec:cell[1, nElem]  - Touch point at time s
  xsTouchOpVec:cell[1, nElem] - Touch point at time s
  projSTimeMat: cell[1, 1] - Projection matrix at time s
  projType:gras.ellapx.enums.EProjType - Projection type
  ltGoodDirNormOrigVec:cell[1, 1] - Norm of the original (not
                                    projected) good direction curve
  lsGoodDirNormOrig:double[1, 1] - Norm of the original (not
                                   projected)good direction at time s
  lsGoodDirOrigVec:cell[1, 1] - Original (not projected) good
                                direction at time s

TODO: correct description of the fields in
    gras.ellapx.smartdb.rels.EllTubeProj
\end{verbatim}


See the description of the following methods in section \ref{secClassDescr:smartdb.relations.ATypifiedStaticRelation}
 for smartdb.relations.ATypifiedStaticRelation:

\begin{list}{}{}
 \item \hyperref[method:smartdb.relations.ATypifiedStaticRelation.addData]{addData}
 \item \hyperref[method:smartdb.relations.ATypifiedStaticRelation.addDataAlongDim]{addDataAlongDim}
 \item \hyperref[method:smartdb.relations.ATypifiedStaticRelation.addTuples]{addTuples}
 \item \hyperref[method:smartdb.relations.ATypifiedStaticRelation.applyGetFunc]{applyGetFunc}
 \item \hyperref[method:smartdb.relations.ATypifiedStaticRelation.applySetFunc]{applySetFunc}
 \item \hyperref[method:smartdb.relations.ATypifiedStaticRelation.applyTupleGetFunc]{applyTupleGetFunc}
 \item \hyperref[method:smartdb.relations.ATypifiedStaticRelation.clearData]{clearData}
 \item \hyperref[method:smartdb.relations.ATypifiedStaticRelation.clone]{clone}
 \item \hyperref[method:smartdb.relations.ATypifiedStaticRelation.copyFrom]{copyFrom}
 \item \hyperref[method:smartdb.relations.ATypifiedStaticRelation.createInstance]{createInstance}
 \item \hyperref[method:smartdb.relations.ATypifiedStaticRelation.dispOnUI]{dispOnUI}
 \item \hyperref[method:smartdb.relations.ATypifiedStaticRelation.display]{display}
 \item \hyperref[method:smartdb.relations.ATypifiedStaticRelation.fromStructList]{fromStructList}
 \item \hyperref[method:smartdb.relations.ATypifiedStaticRelation.getCopy]{getCopy}
 \item \hyperref[method:smartdb.relations.ATypifiedStaticRelation.getFieldDescrList]{getFieldDescrList}
 \item \hyperref[method:smartdb.relations.ATypifiedStaticRelation.getFieldIsNull]{getFieldIsNull}
 \item \hyperref[method:smartdb.relations.ATypifiedStaticRelation.getFieldIsValueNull]{getFieldIsValueNull}
 \item \hyperref[method:smartdb.relations.ATypifiedStaticRelation.getFieldNameList]{getFieldNameList}
 \item \hyperref[method:smartdb.relations.ATypifiedStaticRelation.getFieldProjection]{getFieldProjection}
 \item \hyperref[method:smartdb.relations.ATypifiedStaticRelation.getFieldTypeList]{getFieldTypeList}
 \item \hyperref[method:smartdb.relations.ATypifiedStaticRelation.getFieldTypeSpecList]{getFieldTypeSpecList}
 \item \hyperref[method:smartdb.relations.ATypifiedStaticRelation.getFieldValueSizeMat]{getFieldValueSizeMat}
 \item \hyperref[method:smartdb.relations.ATypifiedStaticRelation.getIsFieldValueNull]{getIsFieldValueNull}
 \item \hyperref[method:smartdb.relations.ATypifiedStaticRelation.getMinDimensionSize]{getMinDimensionSize}
 \item \hyperref[method:smartdb.relations.ATypifiedStaticRelation.getMinDimensionality]{getMinDimensionality}
 \item \hyperref[method:smartdb.relations.ATypifiedStaticRelation.getNElems]{getNElems}
 \item \hyperref[method:smartdb.relations.ATypifiedStaticRelation.getNFields]{getNFields}
 \item \hyperref[method:smartdb.relations.ATypifiedStaticRelation.getNTuples]{getNTuples}
 \item \hyperref[method:smartdb.relations.ATypifiedStaticRelation.getSortIndex]{getSortIndex}
 \item \hyperref[method:smartdb.relations.ATypifiedStaticRelation.getTuples]{getTuples}
 \item \hyperref[method:smartdb.relations.ATypifiedStaticRelation.getTuplesFilteredBy]{getTuplesFilteredBy}
 \item \hyperref[method:smartdb.relations.ATypifiedStaticRelation.getTuplesIndexedBy]{getTuplesIndexedBy}
 \item \hyperref[method:smartdb.relations.ATypifiedStaticRelation.getTuplesJoinedWith]{getTuplesJoinedWith}
 \item \hyperref[method:smartdb.relations.ATypifiedStaticRelation.getUniqueData]{getUniqueData}
 \item \hyperref[method:smartdb.relations.ATypifiedStaticRelation.getUniqueDataAlongDim]{getUniqueDataAlongDim}
 \item \hyperref[method:smartdb.relations.ATypifiedStaticRelation.getUniqueTuples]{getUniqueTuples}
 \item \hyperref[method:smartdb.relations.ATypifiedStaticRelation.initByEmptyDataSet]{initByEmptyDataSet}
 \item \hyperref[method:smartdb.relations.ATypifiedStaticRelation.initByNullDataSet]{initByNullDataSet}
 \item \hyperref[method:smartdb.relations.ATypifiedStaticRelation.isFields]{isFields}
 \item \hyperref[method:smartdb.relations.ATypifiedStaticRelation.isMemberAlongDim]{isMemberAlongDim}
 \item \hyperref[method:smartdb.relations.ATypifiedStaticRelation.isMemberTuples]{isMemberTuples}
 \item \hyperref[method:smartdb.relations.ATypifiedStaticRelation.isUniqueKey]{isUniqueKey}
 \item \hyperref[method:smartdb.relations.ATypifiedStaticRelation.isequal]{isequal}
 \item \hyperref[method:smartdb.relations.ATypifiedStaticRelation.removeDuplicateTuples]{removeDuplicateTuples}
 \item \hyperref[method:smartdb.relations.ATypifiedStaticRelation.removeTuples]{removeTuples}
 \item \hyperref[method:smartdb.relations.ATypifiedStaticRelation.reorderData]{reorderData}
 \item \hyperref[method:smartdb.relations.ATypifiedStaticRelation.saveObj]{saveObj}
 \item \hyperref[method:smartdb.relations.ATypifiedStaticRelation.setData]{setData}
 \item \hyperref[method:smartdb.relations.ATypifiedStaticRelation.setField]{setField}
 \item \hyperref[method:smartdb.relations.ATypifiedStaticRelation.sortBy]{sortBy}
 \item \hyperref[method:smartdb.relations.ATypifiedStaticRelation.sortByAlongDim]{sortByAlongDim}
 \item \hyperref[method:smartdb.relations.ATypifiedStaticRelation.toArray]{toArray}
 \item \hyperref[method:smartdb.relations.ATypifiedStaticRelation.toCell]{toCell}
 \item \hyperref[method:smartdb.relations.ATypifiedStaticRelation.toCellIsNull]{toCellIsNull}
 \item \hyperref[method:smartdb.relations.ATypifiedStaticRelation.toDispCell]{toDispCell}
 \item \hyperref[method:smartdb.relations.ATypifiedStaticRelation.toMat]{toMat}
 \item \hyperref[method:smartdb.relations.ATypifiedStaticRelation.toStruct]{toStruct}
 \item \hyperref[method:smartdb.relations.ATypifiedStaticRelation.unionWith]{unionWith}
 \item \hyperref[method:smartdb.relations.ATypifiedStaticRelation.unionWithAlongDim]{unionWithAlongDim}
 \item \hyperref[method:smartdb.relations.ATypifiedStaticRelation.writeToCSV]{writeToCSV}
 \item \hyperref[method:smartdb.relations.ATypifiedStaticRelation.writeToXLS]{writeToXLS}
\end{list}
\subsection{\texorpdfstring{gras.ellapx.smartdb.rels.EllTubeProj.cut}{cut}}\label{method:gras.ellapx.smartdb.rels.EllTubeProj.cut}
\begin{verbatim}

\end{verbatim}
\subsection{\texorpdfstring{gras.ellapx.smartdb.rels.EllTubeProj.getData}{getData}}\label{method:gras.ellapx.smartdb.rels.EllTubeProj.getData}
\begin{verbatim}
GETDATA - returns an indexed projection of CubeStruct object's content

Input:
  regular:
      self: CubeStruct [1,1] - the object

  optional:

      subIndCVec:
        Case#1: numeric[1,]/numeric[,1]

        Case#2: cell[1,nDims]/cell[nDims,1] of double [nSubElem_i,1]
              for i=1,...,nDims

          -array of indices of field value slices that are selected
          to be returned; if not given (default),
          no indexation is performed

        Note!: numeric components of subIndVec are allowed to contain
           zeros which are be treated as they were references to null
           data slices

      dimVec: numeric[1,nDims]/numeric[nDims,1] - vector of dimension
          numbers corresponding to subIndCVec

  properties:

      fieldNameList: char[1,]/cell[1,nFields] of char[1,]
          list of field names to return

      structNameList: char[1,]/cell[1,nStructs] of char[1,]
          list of internal structures to return (by default it
          is {SData, SIsNull, SIsValueNull}

      replaceNull: logical[1,1] if true, null values are replaced with
          certain default values uniformly across all the cells,
              default value is false

      nullReplacements: cell[1,nReplacedFields]  - list of null
          replacements for each of the fields

      nullReplacementFields: cell[1,nReplacedFields] - list of fields in
         which the nulls are to be replaced with the specified values,
         if not specified it is assumed that all fields are to be
         replaced

         NOTE!: all fields not listed in this parameter are replaced with
         the default values

      checkInputs: logical[1,1] - true by default (input arguments are
         checked for correctness

Output:
  regular:
    SData: struct [1,1] - structure containing values of
        fields at the selected slices, each field is an array
        containing values of the corresponding type

    SIsNull: struct [1,1] - structure containing a nested
        array with is-null indicators for each CubeStruct cell content

    SIsValueNull: struct [1,1] - structure containing a
       logical array [] for each of the fields (true
       means that a corresponding cell doesn't not contain
          any value
\end{verbatim}
\subsection{\texorpdfstring{gras.ellapx.smartdb.rels.EllTubeProj.getEllArray}{getEllArray}}\label{method:gras.ellapx.smartdb.rels.EllTubeProj.getEllArray}
\begin{verbatim}
GETELLARRAY - returns array of matrix's ellipsoid according to
              approxType

Input:
 regular:
    self.
    approxType:char[1,] - type of approximation(internal/external)

Output:
  apprEllMat:double[nDim1,..., nDimN] - array of array of ellipsoid's
           matrices
\end{verbatim}
\subsection{\texorpdfstring{gras.ellapx.smartdb.rels.EllTubeProj.getJoinWith}{getJoinWith}}\label{method:gras.ellapx.smartdb.rels.EllTubeProj.getJoinWith}
\begin{verbatim}
GETJOINWITH - returns a result of INNER join of given relation with
              another relation by the specified key fields

LIMITATION: key fields by which the join is peformed are required to form
a unique key in the given relation

Input:
  regular:
      self:
      otherRel: smartdb.relations.ARelation[1,1]
      keyFieldNameList: char[1,]/cell[1,nFields] of char[1,]

  properties:
      joinType: char[1,] - type of join, can be
          'inner' (DEFAULT)
          'leftOuter'

Output:
  resRel: smartdb.relations.ARelation[1,1] - join result
\end{verbatim}
\subsection{\texorpdfstring{gras.ellapx.smartdb.rels.EllTubeProj.getNoCatOrCutFieldsList}{getNoCatOrCutFieldsList}}\label{method:gras.ellapx.smartdb.rels.EllTubeProj.getNoCatOrCutFieldsList}
\begin{verbatim}

\end{verbatim}
\subsection{\texorpdfstring{gras.ellapx.smartdb.rels.EllTubeProj.getReachTubeNamePrefix}{getReachTubeNamePrefix}}\label{method:gras.ellapx.smartdb.rels.EllTubeProj.getReachTubeNamePrefix}
\begin{verbatim}
GETREACHTUBEANEPREFIX - return prefix of the reach tube

Input:
  regular:
     self.
\end{verbatim}
\subsection{\texorpdfstring{gras.ellapx.smartdb.rels.EllTubeProj.getRegTubeNamePrefix}{getRegTubeNamePrefix}}\label{method:gras.ellapx.smartdb.rels.EllTubeProj.getRegTubeNamePrefix}
\begin{verbatim}
GETREGTUBEANEPREFIX - return prefix of the reg tube

Input:
  regular:
     self.
\end{verbatim}
\subsection{\texorpdfstring{gras.ellapx.smartdb.rels.EllTubeProj.interp}{interp}}\label{method:gras.ellapx.smartdb.rels.EllTubeProj.interp}
\begin{verbatim}

\end{verbatim}
\subsection{\texorpdfstring{gras.ellapx.smartdb.rels.EllTubeProj.isEqual}{isEqual}}\label{method:gras.ellapx.smartdb.rels.EllTubeProj.isEqual}
\begin{verbatim}
ISEQUAL - compares current relation object with other relation object and
          returns true if they are equal, otherwise it returns false


Usage: isEq=isEqual(self,otherObj)

Input:
  regular:
    self: ARelation [1,1] - current relation object
    otherObj: ARelation [1,1] - other relation object

  properties:
    checkFieldOrder/isFieldOrderCheck: logical [1,1] - if true, then fields
        in compared relations must be in the same order, otherwise the
        order is not  important (false by default)
    checkTupleOrder: logical[1,1] -  if true, then the tuples in the
        compared relations are expected to be in the same order,
        otherwise the order is not important (false by default)

    maxTolerance: double [1,1] - maximum allowed tolerance

    compareMetaDataBackwardRef: logical[1,1] if true, the CubeStruct's
        referenced from the meta data objects are also compared

    maxRelativeTolerance: double [1,1] - maximum allowed
    relative tolerance

Output:
  isEq: logical[1,1] - result of comparison
  reportStr: char[1,] - report of comparsion
\end{verbatim}
\subsection{\texorpdfstring{gras.ellapx.smartdb.rels.EllTubeProj.plot}{plot}}\label{method:gras.ellapx.smartdb.rels.EllTubeProj.plot}
\begin{verbatim}
PLOT - displays ellipsoidal tubes using the specified
  RelationDataPlotter

Input:
  regular:
      self:
  optional:
      plObj: smartdb.disp.RelationDataPlotter[1,1] - plotter
          object used for displaying ellipsoidal tubes
  properties:
      fGetColor: function_handle[1, 1] -
          function that specified colorVec for
          ellipsoidal tubes
      fGetAlpha: function_handle[1, 1] -
          function that specified transparency
          value for ellipsoidal tubes
      fGetLineWidth: function_handle[1, 1] -
          function that specified lineWidth for good curves
      fGetFill: function_handle[1, 1] - this
          property not used in this version
      colorFieldList: cell[nColorFields, ] of char[1, ] -
          list of parameters for color function
      alphaFieldList: cell[nAlphaFields, ] of char[1, ] -
          list of parameters for transparency function
      lineWidthFieldList: cell[nLineWidthFields, ]
          of char[1, ] - list of parameters for lineWidth
          function
      fillFieldList: cell[nIsFillFields, ] of char[1, ] -
          list of parameters for fill function
      plotSpecFieldList: cell[nPlotFields, ] of char[1, ] -
          defaul list of parameters. If for any function in
          properties not specified list of parameters,
          this one will be used

Output:
  plObj: smartdb.disp.RelationDataPlotter[1,1] - plotter
          object used for displaying ellipsoidal tubes
\end{verbatim}
\subsection{\texorpdfstring{gras.ellapx.smartdb.rels.EllTubeProj.plotExt}{plotExt}}\label{method:gras.ellapx.smartdb.rels.EllTubeProj.plotExt}
\begin{verbatim}
PLOTEXT - plots external approximation of ellTube.


Usage:
      obj.plotExt() - plots external approximation of ellTube.
      obj.plotExt('Property',PropValue,...) - plots external approximation
                                              of ellTube with setting
                                              properties.

Input:
  regular:
      obj:  EllTubeProj: EllTubeProj object
  optional:
      relDataPlotter:smartdb.disp.RelationDataPlotter[1,1] - relation data plotter object.
      colorSpec: char[1,1] - color specification code, can be 'r','g',
                   etc (any code supported by built-in Matlab function).

  properties:

      fGetColor: function_handle[1, 1] -
          function that specified colorVec for
          ellipsoidal tubes
      fGetAlpha: function_handle[1, 1] -
          function that specified transparency
          value for ellipsoidal tubes
      fGetLineWidth: function_handle[1, 1] -
          function that specified lineWidth for good curves
      fGetFill: function_handle[1, 1] - this
          property not used in this version
      colorFieldList: cell[nColorFields, ] of char[1, ] -
          list of parameters for color function
      alphaFieldList: cell[nAlphaFields, ] of char[1, ] -
          list of parameters for transparency function
      lineWidthFieldList: cell[nLineWidthFields, ]
          of char[1, ] - list of parameters for lineWidth
          function
      fillFieldList: cell[nIsFillFields, ] of char[1, ] -
          list of parameters for fill function
      plotSpecFieldList: cell[nPlotFields, ] of char[1, ] -
          defaul list of parameters. If for any function in
          properties not specified list of parameters,
          this one will be used
      'showDiscrete':logical[1,1]  -
          if true, approximation in 3D will be filled in every time slice
      'nSpacePartPoins': double[1,1] -
          number of points in every time slice.
Output:
  regular:
      plObj: smartdb.disp.RelationDataPlotter[1,1] - returns the relation
      data plotter object.
\end{verbatim}
\subsection{\texorpdfstring{gras.ellapx.smartdb.rels.EllTubeProj.plotInt}{plotInt}}\label{method:gras.ellapx.smartdb.rels.EllTubeProj.plotInt}
\begin{verbatim}
PLOTINT - plots internal approximation of ellTube.


Usage:
      obj.plotInt() - plots internal approximation of ellTube.
      obj.plotInt('Property',PropValue,...) - plots internal approximation
                                              of ellTube with setting
                                              properties.

Input:
  regular:
      obj:  EllTubeProj: EllTubeProj object
  optional:
      relDataPlotter:smartdb.disp.RelationDataPlotter[1,1] - relation data plotter object.
      colorSpec: char[1,1] - color specification code, can be 'r','g',
                   etc (any code supported by built-in Matlab function).

  properties:

      fGetColor: function_handle[1, 1] -
          function that specified colorVec for
          ellipsoidal tubes
      fGetAlpha: function_handle[1, 1] -
          function that specified transparency
          value for ellipsoidal tubes
      fGetLineWidth: function_handle[1, 1] -
          function that specified lineWidth for good curves
      fGetFill: function_handle[1, 1] - this
          property not used in this version
      colorFieldList: cell[nColorFields, ] of char[1, ] -
          list of parameters for color function
      alphaFieldList: cell[nAlphaFields, ] of char[1, ] -
          list of parameters for transparency function
      lineWidthFieldList: cell[nLineWidthFields, ]
          of char[1, ] - list of parameters for lineWidth
          function
      fillFieldList: cell[nIsFillFields, ] of char[1, ] -
          list of parameters for fill function
      plotSpecFieldList: cell[nPlotFields, ] of char[1, ] -
          defaul list of parameters. If for any function in
          properties not specified list of parameters,
          this one will be used
      'showDiscrete':logical[1,1]  -
          if true, approximation in 3D will be filled in every time slice
      'nSpacePartPoins': double[1,1] -
          number of points in every time slice.
Output:
  regular:
      plObj: smartdb.disp.RelationDataPlotter[1,1] - returns the relation
      data plotter object.
\end{verbatim}
\subsection{\texorpdfstring{gras.ellapx.smartdb.rels.EllTubeProj.projMat2str}{projMat2str}}\label{method:gras.ellapx.smartdb.rels.EllTubeProj.projMat2str}
\begin{verbatim}

\end{verbatim}
\subsection{\texorpdfstring{gras.ellapx.smartdb.rels.EllTubeProj.projRow2str}{projRow2str}}\label{method:gras.ellapx.smartdb.rels.EllTubeProj.projRow2str}
\begin{verbatim}

\end{verbatim}
\subsection{\texorpdfstring{gras.ellapx.smartdb.rels.EllTubeProj.sortDetermenistically}{sortDetermenistically}}\label{method:gras.ellapx.smartdb.rels.EllTubeProj.sortDetermenistically}
\begin{verbatim}

\end{verbatim}
\subsection{\texorpdfstring{gras.ellapx.smartdb.rels.EllTubeProj.thinOutTuples}{thinOutTuples}}\label{method:gras.ellapx.smartdb.rels.EllTubeProj.thinOutTuples}
\begin{verbatim}

\end{verbatim}
\section{gras.ellapx.smartdb.rels.EllUnionTube}\label{secClassDescr:gras.ellapx.smartdb.rels.EllUnionTube}
\subsection{\texorpdfstring{gras.ellapx.smartdb.rels.EllUnionTube.EllUnionTube}{EllUnionTube}}\label{method:gras.ellapx.smartdb.rels.EllUnionTube.EllUnionTube}
\begin{verbatim}
EllUionTube - class which keeps ellipsoidal tubes by the instant of
              time

Fields:
  QArray:cell[1, nElem] - Array of ellipsoid matrices
  aMat:cell[1, nElem] - Array of ellipsoid centers
  scaleFactor:double[1, 1] - Tube scale factor
  MArray:cell[1, nElem] - Array of regularization ellipsoid matrices
  dim :double[1, 1] - Dimensionality
  sTime:double[1, 1] - Time s
  approxSchemaName:cell[1,] - Name
  approxSchemaDescr:cell[1,] - Description
  approxType:gras.ellapx.enums.EApproxType - Type of approximation
                (external, internal, not defined
  timeVec:cell[1, m] - Time vector
  calcPrecision:double[1, 1] - Calculation precision
  indSTime:double[1, 1]  - index of sTime within timeVec
  ltGoodDirMat:cell[1, nElem] - Good direction curve
  lsGoodDirVec:cell[1, nElem] - Good direction at time s
  ltGoodDirNormVec:cell[1, nElem] - Norm of good direction curve
  lsGoodDirNorm:double[1, 1] - Norm of good direction at time s
  xTouchCurveMat:cell[1, nElem] - Touch point curve for good
                                  direction
  xTouchOpCurveMat:cell[1, nElem] - Touch point curve for direction
                                    opposite to good direction
  xsTouchVec:cell[1, nElem]  - Touch point at time s
  xsTouchOpVec :cell[1, nElem] - Touch point at time s
  ellUnionTimeDirection:gras.ellapx.enums.EEllUnionTimeDirection -
                     Direction in time along which union is performed
  isLsTouch:logical[1, 1] - Indicates whether a touch takes place
                            along LS
  isLsTouchOp:logical[1, 1] - Indicates whether a touch takes place
                              along LS opposite
  isLtTouchVec:cell[1, nElem] - Indicates whether a touch takes place
                                along LT
  isLtTouchOpVec:cell[1, nElem] - Indicates whether a touch takes
                                  place along LT opposite
  timeTouchEndVec:cell[1, nElem] - Touch point curve for good
                                   direction
  timeTouchOpEndVec:cell[1, nElem] - Touch point curve for good
                                     direction

TODO: correct description of the fields in
    gras.ellapx.smartdb.rels.EllUnionTube
\end{verbatim}


See the description of the following methods in section \ref{secClassDescr:smartdb.relations.ATypifiedStaticRelation}
 for smartdb.relations.ATypifiedStaticRelation:

\begin{list}{}{}
 \item \hyperref[method:smartdb.relations.ATypifiedStaticRelation.addData]{addData}
 \item \hyperref[method:smartdb.relations.ATypifiedStaticRelation.addDataAlongDim]{addDataAlongDim}
 \item \hyperref[method:smartdb.relations.ATypifiedStaticRelation.addTuples]{addTuples}
 \item \hyperref[method:smartdb.relations.ATypifiedStaticRelation.applyGetFunc]{applyGetFunc}
 \item \hyperref[method:smartdb.relations.ATypifiedStaticRelation.applySetFunc]{applySetFunc}
 \item \hyperref[method:smartdb.relations.ATypifiedStaticRelation.applyTupleGetFunc]{applyTupleGetFunc}
 \item \hyperref[method:smartdb.relations.ATypifiedStaticRelation.clearData]{clearData}
 \item \hyperref[method:smartdb.relations.ATypifiedStaticRelation.clone]{clone}
 \item \hyperref[method:smartdb.relations.ATypifiedStaticRelation.copyFrom]{copyFrom}
 \item \hyperref[method:smartdb.relations.ATypifiedStaticRelation.createInstance]{createInstance}
 \item \hyperref[method:smartdb.relations.ATypifiedStaticRelation.dispOnUI]{dispOnUI}
 \item \hyperref[method:smartdb.relations.ATypifiedStaticRelation.display]{display}
 \item \hyperref[method:smartdb.relations.ATypifiedStaticRelation.fromStructList]{fromStructList}
 \item \hyperref[method:smartdb.relations.ATypifiedStaticRelation.getCopy]{getCopy}
 \item \hyperref[method:smartdb.relations.ATypifiedStaticRelation.getFieldDescrList]{getFieldDescrList}
 \item \hyperref[method:smartdb.relations.ATypifiedStaticRelation.getFieldIsNull]{getFieldIsNull}
 \item \hyperref[method:smartdb.relations.ATypifiedStaticRelation.getFieldIsValueNull]{getFieldIsValueNull}
 \item \hyperref[method:smartdb.relations.ATypifiedStaticRelation.getFieldNameList]{getFieldNameList}
 \item \hyperref[method:smartdb.relations.ATypifiedStaticRelation.getFieldProjection]{getFieldProjection}
 \item \hyperref[method:smartdb.relations.ATypifiedStaticRelation.getFieldTypeList]{getFieldTypeList}
 \item \hyperref[method:smartdb.relations.ATypifiedStaticRelation.getFieldTypeSpecList]{getFieldTypeSpecList}
 \item \hyperref[method:smartdb.relations.ATypifiedStaticRelation.getFieldValueSizeMat]{getFieldValueSizeMat}
 \item \hyperref[method:smartdb.relations.ATypifiedStaticRelation.getIsFieldValueNull]{getIsFieldValueNull}
 \item \hyperref[method:smartdb.relations.ATypifiedStaticRelation.getMinDimensionSize]{getMinDimensionSize}
 \item \hyperref[method:smartdb.relations.ATypifiedStaticRelation.getMinDimensionality]{getMinDimensionality}
 \item \hyperref[method:smartdb.relations.ATypifiedStaticRelation.getNElems]{getNElems}
 \item \hyperref[method:smartdb.relations.ATypifiedStaticRelation.getNFields]{getNFields}
 \item \hyperref[method:smartdb.relations.ATypifiedStaticRelation.getNTuples]{getNTuples}
 \item \hyperref[method:smartdb.relations.ATypifiedStaticRelation.getSortIndex]{getSortIndex}
 \item \hyperref[method:smartdb.relations.ATypifiedStaticRelation.getTuples]{getTuples}
 \item \hyperref[method:smartdb.relations.ATypifiedStaticRelation.getTuplesFilteredBy]{getTuplesFilteredBy}
 \item \hyperref[method:smartdb.relations.ATypifiedStaticRelation.getTuplesIndexedBy]{getTuplesIndexedBy}
 \item \hyperref[method:smartdb.relations.ATypifiedStaticRelation.getTuplesJoinedWith]{getTuplesJoinedWith}
 \item \hyperref[method:smartdb.relations.ATypifiedStaticRelation.getUniqueData]{getUniqueData}
 \item \hyperref[method:smartdb.relations.ATypifiedStaticRelation.getUniqueDataAlongDim]{getUniqueDataAlongDim}
 \item \hyperref[method:smartdb.relations.ATypifiedStaticRelation.getUniqueTuples]{getUniqueTuples}
 \item \hyperref[method:smartdb.relations.ATypifiedStaticRelation.initByEmptyDataSet]{initByEmptyDataSet}
 \item \hyperref[method:smartdb.relations.ATypifiedStaticRelation.initByNullDataSet]{initByNullDataSet}
 \item \hyperref[method:smartdb.relations.ATypifiedStaticRelation.isFields]{isFields}
 \item \hyperref[method:smartdb.relations.ATypifiedStaticRelation.isMemberAlongDim]{isMemberAlongDim}
 \item \hyperref[method:smartdb.relations.ATypifiedStaticRelation.isMemberTuples]{isMemberTuples}
 \item \hyperref[method:smartdb.relations.ATypifiedStaticRelation.isUniqueKey]{isUniqueKey}
 \item \hyperref[method:smartdb.relations.ATypifiedStaticRelation.isequal]{isequal}
 \item \hyperref[method:smartdb.relations.ATypifiedStaticRelation.removeDuplicateTuples]{removeDuplicateTuples}
 \item \hyperref[method:smartdb.relations.ATypifiedStaticRelation.removeTuples]{removeTuples}
 \item \hyperref[method:smartdb.relations.ATypifiedStaticRelation.reorderData]{reorderData}
 \item \hyperref[method:smartdb.relations.ATypifiedStaticRelation.saveObj]{saveObj}
 \item \hyperref[method:smartdb.relations.ATypifiedStaticRelation.setData]{setData}
 \item \hyperref[method:smartdb.relations.ATypifiedStaticRelation.setField]{setField}
 \item \hyperref[method:smartdb.relations.ATypifiedStaticRelation.sortBy]{sortBy}
 \item \hyperref[method:smartdb.relations.ATypifiedStaticRelation.sortByAlongDim]{sortByAlongDim}
 \item \hyperref[method:smartdb.relations.ATypifiedStaticRelation.toArray]{toArray}
 \item \hyperref[method:smartdb.relations.ATypifiedStaticRelation.toCell]{toCell}
 \item \hyperref[method:smartdb.relations.ATypifiedStaticRelation.toCellIsNull]{toCellIsNull}
 \item \hyperref[method:smartdb.relations.ATypifiedStaticRelation.toDispCell]{toDispCell}
 \item \hyperref[method:smartdb.relations.ATypifiedStaticRelation.toMat]{toMat}
 \item \hyperref[method:smartdb.relations.ATypifiedStaticRelation.toStruct]{toStruct}
 \item \hyperref[method:smartdb.relations.ATypifiedStaticRelation.unionWith]{unionWith}
 \item \hyperref[method:smartdb.relations.ATypifiedStaticRelation.unionWithAlongDim]{unionWithAlongDim}
 \item \hyperref[method:smartdb.relations.ATypifiedStaticRelation.writeToCSV]{writeToCSV}
 \item \hyperref[method:smartdb.relations.ATypifiedStaticRelation.writeToXLS]{writeToXLS}
\end{list}
\subsection{\texorpdfstring{gras.ellapx.smartdb.rels.EllUnionTube.cut}{cut}}\label{method:gras.ellapx.smartdb.rels.EllUnionTube.cut}
\begin{verbatim}

\end{verbatim}
\subsection{\texorpdfstring{gras.ellapx.smartdb.rels.EllUnionTube.fromEllTubes}{fromEllTubes}}\label{method:gras.ellapx.smartdb.rels.EllUnionTube.fromEllTubes}
\begin{verbatim}
FROMELLTUBES - returns union of the ellipsoidal tubes on time

Input:
   ellTubeRel: smartdb.relation.StaticRelation[1, 1]/
      smartdb.relation.DynamicRelation[1, 1] - relation
      object

Output:
ellUnionTubeRel: ellapx.smartdb.rel.EllUnionTube - union of the
            ellipsoidal tubes
\end{verbatim}
\subsection{\texorpdfstring{gras.ellapx.smartdb.rels.EllUnionTube.getData}{getData}}\label{method:gras.ellapx.smartdb.rels.EllUnionTube.getData}
\begin{verbatim}
GETDATA - returns an indexed projection of CubeStruct object's content

Input:
  regular:
      self: CubeStruct [1,1] - the object

  optional:

      subIndCVec:
        Case#1: numeric[1,]/numeric[,1]

        Case#2: cell[1,nDims]/cell[nDims,1] of double [nSubElem_i,1]
              for i=1,...,nDims

          -array of indices of field value slices that are selected
          to be returned; if not given (default),
          no indexation is performed

        Note!: numeric components of subIndVec are allowed to contain
           zeros which are be treated as they were references to null
           data slices

      dimVec: numeric[1,nDims]/numeric[nDims,1] - vector of dimension
          numbers corresponding to subIndCVec

  properties:

      fieldNameList: char[1,]/cell[1,nFields] of char[1,]
          list of field names to return

      structNameList: char[1,]/cell[1,nStructs] of char[1,]
          list of internal structures to return (by default it
          is {SData, SIsNull, SIsValueNull}

      replaceNull: logical[1,1] if true, null values are replaced with
          certain default values uniformly across all the cells,
              default value is false

      nullReplacements: cell[1,nReplacedFields]  - list of null
          replacements for each of the fields

      nullReplacementFields: cell[1,nReplacedFields] - list of fields in
         which the nulls are to be replaced with the specified values,
         if not specified it is assumed that all fields are to be
         replaced

         NOTE!: all fields not listed in this parameter are replaced with
         the default values

      checkInputs: logical[1,1] - true by default (input arguments are
         checked for correctness

Output:
  regular:
    SData: struct [1,1] - structure containing values of
        fields at the selected slices, each field is an array
        containing values of the corresponding type

    SIsNull: struct [1,1] - structure containing a nested
        array with is-null indicators for each CubeStruct cell content

    SIsValueNull: struct [1,1] - structure containing a
       logical array [] for each of the fields (true
       means that a corresponding cell doesn't not contain
          any value
\end{verbatim}
\subsection{\texorpdfstring{gras.ellapx.smartdb.rels.EllUnionTube.getEllArray}{getEllArray}}\label{method:gras.ellapx.smartdb.rels.EllUnionTube.getEllArray}
\begin{verbatim}
GETELLARRAY - returns array of matrix's ellipsoid according to
              approxType

Input:
 regular:
    self.
    approxType:char[1,] - type of approximation(internal/external)

Output:
  apprEllMat:double[nDim1,..., nDimN] - array of array of ellipsoid's
           matrices
\end{verbatim}
\subsection{\texorpdfstring{gras.ellapx.smartdb.rels.EllUnionTube.getJoinWith}{getJoinWith}}\label{method:gras.ellapx.smartdb.rels.EllUnionTube.getJoinWith}
\begin{verbatim}
GETJOINWITH - returns a result of INNER join of given relation with
              another relation by the specified key fields

LIMITATION: key fields by which the join is peformed are required to form
a unique key in the given relation

Input:
  regular:
      self:
      otherRel: smartdb.relations.ARelation[1,1]
      keyFieldNameList: char[1,]/cell[1,nFields] of char[1,]

  properties:
      joinType: char[1,] - type of join, can be
          'inner' (DEFAULT)
          'leftOuter'

Output:
  resRel: smartdb.relations.ARelation[1,1] - join result
\end{verbatim}
\subsection{\texorpdfstring{gras.ellapx.smartdb.rels.EllUnionTube.getNoCatOrCutFieldsList}{getNoCatOrCutFieldsList}}\label{method:gras.ellapx.smartdb.rels.EllUnionTube.getNoCatOrCutFieldsList}
\begin{verbatim}

\end{verbatim}
\subsection{\texorpdfstring{gras.ellapx.smartdb.rels.EllUnionTube.interp}{interp}}\label{method:gras.ellapx.smartdb.rels.EllUnionTube.interp}
\begin{verbatim}

\end{verbatim}
\subsection{\texorpdfstring{gras.ellapx.smartdb.rels.EllUnionTube.isEqual}{isEqual}}\label{method:gras.ellapx.smartdb.rels.EllUnionTube.isEqual}
\begin{verbatim}
ISEQUAL - compares current relation object with other relation object and
          returns true if they are equal, otherwise it returns false


Usage: isEq=isEqual(self,otherObj)

Input:
  regular:
    self: ARelation [1,1] - current relation object
    otherObj: ARelation [1,1] - other relation object

  properties:
    checkFieldOrder/isFieldOrderCheck: logical [1,1] - if true, then fields
        in compared relations must be in the same order, otherwise the
        order is not  important (false by default)
    checkTupleOrder: logical[1,1] -  if true, then the tuples in the
        compared relations are expected to be in the same order,
        otherwise the order is not important (false by default)

    maxTolerance: double [1,1] - maximum allowed tolerance

    compareMetaDataBackwardRef: logical[1,1] if true, the CubeStruct's
        referenced from the meta data objects are also compared

    maxRelativeTolerance: double [1,1] - maximum allowed
    relative tolerance

Output:
  isEq: logical[1,1] - result of comparison
  reportStr: char[1,] - report of comparsion
\end{verbatim}
\subsection{\texorpdfstring{gras.ellapx.smartdb.rels.EllUnionTube.project}{project}}\label{method:gras.ellapx.smartdb.rels.EllUnionTube.project}
\begin{verbatim}
PROJECT - computes projection of the relation object onto given time
          dependent subspase
Input:
  regular:
      self.
      projType: gras.ellapx.enums.EProjType[1,1] -
          type of the projection, can be
          'Static' and 'DynamicAlongGoodCurve'
      projMatList: cell[1,nProj] of double[nSpDim,nDim] - list of
          projection matrices, not necessarily orthogonal
   fGetProjMat: function_handle[1,1] - function which creates
      vector of the projection
            matrices
       Input:
        regular:
          projMat:double[nDim, mDim] - matrix of the projection at the
            instant of time
          timeVec:double[1, nDim] - time interval
        optional:
           sTime:double[1,1] - instant of time
       Output:
          projOrthMatArray:double[1, nSpDim] - vector of the projection
            matrices
          projOrthMatTransArray:double[nSpDim, 1] - transposed vector of
            the projection matrices
Output:
   ellTubeProjRel: gras.ellapx.smartdb.rels.EllTubeProj[1, 1]/
       gras.ellapx.smartdb.rels.EllTubeUnionProj[1, 1] -
          projected ellipsoidal tube

   indProj2OrigVec:cell[nDim, 1] - index of the line number from
            which is obtained the projection

Example:
  function example
   aMat = [0 1; 0 0]; bMat = eye(2);
   SUBounds = struct();
   SUBounds.center = {'sin(t)'; 'cos(t)'};
   SUBounds.shape = [9 0; 0 2];
   sys = elltool.linsys.LinSysContinuous(aMat, bMat, SUBounds);
   x0EllObj = ell_unitball(2);
   timeVec = [0 10];
   dirsMat = [1 0; 0 1]';
   rsObj = elltool.reach.ReachContinuous(sys, x0EllObj, dirsMat, timeVec);
   ellTubeObj = rsObj.getEllTubeRel();
   unionEllTube = ...
    gras.ellapx.smartdb.rels.EllUnionTube.fromEllTubes(ellTubeObj);
   projMatList = {[1 0;0 1]};
   projType = gras.ellapx.enums.EProjType.Static;
   statEllTubeProj = unionEllTube.project(projType,projMatList,...
      @fGetProjMat);
   plObj=smartdb.disp.RelationDataPlotter();
   statEllTubeProj.plot(plObj);
end

function [projOrthMatArray,projOrthMatTransArray]=fGetProjMat(projMat,...
    timeVec,varargin)
  nTimePoints=length(timeVec);
  projOrthMatArray=repmat(projMat,[1,1,nTimePoints]);
  projOrthMatTransArray=repmat(projMat.',[1,1,nTimePoints]);
 end
\end{verbatim}
\subsection{\texorpdfstring{gras.ellapx.smartdb.rels.EllUnionTube.projectStatic}{projectStatic}}\label{method:gras.ellapx.smartdb.rels.EllUnionTube.projectStatic}
\begin{verbatim}

\end{verbatim}
\subsection{\texorpdfstring{gras.ellapx.smartdb.rels.EllUnionTube.sortDetermenistically}{sortDetermenistically}}\label{method:gras.ellapx.smartdb.rels.EllUnionTube.sortDetermenistically}
\begin{verbatim}

\end{verbatim}
\subsection{\texorpdfstring{gras.ellapx.smartdb.rels.EllUnionTube.thinOutTuples}{thinOutTuples}}\label{method:gras.ellapx.smartdb.rels.EllUnionTube.thinOutTuples}
\begin{verbatim}

\end{verbatim}
\section{gras.ellapx.smartdb.rels.EllUnionTubeStaticProj}\label{secClassDescr:gras.ellapx.smartdb.rels.EllUnionTubeStaticProj}
\subsection{\texorpdfstring{gras.ellapx.smartdb.rels.EllUnionTubeStaticProj.EllUnionTubeStaticProj}{EllUnionTubeStaticProj}}\label{method:gras.ellapx.smartdb.rels.EllUnionTubeStaticProj.EllUnionTubeStaticProj}
\begin{verbatim}
EllUnionTubeStaticProj - class which keeps projection on static plane
                         union of ellipsoid tubes

Fields:
  QArray:cell[1, nElem] - Array of ellipsoid matrices
  aMat:cell[1, nElem] - Array of ellipsoid centers
  scaleFactor:double[1, 1] - Tube scale factor
  MArray:cell[1, nElem] - Array of regularization ellipsoid matrices
  dim :double[1, 1] - Dimensionality
  sTime:double[1, 1] - Time s
  approxSchemaName:cell[1,] - Name
  approxSchemaDescr:cell[1,] - Description
  approxType:gras.ellapx.enums.EApproxType - Type of approximation
                (external, internal, not defined
  timeVec:cell[1, m] - Time vector
  calcPrecision:double[1, 1] - Calculation precision
  indSTime:double[1, 1]  - index of sTime within timeVec
  ltGoodDirMat:cell[1, nElem] - Good direction curve
  lsGoodDirVec:cell[1, nElem] - Good direction at time s
  ltGoodDirNormVec:cell[1, nElem] - Norm of good direction curve
  lsGoodDirNorm:double[1, 1] - Norm of good direction at time s
  xTouchCurveMat:cell[1, nElem] - Touch point curve for good
                                  direction
  xTouchOpCurveMat:cell[1, nElem] - Touch point curve for direction
                                    opposite to good direction
  xsTouchVec:cell[1, nElem]  - Touch point at time s
  xsTouchOpVec :cell[1, nElem] - Touch point at time s
  projSTimeMat: cell[1, 1] - Projection matrix at time s
  projType:gras.ellapx.enums.EProjType - Projection type
  ltGoodDirNormOrigVec:cell[1, 1] - Norm of the original (not
                                    projected) good direction curve
  lsGoodDirNormOrig:double[1, 1] - Norm of the original (not
                                   projected)good direction at time s
  lsGoodDirOrigVec:cell[1, 1] - Original (not projected) good
                                direction at time s
  ellUnionTimeDirection:gras.ellapx.enums.EEllUnionTimeDirection -
                     Direction in time along which union is performed
  isLsTouch:logical[1, 1] - Indicates whether a touch takes place
                            along LS
  isLsTouchOp:logical[1, 1] - Indicates whether a touch takes place
                              along LS opposite
  isLtTouchVec:cell[1, nElem] - Indicates whether a touch takes place
                                along LT
  isLtTouchOpVec:cell[1, nElem] - Indicates whether a touch takes
                                  place along LT opposite
  timeTouchEndVec:cell[1, nElem] - Touch point curve for good
                                   direction
  timeTouchOpEndVec:cell[1, nElem] - Touch point curve for good
                                     direction

  TODO: correct description of the fields in
    gras.ellapx.smartdb.rels.EllUnionTubeStaticProj
\end{verbatim}


See the description of the following methods in section \ref{secClassDescr:smartdb.relations.ATypifiedStaticRelation}
 for smartdb.relations.ATypifiedStaticRelation:

\begin{list}{}{}
 \item \hyperref[method:smartdb.relations.ATypifiedStaticRelation.addData]{addData}
 \item \hyperref[method:smartdb.relations.ATypifiedStaticRelation.addDataAlongDim]{addDataAlongDim}
 \item \hyperref[method:smartdb.relations.ATypifiedStaticRelation.addTuples]{addTuples}
 \item \hyperref[method:smartdb.relations.ATypifiedStaticRelation.applyGetFunc]{applyGetFunc}
 \item \hyperref[method:smartdb.relations.ATypifiedStaticRelation.applySetFunc]{applySetFunc}
 \item \hyperref[method:smartdb.relations.ATypifiedStaticRelation.applyTupleGetFunc]{applyTupleGetFunc}
 \item \hyperref[method:smartdb.relations.ATypifiedStaticRelation.clearData]{clearData}
 \item \hyperref[method:smartdb.relations.ATypifiedStaticRelation.clone]{clone}
 \item \hyperref[method:smartdb.relations.ATypifiedStaticRelation.copyFrom]{copyFrom}
 \item \hyperref[method:smartdb.relations.ATypifiedStaticRelation.createInstance]{createInstance}
 \item \hyperref[method:smartdb.relations.ATypifiedStaticRelation.dispOnUI]{dispOnUI}
 \item \hyperref[method:smartdb.relations.ATypifiedStaticRelation.display]{display}
 \item \hyperref[method:smartdb.relations.ATypifiedStaticRelation.fromStructList]{fromStructList}
 \item \hyperref[method:smartdb.relations.ATypifiedStaticRelation.getCopy]{getCopy}
 \item \hyperref[method:smartdb.relations.ATypifiedStaticRelation.getFieldDescrList]{getFieldDescrList}
 \item \hyperref[method:smartdb.relations.ATypifiedStaticRelation.getFieldIsNull]{getFieldIsNull}
 \item \hyperref[method:smartdb.relations.ATypifiedStaticRelation.getFieldIsValueNull]{getFieldIsValueNull}
 \item \hyperref[method:smartdb.relations.ATypifiedStaticRelation.getFieldNameList]{getFieldNameList}
 \item \hyperref[method:smartdb.relations.ATypifiedStaticRelation.getFieldProjection]{getFieldProjection}
 \item \hyperref[method:smartdb.relations.ATypifiedStaticRelation.getFieldTypeList]{getFieldTypeList}
 \item \hyperref[method:smartdb.relations.ATypifiedStaticRelation.getFieldTypeSpecList]{getFieldTypeSpecList}
 \item \hyperref[method:smartdb.relations.ATypifiedStaticRelation.getFieldValueSizeMat]{getFieldValueSizeMat}
 \item \hyperref[method:smartdb.relations.ATypifiedStaticRelation.getIsFieldValueNull]{getIsFieldValueNull}
 \item \hyperref[method:smartdb.relations.ATypifiedStaticRelation.getMinDimensionSize]{getMinDimensionSize}
 \item \hyperref[method:smartdb.relations.ATypifiedStaticRelation.getMinDimensionality]{getMinDimensionality}
 \item \hyperref[method:smartdb.relations.ATypifiedStaticRelation.getNElems]{getNElems}
 \item \hyperref[method:smartdb.relations.ATypifiedStaticRelation.getNFields]{getNFields}
 \item \hyperref[method:smartdb.relations.ATypifiedStaticRelation.getNTuples]{getNTuples}
 \item \hyperref[method:smartdb.relations.ATypifiedStaticRelation.getSortIndex]{getSortIndex}
 \item \hyperref[method:smartdb.relations.ATypifiedStaticRelation.getTuples]{getTuples}
 \item \hyperref[method:smartdb.relations.ATypifiedStaticRelation.getTuplesFilteredBy]{getTuplesFilteredBy}
 \item \hyperref[method:smartdb.relations.ATypifiedStaticRelation.getTuplesIndexedBy]{getTuplesIndexedBy}
 \item \hyperref[method:smartdb.relations.ATypifiedStaticRelation.getTuplesJoinedWith]{getTuplesJoinedWith}
 \item \hyperref[method:smartdb.relations.ATypifiedStaticRelation.getUniqueData]{getUniqueData}
 \item \hyperref[method:smartdb.relations.ATypifiedStaticRelation.getUniqueDataAlongDim]{getUniqueDataAlongDim}
 \item \hyperref[method:smartdb.relations.ATypifiedStaticRelation.getUniqueTuples]{getUniqueTuples}
 \item \hyperref[method:smartdb.relations.ATypifiedStaticRelation.initByEmptyDataSet]{initByEmptyDataSet}
 \item \hyperref[method:smartdb.relations.ATypifiedStaticRelation.initByNullDataSet]{initByNullDataSet}
 \item \hyperref[method:smartdb.relations.ATypifiedStaticRelation.isFields]{isFields}
 \item \hyperref[method:smartdb.relations.ATypifiedStaticRelation.isMemberAlongDim]{isMemberAlongDim}
 \item \hyperref[method:smartdb.relations.ATypifiedStaticRelation.isMemberTuples]{isMemberTuples}
 \item \hyperref[method:smartdb.relations.ATypifiedStaticRelation.isUniqueKey]{isUniqueKey}
 \item \hyperref[method:smartdb.relations.ATypifiedStaticRelation.isequal]{isequal}
 \item \hyperref[method:smartdb.relations.ATypifiedStaticRelation.removeDuplicateTuples]{removeDuplicateTuples}
 \item \hyperref[method:smartdb.relations.ATypifiedStaticRelation.removeTuples]{removeTuples}
 \item \hyperref[method:smartdb.relations.ATypifiedStaticRelation.reorderData]{reorderData}
 \item \hyperref[method:smartdb.relations.ATypifiedStaticRelation.saveObj]{saveObj}
 \item \hyperref[method:smartdb.relations.ATypifiedStaticRelation.setData]{setData}
 \item \hyperref[method:smartdb.relations.ATypifiedStaticRelation.setField]{setField}
 \item \hyperref[method:smartdb.relations.ATypifiedStaticRelation.sortBy]{sortBy}
 \item \hyperref[method:smartdb.relations.ATypifiedStaticRelation.sortByAlongDim]{sortByAlongDim}
 \item \hyperref[method:smartdb.relations.ATypifiedStaticRelation.toArray]{toArray}
 \item \hyperref[method:smartdb.relations.ATypifiedStaticRelation.toCell]{toCell}
 \item \hyperref[method:smartdb.relations.ATypifiedStaticRelation.toCellIsNull]{toCellIsNull}
 \item \hyperref[method:smartdb.relations.ATypifiedStaticRelation.toDispCell]{toDispCell}
 \item \hyperref[method:smartdb.relations.ATypifiedStaticRelation.toMat]{toMat}
 \item \hyperref[method:smartdb.relations.ATypifiedStaticRelation.toStruct]{toStruct}
 \item \hyperref[method:smartdb.relations.ATypifiedStaticRelation.unionWith]{unionWith}
 \item \hyperref[method:smartdb.relations.ATypifiedStaticRelation.unionWithAlongDim]{unionWithAlongDim}
 \item \hyperref[method:smartdb.relations.ATypifiedStaticRelation.writeToCSV]{writeToCSV}
 \item \hyperref[method:smartdb.relations.ATypifiedStaticRelation.writeToXLS]{writeToXLS}
\end{list}
\subsection{\texorpdfstring{gras.ellapx.smartdb.rels.EllUnionTubeStaticProj.cut}{cut}}\label{method:gras.ellapx.smartdb.rels.EllUnionTubeStaticProj.cut}
\begin{verbatim}

\end{verbatim}
\subsection{\texorpdfstring{gras.ellapx.smartdb.rels.EllUnionTubeStaticProj.fromEllTubes}{fromEllTubes}}\label{method:gras.ellapx.smartdb.rels.EllUnionTubeStaticProj.fromEllTubes}
\begin{verbatim}
FROMELLTUBES - returns union of the ellipsoidal tubes on time

Input:
   ellTubeRel: smartdb.relation.StaticRelation[1, 1]/
      smartdb.relation.DynamicRelation[1, 1] - relation
      object

Output:
ellUnionTubeRel: ellapx.smartdb.rel.EllUnionTube - union of the
            ellipsoidal tubes
\end{verbatim}
\subsection{\texorpdfstring{gras.ellapx.smartdb.rels.EllUnionTubeStaticProj.getData}{getData}}\label{method:gras.ellapx.smartdb.rels.EllUnionTubeStaticProj.getData}
\begin{verbatim}
GETDATA - returns an indexed projection of CubeStruct object's content

Input:
  regular:
      self: CubeStruct [1,1] - the object

  optional:

      subIndCVec:
        Case#1: numeric[1,]/numeric[,1]

        Case#2: cell[1,nDims]/cell[nDims,1] of double [nSubElem_i,1]
              for i=1,...,nDims

          -array of indices of field value slices that are selected
          to be returned; if not given (default),
          no indexation is performed

        Note!: numeric components of subIndVec are allowed to contain
           zeros which are be treated as they were references to null
           data slices

      dimVec: numeric[1,nDims]/numeric[nDims,1] - vector of dimension
          numbers corresponding to subIndCVec

  properties:

      fieldNameList: char[1,]/cell[1,nFields] of char[1,]
          list of field names to return

      structNameList: char[1,]/cell[1,nStructs] of char[1,]
          list of internal structures to return (by default it
          is {SData, SIsNull, SIsValueNull}

      replaceNull: logical[1,1] if true, null values are replaced with
          certain default values uniformly across all the cells,
              default value is false

      nullReplacements: cell[1,nReplacedFields]  - list of null
          replacements for each of the fields

      nullReplacementFields: cell[1,nReplacedFields] - list of fields in
         which the nulls are to be replaced with the specified values,
         if not specified it is assumed that all fields are to be
         replaced

         NOTE!: all fields not listed in this parameter are replaced with
         the default values

      checkInputs: logical[1,1] - true by default (input arguments are
         checked for correctness

Output:
  regular:
    SData: struct [1,1] - structure containing values of
        fields at the selected slices, each field is an array
        containing values of the corresponding type

    SIsNull: struct [1,1] - structure containing a nested
        array with is-null indicators for each CubeStruct cell content

    SIsValueNull: struct [1,1] - structure containing a
       logical array [] for each of the fields (true
       means that a corresponding cell doesn't not contain
          any value
\end{verbatim}
\subsection{\texorpdfstring{gras.ellapx.smartdb.rels.EllUnionTubeStaticProj.getEllArray}{getEllArray}}\label{method:gras.ellapx.smartdb.rels.EllUnionTubeStaticProj.getEllArray}
\begin{verbatim}
GETELLARRAY - returns array of matrix's ellipsoid according to
              approxType

Input:
 regular:
    self.
    approxType:char[1,] - type of approximation(internal/external)

Output:
  apprEllMat:double[nDim1,..., nDimN] - array of array of ellipsoid's
           matrices
\end{verbatim}
\subsection{\texorpdfstring{gras.ellapx.smartdb.rels.EllUnionTubeStaticProj.getJoinWith}{getJoinWith}}\label{method:gras.ellapx.smartdb.rels.EllUnionTubeStaticProj.getJoinWith}
\begin{verbatim}
GETJOINWITH - returns a result of INNER join of given relation with
              another relation by the specified key fields

LIMITATION: key fields by which the join is peformed are required to form
a unique key in the given relation

Input:
  regular:
      self:
      otherRel: smartdb.relations.ARelation[1,1]
      keyFieldNameList: char[1,]/cell[1,nFields] of char[1,]

  properties:
      joinType: char[1,] - type of join, can be
          'inner' (DEFAULT)
          'leftOuter'

Output:
  resRel: smartdb.relations.ARelation[1,1] - join result
\end{verbatim}
\subsection{\texorpdfstring{gras.ellapx.smartdb.rels.EllUnionTubeStaticProj.getNoCatOrCutFieldsList}{getNoCatOrCutFieldsList}}\label{method:gras.ellapx.smartdb.rels.EllUnionTubeStaticProj.getNoCatOrCutFieldsList}
\begin{verbatim}

\end{verbatim}
\subsection{\texorpdfstring{gras.ellapx.smartdb.rels.EllUnionTubeStaticProj.getReachTubeNamePrefix}{getReachTubeNamePrefix}}\label{method:gras.ellapx.smartdb.rels.EllUnionTubeStaticProj.getReachTubeNamePrefix}
\begin{verbatim}
GETREACHTUBEANEPREFIX - return prefix of the reach tube

Input:
  regular:
     self.
\end{verbatim}
\subsection{\texorpdfstring{gras.ellapx.smartdb.rels.EllUnionTubeStaticProj.getRegTubeNamePrefix}{getRegTubeNamePrefix}}\label{method:gras.ellapx.smartdb.rels.EllUnionTubeStaticProj.getRegTubeNamePrefix}
\begin{verbatim}
GETREGTUBEANEPREFIX - return prefix of the reg tube

Input:
  regular:
     self.
\end{verbatim}
\subsection{\texorpdfstring{gras.ellapx.smartdb.rels.EllUnionTubeStaticProj.interp}{interp}}\label{method:gras.ellapx.smartdb.rels.EllUnionTubeStaticProj.interp}
\begin{verbatim}

\end{verbatim}
\subsection{\texorpdfstring{gras.ellapx.smartdb.rels.EllUnionTubeStaticProj.isEqual}{isEqual}}\label{method:gras.ellapx.smartdb.rels.EllUnionTubeStaticProj.isEqual}
\begin{verbatim}
ISEQUAL - compares current relation object with other relation object and
          returns true if they are equal, otherwise it returns false


Usage: isEq=isEqual(self,otherObj)

Input:
  regular:
    self: ARelation [1,1] - current relation object
    otherObj: ARelation [1,1] - other relation object

  properties:
    checkFieldOrder/isFieldOrderCheck: logical [1,1] - if true, then fields
        in compared relations must be in the same order, otherwise the
        order is not  important (false by default)
    checkTupleOrder: logical[1,1] -  if true, then the tuples in the
        compared relations are expected to be in the same order,
        otherwise the order is not important (false by default)

    maxTolerance: double [1,1] - maximum allowed tolerance

    compareMetaDataBackwardRef: logical[1,1] if true, the CubeStruct's
        referenced from the meta data objects are also compared

    maxRelativeTolerance: double [1,1] - maximum allowed
    relative tolerance

Output:
  isEq: logical[1,1] - result of comparison
  reportStr: char[1,] - report of comparsion
\end{verbatim}
\subsection{\texorpdfstring{gras.ellapx.smartdb.rels.EllUnionTubeStaticProj.plot}{plot}}\label{method:gras.ellapx.smartdb.rels.EllUnionTubeStaticProj.plot}
\begin{verbatim}
PLOT - displays ellipsoidal tubes using the specified
  RelationDataPlotter

Input:
  regular:
      self:
  optional:
      plObj: smartdb.disp.RelationDataPlotter[1,1] - plotter
          object used for displaying ellipsoidal tubes
  properties:
      fGetColor: function_handle[1, 1] -
          function that specified colorVec for
          ellipsoidal tubes
      fGetAlpha: function_handle[1, 1] -
          function that specified transparency
          value for ellipsoidal tubes
      fGetLineWidth: function_handle[1, 1] -
          function that specified lineWidth for good curves
      fGetFill: function_handle[1, 1] - this
          property not used in this version
      colorFieldList: cell[nColorFields, ] of char[1, ] -
          list of parameters for color function
      alphaFieldList: cell[nAlphaFields, ] of char[1, ] -
          list of parameters for transparency function
      lineWidthFieldList: cell[nLineWidthFields, ]
          of char[1, ] - list of parameters for lineWidth
          function
      fillFieldList: cell[nIsFillFields, ] of char[1, ] -
          list of parameters for fill function
      plotSpecFieldList: cell[nPlotFields, ] of char[1, ] -
          defaul list of parameters. If for any function in
          properties not specified list of parameters,
          this one will be used

Output:
  plObj: smartdb.disp.RelationDataPlotter[1,1] - plotter
          object used for displaying ellipsoidal tubes
\end{verbatim}
\subsection{\texorpdfstring{gras.ellapx.smartdb.rels.EllUnionTubeStaticProj.plotExt}{plotExt}}\label{method:gras.ellapx.smartdb.rels.EllUnionTubeStaticProj.plotExt}
\begin{verbatim}
PLOTEXT - plots external approximation of ellTube.


Usage:
      obj.plotExt() - plots external approximation of ellTube.
      obj.plotExt('Property',PropValue,...) - plots external approximation
                                              of ellTube with setting
                                              properties.

Input:
  regular:
      obj:  EllTubeProj: EllTubeProj object
  optional:
      relDataPlotter:smartdb.disp.RelationDataPlotter[1,1] - relation data plotter object.
      colorSpec: char[1,1] - color specification code, can be 'r','g',
                   etc (any code supported by built-in Matlab function).

  properties:

      fGetColor: function_handle[1, 1] -
          function that specified colorVec for
          ellipsoidal tubes
      fGetAlpha: function_handle[1, 1] -
          function that specified transparency
          value for ellipsoidal tubes
      fGetLineWidth: function_handle[1, 1] -
          function that specified lineWidth for good curves
      fGetFill: function_handle[1, 1] - this
          property not used in this version
      colorFieldList: cell[nColorFields, ] of char[1, ] -
          list of parameters for color function
      alphaFieldList: cell[nAlphaFields, ] of char[1, ] -
          list of parameters for transparency function
      lineWidthFieldList: cell[nLineWidthFields, ]
          of char[1, ] - list of parameters for lineWidth
          function
      fillFieldList: cell[nIsFillFields, ] of char[1, ] -
          list of parameters for fill function
      plotSpecFieldList: cell[nPlotFields, ] of char[1, ] -
          defaul list of parameters. If for any function in
          properties not specified list of parameters,
          this one will be used
      'showDiscrete':logical[1,1]  -
          if true, approximation in 3D will be filled in every time slice
      'nSpacePartPoins': double[1,1] -
          number of points in every time slice.
Output:
  regular:
      plObj: smartdb.disp.RelationDataPlotter[1,1] - returns the relation
      data plotter object.
\end{verbatim}
\subsection{\texorpdfstring{gras.ellapx.smartdb.rels.EllUnionTubeStaticProj.plotInt}{plotInt}}\label{method:gras.ellapx.smartdb.rels.EllUnionTubeStaticProj.plotInt}
\begin{verbatim}
PLOTINT - plots internal approximation of ellTube.


Usage:
      obj.plotInt() - plots internal approximation of ellTube.
      obj.plotInt('Property',PropValue,...) - plots internal approximation
                                              of ellTube with setting
                                              properties.

Input:
  regular:
      obj:  EllTubeProj: EllTubeProj object
  optional:
      relDataPlotter:smartdb.disp.RelationDataPlotter[1,1] - relation data plotter object.
      colorSpec: char[1,1] - color specification code, can be 'r','g',
                   etc (any code supported by built-in Matlab function).

  properties:

      fGetColor: function_handle[1, 1] -
          function that specified colorVec for
          ellipsoidal tubes
      fGetAlpha: function_handle[1, 1] -
          function that specified transparency
          value for ellipsoidal tubes
      fGetLineWidth: function_handle[1, 1] -
          function that specified lineWidth for good curves
      fGetFill: function_handle[1, 1] - this
          property not used in this version
      colorFieldList: cell[nColorFields, ] of char[1, ] -
          list of parameters for color function
      alphaFieldList: cell[nAlphaFields, ] of char[1, ] -
          list of parameters for transparency function
      lineWidthFieldList: cell[nLineWidthFields, ]
          of char[1, ] - list of parameters for lineWidth
          function
      fillFieldList: cell[nIsFillFields, ] of char[1, ] -
          list of parameters for fill function
      plotSpecFieldList: cell[nPlotFields, ] of char[1, ] -
          defaul list of parameters. If for any function in
          properties not specified list of parameters,
          this one will be used
      'showDiscrete':logical[1,1]  -
          if true, approximation in 3D will be filled in every time slice
      'nSpacePartPoins': double[1,1] -
          number of points in every time slice.
Output:
  regular:
      plObj: smartdb.disp.RelationDataPlotter[1,1] - returns the relation
      data plotter object.
\end{verbatim}
\subsection{\texorpdfstring{gras.ellapx.smartdb.rels.EllUnionTubeStaticProj.projMat2str}{projMat2str}}\label{method:gras.ellapx.smartdb.rels.EllUnionTubeStaticProj.projMat2str}
\begin{verbatim}

\end{verbatim}
\subsection{\texorpdfstring{gras.ellapx.smartdb.rels.EllUnionTubeStaticProj.projRow2str}{projRow2str}}\label{method:gras.ellapx.smartdb.rels.EllUnionTubeStaticProj.projRow2str}
\begin{verbatim}

\end{verbatim}
\subsection{\texorpdfstring{gras.ellapx.smartdb.rels.EllUnionTubeStaticProj.sortDetermenistically}{sortDetermenistically}}\label{method:gras.ellapx.smartdb.rels.EllUnionTubeStaticProj.sortDetermenistically}
\begin{verbatim}

\end{verbatim}
\subsection{\texorpdfstring{gras.ellapx.smartdb.rels.EllUnionTubeStaticProj.thinOutTuples}{thinOutTuples}}\label{method:gras.ellapx.smartdb.rels.EllUnionTubeStaticProj.thinOutTuples}
\begin{verbatim}

\end{verbatim}
\section{elltool.reach.AReach}\label{secClassDescr:elltool.reach.AReach}
\subsection{\texorpdfstring{elltool.reach.AReach.AReach}{AReach}}\label{method:elltool.reach.AReach.AReach}
\begin{verbatim}

\end{verbatim}
\subsection{\texorpdfstring{elltool.reach.AReach.cut}{cut}}\label{method:elltool.reach.AReach.cut}
\begin{verbatim}
CUT - extracts the piece of reach tube from given start time to given
      end time. Given reach set self, find states that are reachable
      within time interval specified by cutTimeVec. If cutTimeVec
      is a scalar, then reach set at given time is returned.

Input:
  regular:
      self.

   cutTimeVec: double[1, 2]/double[1, 1] - time interval to cut.

Output:
  cutObj: elltool.reach.IReach[1, 1] - reach set resulting from the CUT
        operation.

Example:
  aMat = [0 1; 0 0]; bMat = eye(2);
  SUBounds = struct();
  SUBounds.center = {'sin(t)'; 'cos(t)'};
  SUBounds.shape = [9 0; 0 2];
  sys = elltool.linsys.LinSysContinuous(aMat, bMat, SUBounds);
  x0EllObj = ell_unitball(2);
  timeVec = [0 10];
  dirsMat = [1 0; 0 1]';
  rsObj = elltool.reach.ReachContinuous(sys, x0EllObj, dirsMat, timeVec);
  cutObj = rsObj.cut([3 5]);
  dRsObj = elltool.reach.ReachDiscrete(dtsys, x0EllObj, dirsMat, timeVec);
  dCutObj = dRsObj.cut([3 5]);
\end{verbatim}
\subsection{\texorpdfstring{elltool.reach.AReach.dimension}{dimension}}\label{method:elltool.reach.AReach.dimension}
\begin{verbatim}

DIMENSION - returns array of dimensions of given reach set array.

Input:
  regular:
      self - multidimensional array of
             ReachContinuous/ReachDiscrete objects

Output:
  rSdimArr: double[nDim1, nDim2,...] - array of reach set dimensions.
  sSdimArr: double[nDim1, nDim2,...] - array of state space dimensions.

Example:
  aMat = [0 1; 0 0]; bMat = eye(2);
  SUBounds = struct();
  SUBounds.center = {'sin(t)'; 'cos(t)'};
  SUBounds.shape = [9 0; 0 2];
  sys = elltool.linsys.LinSysContinuous(aMat, bMat, SUBounds);
  x0EllObj = ell_unitball(2);
  timeVec = [0 10];
  dirsMat = [1 0; 0 1]';
  rsObj = elltool.reach.ReachContinuous(sys, x0EllObj, dirsMat, timeVec);
  rsObjArr = rsObj.repMat(1,2);
  [rSdim sSdim] = rsObj.dimension()

  rSdim =

           2


  sSdim =

           2

  [rSdim sSdim] = rsObjArr.dimension()

  rSdim =
          [ 2  2 ]

  sSdim =
          [ 2  2 ]
\end{verbatim}
\subsection{\texorpdfstring{elltool.reach.AReach.display}{display}}\label{method:elltool.reach.AReach.display}
\begin{verbatim}

DISPLAY - displays the reach set object.

Input:
  regular:
      self.

Output:
  None.

Example:
  aMat = [0 1; 0 0]; bMat = eye(2);
  SUBounds = struct();
  SUBounds.center = {'sin(t)'; 'cos(t)'};
  SUBounds.shape = [9 0; 0 2];
  sys = elltool.linsys.LinSysContinuous(aMat, bMat, SUBounds);
  x0EllObj = ell_unitball(2);
  timeVec = [0 10];
  dirsMat = [1 0; 0 1]';
  rsObj = elltool.reach.ReachContinuous(sys, x0EllObj, dirsMat, timeVec);
  rsObj.display()

  rsObj =
  Reach set of the continuous-time linear system in R^2 in the time...
       interval [0, 10].

  Initial set at time t0 = 0:
  Ellipsoid with parameters
  Center:
       0
       0

  Shape Matrix:
       1     0
       0     1

  Number of external approximations: 2
  Number of internal approximations: 2
\end{verbatim}
\subsection{\texorpdfstring{elltool.reach.AReach.evolve}{evolve}}\label{method:elltool.reach.AReach.evolve}
\begin{verbatim}

EVOLVE - computes further evolution in time of the
         already existing reach set.

Input:
  regular:
      self.

      newEndTime: double[1, 1] - new end time.

  optional:
      linSys: elltool.linsys.LinSys[1, 1] - new linear system.

Output:
  newReachObj: reach[1, 1] - reach set on time  interval
        [oldT0 newEndTime].

Example:
  aMat = [0 1; 0 0]; bMat = eye(2);
  SUBounds = struct();
  SUBounds.center = {'sin(t)'; 'cos(t)'};
  SUBounds.shape = [9 0; 0 2];
  sys = elltool.linsys.LinSysContinuous(aMat, bMat, SUBounds);
  dsys = elltool.linsys.LinSysDiscrete(aMat, bMat, SUBounds);
  x0EllObj = ell_unitball(2);
  timeVec = [0 10];
  dirsMat = [1 0; 0 1]';
  rsObj = elltool.reach.ReachContinuous(sys, x0EllObj, dirsMat, timeVec);
  dRsObj = elltool.reach.ReachDiscrete(dsys, x0EllObj, dirsMat, timeVec);
  newRsObj = rsObj.evolve(12);
  newDRsObj = dRsObj.evolve(11);
\end{verbatim}
\subsection{\texorpdfstring{elltool.reach.AReach.getAbsTol}{getAbsTol}}\label{method:elltool.reach.AReach.getAbsTol}
\begin{verbatim}
GETABSTOL - gives the array of absTol for all elements
  in rsArr

Input:
  regular:
      rsArr: elltool.reach.AReach[nDim1, nDim2, ...] -
          multidimension array of reach sets
  optional:
      fAbsTolFun: function_handle[1,1] - function that is
          applied to the absTolArr. The default is @min.

Output:
  regular:
      absTolArr: double [absTol1, absTol2, ...] - return
          absTol for each element in rsArr
  optional:
      absTol: double[1,1] - return result of work fAbsTolFun
          with the absTolArr

Usage:
  use [~,absTol] = rsArr.getAbsTol() if you want get only
      absTol,
  use [absTolArr,absTol] = rsArr.getAbsTol() if you want
      get absTolArr and absTol,
  use absTolArr = rsArr.getAbsTol() if you want get only
      absTolArr
\end{verbatim}
\subsection{\texorpdfstring{elltool.reach.AReach.getCopy}{getCopy}}\label{method:elltool.reach.AReach.getCopy}
\begin{verbatim}
Input:
  regular:
      self:
  properties:
      l0Mat: double[nDims,nDirs] - matrix of good
          directions at time s
      isIntExtApxVec: logical[1,2] - two element vector with the
         first element corresponding to internal approximations
        and second - to external ones. An element equal to
         false means that the corresponding approximation type
         is filtered out. Default value is [true,true]
Example:
    aMat = [0 1; 0 0]; bMat = eye(2);
    SUBounds = struct();
    SUBounds.center = {'sin(t)'; 'cos(t)'};
    SUBounds.shape = [9 0; 0 2];
    sys = elltool.linsys.LinSysContinuous(aMat, bMat, SUBounds);
    x0EllObj = ell_unitball(2);
    timeVec = [0 10];
    dirsMat = [1 0; 0 1; 1 1;1 2]';
    rsObj = elltool.reach.ReachContinuous(sys, x0EllObj,...
      dirsMat, timeVec);

    copyRsObj = rsObj.getCopy()

    Reach set of the continuous-time linear system in R^2 in
      the time interval [0, 10].

    Initial set at time k0 = 0:
    Ellipsoid with parameters
    Center:
         0
         0

    Shape Matrix:
         1     0
         0     1

    Number of external approximations: 4
    Number of internal approximations: 4

    copyRsObj = rsObj.getCopy('l0Mat',[0;1],'approxType',...
      [true,false])

    Reach set of the continuous-time linear system in R^2 in
      the time interval [0, 10].

    Initial set at time k0 = 0:
    Ellipsoid with parameters
    Center:
         0
         0

    Shape Matrix:
         1     0
         0     1

    Number of external approximations: 1
    Number of internal approximations: 1
\end{verbatim}
\subsection{\texorpdfstring{elltool.reach.AReach.getEaScaleFactor}{getEaScaleFactor}}\label{method:elltool.reach.AReach.getEaScaleFactor}
\begin{verbatim}
GET_EASCALEFACTOR - return the scale factor for external approximation
                    of reach tube

Input:
  regular:
      self.

Output:
  regular:
      eaScaleFactor: double[1, 1] - scale factor.

Example:
  aMat = [0 1; 0 0]; bMat = eye(2);
  SUBounds = struct();
  SUBounds.center = {'sin(t)'; 'cos(t)'};
  SUBounds.shape = [9 0; 0 2];
  sys = elltool.linsys.LinSysContinuous(aMat, bMat, SUBounds);
  x0EllObj = ell_unitball(2);
  timeVec = [10 0];
  dirsMat = [1 0; 0 1]';
  rsObj = elltool.reach.ReachContinuous(sys, x0EllObj, dirsMat, timeVec);
  rsObj.getEaScaleFactor()

  ans =

      1.0200
\end{verbatim}
\subsection{\texorpdfstring{elltool.reach.AReach.getEllTubeRel}{getEllTubeRel}}\label{method:elltool.reach.AReach.getEllTubeRel}
\begin{verbatim}
Example:
  aMat = [0 1; 0 0]; bMat = eye(2);
  SUBounds = struct();
  SUBounds.center = {'sin(t)'; 'cos(t)'};
  SUBounds.shape = [9 0; 0 2];
  sys = elltool.linsys.LinSysContinuous(aMat, bMat, SUBounds);
  x0EllObj = ell_unitball(2);
  timeVec = [0 10];
  dirsMat = [1 0; 0 1]';
  rsObj = elltool.reach.ReachContinuous(sys, x0EllObj, dirsMat, timeVec);
  rsObj.getEllTubeRel();
\end{verbatim}
\subsection{\texorpdfstring{elltool.reach.AReach.getEllTubeUnionRel}{getEllTubeUnionRel}}\label{method:elltool.reach.AReach.getEllTubeUnionRel}
\begin{verbatim}
Example:
  aMat = [0 1; 0 0]; bMat = eye(2);
  SUBounds = struct();
  SUBounds.center = {'sin(t)'; 'cos(t)'};
  SUBounds.shape = [9 0; 0 2];
  sys = elltool.linsys.LinSysContinuous(aMat, bMat, SUBounds);
  x0EllObj = ell_unitball(2);
  timeVec = [0 10];
  dirsMat = [1 0; 0 1]';
  rsObj = elltool.reach.ReachContinuous(sys, x0EllObj, dirsMat, timeVec);
  getEllTubeUnionRel(rsObj);
\end{verbatim}
\subsection{\texorpdfstring{elltool.reach.AReach.getIaScaleFactor}{getIaScaleFactor}}\label{method:elltool.reach.AReach.getIaScaleFactor}
\begin{verbatim}
GET_IASCALEFACTOR - return the scale factor for internal approximation
                    of reach tube

Input:
  regular:
      self.

Output:
  regular:
      iaScaleFactor: double[1, 1] - scale factor.

Example:
  aMat = [0 1; 0 0]; bMat = eye(2);
  SUBounds = struct();
  SUBounds.center = {'sin(t)'; 'cos(t)'};
  SUBounds.shape = [9 0; 0 2];
  sys = elltool.linsys.LinSysContinuous(aMat, bMat, SUBounds);
  x0EllObj = ell_unitball(2);
  timeVec = [10 0];
  dirsMat = [1 0; 0 1]';
  rsObj = elltool.reach.ReachContinuous(sys, x0EllObj, dirsMat, timeVec);
  rsObj.getIaScaleFactor()

  ans =

      1.0200
\end{verbatim}
\subsection{\texorpdfstring{elltool.reach.AReach.getInitialSet}{getInitialSet}}\label{method:elltool.reach.AReach.getInitialSet}
\begin{verbatim}
GETINITIALSET - return the initial set for linear system, which is solved
                for building reach tube.

Input:
  regular:
      self.

Output:
  regular:
      x0Ell: ellipsoid[1, 1] - ellipsoid x0, which was initial set for
          linear system.

Example:
  aMat = [0 1; 0 0]; bMat = eye(2);
  SUBounds = struct();
  SUBounds.center = {'sin(t)'; 'cos(t)'};
  SUBounds.shape = [9 0; 0 2];
  sys = elltool.linsys.LinSysContinuous(aMat, bMat, SUBounds);
  x0EllObj = ell_unitball(2);
  timeVec = [10 0];
  dirsMat = [1 0; 0 1]';
  rsObj = elltool.reach.ReachContinuous(sys, x0EllObj, dirsMat, timeVec);
  x0Ell = rsObj.getInitialSet()

  x0Ell =

  Center:
       0
       0

  Shape Matrix:
       1     0
       0     1

  Nondegenerate ellipsoid in R^2.
\end{verbatim}
\subsection{\texorpdfstring{elltool.reach.AReach.getNPlot2dPoints}{getNPlot2dPoints}}\label{method:elltool.reach.AReach.getNPlot2dPoints}
\begin{verbatim}
GETNPLOT2DPOINTS - gives array  the same size as rsArr of
  value of nPlot2dPoints property for each element in rsArr -
  array of reach sets

Input:
  regular:
    rsArr:elltool.reach.AReach[nDims1,nDims2,...] - reach
      set array

Output:
  nPlot2dPointsArr:double[nDims1,nDims2,...] - array of
      values of nTimeGridPoints property for each reach set
      in rsArr
\end{verbatim}
\subsection{\texorpdfstring{elltool.reach.AReach.getNPlot3dPoints}{getNPlot3dPoints}}\label{method:elltool.reach.AReach.getNPlot3dPoints}
\begin{verbatim}
GETNPLOT3DPOINTS - gives array  the same size as rsArr of
  value of nPlot3dPoints property for each element in rsArr
  array of reach sets

Input:
  regular:
      rsArr:reach[nDims1,nDims2,...] - reach set array

Output:
  nPlot3dPointsArr:double[nDims1,nDims2,...]- array of values
      of nPlot3dPoints property for each reach set in rsArr
\end{verbatim}
\subsection{\texorpdfstring{elltool.reach.AReach.getNTimeGridPoints}{getNTimeGridPoints}}\label{method:elltool.reach.AReach.getNTimeGridPoints}
\begin{verbatim}
GETNTIMEGRIDPOINTS - gives array  the same size as rsArr of
  value of nTimeGridPoints property for each element in rsArr
  array of reach sets

Input:
  regular:
      rsArr: elltool.reach.AReach [nDims1,nDims2,...] - reach
          set array

Output:
  nTimeGridPointsArr: double[nDims1,nDims2,...]- array of
      values of nTimeGridPoints property for each reach set
      in rsArr
\end{verbatim}
\subsection{\texorpdfstring{elltool.reach.AReach.getRelTol}{getRelTol}}\label{method:elltool.reach.AReach.getRelTol}
\begin{verbatim}
GETRELTOL - gives the array of relTol for all elements in
ellArr

Input:
  regular:
      rsArr: elltool.reach.AReach[nDim1,nDim2, ...] -
          multidimension array of reach sets.
  optional
      fRelTolFun: function_handle[1,1] - function that is
          applied to the relTolArr. The default is @min.

Output:
  regular:
      relTolArr: double [relTol1, relTol2, ...] - return
          relTol for each element in rsArr.
  optional:
      relTol: double[1,1] - return result of work fRelTolFun
          with the relTolArr

Usage:
  use [~,relTol] = rsArr.getRelTol() if you want get only
      relTol,
  use [relTolArr,relTol] = rsArr.getRelTol() if you want get
      relTolArr and relTol,
  use relTolArr = rsArr.getRelTol() if you want get only
      relTolArr
\end{verbatim}
\subsection{\texorpdfstring{elltool.reach.AReach.getSwitchTimeVec}{getSwitchTimeVec}}\label{method:elltool.reach.AReach.getSwitchTimeVec}
\begin{verbatim}

\end{verbatim}
\subsection{\texorpdfstring{elltool.reach.AReach.get\_center}{get\_center}}\label{method:elltool.reach.AReach.getcenter}
\begin{verbatim}

GET_CENTER - returns the trajectory of the center of the reach set.

Input:
  regular:
      self.

Output:
  trCenterMat: double[nDim, nPoints] - array of points that form the
      trajectory of the reach set center, where nDim is reach set
      dimentsion, nPoints - number of points in time grid.

  timeVec: double[1, nPoints] - array of time values.

Example:
  aMat = [0 1; 0 0]; bMat = eye(2);
  SUBounds = struct();
  SUBounds.center = {'sin(t)'; 'cos(t)'};
  SUBounds.shape = [9 0; 0 2];
  sys = elltool.linsys.LinSysContinuous(aMat, bMat, SUBounds);
  x0EllObj = ell_unitball(2);
  timeVec = [0 10];
  dirsMat = [1 0; 0 1]';
  rsObj = elltool.reach.ReachContinuous(sys, x0EllObj, dirsMat, timeVec);
  [trCenterMat timeVec] = rsObj.get_center();
\end{verbatim}
\subsection{\texorpdfstring{elltool.reach.AReach.get\_directions}{get\_directions}}\label{method:elltool.reach.AReach.getdirections}
\begin{verbatim}

GET_DIRECTIONS - returns the values of direction vectors for time grid
                 values.

Input:
  regular:
      self.

Output:
  directionsCVec: cell[1, nPoints] of double [nDim, nDir] - array of
      cells, where each cell is a sequence of direction vector values
      that correspond to the time values of the grid, where nPoints is
      number of points in time grid.

  timeVec: double[1, nPoints] - array of time values.

Example:
  aMat = [0 1; 0 0]; bMat = eye(2);
  SUBounds = struct();
  SUBounds.center = {'sin(t)'; 'cos(t)'};
  SUBounds.shape = [9 0; 0 2];
  sys = elltool.linsys.LinSysContinuous(aMat, bMat, SUBounds);
  x0EllObj = ell_unitball(2);
  timeVec = [0 10];
  dirsMat = [1 0; 0 1]';
  rsObj = elltool.reach.ReachContinuous(sys, x0EllObj, dirsMat, timeVec);
  [directionsCVec timeVec] = rsObj.get_directions();
\end{verbatim}
\subsection{\texorpdfstring{elltool.reach.AReach.get\_ea}{get\_ea}}\label{method:elltool.reach.AReach.getea}
\begin{verbatim}

GET_EA - returns array of ellipsoid objects representing external
         approximation of the reach  tube.

Input:
  regular:
      self.

Output:
  eaEllMat: ellipsoid[nAppr, nPoints] - array of ellipsoids, where nAppr
      is the number of approximations, nPoints is number of points in time
      grid.

   timeVec: double[1, nPoints] - array of time values.
   l0Mat: double[nDirs,nDims] - matrix of good directions at t0

Example:
  aMat = [0 1; 0 0]; bMat = eye(2);
  SUBounds = struct();
  SUBounds.center = {'sin(t)'; 'cos(t)'};
  SUBounds.shape = [9 0; 0 2];
  sys = elltool.linsys.LinSysContinuous(aMat, bMat, SUBounds);
  x0EllObj = ell_unitball(2);
  timeVec = [0 10];
  dirsMat = [1 0; 0 1]';
  rsObj = elltool.reach.ReachContinuous(sys, x0EllObj, dirsMat, timeVec);
  [eaEllMat timeVec] = rsObj.get_ea();

  dsys = elltool.linsys.LinSysDiscrete(aMat, bMat, SUBounds);
  dRsObj = elltool.reach.ReachDiscrete(sys, x0EllObj, dirsMat, timeVec);
  [eaEllMat timeVec] = dRsObj.get_ea();
\end{verbatim}
\subsection{\texorpdfstring{elltool.reach.AReach.get\_goodcurves}{get\_goodcurves}}\label{method:elltool.reach.AReach.getgoodcurves}
\begin{verbatim}
GET_GOODCURVES - returns the 'good curve' trajectories of the reach set.

Input:
  regular:
      self.

Output:
  goodCurvesCVec: cell[1, nPoints] of double [x, y] - array of cells,
      where each cell is array of points that form a 'good curve'.

  timeVec: double[1, nPoints] - array of time values.

Example:
  aMat = [0 1; 0 0]; bMat = eye(2);
  SUBounds = struct();
  SUBounds.center = {'sin(t)'; 'cos(t)'};
  SUBounds.shape = [9 0; 0 2];
  sys = elltool.linsys.LinSysContinuous(aMat, bMat, SUBounds);
  x0EllObj = ell_unitball(2);
  timeVec = [0 10];
  dirsMat = [1 0; 0 1]';
  rsObj = elltool.reach.ReachContinuous(sys, x0EllObj, dirsMat, timeVec);
  [goodCurvesCVec timeVec] = rsObj.get_goodcurves();

  dsys = elltool.linsys.LinSysDiscrete(aMat, bMat, SUBounds);
  dRsObj = elltool.reach.ReachDiscrete(sys, x0EllObj, dirsMat, timeVec);
  [goodCurvesCVec timeVec] = dRsObj.get_goodcurves();
\end{verbatim}
\subsection{\texorpdfstring{elltool.reach.AReach.get\_ia}{get\_ia}}\label{method:elltool.reach.AReach.getia}
\begin{verbatim}

GET_IA - returns array of ellipsoid objects representing internal
         approximation of the  reach tube.

Input:
  regular:
      self.

Output:
  iaEllMat: ellipsoid[nAppr, nPoints] - array of ellipsoids, where nAppr
      is the number of approximations, nPoints is number of points in time
      grid.

  timeVec: double[1, nPoints] - array of time values.

Example:
  aMat = [0 1; 0 0]; bMat = eye(2);
  SUBounds = struct();
  SUBounds.center = {'sin(t)'; 'cos(t)'};
  SUBounds.shape = [9 0; 0 2];
  sys = elltool.linsys.LinSysContinuous(aMat, bMat, SUBounds);
  x0EllObj = ell_unitball(2);
  timeVec = [0 10];
  dirsMat = [1 0; 0 1]';
  rsObj = elltool.reach.ReachContinuous(sys, x0EllObj, dirsMat, timeVec);
  [iaEllMat timeVec] = rsObj.get_ia();
\end{verbatim}
\subsection{\texorpdfstring{elltool.reach.AReach.get\_system}{get\_system}}\label{method:elltool.reach.AReach.getsystem}
\begin{verbatim}

GET_SYSTEM - returns the linear system for which the reach set is
             computed.

Input:
  regular:
      self.

Output:
  linSys: elltool.linsys.LinSys[1, 1] - linear system object.

Example:
  aMat = [0 1; 0 0]; bMat = eye(2);
  SUBounds = struct();
  SUBounds.center = {'sin(t)'; 'cos(t)'};
  SUBounds.shape = [9 0; 0 2];
  sys = elltool.linsys.LinSysContinuous(aMat, bMat, SUBounds);
  x0EllObj = ell_unitball(2);
  timeVec = [0 10];
  dirsMat = [1 0; 0 1]';
  rsObj = elltool.reach.ReachContinuous(sys, x0EllObj, dirsMat, timeVec);
  linSys = rsObj.get_system()

  self =
  A:
       0     1
       0     0


  B:
       1     0
       0     1


  Control bounds:
     2-dimensional ellipsoid with center
      'sin(t)'
      'cos(t)'

     and shape matrix
       9     0
       0     2


  C:
       1     0
       0     1

  2-input, 2-output continuous-time linear time-invariant system of
          dimension 2:
  dx/dt  =  A x(t)  +  B u(t)
   y(t)  =  C x(t)

  dsys = elltool.linsys.LinSysDiscrete(aMat, bMat, SUBounds);
  dRsObj = elltool.reach.ReachDiscrete(sys, x0EllObj, dirsMat, timeVec);
  dRsObj.get_system();
\end{verbatim}
\subsection{\texorpdfstring{elltool.reach.AReach.intersect}{intersect}}\label{method:elltool.reach.AReach.intersect}
\begin{verbatim}
INTERSECT - checks if its external (s = 'e'), or internal (s = 'i')
            approximation intersects with given ellipsoid, hyperplane
            or polytop.

Input:
  regular:
      self.

      intersectObj: ellipsoid[1, 1]/hyperplane[1,1]/polytop[1, 1].

      approxTypeChar: char[1, 1] - 'e' (default) - external approximation,
                                   'i' - internal approximation.

Output:
  isEmptyIntersect: logical[1, 1] -  true - if intersection is nonempty,
                                     false - otherwise.

Example:
  aMat = [0 1; 0 0]; bMat = eye(2);
  SUBounds = struct();
  SUBounds.center = {'sin(t)'; 'cos(t)'};
  SUBounds.shape = [9 0; 0 2];
  sys = elltool.linsys.LinSysContinuous(aMat, bMat, SUBounds);
  x0EllObj = ell_unitball(2);
  timeVec = [0 10];
  dirsMat = [1 0; 0 1]';
  rsObj = elltool.reach.ReachContinuous(sys, x0EllObj, dirsMat, timeVec);
  ellObj = ellipsoid([0; 0], 2*eye(2));
  isEmptyIntersect = intersect(rsObj, ellObj)

  isEmptyIntersect =

                  1
\end{verbatim}
\subsection{\texorpdfstring{elltool.reach.AReach.isEmpty}{isEmpty}}\label{method:elltool.reach.AReach.isEmpty}
\begin{verbatim}

ISEMPTY - checks if given reach set array is an array of empty objects.

Input:
  regular:
      self - multidimensional array of
             ReachContinuous/ReachDiscrete objects

Output:
  isEmptyArr: logical[nDim1, nDim2, nDim3,...] -
              isEmpty(iDim1, iDim2, iDim3,...) = true - if self(iDim1, iDim2, iDim3,...) is empty,
                                               = false - otherwise.

Example:
  aMat = [0 1; 0 0]; bMat = eye(2);
  SUBounds = struct();
  SUBounds.center = {'sin(t)'; 'cos(t)'};
  SUBounds.shape = [9 0; 0 2];
  sys = elltool.linsys.LinSysContinuous(aMat, bMat, SUBounds);
  dsys = elltool.linsys.LinSysContinuous(aMat, bMat, SUBounds);
  x0EllObj = ell_unitball(2);
  timeVec = [0 10];
  dirsMat = [1 0; 0 1]';
  rsObj = elltool.reach.ReachContinuous(sys, x0EllObj, dirsMat, timeVec);
  dRsObj = elltool.reach.ReachRiscrete(dsys, x0EllObj, dirsMat, timeVec);
  rsObjArr = rsObj.repMat(1,2);
  dRsObjArr = dRsObj.repMat(1,2);
  dRsObj.isEmpty();
  rsObj.isEmpty()

  ans =

       0

  dRsObjArr.isEmpty();
  rsObjArr.isEmpty()

  ans =
      [ 0  0 ]
\end{verbatim}
\subsection{\texorpdfstring{elltool.reach.AReach.isEqual}{isEqual}}\label{method:elltool.reach.AReach.isEqual}
\begin{verbatim}
ISEQUAL - checks for equality given reach set objects

Input:
  regular:
      self.
      reachObj:
          elltool.reach.AReach[1, 1] - each set object, which
           compare with self.
  optional:
      indTupleVec: double[1,] - tube numbers that are
          compared
      approxType: gras.ellapx.enums.EApproxType[1, 1] -  type of
          approximation, which will be compared.
  properties:
      notComparedFieldList: cell[1,k] - fields not to compare
          in tubes. Default: LT_GOOD_DIR_*, LS_GOOD_DIR_*,
          IND_S_TIME, S_TIME, TIME_VEC
      areTimeBoundsCompared: logical[1,1] - treat tubes with
          different timebounds as inequal if 'true'.
          Default: false

Output:
  regular:
      ISEQUAL: logical[1, 1] - true - if reach set objects are equal.
          false - otherwise.

Example:
  aMat = [0 1; 0 0]; bMat = eye(2);
  SUBounds = struct();
  SUBounds.center = {'sin(t)'; 'cos(t)'};
  SUBounds.shape = [9 0; 0 2];
  sys = elltool.linsys.LinSysContinuous(aMat, bMat, SUBounds);
  x0EllObj = ell_unitball(2);
  timeVec = [0 10];
  dirsMat = [1 0; 0 1]';
  rsObj = elltool.reach.ReachContinuous(sys, x0EllObj, dirsMat, timeVec);
  copyRsObj = rsObj.getCopy();
  isEqual = isEqual(rsObj, copyRsObj)

  isEqual =

          1
\end{verbatim}
\subsection{\texorpdfstring{elltool.reach.AReach.isbackward}{isbackward}}\label{method:elltool.reach.AReach.isbackward}
\begin{verbatim}
ISBACKWARD - checks if given reach set object was obtained by solving
             the system in reverse time.

Input:
  regular:
      self.

Output:
  regular:
      isBackward: logical[1, 1] - true - if self was obtained by solving
          in reverse time, false - otherwise.

Example:
  aMat = [0 1; 0 0]; bMat = eye(2);
  SUBounds = struct();
  SUBounds.center = {'sin(t)'; 'cos(t)'};
  SUBounds.shape = [9 0; 0 2];
  sys = elltool.linsys.LinSysContinuous(aMat, bMat, SUBounds);
  x0EllObj = ell_unitball(2);
  timeVec = [10 0];
  dirsMat = [1 0; 0 1]';
  rsObj = elltool.reach.ReachContinuous(sys, x0EllObj, dirsMat, timeVec);
  rsObj.isbackward()

  ans =

       1
\end{verbatim}
\subsection{\texorpdfstring{elltool.reach.AReach.iscut}{iscut}}\label{method:elltool.reach.AReach.iscut}
\begin{verbatim}
ISCUT - checks if given array of reach set objects is a cut of
        another reach set object's array.

Input:
  regular:
      self - multidimensional array of
             ReachContinuous/ReachDiscrete objects

Output:
  isCutArr: logical[nDim1, nDim2, nDim3 ...] -
            isCut(iDim1, iDim2, iDim3,..) = true - if self(iDim1, iDim2, iDim3,...) is a cut of the reach set,
                                          = false - otherwise.

Example:
  aMat = [0 1; 0 0]; bMat = eye(2);
  SUBounds = struct();
  SUBounds.center = {'sin(t)'; 'cos(t)'};
  SUBounds.shape = [9 0; 0 2];
  sys = elltool.linsys.LinSysContinuous(aMat, bMat, SUBounds);
  dsys = elltool.linsys.LinSysDiscrete(aMat, bMat, SUBounds);
  x0EllObj = ell_unitball(2);
  timeVec = [0 10];
  dirsMat = [1 0; 0 1]';
  rsObj = elltool.reach.ReachContinuous(sys, x0EllObj, dirsMat, timeVec);
  dRsObj = elltool.reach.ReachRiscrete(dsys, x0EllObj, dirsMat, timeVec);
  cutObj = rsObj.cut([3 5]);
  cutObjArr = cutObj.repMat(2,3,4);
  iscut(cutObj);
  iscut(cutObjArr);
  cutObj = dRsObj.cut([4 8]);
  cutObjArr = cutObj.repMat(1,2);
  iscut(cutObjArr);
  iscut(cutObj);
\end{verbatim}
\subsection{\texorpdfstring{elltool.reach.AReach.isprojection}{isprojection}}\label{method:elltool.reach.AReach.isprojection}
\begin{verbatim}

ISPROJECTION - checks if given array of reach set objects is projections.

Input:
  regular:
      self - multidimensional array of
             ReachContinuous/ReachDiscrete objects

Output:
  isProjArr: logical[nDim1, nDim2, nDim3, ...] -
             isProj(iDim1, iDim2, iDim3,...) = true - if self(iDim1, iDim2, iDim3,...) is projection,
                                             = false - otherwise.


Example:
  aMat = [0 1; 0 0]; bMat = eye(2);
  SUBounds = struct();
  SUBounds.center = {'sin(t)'; 'cos(t)'};
  SUBounds.shape = [9 0; 0 2];
  sys = elltool.linsys.LinSysContinuous(aMat, bMat, SUBounds);
  dsys = elltool.linsys.LinSysDiscrete(aMat, bMat, SUBounds);
  x0EllObj = ell_unitball(2);
  timeVec = [0 10];
  dirsMat = [1 0; 0 1]';
  rsObj = elltool.reach.ReachContinuous(sys, x0EllObj, dirsMat, timeVec);
  dRsObj = elltool.reach.ReachRiscrete(dsys, x0EllObj, dirsMat, timeVec);
  projMat = eye(2);
  projObj = rsObj.projection(projMat);
  projObjArr = projObj.repMat(3,2,2);
  isprojection(projObj);
  isprojection(projObjArr);
  projObj = dRsObj.projection(projMat);
  projObjArr = projObj.repMat(1,2);
  isprojection(projObj);
  isprojection(projObjArr);
\end{verbatim}
\subsection{\texorpdfstring{elltool.reach.AReach.plotByEa}{plotByEa}}\label{method:elltool.reach.AReach.plotByEa}
\begin{verbatim}
plotByEa - plots external approximation of reach tube.


Usage:
      plotByEa(self,'Property',PropValue,...)
      - plots external approximation of reach tube
           with  setting properties

Input:
  regular:
      self: - reach tube

  optional:
      relDataPlotter:smartdb.disp.RelationDataPlotter[1,1] - relation data plotter object.
      charColor: char[1,1]  - color specification code, can be 'r','g',
                     etc (any code supported by built-in Matlab function).
  properties:

      'fill': logical[1,1]  -
              if 1, tube in 2D will be filled with color.
              Default value is true.
      'lineWidth': double[1,1]  -
                   line width for 2D plots. Default value is 2.
      'color': double[1,3] -
               sets default colors in the form [x y z].
                  Default value is [0 0 1].
      'shade': double[1,1]  -
     level of transparency between 0 and 1 (0 - transparent, 1 - opaque).
               Default value is 0.3.

Output:
  regular:
      plObj: smartdb.disp.RelationDataPlotter[1,1] - returns the relation
      data plotter object.
\end{verbatim}
\subsection{\texorpdfstring{elltool.reach.AReach.plotByIa}{plotByIa}}\label{method:elltool.reach.AReach.plotByIa}
\begin{verbatim}
plotByIa - plots internal approximation of reach tube.


Usage:
      plotByIa(self,'Property',PropValue,...)
      - plots internal approximation of reach tube
           with  setting properties

Input:
  regular:
      self: - reach tube

  optional:
      relDataPlotter:smartdb.disp.RelationDataPlotter[1,1] - relation data plotter object.
      charColor: char[1,1]  - color specification code, can be 'r','g',
                     etc (any code supported by built-in Matlab function).
  properties:

      'fill': logical[1,1]  -
              if 1, tube in 2D will be filled with color.
              Default value is true.
      'lineWidth': double[1,1]  -
                   line width for 2D plots. Default value is 2.
      'color': double[1,3] -
               sets default colors in the form [x y z].
                  Default value is [0 1 0].
      'shade': double[1,1]  -
     level of transparency between 0 and 1 (0 - transparent, 1 - opaque).
               Default value is 0.1.

Output:
  regular:
      plObj: smartdb.disp.RelationDataPlotter[1,1] - returns the relation
      data plotter object.
\end{verbatim}
\subsection{\texorpdfstring{elltool.reach.AReach.plotEa}{plotEa}}\label{method:elltool.reach.AReach.plotEa}
\begin{verbatim}

PLOT_EA - plots external approximations of 2D and 3D reach sets.

Input:
  regular:
      self.

  optional:
      colorSpec: char[1, 1] - set color to plot in following way:
                             'r' - red color,
                             'g' - green color,
                             'b' - blue color,
                             'y' - yellow color,
                             'c' - cyan color,
                             'm' - magenta color,
                             'w' - white color.

      OptStruct: struct[1, 1] with fields:
          color: double[1, 3] - sets color of the picture in the form
                [x y z].
          width: double[1, 1] - sets line width for 2D plots.
          shade: double[1, 1] in [0; 1] interval - sets transparency level
                (0 - transparent, 1 - opaque).
           fill: double[1, 1] - if set to 1, reach set will be filled with
                 color.

Output:
  None.

Example:
  aMat = [0 1; 0 0]; bMat = eye(2);
  SUBounds = struct();
  SUBounds.center = {'sin(t)'; 'cos(t)'};
  SUBounds.shape = [9 0; 0 2];
  sys = elltool.linsys.LinSysContinuous(aMat, bMat, SUBounds);
  x0EllObj = ell_unitball(2);
  timeVec = [0 10];
  dirsMat = [1 0; 0 1]';
  rsObj = elltool.reach.ReachContinuous(sys, x0EllObj, dirsMat, timeVec);
  rsObj.plotEa();
  dsys = elltool.linsys.LinSysDiscrete(aMat, bMat, SUBounds);

  dRsObj = elltool.reach.ReachDiscrete(sys, x0EllObj, dirsMat, timeVec);
  dRsObj.plotEa();
\end{verbatim}
\subsection{\texorpdfstring{elltool.reach.AReach.plotIa}{plotIa}}\label{method:elltool.reach.AReach.plotIa}
\begin{verbatim}

PLOTIA - plots internal approximations of 2D and 3D reach sets.

Input:
  regular:
      self.

  optional:
      colorSpec: char[1, 1] - set color to plot in following way:
                             'r' - red color,
                             'g' - green color,
                             'b' - blue color,
                             'y' - yellow color,
                             'c' - cyan color,
                             'm' - magenta color,
                             'w' - white color.

      OptStruct: struct[1, 1] with fields:
          color: double[1, 3] - sets color of the picture in the form
                [x y z].
          width: double[1, 1] - sets line width for 2D plots.
          shade: double[1, 1] in [0; 1] interval - sets transparency level
                (0 - transparent, 1 - opaque).
           fill: double[1, 1] - if set to 1, reach set will be filled with
                color.

Example:
  aMat = [0 1; 0 0]; bMat = eye(2);
  SUBounds = struct();
  SUBounds.center = {'sin(t)'; 'cos(t)'};
  SUBounds.shape = [9 0; 0 2];
  sys = elltool.linsys.LinSysContinuous(aMat, bMat, SUBounds);
  x0EllObj = ell_unitball(2);
  timeVec = [0 10];
  dirsMat = [1 0; 0 1]';
  rsObj = elltool.reach.ReachContinuous(sys, x0EllObj, dirsMat, timeVec);
  rsObj.plotIa();
  dsys = elltool.linsys.LinSysDiscrete(aMat, bMat, SUBounds);
  dRsObj = elltool.reach.ReachDiscrete(sys, x0EllObj, dirsMat, timeVec);
  dRsObj.plotIa();
\end{verbatim}
\subsection{\texorpdfstring{elltool.reach.AReach.projection}{projection}}\label{method:elltool.reach.AReach.projection}
\begin{verbatim}

\end{verbatim}
\subsection{\texorpdfstring{elltool.reach.AReach.refine}{refine}}\label{method:elltool.reach.AReach.refine}
\begin{verbatim}

REFINE - adds new approximations computed for the specified directions
         to the given reach set or to the projection of reach set.

Input:
  regular:
      self.
      l0Mat: double[nDim, nDir] - matrix of directions for new
          approximation

Output:
  regular:
      reachObj: reach[1,1] - refine reach set for the directions
          specified in l0Mat

Example:
  aMat = [0 1; 0 0]; bMat = eye(2);
  SUBounds = struct();
  SUBounds.center = {'sin(t)'; 'cos(t)'};
  SUBounds.shape = [9 0; 0 2];
  sys = elltool.linsys.LinSysContinuous(aMat, bMat, SUBounds);
  x0EllObj = ell_unitball(2);
  timeVec = [0 10];
  dirsMat = [1 0; 0 1]';
  newDirsMat = [1; -1];
  rsObj = elltool.reach.ReachContinuous(sys, x0EllObj, dirsMat, timeVec);
  rsObj = rsObj.refine(newDirsMat);
\end{verbatim}
\subsection{\texorpdfstring{elltool.reach.AReach.repMat}{repMat}}\label{method:elltool.reach.AReach.repMat}
\begin{verbatim}

REPMAT - is analogous to built-in repmat function with one exception - it
         copies the objects, not just the handles

Input:
  regular:
      self.

Output:
  Array of given ReachContinuous/ReachDiscrete object's copies.

 Example:
   aMat = [0 1; 0 0]; bMat = eye(2);
   SUBounds = struct();
   SUBounds.center = {'sin(t)'; 'cos(t)'};
   SUBounds.shape = [9 0; 0 2];
   sys = elltool.linsys.LinSysContinuous(aMat, bMat, SUBounds);
   x0EllObj = ell_unitball(2);
   timeVec = [0 10];
   dirsMat = [1 0; 0 1]';
   reachObj = elltool.reach.ReachContinuous(sys, x0EllObj, dirsMat, timeVec);
   reachObjArr = reachObj.repMat(1,2);

   reachObjArr = 1x2 array of ReachContinuous objects
\end{verbatim}
\section{elltool.reach.ReachContinuous}\label{secClassDescr:elltool.reach.ReachContinuous}
\subsection{\texorpdfstring{elltool.reach.ReachContinuous.ReachContinuous}{ReachContinuous}}\label{method:elltool.reach.ReachContinuous.ReachContinuous}
\begin{verbatim}
ReachContinuous - computes reach set approximation of the continuous
    linear system for the given time interval.
Input:
    regular:
      linSys: elltool.linsys.LinSys object -
          given linear system .
      x0Ell: ellipsoid[1, 1] - ellipsoidal set of
          initial conditions.
      l0Mat: double[nRows, nColumns] - initial good directions
          matrix.
      timeVec: double[1, 2] - time interval.

    properties:
      isRegEnabled: logical[1, 1] - if it is 'true' constructor
          is allowed to use regularization.
      isJustCheck: logical[1, 1] - if it is 'true' constructor
          just check if square matrices are degenerate, if it is
          'false' all degenerate matrices will be regularized.
      regTol: double[1, 1] - regularization precision.

Output:
  regular:
    self - reach set object.

Example:
  aMat = [0 1; 0 0]; bMat = eye(2);
  SUBounds = struct();
  SUBounds.center = {'sin(t)'; 'cos(t)'};
  SUBounds.shape = [9 0; 0 2];
  sys = elltool.linsys.LinSysContinuous(aMat, bMat, SUBounds);
  x0EllObj = ell_unitball(2);
  timeVec = [0 10];
  dirsMat = [1 0; 0 1]';
  rsObj = elltool.reach.ReachContinuous(sys, x0EllObj, dirsMat, timeVec);
\end{verbatim}


See the description of the following methods in section \ref{secClassDescr:elltool.reach.AReach}
 for elltool.reach.AReach:

\begin{list}{}{}
 \item \hyperref[method:elltool.reach.AReach.cut]{cut}
 \item \hyperref[method:elltool.reach.AReach.dimension]{dimension}
 \item \hyperref[method:elltool.reach.AReach.display]{display}
 \item \hyperref[method:elltool.reach.AReach.evolve]{evolve}
 \item \hyperref[method:elltool.reach.AReach.getAbsTol]{getAbsTol}
 \item \hyperref[method:elltool.reach.AReach.getCopy]{getCopy}
 \item \hyperref[method:elltool.reach.AReach.getEaScaleFactor]{getEaScaleFactor}
 \item \hyperref[method:elltool.reach.AReach.getEllTubeRel]{getEllTubeRel}
 \item \hyperref[method:elltool.reach.AReach.getEllTubeUnionRel]{getEllTubeUnionRel}
 \item \hyperref[method:elltool.reach.AReach.getIaScaleFactor]{getIaScaleFactor}
 \item \hyperref[method:elltool.reach.AReach.getInitialSet]{getInitialSet}
 \item \hyperref[method:elltool.reach.AReach.getNPlot2dPoints]{getNPlot2dPoints}
 \item \hyperref[method:elltool.reach.AReach.getNPlot3dPoints]{getNPlot3dPoints}
 \item \hyperref[method:elltool.reach.AReach.getNTimeGridPoints]{getNTimeGridPoints}
 \item \hyperref[method:elltool.reach.AReach.getRelTol]{getRelTol}
 \item \hyperref[method:elltool.reach.AReach.getSwitchTimeVec]{getSwitchTimeVec}
 \item \hyperref[method:elltool.reach.AReach.get_center]{get\_center}
 \item \hyperref[method:elltool.reach.AReach.get_directions]{get\_directions}
 \item \hyperref[method:elltool.reach.AReach.get_ea]{get\_ea}
 \item \hyperref[method:elltool.reach.AReach.get_goodcurves]{get\_goodcurves}
 \item \hyperref[method:elltool.reach.AReach.get_ia]{get\_ia}
 \item \hyperref[method:elltool.reach.AReach.get_system]{get\_system}
 \item \hyperref[method:elltool.reach.AReach.intersect]{intersect}
 \item \hyperref[method:elltool.reach.AReach.isEmpty]{isEmpty}
 \item \hyperref[method:elltool.reach.AReach.isEqual]{isEqual}
 \item \hyperref[method:elltool.reach.AReach.isbackward]{isbackward}
 \item \hyperref[method:elltool.reach.AReach.iscut]{iscut}
 \item \hyperref[method:elltool.reach.AReach.isprojection]{isprojection}
 \item \hyperref[method:elltool.reach.AReach.plotByEa]{plotByEa}
 \item \hyperref[method:elltool.reach.AReach.plotByIa]{plotByIa}
 \item \hyperref[method:elltool.reach.AReach.plotEa]{plotEa}
 \item \hyperref[method:elltool.reach.AReach.plotIa]{plotIa}
 \item \hyperref[method:elltool.reach.AReach.projection]{projection}
 \item \hyperref[method:elltool.reach.AReach.refine]{refine}
 \item \hyperref[method:elltool.reach.AReach.repMat]{repMat}
\end{list}
\section{elltool.reach.ReachDiscrete}\label{secClassDescr:elltool.reach.ReachDiscrete}
\subsection{\texorpdfstring{elltool.reach.ReachDiscrete.ReachDiscrete}{ReachDiscrete}}\label{method:elltool.reach.ReachDiscrete.ReachDiscrete}
\begin{verbatim}
ReachDiscrete - computes reach set approximation of the discrete linear
                system for the given time interval.


Input:
    linSys: elltool.linsys.LinSys object - given linear system
    x0Ell: ellipsoid[1, 1] - ellipsoidal set of initial conditions
    l0Mat: double[nRows, nColumns] - initial good directions
          matrix.
    timeVec: double[1, 2] - time interval
    properties:
      isRegEnabled: logical[1, 1] - if it is 'true' constructor
          is allowed to use regularization.
      isJustCheck: logical[1, 1] - if it is 'true' constructor
          just check if square matrices are degenerate, if it is
          'false' all degenerate matrices will be regularized.
      regTol: double[1, 1] - regularization precision.
      minmax: logical[1, 1] - field, which:
          = 1 compute minmax reach set,
          = 0 (default) compute maxmin reach set.

Output:
  regular:
    self - reach set object.
Example:
  adMat = [0 1; -1 -0.5];
  bdMat = [0; 1];
  udBoundsEllObj  = ellipsoid(1);
  dtsys = elltool.linsys.LinSysDiscrete(adMat, bdMat, udBoundsEllObj);
  x0EllObj = ell_unitball(2);
  timeVec = [0 10];
  dirsMat = [1 0; 0 1]';
  dRsObj = elltool.reach.ReachDiscrete(dtsys, x0EllObj, dirsMat, timeVec);
\end{verbatim}


See the description of the following methods in section \ref{secClassDescr:elltool.reach.AReach}
 for elltool.reach.AReach:

\begin{list}{}{}
 \item \hyperref[method:elltool.reach.AReach.cut]{cut}
 \item \hyperref[method:elltool.reach.AReach.dimension]{dimension}
 \item \hyperref[method:elltool.reach.AReach.display]{display}
 \item \hyperref[method:elltool.reach.AReach.evolve]{evolve}
 \item \hyperref[method:elltool.reach.AReach.getAbsTol]{getAbsTol}
 \item \hyperref[method:elltool.reach.AReach.getCopy]{getCopy}
 \item \hyperref[method:elltool.reach.AReach.getEaScaleFactor]{getEaScaleFactor}
 \item \hyperref[method:elltool.reach.AReach.getEllTubeRel]{getEllTubeRel}
 \item \hyperref[method:elltool.reach.AReach.getEllTubeUnionRel]{getEllTubeUnionRel}
 \item \hyperref[method:elltool.reach.AReach.getIaScaleFactor]{getIaScaleFactor}
 \item \hyperref[method:elltool.reach.AReach.getInitialSet]{getInitialSet}
 \item \hyperref[method:elltool.reach.AReach.getNPlot2dPoints]{getNPlot2dPoints}
 \item \hyperref[method:elltool.reach.AReach.getNPlot3dPoints]{getNPlot3dPoints}
 \item \hyperref[method:elltool.reach.AReach.getNTimeGridPoints]{getNTimeGridPoints}
 \item \hyperref[method:elltool.reach.AReach.getRelTol]{getRelTol}
 \item \hyperref[method:elltool.reach.AReach.getSwitchTimeVec]{getSwitchTimeVec}
 \item \hyperref[method:elltool.reach.AReach.get_center]{get\_center}
 \item \hyperref[method:elltool.reach.AReach.get_directions]{get\_directions}
 \item \hyperref[method:elltool.reach.AReach.get_ea]{get\_ea}
 \item \hyperref[method:elltool.reach.AReach.get_goodcurves]{get\_goodcurves}
 \item \hyperref[method:elltool.reach.AReach.get_ia]{get\_ia}
 \item \hyperref[method:elltool.reach.AReach.get_system]{get\_system}
 \item \hyperref[method:elltool.reach.AReach.intersect]{intersect}
 \item \hyperref[method:elltool.reach.AReach.isEmpty]{isEmpty}
 \item \hyperref[method:elltool.reach.AReach.isEqual]{isEqual}
 \item \hyperref[method:elltool.reach.AReach.isbackward]{isbackward}
 \item \hyperref[method:elltool.reach.AReach.iscut]{iscut}
 \item \hyperref[method:elltool.reach.AReach.isprojection]{isprojection}
 \item \hyperref[method:elltool.reach.AReach.plotByEa]{plotByEa}
 \item \hyperref[method:elltool.reach.AReach.plotByIa]{plotByIa}
 \item \hyperref[method:elltool.reach.AReach.plotEa]{plotEa}
 \item \hyperref[method:elltool.reach.AReach.plotIa]{plotIa}
 \item \hyperref[method:elltool.reach.AReach.projection]{projection}
 \item \hyperref[method:elltool.reach.AReach.refine]{refine}
 \item \hyperref[method:elltool.reach.AReach.repMat]{repMat}
\end{list}
\section{elltool.reach.ReachFactory}\label{secClassDescr:elltool.reach.ReachFactory}
\subsection{\texorpdfstring{elltool.reach.ReachFactory.ReachFactory}{ReachFactory}}\label{method:elltool.reach.ReachFactory.ReachFactory}
\begin{verbatim}
Example:
  import elltool.reach.ReachFactory;
  crm=gras.ellapx.uncertcalc.test.regr.conf.ConfRepoMgr();
  crmSys=gras.ellapx.uncertcalc.test.regr.conf.sysdef.ConfRepoMgr();
  rsObj =  ReachFactory('demo3firstTest', crm, crmSys, false, false);
\end{verbatim}
\subsection{\texorpdfstring{elltool.reach.ReachFactory.createInstance}{createInstance}}\label{method:elltool.reach.ReachFactory.createInstance}
\begin{verbatim}
Example:
  import elltool.reach.ReachFactory;
  crm=gras.ellapx.uncertcalc.test.regr.conf.ConfRepoMgr();
  crmSys=gras.ellapx.uncertcalc.test.regr.conf.sysdef.ConfRepoMgr();
  rsObj =  ReachFactory('demo3firstTest', crm, crmSys, false, false);
  reachObj = rsObj.createInstance();
\end{verbatim}
\subsection{\texorpdfstring{elltool.reach.ReachFactory.createSysInstance}{createSysInstance}}\label{method:elltool.reach.ReachFactory.createSysInstance}
\begin{verbatim}

\end{verbatim}
\subsection{\texorpdfstring{elltool.reach.ReachFactory.getDim}{getDim}}\label{method:elltool.reach.ReachFactory.getDim}
\begin{verbatim}
Example:
  import elltool.reach.ReachFactory;
  crm=gras.ellapx.uncertcalc.test.regr.conf.ConfRepoMgr();
  crmSys=gras.ellapx.uncertcalc.test.regr.conf.sysdef.ConfRepoMgr();
  rsObj =  ReachFactory('demo3firstTest', crm, crmSys, false, false);
  dim = rsObj.getDim();
\end{verbatim}
\subsection{\texorpdfstring{elltool.reach.ReachFactory.getL0Mat}{getL0Mat}}\label{method:elltool.reach.ReachFactory.getL0Mat}
\begin{verbatim}
Example:
  import elltool.reach.ReachFactory;
  crm=gras.ellapx.uncertcalc.test.regr.conf.ConfRepoMgr();
  crmSys=gras.ellapx.uncertcalc.test.regr.conf.sysdef.ConfRepoMgr();
  rsObj =  ReachFactory('demo3firstTest', crm, crmSys, false, false);
  l0Mat = rsObj.getL0Mat()

  l0Mat =

       1     0
       0     1
\end{verbatim}
\subsection{\texorpdfstring{elltool.reach.ReachFactory.getLinSys}{getLinSys}}\label{method:elltool.reach.ReachFactory.getLinSys}
\begin{verbatim}
Example:
  import elltool.reach.ReachFactory;
  crm=gras.ellapx.uncertcalc.test.regr.conf.ConfRepoMgr();
  crmSys=gras.ellapx.uncertcalc.test.regr.conf.sysdef.ConfRepoMgr();
  rsObj =  ReachFactory('demo3firstTest', crm, crmSys, false, false);
  linSys = rsObj.getLinSys();
\end{verbatim}
\subsection{\texorpdfstring{elltool.reach.ReachFactory.getTVec}{getTVec}}\label{method:elltool.reach.ReachFactory.getTVec}
\begin{verbatim}
Example:
  import elltool.reach.ReachFactory;
  crm=gras.ellapx.uncertcalc.test.regr.conf.ConfRepoMgr();
  crmSys=gras.ellapx.uncertcalc.test.regr.conf.sysdef.ConfRepoMgr();
  rsObj =  ReachFactory('demo3firstTest', crm, crmSys, false, false);
  tVec = rsObj.getTVec()

  tVec =

       0    10
\end{verbatim}
\subsection{\texorpdfstring{elltool.reach.ReachFactory.getX0Ell}{getX0Ell}}\label{method:elltool.reach.ReachFactory.getX0Ell}
\begin{verbatim}
Example:
  import elltool.reach.ReachFactory;
  crm=gras.ellapx.uncertcalc.test.regr.conf.ConfRepoMgr();
  crmSys=gras.ellapx.uncertcalc.test.regr.conf.sysdef.ConfRepoMgr();
  rsObj =  ReachFactory('demo3firstTest', crm, crmSys, false, false);
  X0Ell = rsObj.getX0Ell()

  X0Ell =

  Center:
       0
       0

  Shape Matrix:
      0.0100         0
           0    0.0100

  Nondegenerate ellipsoid in R^2.
\end{verbatim}
\section{elltool.linsys.ALinSys}\label{secClassDescr:elltool.linsys.ALinSys}
\subsection{\texorpdfstring{elltool.linsys.ALinSys.ALinSys}{ALinSys}}\label{method:elltool.linsys.ALinSys.ALinSys}
\begin{verbatim}
ALinSys - constructor abstract class of linear system.

Continuous-time linear system:
          dx/dt  =  A(t) x(t)  +  B(t) u(t)  +  C(t) v(t)

Discrete-time linear system:
          x[k+1]  =  A[k] x[k]  +  B[k] u[k]  +  C[k] v[k]

Input:
  regular:
      atInpMat: double[nDim, nDim]/cell[nDim, nDim] - matrix A.

      btInpMat: double[nDim, kDim]/cell[nDim, kDim] - matrix B.

      uBoundsEll: ellipsoid[1, 1]/struct[1, 1] - control bounds
          ellipsoid.

      ctInpMat: double[nDim, lDim]/cell[nDim, lDim] - matrix G.

      vBoundsEll: ellipsoid[1, 1]/struct[1, 1] - disturbance bounds
          ellipsoid.
      discrFlag: char[1, 1] - if discrFlag set:
          'd' - to discrete-time linSys
          not 'd' - to continuous-time linSys.

Output:
  self: elltool.linsys.ALinSys[1, 1] -
      linear system.
\end{verbatim}
\subsection{\texorpdfstring{elltool.linsys.ALinSys.dimension}{dimension}}\label{method:elltool.linsys.ALinSys.dimension}
\begin{verbatim}

DIMENSION - returns dimensions of state, input, output and disturbance
            spaces.
Input:
  regular:
      self: elltool.linsys.LinSys[nDims1, nDims2,...] - an array of
            linear systems.

Output:
  stateDimArr: double[nDims1, nDims2,...] - array of state space
      dimensions.

  inpDimArr: double[nDims1, nDims2,...] - array of input dimensions.

  distDimArr: double[nDims1, nDims2,...] - array of disturbance
        dimensions.

Examples:
  aMat = [0 1; 0 0]; bMat = eye(2);
  SUBounds = struct();
  SUBounds.center = {'sin(t)'; 'cos(t)'};
  SUBounds.shape = [9 0; 0 2];
  sys = elltool.linsys.LinSysContinuous(aMat, bMat, SUBounds);
  [stateDimArr, inpDimArr, outDimArr, distDimArr] = sys.dimension()

  stateDimArr =

       2


  inpDimArr =

       2



  distDimArr =

       0

  dsys = elltool.linsys.LinSysDiscrete(aMat, bMat, SUBounds);
  dsys.dimension();
\end{verbatim}
\subsection{\texorpdfstring{elltool.linsys.ALinSys.display}{display}}\label{method:elltool.linsys.ALinSys.display}
\begin{verbatim}
DISPLAY - displays the details of linear system object.

Input:
  regular:
      self: elltool.linsys.ALinSys[1, 1] - linear system.

Output:
  None.
\end{verbatim}
\subsection{\texorpdfstring{elltool.linsys.ALinSys.getAbsTol}{getAbsTol}}\label{method:elltool.linsys.ALinSys.getAbsTol}
\begin{verbatim}

GETABSTOL - gives array the same size as linsysArr with values of absTol
            properties for each hyperplane in hplaneArr.


Input:
  regular:
      self: elltool.linsys.LinSys[nDims1, nDims2,...] - an array of linear
            systems.

Output:
  absTolArr: double[nDims1, nDims2,...] - array of absTol properties for
      linear systems in self.

Examples:
  aMat = [0 1; 0 0]; bMat = eye(2);
  SUBounds = struct();
  SUBounds.center = {'sin(t)'; 'cos(t)'};
  SUBounds.shape = [9 0; 0 2];
  sys = elltool.linsys.LinSysContinuous(aMat, bMat, SUBounds);
  sys.getAbsTol();
  dsys = elltool.linsys.LinSysDiscrete(aMat, bMat, SUBounds);
  dsys.getAbsTol();
\end{verbatim}
\subsection{\texorpdfstring{elltool.linsys.ALinSys.getAtMat}{getAtMat}}\label{method:elltool.linsys.ALinSys.getAtMat}
\begin{verbatim}

Input:
  regular:
      self: elltool.linsys.ILinSys[1, 1] - linear system.

Output:
  aMat: double[aMatDim, aMatDim]/cell[nDim, nDim] - matrix A.

Examples:
  aMat = [0 1; 0 0]; bMat = eye(2);
  SUBounds = struct();
  SUBounds.center = {'sin(t)'; 'cos(t)'};
  SUBounds.shape = [9 0; 0 2];
  sys = elltool.linsys.LinSysContinuous(aMat, bMat, SUBounds);
  dsys = elltool.linsys.LinSysDiscrete(aMat, bMat, SUBounds);
  aMat = dsys.getAtMat();
\end{verbatim}
\subsection{\texorpdfstring{elltool.linsys.ALinSys.getBtMat}{getBtMat}}\label{method:elltool.linsys.ALinSys.getBtMat}
\begin{verbatim}

Input:
  regular:
      self: elltool.linsys.ILinSys[1, 1] - linear system.

Output:
  bMat: double[bMatDim, bMatDim]/cell[bMatDim, bMatDim] - matrix B.

Examples:
  aMat = [0 1; 0 0]; bMat = eye(2);
  SUBounds = struct();
  SUBounds.center = {'sin(t)'; 'cos(t)'};
  SUBounds.shape = [9 0; 0 2];
  sys = elltool.linsys.LinSysContinuous(aMat, bMat, SUBounds);
  dsys = elltool.linsys.LinSysDiscrete(aMat, bMat, SUBounds);
  bMat = dsys.getBtMat();
\end{verbatim}
\subsection{\texorpdfstring{elltool.linsys.ALinSys.getCopy}{getCopy}}\label{method:elltool.linsys.ALinSys.getCopy}
\begin{verbatim}

GETCOPY - gives array the same size as linsysArr with with copies of
          elements of self.

Input:
  regular:
      self: elltool.linsys.ALinSys[nDims1, nDims2,...] - an array of
            linear systems.

Output:
  copyLinSysArr: elltool.linsys.LinSys[nDims1, nDims2,...] -  an array of
     copies of elements of self.

Examples:
  aMat = [0 1; 0 0]; bMat = eye(2);
  SUBounds = struct();
  SUBounds.center = {'sin(t)'; 'cos(t)'};
  SUBounds.shape = [9 0; 0 2];
  sys = elltool.linsys.LinSysContinuous(aMat, bMat, SUBounds);
  newSys = sys.getCopy();
  dsys = elltool.linsys.LinSysDiscrete(aMat, bMat, SUBounds);
  newDSys = dsys.getCopy();
\end{verbatim}
\subsection{\texorpdfstring{elltool.linsys.ALinSys.getCtMat}{getCtMat}}\label{method:elltool.linsys.ALinSys.getCtMat}
\begin{verbatim}

Input:
  regular:
      self: elltool.linsys.ILinSys[1, 1] - linear system.

Output:
  cMat: double[cMatDim, cMatDim]/cell[cMatDim, cMatDim] - matrix C.

Examples:
  aMat = [0 1; 0 0]; bMat = eye(2);
  SUBounds = struct();
  SUBounds.center = {'sin(t)'; 'cos(t)'};
  SUBounds.shape = [9 0; 0 2];
  sys = elltool.linsys.LinSysContinuous(aMat, bMat, SUBounds);
  dsys = elltool.linsys.LinSysDiscrete(aMat, bMat, SUBounds);
  gMat = sys.getCtMat();
\end{verbatim}
\subsection{\texorpdfstring{elltool.linsys.ALinSys.getDistBoundsEll}{getDistBoundsEll}}\label{method:elltool.linsys.ALinSys.getDistBoundsEll}
\begin{verbatim}

Input:
  regular:
      self: elltool.linsys.ILinSys[1, 1] - linear system.

Output:
  distEll: ellipsoid[1, 1]/struct[1, 1] - disturbance bounds ellipsoid.

Examples:
  aMat = [0 1; 0 0]; bMat = eye(2);
  SUBounds = struct();
  SUBounds.center = {'sin(t)'; 'cos(t)'};
  SUBounds.shape = [9 0; 0 2];
  sys = elltool.linsys.LinSysContinuous(aMat, bMat, SUBounds);
  dsys = elltool.linsys.LinSysDiscrete(aMat, bMat, SUBounds);
  distEll = sys.getDistBoundsEll();
\end{verbatim}
\subsection{\texorpdfstring{elltool.linsys.ALinSys.getUBoundsEll}{getUBoundsEll}}\label{method:elltool.linsys.ALinSys.getUBoundsEll}
\begin{verbatim}

Input:
  regular:
      self: elltool.linsys.ILinSys[1, 1] - linear system.

Output:
  uEll: ellipsoid[1, 1]/struct[1, 1] - control bounds ellipsoid.

Examples:
  aMat = [0 1; 0 0]; bMat = eye(2);
  SUBounds = struct();
  SUBounds.center = {'sin(t)'; 'cos(t)'};
  SUBounds.shape = [9 0; 0 2];
  sys = elltool.linsys.LinSysContinuous(aMat, bMat, SUBounds);
  dsys = elltool.linsys.LinSysDiscrete(aMat, bMat, SUBounds);
  uEll = dsys.getUBoundsEll();
\end{verbatim}
\subsection{\texorpdfstring{elltool.linsys.ALinSys.hasDisturbance}{hasDisturbance}}\label{method:elltool.linsys.ALinSys.hasDisturbance}
\begin{verbatim}
HASDISTURBANCE - returns true if system has disturbance

Input:
  regular:
      self: elltool.linsys.LinSys[nDims1, nDims2,...] - an array of
            linear systems.
  optional:
      isMeaningful: logical[1,1] - if true(default), treat constant
                    disturbance vector as absence of disturbance

Output:
  isDisturbanceArr: logical[nDims1, nDims2,...] - array such that it's
      element at each position is true if corresponding linear system
      has disturbance, and false otherwise.

Examples:
  aMat = [0 1; 0 0]; bMat = eye(2);
  SUBounds = struct();
  SUBounds.center = {'sin(t)'; 'cos(t)'};
  SUBounds.shape = [9 0; 0 2];
  sys = elltool.linsys.LinSysContinuous(aMat, bMat, SUBounds);
  sys.hasDisturbance()

  ans =

       0
  dsys = elltool.linsys.LinSysDiscrete(aMat, bMat, SUBounds);
  dsys.hasDisturbance();
\end{verbatim}
\subsection{\texorpdfstring{elltool.linsys.ALinSys.isEmpty}{isEmpty}}\label{method:elltool.linsys.ALinSys.isEmpty}
\begin{verbatim}


ISEMPTY - checks if linear system is empty.

Input:
  regular:
      self: elltool.linsys.LinSys[nDims1, nDims2,...] - an array of linear
            systems.

Output:
  isEmptyMat: logical[nDims1, nDims2,...] - array such that it's element at
      each position is true if corresponding linear system is empty, and
      false otherwise.

Examples:
  aMat = [0 1; 0 0]; bMat = eye(2);
  SUBounds = struct();
  SUBounds.center = {'sin(t)'; 'cos(t)'};
  SUBounds.shape = [9 0; 0 2];
  sys = elltool.linsys.LinSysContinuous(aMat, bMat, SUBounds);
  sys.isEmpty()

  ans =

       0
  dsys = elltool.linsys.LinSysDiscrete(aMat, bMat, SUBounds);
  dsys.isEmpty();
\end{verbatim}
\subsection{\texorpdfstring{elltool.linsys.ALinSys.isEqual}{isEqual}}\label{method:elltool.linsys.ALinSys.isEqual}
\begin{verbatim}

ISEQUAL - produces produces logical array the same size as
          self/compLinSysArr (if they have the same).
          isEqualArr[iDim1, iDim2,...] is true if corresponding
          linear systems are equal and false otherwise.

Input:
  regular:
      self: elltool.linsys.ILinSys[nDims1, nDims2,...] -  an array of
           linear systems.
      compLinSysArr: elltool.linsys.ILinSys[nDims1,...nDims2,...] - an
           array of linear systems.

Output:
  isEqualArr: elltool.linsys.LinSys[nDims1, nDims2,...] - an array of
      logical values.
      isEqualArr[iDim1, iDim2,...] is true if corresponding linear systems
      are equal and false otherwise.

Examples:
  aMat = [0 1; 0 0]; bMat = eye(2);
  SUBounds = struct();
  SUBounds.center = {'sin(t)'; 'cos(t)'};
  SUBounds.shape = [9 0; 0 2];
  sys = elltool.linsys.LinSysContinuous(aMat, bMat, SUBounds);
  newSys = sys.getCopy();
  isEqual = sys.isEqual(newSys)

  isEqual =

       1
  dsys = elltool.linsys.LinSysDiscrete(aMat, bMat, SUBounds);
  newDSys = sys.getCopy();
  isEqual = dsys.isEqual(newDSys)

  isEqual =

       1
\end{verbatim}
\subsection{\texorpdfstring{elltool.linsys.ALinSys.isLti}{isLti}}\label{method:elltool.linsys.ALinSys.isLti}
\begin{verbatim}

ISLTI - checks if linear system is time-invariant.

Input:
  regular:
      self: elltool.linsys.LinSys[nDims1, nDims2,...] - an array of linear
            systems.

Output:
  isLtiMat: logical[nDims1, nDims2,...] -array such that it's element at
      each position is true if corresponding linear system is
      time-invariant, and false otherwise.

Examples:
  aMat = [0 1; 0 0]; bMat = eye(2);
  SUBounds = struct();
  SUBounds.center = {'sin(t)'; 'cos(t)'};
  SUBounds.shape = [9 0; 0 2];
  sys = elltool.linsys.LinSysContinuous(aMat, bMat, SUBounds);
  isLtiArr = sys.isLti();
  dsys = elltool.linsys.LinSysDiscrete(aMat, bMat, SUBounds);
  isLtiArr = dsys.isLti();
\end{verbatim}
\section{elltool.linsys.LinSysContinuous}\label{secClassDescr:elltool.linsys.LinSysContinuous}
\subsection{\texorpdfstring{elltool.linsys.LinSysContinuous.LinSysContinuous}{LinSysContinuous}}\label{method:elltool.linsys.LinSysContinuous.LinSysContinuous}
\begin{verbatim}
LINSYSCONTINUOUS - Constructor of continuous linear system object.

Continuous-time linear system:
          dx/dt  =  A(t) x(t)  +  B(t) u(t)  +  C(t) v(t)

Input:
  regular:
      atInpMat: double[nDim, nDim]/cell[nDim, nDim] - matrix A.

      btInpMat: double[nDim, kDim]/cell[nDim, kDim] - matrix B.

  optional:
      uBoundsEll: ellipsoid[1, 1]/struct[1, 1] - control bounds
            ellipsoid.

      ctInpMat: double[nDim, lDim]/cell[nDim, lDim] - matrix G.

      distBoundsEll: ellipsoid[1, 1]/struct[1, 1] - disturbance
            bounds ellipsoid.

      discrFlag: char[1, 1] - if discrFlag set:
             'd' - to discrete-time linSys,
             not 'd' - to continuous-time linSys.


Output:
  self: elltool.linsys.LinSysContinuous[1, 1] - continuous linear
            system.

Example:
  aMat = [0 1; 0 0]; bMat = eye(2);
  SUBounds = struct();
  SUBounds.center = {'sin(t)'; 'cos(t)'};
  SUBounds.shape = [9 0; 0 2];
  sys = elltool.linsys.LinSysContinuous(aMat, bMat, SUBounds);
\end{verbatim}


See the description of the following methods in section \ref{secClassDescr:elltool.linsys.ALinSys}
 for elltool.linsys.ALinSys:

\begin{list}{}{}
 \item \hyperref[method:elltool.linsys.ALinSys.dimension]{dimension}
 \item \hyperref[method:elltool.linsys.ALinSys.display]{display}
 \item \hyperref[method:elltool.linsys.ALinSys.getAbsTol]{getAbsTol}
 \item \hyperref[method:elltool.linsys.ALinSys.getAtMat]{getAtMat}
 \item \hyperref[method:elltool.linsys.ALinSys.getBtMat]{getBtMat}
 \item \hyperref[method:elltool.linsys.ALinSys.getCopy]{getCopy}
 \item \hyperref[method:elltool.linsys.ALinSys.getCtMat]{getCtMat}
 \item \hyperref[method:elltool.linsys.ALinSys.getDistBoundsEll]{getDistBoundsEll}
 \item \hyperref[method:elltool.linsys.ALinSys.getUBoundsEll]{getUBoundsEll}
 \item \hyperref[method:elltool.linsys.ALinSys.hasDisturbance]{hasDisturbance}
 \item \hyperref[method:elltool.linsys.ALinSys.isEmpty]{isEmpty}
 \item \hyperref[method:elltool.linsys.ALinSys.isEqual]{isEqual}
 \item \hyperref[method:elltool.linsys.ALinSys.isLti]{isLti}
\end{list}
\section{elltool.linsys.LinSysDiscrete}\label{secClassDescr:elltool.linsys.LinSysDiscrete}
\subsection{\texorpdfstring{elltool.linsys.LinSysDiscrete.LinSysDiscrete}{LinSysDiscrete}}\label{method:elltool.linsys.LinSysDiscrete.LinSysDiscrete}
\begin{verbatim}
LINSYSDISCRETE - constructor of discrete linear system object.

Discrete-time linear system:
          x[k+1]  =  A[k] x[k]  +  B[k] u[k]  +  C[k] v[k]

Input:
  regular:
      atInpMat: double[nDim, nDim]/cell[nDim, nDim] - matrix A.

      btInpMat: double[nDim, kDim]/cell[nDim, kDim] - matrix B.
  optional:
      uBoundsEll: ellipsoid[1, 1]/struct[1, 1] - control bounds
          ellipsoid.

      ctInpMat: double[nDim, lDim]/cell[nDim, lDim] - matrix G.

      distBoundsEll: ellipsoid[1, 1]/struct[1, 1] - disturbance bounds
          ellipsoid.

      discrFlag: char[1, 1] - if discrFlag set:
           'd' - to discrete-time linSys
           not 'd' - to continuous-time linSys.

Output:
  self: elltool.linsys.LinSysDiscrete[1, 1] - discrete linear system.

Example:
  for k = 1:20
     atMat = {'0' '1 + cos(pi*k/2)'; '-2' '0'};
     btMat =  [0; 1];
     uBoundsEllObj = ellipsoid(4);
     ctMat = [1; 0];
     distBounds = 1/(k+1);
     lsys = elltool.linsys.LinSysDiscrete(atMat, btMat,...
         uBoundsEllObj, ctMat,distBounds);
  end
\end{verbatim}


See the description of the following methods in section \ref{secClassDescr:elltool.linsys.ALinSys}
 for elltool.linsys.ALinSys:

\begin{list}{}{}
 \item \hyperref[method:elltool.linsys.ALinSys.dimension]{dimension}
 \item \hyperref[method:elltool.linsys.ALinSys.display]{display}
 \item \hyperref[method:elltool.linsys.ALinSys.getAbsTol]{getAbsTol}
 \item \hyperref[method:elltool.linsys.ALinSys.getAtMat]{getAtMat}
 \item \hyperref[method:elltool.linsys.ALinSys.getBtMat]{getBtMat}
 \item \hyperref[method:elltool.linsys.ALinSys.getCopy]{getCopy}
 \item \hyperref[method:elltool.linsys.ALinSys.getCtMat]{getCtMat}
 \item \hyperref[method:elltool.linsys.ALinSys.getDistBoundsEll]{getDistBoundsEll}
 \item \hyperref[method:elltool.linsys.ALinSys.getUBoundsEll]{getUBoundsEll}
 \item \hyperref[method:elltool.linsys.ALinSys.hasDisturbance]{hasDisturbance}
 \item \hyperref[method:elltool.linsys.ALinSys.isEmpty]{isEmpty}
 \item \hyperref[method:elltool.linsys.ALinSys.isEqual]{isEqual}
 \item \hyperref[method:elltool.linsys.ALinSys.isLti]{isLti}
\end{list}
\section{elltool.linsys.LinSysFactory}\label{secClassDescr:elltool.linsys.LinSysFactory}
\subsection{\texorpdfstring{elltool.linsys.LinSysFactory.LinSysFactory}{LinSysFactory}}\label{method:elltool.linsys.LinSysFactory.LinSysFactory}
\begin{verbatim}
Factory class of linear system objects of the Ellipsoidal Toolbox.
\end{verbatim}
\subsection{\texorpdfstring{elltool.linsys.LinSysFactory.create}{create}}\label{method:elltool.linsys.LinSysFactory.create}
\begin{verbatim}
CREATE - returns linear system object.

Continuous-time linear system:
          dx/dt  =  A(t) x(t)  +  B(t) u(t)  +  C(t) v(t)

Discrete-time linear system:
          x[k+1]  =  A[k] x[k]  +  B[k] u[k]  +  C[k] v[k]

Input:
  regular:
      atInpMat: double[nDim, nDim]/cell[nDim, nDim] - matrix A.

      btInpMat: double[nDim, kDim]/cell[nDim, kDim] - matrix B.

      uBoundsEll: ellipsoid[1, 1]/struct[1, 1] - control bounds
          ellipsoid.

      ctInpMat: double[nDim, lDim]/cell[nDim, lDim] - matrix G.

      distBoundsEll: ellipsoid[1, 1]/struct[1, 1] - disturbance bounds
          ellipsoid.

      discrFlag: char[1, 1] - if discrFlag set:
          'd' - to discrete-time linSys
          not 'd' - to continuous-time linSys.

Output:
  linSys: elltool.linsys.LinSysContinuous[1, 1]/
      elltool.linsys.LinSysDiscrete[1, 1] - linear system.

Examples:
  aMat = [0 1; 0 0]; bMat = eye(2);
  SUBounds = struct();
  SUBounds.center = {'sin(t)'; 'cos(t)'};
  SUBounds.shape = [9 0; 0 2];
  sys = elltool.linsys.LinSysFactory.create(aMat, bMat,SUBounds);
\end{verbatim}
