\section{Properties}
Functions of the {\it Ellipsoidal Toolbox} can be called with
user-specified values of certain global parameters. System of the parameters
are configured using xml files, which  available from a set of command-line
utilities:
\verbmcodef[Configuration download]
{mcodesnippets/s_chapter06_section01_snippet01.m}
Here we list system parameters available from the 'default' configuration:
\begin{enumerate}
\item {\tt version = '1.4dev'} - current version of {\it ET}.
\item {\tt isVerbose = 0} - makes all the calls to {\it ET}
routines silent, and no information except errors is displayed.
\item {\tt abs\_Tol = 1e-7} - absolute tolerance.
\item {\tt rel\_Tol = 1e-5} - relative tolerance.
\item {\tt nTimeGridPoints = 200} - density of the time grid for the
continuous time reach set computation.
This parameter directly affects the number of ellipsoids to
be stored in the {\tt reach} object.
\item {\tt ODESolverName = ode45} - specifies the ODE solver for continuous time
reach set computation.
\item {\tt isODENormControl = 'on'} - switches on and off the norm control
in the ODE solver. When turned on, it slows down the computation, but improves
the accuracy.
\item {\tt isEnabledOdeSolverOptions = 0} - when set to $0$, calls the ODE solver
without any additional options like norm control. It makes the computation
faster but less accurate. Otherwise, it is assumed to be $1$, and only in this
case the previous option makes a difference.
\item {\tt nPlot2dPoints = 200} - the number of points used to plot a
2D ellipsoid. This parameter also affects the quality of 2D reach tube
and reach set plots.
\item {\tt nPlot3dPoints = 200} - the number of points used to plot
a 3D ellipsoid. This parameter also affects the quality of 3D reach set plots.
\end{enumerate}
Once the configuration is loaded, the system parameters are available through
{\tt elltool.conf.Properties}.
{\tt elltool.conf.Properties} is a static class, providing emulation of static
properties for toolbox. It has two function types: setters and getters.
Using getters we obtain system parameters.
\verbmcodef[Getting parameters]
{mcodesnippets/s_chapter06_section01_snippet02.m}
 Some of the parameters can be changed
in run-time via setters.
\verbmcodef[Changing parameters]
{mcodesnippets/s_chapter06_section01_snippet03.m}




\section{ellipsoid}
The main object of the {\it Ellipsoidal Toolbox}, {\tt ellipsoid}, is
very simple. In accordance with definition \ref{ellipsoiddef}, it contains
two fields:
\begin{itemize}
\item {\tt center} - $n$-dimensional vector specifying the center
of the ellipsoid;
\item {\tt shape} - $(n\times n)$-dimensional symmetric positive
semidefinite matrix.
\end{itemize}
These fields cannot be accessed by the user directly. Their values
can be obtained through {\tt ellipsoid/parameters} function, but they cannot
be modified except by some allowed operation with the {\tt ellipsoid}
object. For the list of {\tt ellipsoid} methods, see appendix A.1.



\section{hyperplane}
According to  definition \ref{hyperplanedef}, the hyperplane object
contains two fields:
\begin{itemize}
\item {\tt normal} - $n$-dimensional vector specifying the normal to
the hyperplane ($c$ in \ref{hyperplane});
\item {\tt shift} - the scalar ($\gamma$ in \ref{hyperplane}).
\end{itemize}
These fields cannot be accessed by the user directly. Their values
can be obtained through {\tt hyperplane/parameters} function, but they cannot
be modified other than by some allowed operation with the {\tt hyperplane}
object. For the list of {\tt hyperplane} methods, see appendix A.2.

In some {\it ET} functions, for example, in {\tt ellipsoid/intersection\_ea}
and {\tt ellipsoid/intersection\_ia}, the hyperplane specifies the halfspace.
It is assumed that the halfspace is
\[ \{ x\in{\bf R}^n ~|~ \langle{\tt normal},x\rangle\leq{\tt shift}\}. \]



\section{linsys}
{\it Ellipsoidal Toolbox} supports both types of linear (affine)
dynamical systems: continuous-time,
\begin{eqnarray*}
\dot{x}(t) & = & A(t)x(t) + B(t)u(t) + G(t)v(t),\\
y(t) & = & C(t)x(t) + w(t);
\end{eqnarray*}
and discrete-time,
\begin{eqnarray*}
x[k+1] & = & A[k]x[k] + B[k]u[k] + G[k]v[k], \\
y[k] & = & C[k]x[k] + w[k].
\end{eqnarray*}
Both can be time-invariant (have constant matrices $A$, $B$, $G$, $C$)
or time-variant.
\newline
Both the {\tt LinSysContinuous} and the {\tt LinSysDiscrete} objects contain the fields:
\begin{itemize}
\item {\tt atInpMat} - $(n\times n)$-dimensional matrix $A$ of type {\tt double}
if constant, or {\tt cell} to symbolically represent $A(t)$ or $A[k]$.
\item {\tt btInpMat} - $(n\times m)$-dimensional matrix $B$, {\tt double} if constant,
{\tt cell} if symbolic.
\item {\tt uBoundsEll} - ellipsoidal bounds on control $u$,
either an {\tt ellipsoid}
object of dimension $m$, or structure {\tt SUBounds} with fields {\tt SUBounds.center} and
{\tt SUBounds.shape} to represent the ellipsoid that depends on time. For example,
\verbmcodef[Definition of the ellipsoidal bounds]
{mcodesnippets/s_chapter06_section04_snippet01.m}

This example defines ellipsoid $\EE(p, P(t))$ with $p=\left[\begin{array}{c}
0\\
1\end{array}\right]$ and $P(t) = \left[\begin{array}{cc}
4 & \cos(t)\\
\cos(t) & 1\end{array}\right]$.
\item {\tt gtInpMat} - $(n\times d)$-dimensional matrix $G$ of type {\tt double}
if constant, or {\tt cell} if symbolic. Can be empty if the system has
no disturbance or affine term.
\item {\tt distBoundsEll} - ellipsoidal bounds on disturbance $v$, either an
{\tt ellipsoid} object of dimension $d$, or structure {\tt V} with
fields {\tt V.center} and {\tt V.shape} for symbolic representation of
ellipsoid, similar to the {\tt control} field.
This field can be also a single $d$-dimensional vector - constant or symbolic -
to represent an affine term.
\item {\tt ctInpMat} - $(r\times n)$-dimensional matrix $C$ of type {\tt double}
if constant, or {\tt cell} if symbolic.
\item {\tt noiseBoundsEll} - ellipsoidal bounds on the noise $w$, either an
{\tt ellipsoid} object of dimension $r$, or structure {\tt W} with
fields {\tt W.center} and {\tt W.shape} for symbolic representation of
ellipsoid, similar to the {\tt control} and {\tt disturbance} fields.

This field can be also a single $r$-dimensional vector, constant or symbolic,
to represent an affine term.

\item {\tt discrFlag} - {\tt 'd'} if the system is discrete-time, not {\tt 'd'}- otherwise.
\end{itemize}
The fields of {\tt LinSysContinuous} or {\tt LinSysDiscrete} object can be accessed but cannot be modified
directly by the user. The only way to modify these fields is through
{\tt linsys/LinSysContinuous} or {\tt linsys/LinSysDiscrete} constructor.

There is {\tt LinSysFactory} - the factory class of linear system objects. Through methods of this class
user creates {\tt LinSysDiscrete} object if {\tt discrFlag} is equal {\tt 'd'} and {\tt LinSysContinuous} object
the other way.

For the list of {\tt linsys} methods, see appendix A.3.



\section{reach}
The {\tt ReachContinuous} object represents the reach (or backward reach)
set of an affine continuous-time system, the {\tt ReachDiscrete} object represents
the reach (or backward reach) set of a discrete-time system . Both contain the fields:
\begin{itemize}
\item {\tt linSys} - the description of the system, for which the reach set
is computed, in the form of {\tt LinSysContinuous} or {\tt LinSysDiscrete} object.

\item {\tt x0Ell} - the set of initial (or terminating, in case of backward
reachability) conditions in the form of {\tt ellipsoid}
object.
\item {\tt dirsMat} - matrix whose columns represent the values
of direction vector $l$ (see chapter 3), for which the ellipsoidal
approximations of the reach set are computed.
\item {\tt timeVec} - time interval, for which the reach set is computed,
is split into the number of segments specified by the 
{\tt nTimeGridPoints} parameter. This field contains the values
of the time grid. If the last value of this array is less than the first 
value of this array, then the reach set is in fact backward reach set.
\end{itemize}
These fields can be accessed and modified only through {\tt ReachContinuous} or
{\tt ReachDiscrete} methods.
For the list of {\tt reach} methods, see appendix A.4.
