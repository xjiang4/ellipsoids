\section{Properties}
Functions of the {\it Ellipsoidal Toolbox} can be called with
user-specified values of certain global parameters. System of the parameters
are configured using xml files, which  available from a set of command-line
utilities:
\verbmcodef[Configuration download]
{mcodesnippets/s_chapter06_section01_snippet01.m}
Here we list system parameters available from the 'default' configuration:
\begin{enumerate}
\item {\tt version = '1.4dev'} - current version of {\it ET}.
\item {\tt isVerbose = 0} - makes all the calls to {\it ET}
routines silent, and no information except errors is displayed.
\item {\tt abs\_tol = 1e-7} - absolute tolerance.
\item {\tt rel\_tol = 1e-5} - relative tolerance.
\item {\tt nTimeGridPoints = 200} - density of the time grid for the
continuous time reach set computation.
This parameter directly affects the number of ellipsoids to
be stored in the {\tt reach} object.
\item {\tt ODESolverName = ode45} - specifies the ODE solver for continuous time
reach set computation.
\item {\tt isODENormControl = 'on'} - switches on and off the norm control
in the ODE solver. When turned on, it slows down the computation, but improves
the accuracy.
\item {\tt isEnabledOdeSolverOptions = 0} - when set to $0$, calls the ODE solver
without any additional options like norm control. It makes the computation
faster but less accurate. Otherwise, it is assumed to be $1$, and only in this
case the previous option makes a difference.
\item {\tt nPlot2dPoints = 200} - the number of points used to plot a
2D ellipsoid. This parameter also affects the quality of 2D reach tube
and reach set plots.
\item {\tt nPlot3dPoints = 200} - the number of points used to plot
a 3D ellipsoid. This parameter also affects the quality of 3D reach set plots.
\end{enumerate}
Once the configuration is loaded, the system parameters are available through
elltool.conf.Properties.
Elltool.conf.Properties is a static class, providing emulation of static
properties for toolbox. It has two function types: setters and getters.
Using getters we obtain system parameters.
\verbmcodef[Getting parameters]
{mcodesnippets/s_chapter06_section01_snippet02.m}
 Some of the parameters can be changed
in run-time via setters.
\verbmcodef[Changing parameters]
{mcodesnippets/s_chapter06_section01_snippet03.m}




\section{ellipsoid}
The main object of the {\it Ellipsoidal Toolbox}, {\tt ellipsoid}, is
very simple. In accordance with definition \ref{ellipsoiddef}, it contains
two fields:
\begin{itemize}
\item {\tt center} - $n$-dimensional vector specifying the center
of the ellipsoid;
\item {\tt shape} - $(n\times n)$-dimensional symmetric positive
semidefinite matrix.
\end{itemize}
These fields cannot be accessed by the user directly. Their values
can be obtained through {\tt ellipsoid/parameters} function, but they cannot
be modified except by some allowed operation with the {\tt ellipsoid}
object. For the list of {\tt ellipsoid} methods, see appendix A.1.



\section{hyperplane}
According to  definition \ref{hyperplanedef}, the hyperplane object
contains two fields:
\begin{itemize}
\item {\tt normal} - $n$-dimensional vector specifying the normal to
the hyperplane ($c$ in \ref{hyperplane});
\item {\tt shift} - the scalar ($\gamma$ in \ref{hyperplane}).
\end{itemize}
These fields cannot be accessed by the user directly. Their values
can be obtained through {\tt hyperplane/parameters} function, but they cannot
be modified other than by some allowed operation with the {\tt hyperplane}
object. For the list of {\tt hyperplane} methods, see appendix A.2.

In some {\it ET} functions, for example, in {\tt ellipsoid/intersection\_ea}
and {\tt ellipsoid/intersection\_ia}, the hyperplane specifies the halfspace.
It is assumed that the halfspace is
\[ \{ x\in{\bf R}^n ~|~ \langle{\tt normal},x\rangle\leq{\tt shift}\}. \]



\section{linsys}
{\it Ellipsoidal Toolbox}  supports both types of linear (affine)
dynamical systems: continuous-time,
\begin{eqnarray*}
\dot{x}(t) & = & A(t)x(t) + B(t)u(t) + G(t)v(t),\\
y(t) & = & C(t)x(t) + D(t)u(t) + w(t);
\end{eqnarray*}
and discrete-time,
\begin{eqnarray*}
x[k+1] & = & A[k]x[k] + B[k]u[k] + G[k]v[k], \\
y[k] & = & C[k]x[k] + D[k]u[k] + w[k].
\end{eqnarray*}
Both can be time-invariant (have constant matrices $A$, $B$, $G$, $C$, $D$)
or time-variant.
\newline
The {\tt linsys} object contains the fields:
\begin{itemize}
\item {\tt A} - $(n\times n)$-dimensional matrix $A$ of type {\tt double}
if constant, or {\tt cell} to symbolically represent $A(t)$ or $A[k]$.
\item {\tt B} - $(n\times m)$-dimensional matrix $B$, {\tt double} if constant,
{\tt cell} if symbolic.
\item {\tt control} - ellipsoidal bounds on control $u$,
either an {\tt ellipsoid}
object of dimension $m$, or structure {\tt uBoundsEllObj} with fields {\tt U.center} and
{\tt U.shape} to represent the ellipsoid that depends on time. For example,
\verbmcodef[Definition of the ellipsoid]
{mcodesnippets/s_chapter06_section04_snippet01.m}

This example defines ellipsoid $\EE(p, P(t))$ with $p=\left[\begin{array}{c}
0\\
1\end{array}\right]$ and $P(t) = \left[\begin{array}{cc}
4 & \cos(t)\\
\cos(t) & 1\end{array}\right]$.
\item {\tt G} - $(n\times d)$-dimensional matrix $G$ of type {\tt double}
if constant, or {\tt cell} if symbolic. Can be empty if the system has
no disturbance or affine term.
\item {\tt disturbance} - ellipsoidal bounds on disturbance $v$, either an
{\tt ellipsoid} object of dimension $d$, or structure {\tt V} with
fields {\tt V.center} and {\tt V.shape} for symbolic representation of
ellipsoid, similar to the {\tt control} field.
This field can be also a single $d$-dimensional vector - constant or symbolic -
to represent an affine term.
\item {\tt C} - $(r\times n)$-dimensional matrix $C$ of type {\tt double}
if constant, or {\tt cell} if symbolic.
\item {\tt D} - $(r\times m)$-dimensional matrix $D$ of type {\tt double}
if constant, or {\tt cell} if symbolic. Can be empty.
\item {\tt noise} - ellipsoidal bounds on the noise $w$, either an
{\tt ellipsoid} object of dimension $r$, or structure {\tt W} with
fields {\tt W.center} and {\tt W.shape} for symbolic representation of
ellipsoid, similar to the {\tt control} and {\tt disturbance} fields.

This field can be also a single $r$-dimensional vector, constant or symbolic,
to represent an affine term.
\item {\tt lti} - $1$ if the system is time-invariant, $0$ - otherwise.
\item {\tt dt} - $1$ if the system is discrete-time, $0$ - otherwise.
\item {\tt constantbounds} - indicates if the bounds on control, disturbance
and noise are constant.
\end{itemize}
The fields of {\tt linsys} object can be accessed but cannot be modified
directly by the user. The only way to modify these fields is through
{\tt linsys/linsys} constructor.
For the list of {\tt linsys} methods, see appendix A.3.



\section{reach}
The {\tt reach} object represents the reach (or backward reach)
set of an affine system. It contains the fields:
\begin{itemize}
\item {\tt system} - the description of the system, for which the reach set
is computed, in the form of {\tt linsys} object.
\item {\tt t0} - initial time value.
\item {\tt X0} - the set of initial (or terminating, in case of backward
reachability) conditions in the form of {\tt ellipsoid}
object.
\item {\tt initial\_directions} - matrix whose columns represent the values
of direction vector $l$ (see chapter 3), for which the ellipsoidal
approximations of the reach set are computed.
\item {\tt time\_values} - time interval, for which the reach set is computed,
is split into the number of segments specified by the global
{\tt ellOptions.time\_grid} parameter. This field contains the values
of the time grid. If the last value of this array is less than the value
of {\tt t0}, then the reach set is in fact backward reach set.
\item {\tt center\_values} - matrix whose columns are values of the reach set
center trajectory evaluated at times specified by {\tt time\_values}.
\item {\tt l\_values} - array of directions vectors evaluated at times
specified by {\tt time\_values}.
\item {\tt ea\_values} - array of the shape matrices of the external ellipsoids
evaluated at times specified by {\tt time\_values}.
\item {\tt ia\_values} - array of the shape matrices of the internal ellipsoids
evaluated at times specified by {\tt time\_values}.
\item {\tt projection\_basis} - if the reach set is projected onto the given
orthonormal basis, the columns of this field are the basis vectors, otherwise,
this field is empty.
\item {\tt calc\_data} - this field is empty unless the reach set is computed
with the {\tt save\_all} option set to $1$. This field then contains the
intermediate calculation data, which can be used for the approximation
refinement. For more detail, see description of the function
{\tt refine} in appendix A.4.
\end{itemize}
These fields can be accessed and modified only through {\tt reach} methods.
For the list of {\tt reach} methods, see appendix A.4.
