\section{List of all functions}
\begin{enumerate}
\item {ellipsoid.ellipsoid}
\fontfamily{pcr}
\selectfont
\begin{lstlisting}
ELLIPSOID - constructor of the ellipsoid object.

  Ellipsoid E = { x in R^n : <(x - q), Q^(-1)(x - q)> <= 1 }, with current
      "Properties". Here q is a vector in R^n, and Q in R^(nxn) is positive
      semi-definite matrix

  ell = ELLIPSOID - Creates an empty ellipsoid

  ell = ELLIPSOID(shMat) - creates an ellipsoid with shape matrix shMat,
      centered at 0

   ell = ELLIPSOID(centVec, shMat) - creates an ellipsoid with shape matrix
      shMat and center centVec

  ell = ELLIPSOID(centVec, shMat, 'propName1', propVal1,...,
      'propNameN',propValN) - creates an ellipsoid with shape
      matrix shMat, center centVec and propName1 = propVal1,...,
      propNameN = propValN. In other cases "Properties"
      are taken from current values stored in
      elltool.conf.Properties.
  ellMat = Ellipsoid(centVecArray, shMatArray,
      ['propName1', propVal1,...,'propNameN',propValN]) -
      creates an array (possibly multidimensional) of
      ellipsoids with centers centVecArray(:,dim1,...,dimn)
      and matrices shMatArray(:,:,dim1,...dimn) with
      properties if given.

  These parameters can be accessed by DOUBLE(E) function call.
  Also, DIMENSION(E) function call returns the dimension of
  the space in which ellipsoid E is defined and the actual
  dimension of the ellipsoid; function ISEMPTY(E) checks if
  ellipsoid E is empty; function ISDEGENERATE(E) checks if
  ellipsoid E is degenerate.

Input:
  Case1:
    regular:
      shMatArray: double [nDim, nDim] /
          double [nDim, nDim, nDim1,...,nDimn] -
          shape matrices array

  Case2:
    regular:
      centVecArray: double [nDim,1] /
          double [nDim, 1, nDim1,...,nDimn] -
          centers array
      shMatArray: double [nDim, nDim] /
          double [nDim, nDim, nDim1,...,nDimn] -
          shape matrices array


  properties:
      absTol: double [1,1] - absolute tolerance with default value 10^(-7)
      relTol: double [1,1] - relative tolerance with default value 10^(-5)
      nPlot2dPoints: double [1,1] - number of points for 2D plot with
          default value 200
      nPlot3dPoints: double [1,1] - number of points for 3D plot with
           default value 200.

Output:
  ellMat: ellipsoid [1,1] / ellipsoid [nDim1,...nDimn] -
      ellipsoid with specified properties
      or multidimensional array of ellipsoids.

Example:
ellObj = ellipsoid([1 0 -1 6]', 9*eye(4));



\end{lstlisting}
\fontfamily{\familydefault}
\selectfont
\item {ellipsoid.checkIsMe}
\fontfamily{pcr}
\selectfont
\begin{lstlisting}
Example:
ellObj = ellipsoid([1; 2], eye(2));
ellipsoid.checkIsMe(ellObj)




\end{lstlisting}
\fontfamily{\familydefault}
\selectfont
\item {ellipsoid.contains}
\fontfamily{pcr}
\selectfont
\begin{lstlisting}
CONTAINS - checks if one ellipsoid contains the other.
           The condition for E1 = firstEllArr to contain
           E2 = secondEllArr is
           min(rho(l | E1) - rho(l | E2)) > 0, subject to <l, l> = 1.

Input:
  regular:
      firstEllArr: ellipsoid [nDims1,nDims2,...,nDimsN]/[1,1] - first
          array of ellipsoids.
      secondEllArr: ellipsoid [nDims1,nDims2,...,nDimsN]/[1,1] - second
          array of ellipsoids.

Output:
  resArr: logical[nDims1,nDims2,...,nDimsN],
      resArr(iCount) = true - firstEllArr(iCount)
      contains secondEllArr(iCount), false - otherwise.

Example:
firstEllObj = ellipsoid([-2; -1], [2 -1; -1 1]);
secEllObj = ellipsoid([-1;0], eye(2));
contains(firstEllObj,secEllObj)

ans =

     0




\end{lstlisting}
\fontfamily{\familydefault}
\selectfont
\item {ellipsoid.contents}
\fontfamily{pcr}
\selectfont
\begin{lstlisting}
Ellipsoid library of the Ellipsoidal Toolbox.


Constructor and data accessing functions:
-----------------------------------------
 ellipsoid    - Constructor of ellipsoid object.
 double       - Returns parameters of ellipsoid, i.e. center and shape
                matrix.
 parameters   - Same function as 'double'(legacy matter).
 dimension    - Returns dimension of ellipsoid and its rank.
 isdegenerate - Checks if ellipsoid is degenerate.
 isempty      - Checks if ellipsoid is empty.
 maxeig       - Returns the biggest eigenvalue of the ellipsoid.
 mineig       - Returns the smallest eigenvalue of the ellipsoid.
 trace        - Returns the trace of the ellipsoid.
 volume       - Returns the volume of the ellipsoid.


Overloaded operators and functions:
-----------------------------------
 eq      - Checks if two ellipsoids are equal.
 ne      - The opposite of 'eq'.
 gt, ge  - E1 > E2 (E1 >= E2) checks if, given the same center ellipsoid
           E1 contains E2.
 lt, le  - E1 < E2 (E1 <= E2) checks if, given the same center ellipsoid
           E2 contains E1.
 mtimes  - Given matrix A in R^(mxn) and ellipsoid E in R^n, returns
           (A * E).
 plus    - Given vector b in R^n and ellipsoid E in R^n, returns (E + b).
 minus   - Given vector b in R^n and ellipsoid E in R^n, returns (E - b).
 uminus  - Changes the sign of the center of ellipsoid.
 display - Displays the details about given ellipsoid object.
 inv     - inverts the shape matrix of the ellipsoid.
 plot    - Plots ellipsoid in 1D, 2D and 3D.


Geometry functions:
-------------------
 move2origin        - Moves the center of ellipsoid to the origin.
 shape              - Same as 'mtimes', but modifies only shape matrix of
                      the ellipsoid leaving its center as is.
 rho                - Computes the value of support function and
                      corresponding boundary point of the ellipsoid in
                      the given direction.
 polar              - Computes the polar ellipsoid to an ellipsoid that
                      contains the origin.
 projection         - Projects the ellipsoid onto a subspace specified
                      by  orthogonal basis vectors.
 minksum            - Computes and plots the geometric (Minkowski) sum of
                      given ellipsoids in 1D, 2D and 3D.
 minksum_ea         - Computes the external ellipsoidal approximation of
                      geometric sum of given ellipsoids in given
                      direction.
 minksum_ia         - Computes the internal ellipsoidal approximation of
                      geometric sum of given ellipsoids in given
                      direction.
 minkdiff           - Computes and plots the geometric (Minkowski)
                      difference of given ellipsoids in 1D, 2D and 3D.
 minkdiff_ea        - Computes the external ellipsoidal approximation of
                      geometric difference of two ellipsoids in given
                      direction.
 minkdiff_ia        - Computes the internal ellipsoidal approximation of
                      geometric difference of two ellipsoids in given
                      direction
 minkpm             - Computes and plots the geometric (Minkowski)
                      difference of a geometric sum of ellipsoids and a
                      single ellipsoid in 1D, 2D and 3D.
 minkpm_ea          - Computes the external ellipsoidal approximation of
                      the geometric difference of a geometric sum of
                      ellipsoids and a single ellipsoid in given
                      direction.
 minkpm_ia          - Computes the internal ellipsoidal approximation of
                      the geometric difference of a geometric sum of
                      ellipsoids and a single ellipsoid in given
                      direction.
 minkmp             - Computes and plots the geometric (Minkowski) sum of
                      a geometric difference of two single ellipsoids and
                      a geometric sum of ellipsoids in 1D, 2D and 3D.
 minkmp_ea          - Computes the external ellipsoidal approximation of
                      the geometric sum of a geometric difference of two
                      single ellipsoids and a geometric sum of ellipsoids
                      in given direction.
minkmp_ia          -  Computes the internal ellipsoidal approximation of
                      the geometric sum of a geometric difference of
                      two single ellipsoids and a geometric sum of
                      ellipsoids in given direction.
 isbaddirection     - Checks if ellipsoidal approximation of geometric
                      difference of two ellipsoids in the given direction
                      can be computed.
 isinside           - Checks if the union or intersection of ellipsoids
                      or polytopes lies inside the intersection of given
                      ellipsoids.
 isinternal         - Checks if given vector belongs to the union or
                      intersection  of given ellipsoids.
 distance           - Computes the distance from ellipsoid to given point,
                      ellipsoid, hyperplane or polytope.
 intersect          - Checks if the union or intersection of ellipsoids
                      intersects  with given ellipsoid, hyperplane,
                      or polytope.
 intersection_ea    - Computes the minimal volume ellipsoid containing
                      intersection of two ellipsoids, ellipsoid and
                      halfspace, or ellipsoid and polytope.
 intersection_ia    - Computes the maximal ellipsoid contained inside the
                      intersection of two ellipsoids, ellipsoid and
                      halfspace  or ellipsoid and polytope.
 ellintersection_ia - Computes maximum volume ellipsoid that is contained
                      in the intersection of given ellipsoids
                      (can be more than 2).
 ellunion_ea        - Computes minimum volume ellipsoid that contains the
                      union of given ellipsoids.
 hpintersection     - Computes the intersection of ellipsoid with
                      hyperplane.



\end{lstlisting}
\fontfamily{\familydefault}
\selectfont
\item {ellipsoid.dimension}
\fontfamily{pcr}
\selectfont
\begin{lstlisting}
 DIMENSION - returns the dimension of the space in which the ellipsoid is
             defined and the actual dimension of the ellipsoid.

Input:
   regular:
     myEllArr: ellipsoid[nDims1,nDims2,...,nDimsN] - array of ellipsoids.

 Output:
   regular:
     dimArr: double[nDims1,nDims2,...,nDimsN] - space dimensions.

   optional:
     rankArr: double[nDims1,nDims2,...,nDimsN] - dimensions of the
            ellipsoids in myEllArr.

 Example:
 firstEllObj = ellipsoid();
 tempMatObj = [3 1; 0 1; -2 1];
 secEllObj = ellipsoid([1; -1; 1], tempMatObj*tempMatObj');
 thirdEllObj = ellipsoid(eye(2));
 fourthEllObj = ellipsoid(0);
 ellMat = [firstEllObj secEllObj; thirdEllObj fourthEllObj];
 [dimMat, rankMat] = ellMat.dimension()

 dimMat =

    0     3
    2     1

 rankMat =

    0     2
    2     0





\end{lstlisting}
\fontfamily{\familydefault}
\selectfont
\item {ellipsoid.disp}
\fontfamily{pcr}
\selectfont
\begin{lstlisting}
DISP - Displays ellipsoid object.

Input:
  regular:
    myEllMat: ellipsoid [mRows, nCols] - matrix of ellipsoids.

Example:
ellObj = ellipsoid([-2; -1], [2 -1; -1 1]);
disp(ellObj)

Ellipsoid with parameters
Center:
    -2
    -1

Shape Matrix:
     2    -1
    -1     1




\end{lstlisting}
\fontfamily{\familydefault}
\selectfont
\item {ellipsoid.display}
\fontfamily{pcr}
\selectfont
\begin{lstlisting}
DISPLAY - Displays the details of the ellipsoid object.

Input:
  regular:
      myEllMat: ellipsoid [mRows, nCols] - matrix of ellipsoids.

Example:
ellObj = ellipsoid([-2; -1], [2 -1; -1 1]);
display(ellObj)

ellObj =

Center:
    -2
    -1

Shape Matrix:
     2    -1
    -1     1

Nondegenerate ellipsoid in R^2.




\end{lstlisting}
\fontfamily{\familydefault}
\selectfont
\item {ellipsoid.distance}
\fontfamily{pcr}
\selectfont
\begin{lstlisting}
DISTANCE - computes distance between the given ellipsoid (or array of
           ellipsoids) to the specified object (or arrays of objects):
           vector, ellipsoid, hyperplane or polytope.

Input:
  regular:
      ellObjArr: ellipsoid [nDims1, nDims2,..., nDimsN] -  array of
         ellipsoids of the same dimension.
      objArray: double / ellipsoid / hyperplane / polytope [nDims1,
          nDims2,..., nDimsN] - array of vectors or ellipsoids or
          hyperplanes or polytopes. If number of elements in objArray
          is more than 1, then it must be equal to the number of elements
          in ellObjArr.

  optional:
      isFlagOn: logical[1,1] - if true then distance is  computed in
          ellipsoidal metric, if false - in Euclidean metric (by default
          isFlagOn=false).

Output:
  regular:
    distValArray: double [nDims1, nDims2,..., nDimsN] - array of pairwise
          calculated distances.
          Negative distance value means
              for ellipsoid and vector: vector belongs to the ellipsoid,
              for ellipsoid and hyperplane: ellipsoid intersects the
                  hyperplane.
              Zero distance value means for ellipsoid and vector: vector
                  is aboundary point of the ellipsoid,
              for ellipsoid and hyperplane: ellipsoid  touches the
                  hyperplane.
  optional:
      statusArray: double [nDims1, nDims2,..., nDimsN] - array of time of
          computation of ellipsoids-vectors or ellipsoids-ellipsoids
          distances, or status of cvx solver for ellipsoids-polytopes
          distances.

Example:
ellObj = ellipsoid([-2; -1], [4 -1; -1 1]);
tempMat = [1 1; 1 -1; -1 1; -1 -1]';
distVec = ellObj.distance(tempMat)

distVec =

     2.3428    1.0855    1.3799    -1.0000




\end{lstlisting}
\fontfamily{\familydefault}
\selectfont
\item {ellipsoid.double}
\fontfamily{pcr}
\selectfont
\begin{lstlisting}
DOUBLE - returns parameters of the ellipsoid.

Input:
  regular:
      myEll: ellipsoid [1, 1] - single ellipsoid of dimention nDims.


Output:
  myEllCentVec: double[nDims, 1] - center of the ellipsoid myEll.

  myEllShMat: double[nDims, nDims] - shape matrix of the ellipsoid myEll.

Example:
ellObj = ellipsoid([-2; -1], [2 -1; -1 1]);
[centVec, shapeMat] = double(ellObj)
centVec =

    -2
    -1


shapeMat =

     2    -1
    -1     1




\end{lstlisting}
\fontfamily{\familydefault}
\selectfont
\item {ellipsoid.ellintersection\_ia}
\fontfamily{pcr}
\selectfont
\begin{lstlisting}
ELLINTERSECTION_IA - computes maximum volume ellipsoid that is contained
                     in the intersection of given ellipsoids.


Input:
  regular:
      inpEllArr: ellipsoid [nDims1,nDims2,...,nDimsN] - array of
          ellipsoids of the same dimentions.

Output:
  outEll: ellipsoid [1, 1] - resulting maximum volume ellipsoid.

Example:
firstEllObj = ellipsoid([-1; 1], [2 0; 0 3]);
secEllObj = ellipsoid([1 2], eye(2);
ellVec = [firstEllObj secEllObj];
resEllObj = ellintersection_ia(ellVec)
resEllObj =

Center:
    0.1847
    1.6914

Shape Matrix:
    0.0340   -0.0607
   -0.0607    0.1713

Nondegenerate ellipsoid in R^2.




\end{lstlisting}
\fontfamily{\familydefault}
\selectfont
\item {ellipsoid.ellunion\_ea}
\fontfamily{pcr}
\selectfont
\begin{lstlisting}
ELLUNION_EA - computes minimum volume ellipsoid that contains union
              of given ellipsoids.

Input:
  regular:
      inpEllMat: ellipsoid [nDims1,nDims2,...,nDimsN] - array of
          ellipsoids of the same dimentions.

Output:
  outEll: ellipsoid [1, 1] - resulting minimum volume ellipsoid.

Example:
firstEllObj = ellipsoid([-1; 1], [2 0; 0 3]);
secEllObj = ellipsoid([1 2], eye(2));
ellVec = [firstEllObj secEllObj];
resEllObj = ellunion_ea(ellVec)
resEllObj =

Center:
   -0.3188
    1.2936

Shape Matrix:
    5.4573    1.3386
    1.3386    4.1037

Nondegenerate ellipsoid in R^2.




\end{lstlisting}
\fontfamily{\familydefault}
\selectfont
\item {ellipsoid.getAbsTol}
\fontfamily{pcr}
\selectfont
\begin{lstlisting}
GETABSTOL - gives the array of absTol for all elements in ellArr

Input:
  regular:
      ellArr: ellipsoid[nDim1, nDim2, ...] - multidimension array
          of ellipsoids
  optional
      fAbsTolFun: function_handle[1,1] - function that apply
          to the absTolArr. The default is @min.

Output:
  regular:
      absTolArr: double [absTol1, absTol2, ...] - return absTol for
          each element in ellArr
  optional:
      absTol: double[1,1] - return result of work fAbsTolFun with
          the absTolArr

Usage:
  use [~,absTol] = ellArr.getAbsTol() if you want get only
      absTol,
  use [absTolArr,absTol] = ellArr.getAbsTol() if you want get
      absTolArr and absTol,
  use absTolArr = ellArr.getAbsTol() if you want get only absTolArr

Example:
firstEllObj = ellipsoid([-1; 1], [2 0; 0 3]);
secEllObj = ellipsoid([1 2], eye(2));
ellVec = [firstEllObj secEllObj];
absTolVec = ellVec.getAbsTol()

absTolVec =

   1.0e-07 *

    1.0000    1.0000




\end{lstlisting}
\fontfamily{\familydefault}
\selectfont
\item {ellipsoid.getCopy}
\fontfamily{pcr}
\selectfont
\begin{lstlisting}
GETCOPY - gives array the same size as ellArr with copies of
          elements of ellArr.

Input:
  regular:
      ellArr: ellipsoid[nDim1, nDim2,...] - multidimensional array
          of ellipsoids.

Output:
  copyEllArr: ellipsoid[nDim1, nDim2,...] - multidimension array of
      copies of elements of ellArr.

Example:
firstEllObj = ellipsoid([-1; 1], [2 0; 0 3]);
secEllObj = ellipsoid([1 2], eye(2));
ellVec = [firstEllObj secEllObj];
copyEllVec = getCopy(ellVec)

copyEllVec =
1x2 array of ellipsoids.




\end{lstlisting}
\fontfamily{\familydefault}
\selectfont
\item {ellipsoid.getNPlot2dPoints}
\fontfamily{pcr}
\selectfont
\begin{lstlisting}
GETNPLOT2DPOINTS - gives value of nPlot2dPoints property
                   of ellipsoids in ellArr

Input:
  regular:
      ellArr: ellipsoid[nDim1, nDim2,...] - mltidimensional array
          of ellipsoids

Output:
      nPlot2dPointsArr: double[nDim1, nDim2,...] - multidimension array
          of nPlot2dPoints property for ellipsoids in ellArr
Example:
firstEllObj = ellipsoid([-1; 1], [2 0; 0 3]);
secEllObj = ellipsoid([1 ;2], eye(2));
ellVec = [firstEllObj secEllObj];
ellVec.getNPlot2dPoints()

ans =

   200   200



\end{lstlisting}
\fontfamily{\familydefault}
\selectfont
\item {ellipsoid.getNPlot3dPoints}
\fontfamily{pcr}
\selectfont
\begin{lstlisting}
GETNPLOT3DPOINTS - gives value of nPlot3dPoints property
                   of ellipsoids in ellArr

Input:
  regular:
      ellArr: ellipsoid[nDim1, nDim2,...] - mltidimensional array
          of ellipsoids

Output:
      nPlot2dPointsArr: double[nDim1, nDim2,...] - multidimension array
          of nPlot3dPoints property for ellipsoids in ellArr

Example:
firstEllObj = ellipsoid([-1; 1], [2 0; 0 3]);
secEllObj = ellipsoid([1 ;2], eye(2));
ellVec = [firstEllObj secEllObj];
ellVec.getNPlot3dPoints()

ans =

   200   200



\end{lstlisting}
\fontfamily{\familydefault}
\selectfont
\item {ellipsoid.getRelTol}
\fontfamily{pcr}
\selectfont
\begin{lstlisting}
GETRELTOL - gives the array of relTol for all elements in ellArr

Input:
  regular:
      ellArr: ellipsoid[nDim1, nDim2, ...] - multidimension array
          of ellipsoids
  optional:
      fRelTolFun: function_handle[1,1] - function that apply
          to the relTolArr. The default is @min.
Output:
  regular:
      relTolArr: double [relTol1, relTol2, ...] - return relTol for
          each element in ellArr
  optional:
      relTol: double[1,1] - return result of work fRelTolFun with
          the relTolArr

Usage:
  use [~,relTol] = ellArr.getRelTol() if you want get only
      relTol,
  use [relTolArr,relTol] = ellArr.getRelTol() if you want get
      relTolArr and relTol,
  use relTolArr = ellArr.getRelTol() if you want get only relTolArr

Example:
firstEllObj = ellipsoid([-1; 1], [2 0; 0 3]);
secEllObj = ellipsoid([1 ;2], eye(2));
ellVec = [firstEllObj secEllObj];
ellVec.getRelTol()

ans =

   1.0e-05 *

    1.0000    1.0000




\end{lstlisting}
\fontfamily{\familydefault}
\selectfont
\item {ellipsoid.hpintersection}
\fontfamily{pcr}
\selectfont
\begin{lstlisting}
HPINTERSECTION - computes the intersection of ellipsoid with hyperplane.

Input:
  regular:
      myEllArr: ellipsoid [nDims1,nDims2,...,nDimsN]/[1,1] - array
          of ellipsoids.
      myHypArr: hyperplane [nDims1,nDims2,...,nDimsN]/[1,1] - array
          of hyperplanes of the same size.

Output:
  intEllArr: ellipsoid [nDims1,nDims2,...,nDimsN] - array of ellipsoids
      resulting from intersections.

  isnIntersectedArr: logical [nDims1,nDims2,...,nDimsN].
      isnIntersectedArr(iCount) = true, if myEllArr(iCount)
      doesn't intersect myHipArr(iCount),
      isnIntersectedArr(iCount) = false, otherwise.

Example:
ellObj = ellipsoid([-2; -1], [4 -1; -1 1]);
hypMat = [hyperplane([0 -1; -1 0]', 1); hyperplane([0 -2; -1 0]', 1)];
ellMat = ellObj.hpintersection(hypMat)

ellMat =
2x2 array of ellipsoids.



\end{lstlisting}
\fontfamily{\familydefault}
\selectfont
\item {ellipsoid.intersect}
\fontfamily{pcr}
\selectfont
\begin{lstlisting}
INTERSECT - checks if the union or intersection of ellipsoids intersects
            given ellipsoid, hyperplane or polytope.

  resArr = INTERSECT(myEllArr, objArr, mode) - Checks if the union
      (mode = 'u') or intersection (mode = 'i') of ellipsoids
      in myEllArr intersects with objects in objArr.
      objArr can be array of ellipsoids, array of hyperplanes,
      or array of polytopes.
      Ellipsoids, hyperplanes or polytopes in objMat must have
      the same dimension as ellipsoids in myEllArr.
      mode = 'u' (default) - union of ellipsoids in myEllArr.
      mode = 'i' - intersection.

  If we need to check the intersection of union of ellipsoids in
  myEllArr (mode = 'u'), or if myEllMat is a single ellipsoid,
  it can be done by calling distance function for each of the
  ellipsoids in myEllArr and objMat, and if it returns negative value,
  the intersection is nonempty. Checking if the intersection of
  ellipsoids in myEllArr (with size of myEllMat greater than 1)
  intersects with ellipsoids or hyperplanes in objArr is more
  difficult. This problem can be formulated as quadratically
  constrained quadratic programming (QCQP) problem.

  Let objArr(iObj) = E(q, Q) be an ellipsoid with center q and shape
  matrix Q. To check if this ellipsoid intersects (or touches) the
  intersection of ellipsoids in meEllArr: E(q1, Q1), E(q2, Q2), ...,
  E(qn, Qn), we define the QCQP problem:
                    J(x) = <(x - q), Q^(-1)(x - q)> --> min
  with constraints:
                     <(x - q1), Q1^(-1)(x - q1)> <= 1   (1)
                     <(x - q2), Q2^(-1)(x - q2)> <= 1   (2)
                     ................................
                     <(x - qn), Qn^(-1)(x - qn)> <= 1   (n)

  If this problem is feasible, i.e. inequalities (1)-(n) do not
  contradict, or, in other words, intersection of ellipsoids
  E(q1, Q1), E(q2, Q2), ..., E(qn, Qn) is nonempty, then we can find
  vector y such that it satisfies inequalities (1)-(n) and minimizes
  function J. If J(y) <= 1, then ellipsoid E(q, Q) intersects or touches
  the given intersection, otherwise, it does not. To check if E(q, Q)
  intersects the union of E(q1, Q1), E(q2, Q2), ..., E(qn, Qn),
  we compute the distances from this ellipsoids to those in the union.
  If at least one such distance is negative,
  then E(q, Q) does intersect the union.

  If we check the intersection of ellipsoids with hyperplane
  objArr = H(v, c), it is enough to check the feasibility
  of the problem
                      1'x --> min
  with constraints (1)-(n), plus
                    <v, x> - c = 0.

  Checking the intersection of ellipsoids with polytope
  objArr = P(A, b) reduces to checking if there any x, satisfying
  constraints (1)-(n) and
                       Ax <= b.

Input:
  regular:
      myEllArr: ellipsoid [nDims1,nDims2,...,nDimsN] - array of
           ellipsoids.
      objArr: ellipsoid / hyperplane /
          / polytope [nDims1,nDims2,...,nDimsN] - array of ellipsoids or
          hyperplanes or polytopes of the same sizes.

  optional:
      mode: char[1, 1] - 'u' or 'i', go to description.

          note: If mode == 'u', then mRows, nCols should be equal to 1.

Output:
  resArr: double[nDims1,nDims2,...,nDimsN] - return:
      resArr(iCount) = -1 in case parameter mode is set
          to 'i' and the intersection of ellipsoids in myEllArr
          is empty.
      resArr(iCount) = 0 if the union or intersection of
          ellipsoids in myEllArr does not intersect the object
          in objArr(iCount).
      resArr(iCount) = 1 if the union or intersection of
          ellipsoids in myEllArr and the object in objArr(iCount)
          have nonempty intersection.
  statusArr: double[0, 0]/double[nDims1,nDims2,...,nDimsN] - status
      variable. statusArr is empty if mode = 'u'.

Example:
firstEllObj = ellipsoid([-2; -1], [4 -1; -1 1]);
secEllObj = firstEllObj + [5; 5];
hypObj  = hyperplane([1; -1]);
ellVec = [firstEllObj secEllObj];
ellVec.intersect(hypObj)

ans =

     1

ellVec.intersect(hypObj, 'i')

ans =

    -1




\end{lstlisting}
\fontfamily{\familydefault}
\selectfont
\item {ellipsoid.intersection\_ea}
\fontfamily{pcr}
\selectfont
\begin{lstlisting}
INTERSECTION_EA - external ellipsoidal approximation of the
                  intersection of two ellipsoids, or ellipsoid and
                  halfspace, or ellipsoid and polytope.

  outEllArr = INTERSECTION_EA(myEllArr, objArr) Given two ellipsoidal
      matrixes of equal sizes, myEllArr and objArr = ellArr, or,
      alternatively, myEllArr or ellMat must be a single ellipsoid,
      computes the ellipsoid that contains the intersection of two
      corresponding ellipsoids from myEllArr and from ellArr.
  outEllArr = INTERSECTION_EA(myEllArr, objArr) Given matrix of
      ellipsoids myEllArr and matrix of hyperplanes objArr = hypArr
      whose sizes match, computes the external ellipsoidal
      approximations of intersections of ellipsoids
      and halfspaces defined by hyperplanes in hypArr.
      If v is normal vector of hyperplane and c - shift,
      then this hyperplane defines halfspace
              <v, x> <= c.
  outEllArr = INTERSECTION_EA(myEllArr, objArr) Given matrix of
      ellipsoids myEllArr and matrix of polytopes objArr = polyArr
      whose sizes match, computes the external ellipsoidal
      approximations of intersections of ellipsoids myEllMat and
      polytopes polyArr.

  The method used to compute the minimal volume overapproximating
  ellipsoid is described in "Ellipsoidal Calculus Based on
  Propagation and Fusion" by Lluis Ros, Assumpta Sabater and
  Federico Thomas; IEEE Transactions on Systems, Man and Cybernetics,
  Vol.32, No.4, pp.430-442, 2002. For more information, visit
  http://www-iri.upc.es/people/ros/ellipsoids.html

  For polytopes this method won't give the minimal volume
  overapproximating ellipsoid, but just some overapproximating ellipsoid.

Input:
  regular:
      myEllArr: ellipsoid [nDims1,nDims2,...,nDimsN]/[1,1] - array
          of ellipsoids.
      objArr: ellipsoid / hyperplane /
          / polytope [nDims1,nDims2,...,nDimsN]/[1,1]  - array of
          ellipsoids or hyperplanes or polytopes of the same sizes.

Example:
firstEllObj = ellipsoid([-2; -1], [4 -1; -1 1]);
secEllObj = firstEllObj + [5; 5];
ellVec = [firstEllObj secEllObj];
thirdEllObj  = ell_unitball(2);
externalEllVec = ellVec.intersection_ea(thirdEllObj)

externalEllVec =
1x2 array of ellipsoids.




\end{lstlisting}
\fontfamily{\familydefault}
\selectfont
\item {ellipsoid.intersection\_ia}
\fontfamily{pcr}
\selectfont
\begin{lstlisting}
INTERSECTION_IA - internal ellipsoidal approximation of the
                  intersection of ellipsoid and ellipsoid,
                  or ellipsoid and halfspace, or ellipsoid
                  and polytope.

  outEllArr = INTERSECTION_IA(myEllArr, objArr) - Given two
      ellipsoidal matrixes of equal sizes, myEllArr and
      objArr = ellArr, or, alternatively, myEllMat or ellMat must be
      a single ellipsoid, comuptes the internal ellipsoidal
      approximations of intersections of two corresponding ellipsoids
      from myEllMat and from ellMat.
  outEllArr = INTERSECTION_IA(myEllArr, objArr) - Given matrix of
      ellipsoids myEllArr and matrix of hyperplanes objArr = hypArr
      whose sizes match, computes the internal ellipsoidal
      approximations of intersections of ellipsoids and halfspaces
      defined by hyperplanes in hypMat.
      If v is normal vector of hyperplane and c - shift,
      then this hyperplane defines halfspace
                 <v, x> <= c.
  outEllArr = INTERSECTION_IA(myEllArr, objArr) - Given matrix of
      ellipsoids  myEllArr and matrix of polytopes objArr = polyArr
      whose sizes match, computes the internal ellipsoidal
      approximations of intersections of ellipsoids myEllArr
      and polytopes polyArr.

  The method used to compute the minimal volume overapproximating
  ellipsoid is described in "Ellipsoidal Calculus Based on
  Propagation and Fusion" by Lluis Ros, Assumpta Sabater and
  Federico Thomas; IEEE Transactions on Systems, Man and Cybernetics,
  Vol.32, No.4, pp.430-442, 2002. For more information, visit
  http://www-iri.upc.es/people/ros/ellipsoids.html

  The method used to compute maximum volume ellipsoid inscribed in
  intersection of ellipsoid and polytope, is modified version of
  algorithm of finding maximum volume ellipsoid inscribed in intersection
  of ellipsoids discribed in Stephen Boyd and Lieven Vandenberghe "Convex
  Optimization". It works properly for nondegenerate ellipsoid, but for
  degenerate ellipsoid result would not lie in this ellipsoid. The result
  considered as empty ellipsoid, when maximum absolute velue of element
  in its matrix is less than myEllipsoid.getAbsTol().

Input:
  regular:
      myEllArr: ellipsoid [nDims1,nDims2,...,nDimsN]/[1,1] - array
          of ellipsoids.
      objArr: ellipsoid / hyperplane /
          / polytope [nDims1,nDims2,...,nDimsN]/[1,1]  - array of
          ellipsoids or hyperplanes or polytopes of the same sizes.

Output:
   outEllArr: ellipsoid [nDims1,nDims2,...,nDimsN] - array of internal
      approximating ellipsoids; entries can be empty ellipsoids
      if the corresponding intersection is empty.

Example:
firstEllObj = ellipsoid([-2; -1], [4 -1; -1 1]);
secEllObj = firstEllObj + [5; 5];
ellVec = [firstEllObj secEllObj];
thirdEllObj  = ell_unitball(2);
internalEllVec = ellVec.intersection_ia(thirdEllObj)

internalEllVec =
1x2 array of ellipsoids.




\end{lstlisting}
\fontfamily{\familydefault}
\selectfont
\item {ellipsoid.inv}
\fontfamily{pcr}
\selectfont
\begin{lstlisting}
INV - inverts shape matrices of ellipsoids in the given array.

  invEllArr = INV(myEllArr)  Inverts shape matrices of ellipsoids
      in the array myEllMat. In case shape matrix is sigular, it is
      regularized before inversion.

Input:
  regular:
    myEllArr: ellipsoid [nDims1,nDims2,...,nDimsN] - array of ellipsoids.

Output:
   invEllArr: ellipsoid [nDims1,nDims2,...,nDimsN] - array of ellipsoids
      with inverted shape matrices.

Example:
ellObj = ellipsoid([1; 1], [4 -1; -1 5]);
ellObj.inv()

ans =

Center:
     1
     1

Shape Matrix:
    0.2632    0.0526
    0.0526    0.2105

Nondegenerate ellipsoid in R^2.



\end{lstlisting}
\fontfamily{\familydefault}
\selectfont
\item {ellipsoid.isEqual}
\fontfamily{pcr}
\selectfont
\begin{lstlisting}
ISEQUAL - produces logical array the same size as
          ell1Arr/ell1Arr (if they have the same).
          isEqualArr[iDim1, iDim2,...] is true if corresponding
          ellipsoids are equal and false otherwise.


Input:
  regular:
      ell1Arr: ellipsoid[nDim1, nDim2,...] - multidimensional array
          of ellipsoids.
      ell2Arr: ellipsoid[nDim1, nDim2,...] - multidimensional array
          of ellipsoids.

Output:
  isEqualArr: logical[nDim1, nDim2,...] - multidimension array of
      logical values. isEqualArr[iDim1, iDim2,...] is true if
      corresponding ellipsoids are equal and false otherwise.

Example:
firstEllObj = ellipsoid([-1; 1], [2 0; 0 3]);
secEllObj = ellipsoid([1 2], eye(2));
isEqual(firstEllObj, secEllObj)

ans =

     0




\end{lstlisting}
\fontfamily{\familydefault}
\selectfont
\item {ellipsoid.isbaddirection}
\fontfamily{pcr}
\selectfont
\begin{lstlisting}
ISBADDIRECTION - checks if ellipsoidal approximations of geometric
                 difference of two ellipsoids can be computed for
                 given directions.
  isBadDirVec = ISBADDIRECTION(fstEll, secEll, dirsMat) - Checks if
      it is possible to build ellipsoidal approximation of the
      geometric difference of two ellipsoids fstEll - secEll in
      directions specified by matrix dirsMat (columns of dirsMat
      are direction vectors). Type 'help minkdiff_ea' or
      'help minkdiff_ia' for more information.

Input:
  regular:
      fstEll: ellipsoid [1, 1] - first ellipsoid. Suppose nDim - space
          dimension.
      secEll: ellipsoid [1, 1] - second ellipsoid of the same dimention.
      dirsMat: numeric[nDims, nCols] - matrix whose columns are
          direction vectors that need to be checked.
      absTol: double [1,1] - absolute tolerance

Output:
   isBadDirVec: logical[1, nCols] - array of true or false with length
      being equal to the number of columns in matrix dirsMat.
      true marks direction vector as bad - ellipsoidal approximation
      cannot be computed for this direction. false means the opposite.

Example:
firstEllObj = ellipsoid([-2; -1], [4 -1; -1 1]);
secEllObj = 3*ell_unitball(2);
dirsMat = [1 0; 1 1; 0 1; -1 1]';
absTol = getAbsTol(secEllObj);
secEllObj.isbaddirection(firstEllObj, dirsMat, absTol)

ans =

     0     1     1     0



\end{lstlisting}
\fontfamily{\familydefault}
\selectfont
\item {ellipsoid.isbigger}
\fontfamily{pcr}
\selectfont
\begin{lstlisting}
ISBIGGER - checks if one ellipsoid would contain the other if their
           centers would coincide.

  isPositive = ISBIGGER(fstEll, secEll) - Given two single ellipsoids
      of the same dimension, fstEll and secEll, check if fstEll
      would contain secEll inside if they were both
      centered at origin.

Input:
  regular:
      fstEll: ellipsoid [1, 1] - first ellipsoid.
      secEll: ellipsoid [1, 1] - second ellipsoid
          of the same dimention.

Output:
  isPositive: logical[1, 1], true - if ellipsoid fstEll
      would contain secEll inside, false - otherwise.

Example:
firstEllObj = ellipsoid([1; 1], eye(2));
secEllObj = ellipsoid([1; 1], [4 -1; -1 5]);
isbigger(firstEllObj, secEllObj)

ans =

     0



\end{lstlisting}
\fontfamily{\familydefault}
\selectfont
\item {ellipsoid.isdegenerate}
\fontfamily{pcr}
\selectfont
\begin{lstlisting}
ISDEGENERATE - checks if the ellipsoid is degenerate.

Input:
  regular:
      myEllArr: ellipsoid[nDims1,nDims2,...,nDimsN] - array of ellipsoids.

Output:
  isPositiveArr: logical[nDims1,nDims2,...,nDimsN],
      isPositiveArr(iCount) = true if ellipsoid myEllMat(iCount)
      is degenerate, false - otherwise.

Example:
ellObj = ellipsoid([1; 1], eye(2));
isdegenerate(ellObj)

ans =

     0



\end{lstlisting}
\fontfamily{\familydefault}
\selectfont
\item {ellipsoid.isempty}
\fontfamily{pcr}
\selectfont
\begin{lstlisting}
ISEMPTY - checks if the ellipsoid object is empty.

Input:
  regular:
      myEllArr: ellipsoid [nDims1,nDims2,...,nDimsN] - array of
           ellipsoids.

Output:
  isPositiveArr: logical[nDims1,nDims2,...,nDimsN],
      isPositiveArr(iCount) = true - if ellipsoid
      myEllMat(iCount) is empty, false - otherwise.

Example:
ellObj = ellipsoid();
isempty(ellObj)

ans =

     1



\end{lstlisting}
\fontfamily{\familydefault}
\selectfont
\item {ellipsoid.isinside}
\fontfamily{pcr}
\selectfont
\begin{lstlisting}
ISINSIDE - checks if the intersection of ellipsoids contains the
           union or intersection of given ellipsoids or polytopes.

  res = ISINSIDE(fstEllArr, secEllArr, mode) Checks if the union
      (mode = 'u') or intersection (mode = 'i') of ellipsoids in
      secEllArr lies inside the intersection of ellipsoids in
      fstEllArr. Ellipsoids in fstEllArr and secEllArr must be
      of the same dimension. mode = 'u' (default) - union of
      ellipsoids in secEllArr. mode = 'i' - intersection.
  res = ISINSIDE(fstEllArr, secPolyArr, mode) Checks if the union
      (mode = 'u') or intersection (mode = 'i')  of polytopes in
      secPolyArr lies inside the intersection of ellipsoids in
      fstEllArr. Ellipsoids in fstEllArr and polytopes in secPolyArr
      must be of the same dimension. mode = 'u' (default) - union of
      polytopes in secPolyMat. mode = 'i' - intersection.

  To check if the union of ellipsoids secEllArr belongs to the
  intersection of ellipsoids fstEllArr, it is enough to check that
  every ellipsoid of secEllMat is contained in every
  ellipsoid of fstEllArr.
  Checking if the intersection of ellipsoids in secEllMat is inside
  intersection fstEllMat can be formulated as quadratically
  constrained quadratic programming (QCQP) problem.

  Let fstEllArr(iEll) = E(q, Q) be an ellipsoid with center q and shape
  matrix Q. To check if this ellipsoid contains the intersection of
  ellipsoids in secObjArr:
  E(q1, Q1), E(q2, Q2), ..., E(qn, Qn), we define the QCQP problem:
                    J(x) = <(x - q), Q^(-1)(x - q)> --> max
  with constraints:
                    <(x - q1), Q1^(-1)(x - q1)> <= 1   (1)
                    <(x - q2), Q2^(-1)(x - q2)> <= 1   (2)
                    ................................
                    <(x - qn), Qn^(-1)(x - qn)> <= 1   (n)

  If this problem is feasible, i.e. inequalities (1)-(n) do not
  contradict, or, in other words, intersection of ellipsoids
  E(q1, Q1), E(q2, Q2), ..., E(qn, Qn) is nonempty, then we can find
  vector y such that it satisfies inequalities (1)-(n)
  and maximizes function J. If J(y) <= 1, then ellipsoid E(q, Q)
  contains the given intersection, otherwise, it does not.

  The intersection of polytopes is a polytope, which is computed
  by the standard routine of MPT. If the vertices of this polytope
  belong to the intersection of ellipsoids, then the polytope itself
  belongs to this intersection.
  Checking if the union of polytopes belongs to the intersection
  of ellipsoids is the same as checking if its convex hull belongs
  to this intersection.

Input:
  regular:
      fstEllArr: ellipsoid [nDims1,nDims2,...,nDimsN] - array of
          ellipsoids of the same size.
      secEllArr: ellipsoid /
          polytope [nDims1,nDims2,...,nDimsN] - array of ellipsoids or
          polytopes of the same sizes.

          note: if mode == 'i', then fstEllArr, secEllVec should be
              array.

  optional:
      mode: char[1, 1] - 'u' or 'i', go to description.

Output:
  res: double[1, 1] - result:
      -1 - problem is infeasible, for example, if s = 'i',
          but the intersection of ellipsoids in E2 is an empty set;
      0 - intersection is empty;
      1 - if intersection is nonempty.
  status: double[0, 0]/double[1, 1] - status variable. status is empty
      if mode == 'u' or mSecRows == nSecCols == 1.

Example:
firstEllObj = ellipsoid([-2; -1], [4 -1; -1 1]);
secEllObj = ell_unitball(2);
firstEllObj.isinside([firstEllObj secEllObj], 'i')

ans =

     1



\end{lstlisting}
\fontfamily{\familydefault}
\selectfont
\item {ellipsoid.isinternal}
\fontfamily{pcr}
\selectfont
\begin{lstlisting}
ISINTERNAL - checks if given points belong to the union or intersection
             of ellipsoids in the given array.

  isPositiveVec = ISINTERNAL(myEllArr,  matrixOfVecMat, mode) - Checks
      if vectors specified as columns of matrix matrixOfVecMat
      belong to the union (mode = 'u'), or intersection (mode = 'i')
      of the ellipsoids in myEllArr. If myEllArr is a single
      ellipsoid, then this function checks if points in matrixOfVecMat
      belong to myEllArr or not. Ellipsoids in myEllArr must be
      of the same dimension. Column size of matrix  matrixOfVecMat
      should match the dimension of ellipsoids.

   Let myEllArr(iEll) = E(q, Q) be an ellipsoid with center q and shape
   matrix Q. Checking if given vector matrixOfVecMat = x belongs
   to E(q, Q) is equivalent to checking if inequality
                   <(x - q), Q^(-1)(x - q)> <= 1
   holds.
   If x belongs to at least one of the ellipsoids in the array, then it
   belongs to the union of these ellipsoids. If x belongs to all
   ellipsoids in the array,
   then it belongs to the intersection of these ellipsoids.
   The default value of the specifier s = 'u'.

   WARNING: be careful with degenerate ellipsoids.

Input:
  regular:
      myEllArr: ellipsoid [nDims1,nDims2,...,nDimsN] - array
          of ellipsoids.
      matrixOfVecMat: double [mRows, nColsOfVec] - matrix which
          specifiy points.

  optional:
      mode: char[1, 1] - 'u' or 'i', go to description.

Output:
   isPositiveVec: logical[1, nColsOfVec] -
      true - if vector belongs to the union or intersection
      of ellipsoids, false - otherwise.

Example:
firstEllObj = ellipsoid([-2; -1], [4 -1; -1 1]);
secEllObj = firstEllObj + [5; 5];
ellVec = [firstEllObj secEllObj];
ellVec.isinternal([-2 3; -1 4], 'i')

ans =

     0     0

ellVec.isinternal([-2 3; -1 4])

ans =

     1     1



\end{lstlisting}
\fontfamily{\familydefault}
\selectfont
\item {ellipsoid.maxeig}
\fontfamily{pcr}
\selectfont
\begin{lstlisting}
MAXEIG - return the maximal eigenvalue of the ellipsoid.

Input:
  regular:
      inpEllArr: ellipsoid [nDims1,nDims2,...,nDimsN] - array of
           ellipsoids.

Output:
  maxEigArr: double[nDims1,nDims2,...,nDimsN] - array of maximal
      eigenvalues of ellipsoids in the input matrix inpEllMat.

Example:
ellObj = ellipsoid([-2; 4], [4 -1; -1 5]);
maxEig = maxeig(ellObj)

maxEig =

    5.6180



\end{lstlisting}
\fontfamily{\familydefault}
\selectfont
\item {ellipsoid.mineig}
\fontfamily{pcr}
\selectfont
\begin{lstlisting}
MINEIG - return the minimal eigenvalue of the ellipsoid.

Input:
   regular:
      inpEllArr: ellipsoid [nDims1,nDims2,...,nDimsN] - array of
        ellipsoids.

Output:
   minEigArr: double[nDims1,nDims2,...,nDimsN] - array of minimal
      eigenvalues of ellipsoids in the input array inpEllMat.

Example:
ellObj = ellipsoid([-2; 4], [4 -1; -1 5]);
minEig = mineig(ellObj)

minEig =

    3.3820



\end{lstlisting}
\fontfamily{\familydefault}
\selectfont
\item {ellipsoid.minkdiff}
\fontfamily{pcr}
\selectfont
\begin{lstlisting}
MINKDIFF - computes geometric (Minkowski) difference of two
           ellipsoids in 2D or 3D.

  MINKDIFF(fstEll, secEll, Options) - Computes geometric difference
      of two ellipsoids fstEll - secEll, if 1 <= dimension(fstEll) =
      = dimension(secEll) <= 3, and plots it if no output arguments
      are specified.
  [centVec, boundPointMat] = MINKDIFF(fstEll, secEll)  Computes
      geometric difference of two ellipsoids fstEll - secEll.
      Here centVec is the center, and boundPointMat - matrix
      whose colums are boundary points.
  MINKDIFF(fstEll, secEll)  Plots geometric difference of two
      ellipsoids fstEll - secEll in default (red) color.
  MINKDIFF(fstEll, secEll, Options)  Plots geometric difference
      fstEll - secEll using options given in the Options structure.

  In order for the geometric difference to be nonempty set,
  ellipsoid fstEll must be bigger than secEll in the sense that
  if fstEll and secEll had the same center, secEll would be
  contained inside fstEll.

Input:
  regular:
      fstEll: ellipsoid [1, 1] - first ellipsoid. Suppose
          nDim - space dimension, nDim = 2 or 3.
      secEll: ellipsoid [1, 1] - second ellipsoid
          of the same dimention.

  optional:
      Options: structure[1, 1] - fields:
          show_all: double[1, 1] - if 1, displays
              also ellipsoids fstEll and secEll.
          newfigure: double[1, 1] - if 1, each plot
              command will open a new figure window.
          fill: double[1, 1] - if 1, the resulting
              set in 2D will be filled with color.
          color: double[1, 3] - sets default colors
              in the form [x y z].
          shade: double[1, 1] = 0-1 - level of transparency
              (0 - transparent, 1 - opaque).

Output:
  centVec: double[nDim, 1]/double[0, 0] - center of the resulting set.
      centVec may be empty if ellipsoid fsrEll isn't bigger
      than secEll.
  boundPointMat: double[nDim, nBoundPoints]/double[0, 0] - set of
      boundary points (vertices) of resulting set. boundPointMat
      may be empty if  ellipsoid fstEll isn't bigger than secEll.

Example:
firstEllObj = ellipsoid([-1; 1], [2 0; 0 3]);
secEllObj = ellipsoid([1 2], eye(2));
[centVec, boundPointMat] = minkdiff(firstEllObj, secEllObj);




\end{lstlisting}
\fontfamily{\familydefault}
\selectfont
\item {ellipsoid.minkdiff\_ea}
\fontfamily{pcr}
\selectfont
\begin{lstlisting}
MINKDIFF_EA - computation of external approximating ellipsoids
              of the geometric difference of two ellipsoids along
              given directions.

  extApprEllVec = MINKDIFF_EA(fstEll, secEll, directionsMat) -
      Computes external approximating ellipsoids of the
      geometric difference of two ellipsoids fstEll - secEll
      along directions specified by columns of matrix directionsMat

  First condition for the approximations to be computed, is that
  ellipsoid fstEll = E1 must be bigger than ellipsoid secEll = E2
  in the sense that if they had the same center, E2 would be contained
  inside E1. Otherwise, the geometric difference E1 - E2
  is an empty set.
  Second condition for the approximation in the given direction l
  to exist, is the following. Given
      P = sqrt(<l, Q1 l>)/sqrt(<l, Q2 l>)
  where Q1 is the shape matrix of ellipsoid E1, and
  Q2 - shape matrix of E2, and R being minimal root of the equation
      det(Q1 - R Q2) = 0,
  parameter P should be less than R.
  If both of these conditions are satisfied, then external
  approximating ellipsoid is defined by its shape matrix
      Q = (Q1^(1/2) + S Q2^(1/2))' (Q1^(1/2) + S Q2^(1/2)),
  where S is orthogonal matrix such that vectors
      Q1^(1/2)l and SQ2^(1/2)l
  are parallel, and its center
      q = q1 - q2,
  where q1 is center of ellipsoid E1 and q2 - center of E2.

Input:
  regular:
      fstEll: ellipsoid [1, 1] - first ellipsoid. Suppose
          nDim - space dimension.
      secEll: ellipsoid [1, 1] - second ellipsoid
          of the same dimention.
      directionsMat: double[nDim, nCols] - matrix whose columns
          specify the directions for which the approximations
          should be computed.

Output:
  extApprEllVec: ellipsoid [1, nCols] - array of external
      approximating ellipsoids (empty, if for all specified
      directions approximations cannot be computed).

Example:
firstEllObj= ellipsoid([-2; -1], [4 -1; -1 1]);
secEllObj = 3*ell_unitball(2);
dirsMat = [1 0; 1 1; 0 1; -1 1]';
externalEllVec = secEllObj.minkdiff_ea(firstEllObj, dirsMat)

externalEllVec =
1x2 array of ellipsoids.




\end{lstlisting}
\fontfamily{\familydefault}
\selectfont
\item {ellipsoid.minkdiff\_ia}
\fontfamily{pcr}
\selectfont
\begin{lstlisting}
MINKDIFF_IA - computation of internal approximating ellipsoids
              of the geometric difference of two ellipsoids along
              given directions.

  intApprEllVec = MINKDIFF_IA(fstEll, secEll, directionsMat) -
      Computes internal approximating ellipsoids of the geometric
      difference of two ellipsoids fstEll - secEll along directions
      specified by columns of matrix directionsMat.

  First condition for the approximations to be computed, is that
  ellipsoid fstEll = E1 must be bigger than ellipsoid secEll = E2
  in the sense that if they had the same center, E2 would be contained
  inside E1. Otherwise, the geometric difference E1 - E2 is an
  empty set. Second condition for the approximation in the given
  direction l to exist, is the following. Given
      P = sqrt(<l, Q1 l>)/sqrt(<l, Q2 l>)
  where Q1 is the shape matrix of ellipsoid E1,
  and Q2 - shape matrix of E2, and R being minimal root of the equation
      det(Q1 - R Q2) = 0,
  parameter P should be less than R.
  If these two conditions are satisfied, then internal approximating
  ellipsoid for the geometric difference E1 - E2 along the
  direction l is defined by its shape matrix
      Q = (1 - (1/P)) Q1 + (1 - P) Q2
  and its center
      q = q1 - q2,
  where q1 is center of E1 and q2 - center of E2.

Input:
  regular:
      fstEll: ellipsoid [1, 1] - first ellipsoid. Suppose
          nDim - space dimension.
      secEll: ellipsoid [1, 1] - second ellipsoid
          of the same dimention.
      directionsMat: double[nDim, nCols] - matrix whose columns
          specify the directions for which the approximations
          should be computed.

Output:
  intApprEllVec: ellipsoid [1, nCols] - array of internal
      approximating ellipsoids (empty, if for all specified directions
      approximations cannot be computed).

Example:
firstEllObj = ellipsoid([-2; -1], [4 -1; -1 1]);
secEllObj = 3*ell_unitball(2);
dirsMat = [1 0; 1 1; 0 1; -1 1]';
internalEllVec = secEllObj.minkdiff_ia(firstEllObj, dirsMat)

internalEllVec =
1x2 array of ellipsoids.



\end{lstlisting}
\fontfamily{\familydefault}
\selectfont
\item {ellipsoid.minkmp}
\fontfamily{pcr}
\selectfont
\begin{lstlisting}
MINKMP - computes and plots geometric (Minkowski) sum of the
         geometric difference of two ellipsoids and the geometric
         sum of n ellipsoids in 2D or 3D:
         (E - Em) + (E1 + E2 + ... + En),
         where E = firstEll, Em = secondEll,
         E1, E2, ..., En - are ellipsoids in sumEllArr

  MINKMP(firstEll, secondEll, sumEllArr, Options) - Computes
      geometric sum of the geometric difference of two ellipsoids
      firstEll - secondEll and the geometric sum of ellipsoids in
      the ellipsoidal array sumEllArr, if
      1 <= dimension(firstEll) = dimension(secondEll) =
      = dimension(sumEllArr) <= 3, and plots it if no output
      arguments are specified.

  [centVec, boundPntMat] = MINKMP(firstEll, secondEll, sumEllArr) -
      computes: (firstEll - secondEll) +
      + (geometric sum of ellipsoids in sumEllArr).
      Here centVec is the center, and
      boundPntMat - array of boundary points.
  MINKMP(firstEll, secondEll, sumEllArr) - plots
      (firstEll - secondEll) +
      +(geometric sum of ellipsoids in sumEllArr)
      in default (red) color.
  MINKMP(firstEll, secondEll, sumEllArr, Options) - plots
      (firstEll - secondEll) +
      +(geometric sum of ellipsoids in sumEllArr)
      using options given in the Options structure.

Input:
  regular:
      firstEll: ellipsoid [1, 1] - first ellipsoid. Suppose
          nDim - space dimension, nDim = 2 or 3.
      secondEll: ellipsoid [1, 1] - second ellipsoid
          of the same dimention.
      sumEllArr: ellipsoid [nDims1, nDims2,...,nDimsN] - array of
          ellipsoids.

  optional:
      Options: structure[1, 1] - fields:
          show_all: double[1, 1] - if 1, displays
              also ellipsoids fstEll and secEll.
          newfigure: double[1, 1] - if 1, each plot
              command will open a new figure window.
          fill: double[1, 1] - if 1, the resulting
              set in 2D will be filled with color.
          color: double[1, 3] - sets default colors
              in the form [x y z].
          shade: double[1, 1] = 0-1 - level of transparency
              (0 - transparent, 1 - opaque).

Output:
  centerVec: double[nDim, 1] - center of the resulting set.
  boundarPointsMat: double[nDim, nBoundPoints] - set of boundary
      points (vertices) of resulting set.

Example:
firstEllObj = ellipsoid([-2; -1], [2 -1; -1 1]);
secEllObj = ell_unitball(2);
ellVec = [firstEllObj secEllObj ellipsoid([-3; 1], eye(2))];
minkmp(firstEllObj, secEllObj, ellVec);



\end{lstlisting}
\fontfamily{\familydefault}
\selectfont
\item {ellipsoid.minkmp\_ea}
\fontfamily{pcr}
\selectfont
\begin{lstlisting}
MINKMP_EA - computation of external approximating ellipsoids
            of (E - Em) + (E1 + ... + En) along given directions.
            where E = fstEll, Em = secEll,
            E1, E2, ..., En - are ellipsoids in sumEllArr

  extApprEllVec = MINKMP_EA(fstEll, secEll, sumEllArr, dirMat) -
      Computes external approximating
      ellipsoids of (E - Em) + (E1 + E2 + ... + En),
      where E1, E2, ..., En are ellipsoids in array sumEllArr,
      E = fstEll, Em = secEll,
      along directions specified by columns of matrix dirMat.

Input:
  regular:
      fstEll: ellipsoid [1, 1] - first ellipsoid. Suppose
          nDims - space dimension.
      secEll: ellipsoid [1, 1] - second ellipsoid
          of the same dimention.
      sumEllArr: ellipsoid [nDims1, nDims2,...,nDimsN] - array of
          ellipsoids of the same dimentions nDims.
      dirMat: double[nDims, nCols] - matrix whose columns specify the
          directions for which the approximations should be computed.

Output:
  extApprEllVec: ellipsoid [1, nCols] - array of external
      approximating ellipsoids (empty, if for all specified
      directions approximations cannot be computed).

Example:
firstEllObj = ellipsoid([-2; -1], [4 -1; -1 1]);
secEllObj = 3*ell_unitball(2);
dirsMat = [1 0; 1 1; 0 1; -1 1]';
bufEllVec = [secEllObj firstEllObj];
externalEllVec = secEllObj.minkmp_ea(firstEllObj, bufEllVec, dirsMat)

externalEllVec =
1x2 array of ellipsoids.



\end{lstlisting}
\fontfamily{\familydefault}
\selectfont
\item {ellipsoid.minkmp\_ia}
\fontfamily{pcr}
\selectfont
\begin{lstlisting}
MINKMP_IA - computation of internal approximating ellipsoids
            of (E - Em) + (E1 + ... + En) along given directions.
            where E = fstEll, Em = secEll,
            E1, E2, ..., En - are ellipsoids in sumEllArr

  intApprEllVec = MINKMP_IA(fstEll, secEll, sumEllArr, dirMat) -
      Computes internal approximating
      ellipsoids of (E - Em) + (E1 + E2 + ... + En),
      where E1, E2, ..., En are ellipsoids in array sumEllArr,
      E = fstEll, Em = secEll,
      along directions specified by columns of matrix dirMat.

Input:
  regular:
      fstEll: ellipsoid [1, 1] - first ellipsoid. Suppose
          nDim - space dimension.
      secEll: ellipsoid [1, 1] - second ellipsoid
          of the same dimention.
      sumEllArr: ellipsoid [nDims1, nDims2,...,nDimsN] - array of
          ellipsoids of the same dimentions.
      dirMat: double[nDim, nCols] - matrix whose columns specify the
          directions for which the approximations should be computed.

Output:
  intApprEllVec: ellipsoid [1, nCols] - array of internal
      approximating ellipsoids (empty, if for all specified
      directions approximations cannot be computed).

Example:
firstEllObj = ellipsoid([-2; -1], [4 -1; -1 1]);
secEllObj = 3*ell_unitball(2);
dirsMat = [1 0; 1 1; 0 1; -1 1]';
bufEllVec = [secEllObj firstEllObj];
internalEllVec = secEllObj.minkmp_ia(firstEllObj, bufEllVec, dirsMat)

internalEllVec =
1x2 array of ellipsoids.



\end{lstlisting}
\fontfamily{\familydefault}
\selectfont
\item {ellipsoid.minkpm}
\fontfamily{pcr}
\selectfont
\begin{lstlisting}
MINKPM - computes and plots geometric (Minkowski) difference
         of the geometric sum of ellipsoids and a single ellipsoid
         in 2D or 3D: (E1 + E2 + ... + En) - E,
         where E = inpEll,
         E1, E2, ... En - are ellipsoids in inpEllArr.

  MINKPM(inpEllArr, inpEll, OPTIONS)  Computes geometric difference
      of the geometric sum of ellipsoids in inpEllArr and
      ellipsoid inpEll, if
      1 <= dimension(inpEllArr) = dimension(inpArr) <= 3,
      and plots it if no output arguments are specified.

  [centVec, boundPointMat] = MINKPM(inpEllArr, inpEll) - computes
      (geometric sum of ellipsoids in inpEllArr) - inpEll.
      Here centVec is the center, and boundPointMat - array
      of boundary points.
  MINKPM(inpEllArr, inpEll) - plots (geometric sum of ellipsoids
      in inpEllArr) - inpEll in default (red) color.
  MINKPM(inpEllArr, inpEll, Options) - plots
      (geometric sum of ellipsoids in inpEllArr) - inpEll using
      options given in the Options structure.

Input:
  regular:
      inpEllArr: ellipsoid [nDims1, nDims2,...,nDimsN] - array of
          ellipsoids of the same dimentions 2D or 3D.
      inpEll: ellipsoid [1, 1] - ellipsoid of the same
          dimention 2D or 3D.

  optional:
      Options: structure[1, 1] - fields:
          show_all: double[1, 1] - if 1, displays
              also ellipsoids fstEll and secEll.
          newfigure: double[1, 1] - if 1, each plot
              command will open a new figure window.
          fill: double[1, 1] - if 1, the resulting
              set in 2D will be filled with color.
          color: double[1, 3] - sets default colors
              in the form [x y z].
          shade: double[1, 1] = 0-1 - level of transparency
              (0 - transparent, 1 - opaque).

Output:
   centVec: double[nDim, 1]/double[0, 0] - center of the resulting set.
      centerVec may be empty.
   boundPointMat: double[nDim, ]/double[0, 0] - set of boundary
      points (vertices) of resulting set. boundPointMat may be empty.

Example:
firstEllObj = ellipsoid([-2; -1], [2 -1; -1 1]);
secEllObj = ell_unitball(2);
thirdEllObj = ell_unitball(2);
ellVec = [firstEllObj secEllObj];
minkpm(ellVec, thirdEllObj);




\end{lstlisting}
\fontfamily{\familydefault}
\selectfont
\item {ellipsoid.minkpm\_ea}
\fontfamily{pcr}
\selectfont
\begin{lstlisting}
MINKPM_EA - computation of external approximating ellipsoids
            of (E1 + E2 + ... + En) - E along given directions.
            where E = inpEll,
            E1, E2, ... En - are ellipsoids in inpEllArr.

  ExtApprEllVec = MINKPM_EA(inpEllArr, inpEll, dirMat) - Computes
      external approximating ellipsoids of
      (E1 + E2 + ... + En) - E, where E1, E2, ..., En are ellipsoids
      in array inpEllArr, E = inpEll,
      along directions specified by columns of matrix dirMat.

Input:
  regular:
      inpEllArr: ellipsoid [nDims1, nDims2,...,nDimsN] -
          array of ellipsoids of the same dimentions.
      inpEll: ellipsoid [1, 1] - ellipsoid of the same dimention.
      dirMat: double[nDim, nCols] - matrix whose columns specify
          the directions for which the approximations
          should be computed.

Output:
  extApprEllVec: ellipsoid [1, nCols]/[0, 0] - array of external
      approximating ellipsoids. Empty, if for all specified
      directions approximations cannot be computed.

Example:
firstEllObj = ellipsoid([2; -1], [9 -5; -5 4]);
secEllObj = ellipsoid([-2; -1], [4 -1; -1 1]);
thirdEllObj = ell_unitball(2);
dirsMat = [1 0; 1 1; 0 1; -1 1]';
ellVec = [thirdEllObj firstEllObj];
externalEllVec = ellVec.minkpm_ea(secEllObj, dirsMat)

externalEllVec =
1x4 array of ellipsoids.



\end{lstlisting}
\fontfamily{\familydefault}
\selectfont
\item {ellipsoid.minkpm\_ia}
\fontfamily{pcr}
\selectfont
\begin{lstlisting}
MINKPM_IA - computation of internal approximating ellipsoids
            of (E1 + E2 + ... + En) - E along given directions.
            where E = inpEll,
            E1, E2, ... En - are ellipsoids in inpEllArr.

  intApprEllVec = MINKPM_IA(inpEllArr, inpEll, dirMat) - Computes
      internal approximating ellipsoids of
      (E1 + E2 + ... + En) - E, where E1, E2, ..., En are ellipsoids
      in array inpEllArr, E = inpEll,
      along directions specified by columns of matrix dirArr.

Input:
  regular:
      inpEllArr: ellipsoid [nDims1, nDims2,...,nDimsN] -
          array of ellipsoids of the same dimentions.
      inpEll: ellipsoid [1, 1] - ellipsoid of the same dimention.
      dirMat: double[nDim, nCols] - matrix whose columns specify
          the directions for which the approximations
          should be computed.

Output:
  intApprEllVec: ellipsoid [1, nCols]/[0, 0] - array of internal
      approximating ellipsoids. Empty, if for all specified
      directions approximations cannot be computed.

Example:
firstEllObj = ellipsoid([2; -1], [9 -5; -5 4]);
secEllObj = ellipsoid([-2; -1], [4 -1; -1 1]);
thirdEllObj = ell_unitball(2);
ellVec = [thirdEllObj firstEllObj];
dirsMat = [1 0; 1 1; 0 1; -1 1]';
internalEllVec = ellVec.minkpm_ia(secEllObj, dirsMat)

internalEllVec =
1x3 array of ellipsoids.



\end{lstlisting}
\fontfamily{\familydefault}
\selectfont
\item {ellipsoid.minksum}
\fontfamily{pcr}
\selectfont
\begin{lstlisting}
MINKSUM - computes geometric (Minkowski) sum of ellipsoids in 2D or 3D.

  MINKSUM(inpEllArr, Options) - Computes geometric sum of ellipsoids
      in the array inpEllArr, if
      1 <= min(dimension(inpEllArr)) = max(dimension(inpEllArr)) <= 3,
      and plots it if no output arguments are specified.

  [centVec, boundPointMat] = MINKSUM(inpEllArr) - Computes
      geometric sum of ellipsoids in inpEllArr. Here centVec is
      the center, and boundPointMat - array of boundary points.
  MINKSUM(inpEllArr) - Plots geometric sum of ellipsoids in
      inpEllArr in default (red) color.
  MINKSUM(inpEllArr, Options) - Plots geometric sum of inpEllMat
      using options given in the Options structure.

Input:
  regular:
      inpEllArr: ellipsoid [nDims1, nDims2,...,nDimsN] - array of
          ellipsoids of the same dimentions 2D or 3D.

  optional:
      Options: structure[1, 1] - fields:
          show_all: double[1, 1] - if 1, displays
              also ellipsoids fstEll and secEll.
          newfigure: double[1, 1] - if 1, each plot
              command will open a new figure window.
          fill: double[1, 1] - if 1, the resulting
              set in 2D will be filled with color.
          color: double[1, 3] - sets default colors
              in the form [x y z].
          shade: double[1, 1] = 0-1 - level of transparency
              (0 - transparent, 1 - opaque).

Output:
  centVec: double[nDim, 1] - center of the resulting set.
  boundPointMat: double[nDim, nBoundPoints] - set of boundary
      points (vertices) of resulting set.

Example:
firstEllObj = ellipsoid([-2; -1], [2 -1; -1 1]);
secEllObj = ell_unitball(2);
ellVec = [firstEllObj, secellObj]
sumVec = minksum(ellVec);


\end{lstlisting}
\fontfamily{\familydefault}
\selectfont
\item {ellipsoid.minksum\_ea}
\fontfamily{pcr}
\selectfont
\begin{lstlisting}
MINKSUM_EA - computation of external approximating ellipsoids
             of the geometric sum of ellipsoids along given directions.

  extApprEllVec = MINKSUM_EA(inpEllArr, dirMat) - Computes
      tight external approximating ellipsoids for the geometric
      sum of the ellipsoids in the array inpEllArr along directions
      specified by columns of dirMat.
      If ellipsoids in inpEllArr are n-dimensional, matrix
      dirMat must have dimension (n x k) where k can be
      arbitrarily chosen.
      In this case, the output of the function will contain k
      ellipsoids computed for k directions specified in dirMat.

  Let inpEllArr consists of E(q1, Q1), E(q2, Q2), ..., E(qm, Qm) -
  ellipsoids in R^n, and dirMat(:, iCol) = l - some vector in R^n.
  Then tight external approximating ellipsoid E(q, Q) for the
  geometric sum E(q1, Q1) + E(q2, Q2) + ... + E(qm, Qm)
  along direction l, is such that
      rho(l | E(q, Q)) = rho(l | (E(q1, Q1) + ... + E(qm, Qm)))
  and is defined as follows:
      q = q1 + q2 + ... + qm,
      Q = (p1 + ... + pm)((1/p1)Q1 + ... + (1/pm)Qm),
  where
      p1 = sqrt(<l, Q1l>), ..., pm = sqrt(<l, Qml>).

Input:
  regular:
      inpEllArr: ellipsoid [nDims1, nDims2,...,nDimsN] - array
          of ellipsoids of the same dimentions.
      dirMat: double[nDims, nCols] - matrix whose columns specify
          the directions for which the approximations
          should be computed.

Output:
  extApprEllVec: ellipsoid [1, nCols] - array of external
      approximating ellipsoids.

Example:
firstEllObj = ellipsoid([-2; -1], [4 -1; -1 1]);
secEllObj = ell_unitball(2);
ellVec = [firstEllObj secEllObj firstEllObj.inv()];
dirsMat = [1 0; 1 1; 0 1; -1 1]';
externalEllVec = ellVec.minksum_ea(dirsMat)

externalEllVec =
1x4 array of ellipsoids.



\end{lstlisting}
\fontfamily{\familydefault}
\selectfont
\item {ellipsoid.minksum\_ia}
\fontfamily{pcr}
\selectfont
\begin{lstlisting}
MINKSUM_IA - computation of internal approximating ellipsoids
             of the geometric sum of ellipsoids along given directions.

  intApprEllVec = MINKSUM_IA(inpEllArr, dirMat) - Computes
      tight internal approximating ellipsoids for the geometric
      sum of the ellipsoids in the array inpEllArr along directions
      specified by columns of dirMat. If ellipsoids in
      inpEllArr are n-dimensional, matrix dirMat must have
      dimension (n x k) where k can be arbitrarily chosen.
      In this case, the output of the function will contain k
      ellipsoids computed for k directions specified in dirMat.

  Let inpEllArr consist of E(q1, Q1), E(q2, Q2), ..., E(qm, Qm) -
  ellipsoids in R^n, and dirMat(:, iCol) = l - some vector in R^n.
  Then tight internal approximating ellipsoid E(q, Q) for the
  geometric sum E(q1, Q1) + E(q2, Q2) + ... + E(qm, Qm) along
  direction l, is such that
      rho(l | E(q, Q)) = rho(l | (E(q1, Q1) + ... + E(qm, Qm)))
  and is defined as follows:
      q = q1 + q2 + ... + qm,
      Q = (S1 Q1^(1/2) + ... + Sm Qm^(1/2))' *
          * (S1 Q1^(1/2) + ... + Sm Qm^(1/2)),
  where S1 = I (identity), and S2, ..., Sm are orthogonal
  matrices such that vectors
  (S1 Q1^(1/2) l), ..., (Sm Qm^(1/2) l) are parallel.

Input:
  regular:
      inpEllArr: ellipsoid [nDims1, nDims2,...,nDimsN] - array
          of ellipsoids of the same dimentions.
      dirMat: double[nDim, nCols] - matrix whose columns specify the
          directions for which the approximations should be computed.

Output:
  intApprEllVec: ellipsoid [1, nCols] - array of internal
      approximating ellipsoids.

Example:
firstEllObj = ellipsoid([-2; -1], [4 -1; -1 1]);
secEllObj = ell_unitball(2);
ellVec = [firstEllObj secEllObj firstEllObj.inv()];
dirsMat = [1 0; 1 1; 0 1; -1 1]';
internalEllVec = ellVec.minksum_ia(dirsMat)

internalEllVec =
1x4 array of ellipsoids.



\end{lstlisting}
\fontfamily{\familydefault}
\selectfont
\item {ellipsoid.minus}
\fontfamily{pcr}
\selectfont
\begin{lstlisting}
MINUS - overloaded operator '-'

  outEllArr = MINUS(inpEllArr, inpVec) implements E(q, Q) - b
      for each ellipsoid E(q, Q) in inpEllArr.
  outEllArr = MINUS(inpVec, inpEllArr) implements b - E(q, Q)
      for each ellipsoid E(q, Q) in inpEllArr.

  Operation E - b where E = inpEll is an ellipsoid in R^n,
  and b = inpVec - vector in R^n. If E(q, Q) is an ellipsoid
  with center q and shape matrix Q, then
  E(q, Q) - b = E(q - b, Q).

Input:
  regular:
      inpEllArr: ellipsoid [nDims1,nDims2,...,nDimsN] - array of
          ellipsoids of the same dimentions nDims.
      inpVec: double[nDims, 1] - vector.

Output:
   outEllVec: ellipsoid [nDims1,nDims2,...,nDimsN] - array of ellipsoids
      with same shapes as inpEllVec, but with centers shifted by vectors
      in -inpVec.

Example:
ellVec  = [ellipsoid([-2; -1], [4 -1; -1 1]) ell_unitball(2)];
outEllVec = ellVec - [1; 1];
outEllVec(1)

ans =

Center:
    -3
    -2

Shape:
     4    -1
    -1     1

Nondegenerate ellipsoid in R^2.

outEllVec(2)

ans =

Center:
    -1
    -1

Shape:
     1     0
     0     1

Nondegenerate ellipsoid in R^2.




\end{lstlisting}
\fontfamily{\familydefault}
\selectfont
\item {ellipsoid.move2origin}
\fontfamily{pcr}
\selectfont
\begin{lstlisting}
MOVE2ORIGIN - moves ellipsoids in the given array to the origin.

  outEllArr = MOVE2ORIGIN(inpEll) - Replaces the centers of
      ellipsoids in inpEllArr with zero vectors.

Input:
  regular:
      inpEllArr: ellipsoid [nDims1,nDims2,...,nDimsN] - array of
          ellipsoids.

Output:
  outEllArr: ellipsoid [nDims1,nDims2,...,nDimsN] - array of ellipsoids
      with the same shapes as in inpEllArr centered at the origin.

Example:
ellObj = ellipsoid([-2; -1], [4 -1; -1 1]);
outEllObj = ellObj.move2origin()

outEllObj =

Center:
     0
     0

Shape:
     4    -1
    -1     1

Nondegenerate ellipsoid in R^2.




\end{lstlisting}
\fontfamily{\familydefault}
\selectfont
\item {ellipsoid.mtimes}
\fontfamily{pcr}
\selectfont
\begin{lstlisting}
MTIMES - overloaded operator '*'.

  Multiplication of the ellipsoid by a matrix or a scalar.
  If inpEllVec(iEll) = E(q, Q) is an ellipsoid, and
  multMat = A - matrix of suitable dimensions,
  then A E(q, Q) = E(Aq, AQA').

Input:
  regular:
      multMat: double[mRows, nDims]/[1, 1] - scalar or
          matrix in R^{mRows x nDim}
      inpEllVec: ellipsoid [1, nCols] - array of ellipsoids.

Output:
  outEllVec: ellipsoid [1, nCols] - resulting ellipsoids.

Example:
ellObj = ellipsoid([-2; -1], [4 -1; -1 1]);
tempMat = [0 1; -1 0];
outEllObj = tempMat*ellObj

outEllObj =

Center:
    -1
     2

Shape:
     1     1
     1     4

Nondegenerate ellipsoid in R^2.



\end{lstlisting}
\fontfamily{\familydefault}
\selectfont
\item {ellipsoid.parameters}
\fontfamily{pcr}
\selectfont
\begin{lstlisting}
PARAMETERS - returns parameters of the ellipsoid.

Input:
  regular:
      myEll: ellipsoid [1, 1] - single ellipsoid of dimention nDims.

Output:
  myEllCenterVec: double[nDims, 1] - center of the ellipsoid myEll.
  myEllShapeMat: double[nDims, nDims] - shape matrix
      of the ellipsoid myEll.

Example:
ellObj = ellipsoid([-2; 4], [4 -1; -1 5]);
[centVec shapeMat] = parameters(ellObj)
centVec =

    -2
     4


shapeMat =

     4    -1
    -1     5




\end{lstlisting}
\fontfamily{\familydefault}
\selectfont
\item {ellipsoid.plot}
\fontfamily{pcr}
\selectfont
\begin{lstlisting}
PLOT - plots ellipsoids in 2D or 3D.


Description:
------------

PLOT(E, OPTIONS) plots ellipsoid E if 1 <= dimension(E) <= 3.

                 PLOT(E)  Plots E in default (red) color.
             PLOT(EA, E)  Plots array of ellipsoids EA and single
                          ellipsoid E.
  PLOT(E1, 'g', E2, 'b')  Plots E1 in green and E2 in blue color.
       PLOT(EA, Options)  Plots EA using options given in the Options
                          structure.


Options.newfigure    - if 1, each plot command will open a new figure
                       window.
Options.fill         - if 1, ellipsoids in 2D will be filled with color.
Options.width        - line width for 1D and 2D plots.
Options.color        - sets default colors in the form [x y z].
Options.shade = 0-1  - level of transparency (0 - transparent, 1 - opaque).


Output:
-------

   None.


See also:
---------

   ELLIPSOID/ELLIPSOID.




\end{lstlisting}
\fontfamily{\familydefault}
\selectfont
\item {ellipsoid.plot3}
\fontfamily{pcr}
\selectfont
\begin{lstlisting}
PLOT3 - plots ellipsoids in 2D or 3D.


Description:
------------

PLOT3(E, OPTIONS) plots ellipsoid E if
                1 <= dimension(E) <= 3.

PLOT3(E)  Plots E in default (red) color.
PLOT3(EA, E)  Plots array of ellipsoids EA and single
              ellipsoid E.
PLOT3(E1, 'g', E2, 'b')  Plots E1 in green and E2
                         in blue color.
      PLOT3(EA, Options)  Plots EA using options given in
                          the Options structure.


Options.newfigure    - if 1, each plot command will open
                       a new figure window.
Options.fill         - if 1, ellipsoids in 2D will be
                       filled with color.
Options.width        - line width for 1D and 2D plots.
Options.color        - sets default colors in the
                       form [x y z].
Options.shade = 0-1  - level of transparency
                       (0 - transparent, 1 - opaque).


Output:
-------

   None.


See also:
---------

   ELLIPSOID/ELLIPSOID, ELLIPSOID/PLOT.




\end{lstlisting}
\fontfamily{\familydefault}
\selectfont
\item {ellipsoid.plus}
\fontfamily{pcr}
\selectfont
\begin{lstlisting}
PLUS - overloaded operator '+'

  outEllArr = PLUS(inpEllArr, inpVec) implements E(q, Q) + b
      for each ellipsoid E(q, Q) in inpEllArr.
  outEllArr = PLUS(inpVec, inpEllArr) implements b + E(q, Q)
      for each ellipsoid E(q, Q) in inpEllArr.

   Operation E + b (or b+E) where E = inpEll is an ellipsoid in R^n,
  and b=inpVec - vector in R^n. If E(q, Q) is an ellipsoid
  with center q and shape matrix Q, then
  E(q, Q) + b = b + E(q,Q) = E(q + b, Q).

Input:
  regular:
      ellArr: ellipsoid [nDims1,nDims2,...,nDimsN] - array of ellipsoids
          of the same dimentions nDims.
      bVec: double[nDims, 1] - vector.

Output:
  outEllArr: ellipsoid [nDims1,nDims2,...,nDimsN] - array of ellipsoids
      with same shapes as ellVec, but with centers shifted by vectors
      in inpVec.

Example:
ellVec  = [ellipsoid([-2; -1], [4 -1; -1 1]) ell_unitball(2)];
outEllVec = ellVec + [1; 1];
outEllVec(1)

ans =

Center:
    -1
     0

Shape:
     4    -1
    -1     1

Nondegenerate ellipsoid in R^2.

outEllVec(2)

ans =

Center:
     1
     1

Shape:
     1     0
     0     1

Nondegenerate ellipsoid in R^2.




\end{lstlisting}
\fontfamily{\familydefault}
\selectfont
\item {ellipsoid.polar}
\fontfamily{pcr}
\selectfont
\begin{lstlisting}
POLAR - computes the polar ellipsoids.

  polEllArr = POLAR(ellArr)  Computes the polar ellipsoids for those
      ellipsoids in ellArr, for which the origin is an interior point.
      For those ellipsoids in E, for which this condition does not hold,
      an empty ellipsoid is returned.

  Given ellipsoid E(q, Q) where q is its center, and Q - its shape matrix,
  the polar set to E(q, Q) is defined as follows:
  P = { l in R^n  | <l, q> + sqrt(<l, Q l>) <= 1 }
  If the origin is an interior point of ellipsoid E(q, Q),
  then its polar set P is an ellipsoid.

Input:
  regular:
      ellArr: ellipsoid [nDims1,nDims2,...,nDimsN] - array
          of ellipsoids.

Output:
  polEllArr: ellipsoid [nDims1,nDims2,...,nDimsN] - array of
       polar ellipsoids.

Example:
ellObj = ellipsoid([4 -1; -1 1]);
ellObj.polar() == ellObj.inv()

ans =

     1



\end{lstlisting}
\fontfamily{\familydefault}
\selectfont
\item {ellipsoid.projection}
\fontfamily{pcr}
\selectfont
\begin{lstlisting}
PROJECTION - computes projection of the ellipsoid onto the given subspace.

  projEllArr = projection(ellArr, basisMat)  Computes projection of the
      ellipsoid ellArr onto a subspace, specified by orthogonal
      basis vectors basisMat. ellArr can be an array of ellipsoids of
      the same dimension. Columns of B must be orthogonal vectors.

Input:
  regular:
      ellArr: ellipsoid [nDims1,nDims2,...,nDimsN] - array
          of ellipsoids.
      basisMat: double[nDim, nSubSpDim] - matrix of orthogonal basis
          vectors

Output:
  projEllArr: ellipsoid [nDims1,nDims2,...,nDimsN] - array of
      projected ellipsoids, generally, of lower dimension.

Example:
ellObj = ellipsoid([-2; -1; 4], [4 -1 0; -1 1 0; 0 0 9]);
basisMat = [0 1 0; 0 0 1]';
outEllObj = ellObj.projection(basisMat)

outEllObj =

Center:
    -1
     4

Shape:
     1     0
     0     9

Nondegenerate ellipsoid in R^2.



\end{lstlisting}
\fontfamily{\familydefault}
\selectfont
\item {ellipsoid.rho}
\fontfamily{pcr}
\selectfont
\begin{lstlisting}
RHO - computes the values of the support function for given ellipsoid
   and given direction.

   supArr = RHO(ellArr, dirsMat)  Computes the support function of the
      ellipsoid ellArr in directions specified by the columns of matrix
      dirsMat. Or, if ellArr is array of ellipsoids, dirsMat is expected
      to be a single vector.

   [supArr, bpMat] = RHO(ellArr, dirstMat)  Computes the support function
      of the ellipsoid ellArr in directions specified by the columns of
      matrix dirsMat, and boundary points bpMat of this ellipsoid that
      correspond to directions in dirsMat. Or, if ellArr is array of
      ellipsoids, and dirsMat - single vector, then support functions and
      corresponding boundary points are computed for all the given
      ellipsoids in the array in the specified direction dirsMat.

   The support function is defined as
  (1)  rho(l | E) = sup { <l, x> : x belongs to E }.
   For ellipsoid E(q,Q), where q is its center and Q - shape matrix,
  it is simplified to
  (2)  rho(l | E) = <q, l> + sqrt(<l, Ql>)
  Vector x, at which the maximum at (1) is achieved is defined by
  (3)  q + Ql/sqrt(<l, Ql>)

Input:
  regular:
      ellArr: ellipsoid [nDims1,nDims2,...,nDimsN]/[1,1] - array
          of ellipsoids.
      dirsMat: double[nDim,nDirs]/[nDim,1] - matrix of directions.

Output:
   supArr: double [nDims1,nDims2,...,nDimsN]/[1,nDirs] - support function
      of the ellArr in directions specified by the columns of matrix
      dirsMat. Or, if ellArr is array of ellipsoids, support function of
      each ellipsoid in ellArr specified by dirsMat direction.

  bpMat: double [nDim,nDims1*nDims2*...*nDimsN]/[nDim,nDirs] - matrix of
      boundary points

Example:
ellObj = ellipsoid([-2; 4], [4 -1; -1 1]);
dirsMat = [-2 5; 5 1];
suppFuncVec = rho(ellObj, dirsMat)

suppFuncVec =

   31.8102    3.5394



\end{lstlisting}
\fontfamily{\familydefault}
\selectfont
\item {ellipsoid.shape}
\fontfamily{pcr}
\selectfont
\begin{lstlisting}
SHAPE - modifies the shape matrix of the ellipsoid without
  changing its center.

   modEllArr = SHAPE(ellArr, modMat)  Modifies the shape matrices of
      the ellipsoids in the ellipsoidal array ellArr. The centers
      remain untouched - that is the difference of the function SHAPE and
      linear transformation modMat*ellArr. modMat is expected to be a
      scalar or a square matrix of suitable dimension.

Input:
  regular:
      ellArr: ellipsoid [nDims1,nDims2,...,nDimsN] - array
          of ellipsoids.
      modMat: double[nDim, nDim]/[1,1] - square matrix or scalar

Output:
   modEllArr: ellipsoid [nDims1,nDims2,...,nDimsN] - array of modified
      ellipsoids.

Example:
ellObj = ellipsoid([-2; -1], [4 -1; -1 1]);
tempMat = [0 1; -1 0];
outEllObj = shape(ellObj, tempMat)

outEllObj =

Center:
    -2
    -1

Shape:
     1     1
     1     4

Nondegenerate ellipsoid in R^2.



\end{lstlisting}
\fontfamily{\familydefault}
\selectfont
\item {ellipsoid.trace}
\fontfamily{pcr}
\selectfont
\begin{lstlisting}
TRACE - returns the trace of the ellipsoid.

   trArr = TRACE(ellArr)  Computes the trace of ellipsoids in
      ellipsoidal array ellArr.

Input:
  regular:
      ellArr: ellipsoid [nDims1,nDims2,...,nDimsN] - array
          of ellipsoids.

Output:
   trArr: double [nDims1,nDims2,...,nDimsN] - array of trace values,
      same size as ellArr.

Example:
firstEllObj = ellipsoid([4 -1; -1 1]);
secEllObj = ell_unitball(2);
ellVec = [firstEllObj secEllObj];
trVec = ellVec.trace()

trVec =

    5     2



\end{lstlisting}
\fontfamily{\familydefault}
\selectfont
\item {ellipsoid.uminus}
\fontfamily{pcr}
\selectfont
\begin{lstlisting}
UMINUS - changes the sign of the center of ellipsoid.

Input:
   regular:
      ellArr: ellipsoid [nDims1,nDims2,...,nDimsN] - array of ellipsoids.


Output:
   outEllArr: ellipsoid [nDims1,nDims2,...,nDimsN] - array of ellipsoids,
       same size as ellArr.

Example:
ellObj = -ellipsoid([-2; -1], [4 -1; -1 1])

ellObj =

Center:
     2
     1

Shape:
     4    -1
    -1     1

Nondegenerate ellipsoid in R^2.




\end{lstlisting}
\fontfamily{\familydefault}
\selectfont
\item {ellipsoid.volume}
\fontfamily{pcr}
\selectfont
\begin{lstlisting}
VOLUME - returns the volume of the ellipsoid.

   volArr = VOLUME(ellArr)  Computes the volume of ellipsoids in
      ellipsoidal array ellArr.

   The volume of ellipsoid E(q, Q) with center q and shape matrix Q
   is given by V = S sqrt(det(Q)) where S is the volume of unit ball.

Input:
  regular:
      ellArr: ellipsoid [nDims1,nDims2,...,nDimsN] - array
          of ellipsoids.

Output:
   volArr: double [nDims1,nDims2,...,nDimsN] - array of
      volume values, same size as ellArr.

Example:
firstEllObj = ellipsoid([4 -1; -1 1]);
secEllObj = ell_unitball(2);
ellVec = [firstEllObj secEllObj]
volVec = ellVec.volume()

volVec =

    5.4414     3.1416



\end{lstlisting}
\fontfamily{\familydefault}
\selectfont
\item {hyperplane.hyperplane}
\fontfamily{pcr}
\selectfont
\begin{lstlisting}
HYPERPLANE - creates hyperplane structure
             (or array of hyperplane structures).

  Hyperplane H = { x in R^n : <v, x> = c },
  with current "Properties"..
  Here v must be vector in R^n, and c - scalar.

  hypH = HYPERPLANE - create empty hyperplane.

  hypH = HYPERPLANE(hypNormVec) - create
      hyperplane object hypH with properties:
          hypH.normal = hypNormVec,
          hypH.shift = 0.

  hypH = HYPERPLANE(hypNormVec, hypConst) - create
      hyperplane object hypH with properties:
          hypH.normal = hypNormVec,
          hypH.shift = hypConst.

  hypH = HYPERPLANE(hypNormVec, hypConst, ...
      'absTol', absTolVal) - create
      hyperplane object hypH with properties:
          hypH.normal = hypNormVec,
          hypH.shift = hypConst.
          hypH.absTol = absTolVal

  hypObjArr = HYPERPLANE(hypNormArr, hypConstArr) - create
      array of hyperplanes object just as
      hyperplane(hypNormVec, hypConst).

  hypObjArr = HYPERPLANE(hypNormArr, hypConstArr, ...
      'absTol', absTolValArr) - create
      array of hyperplanes object just as
      hyperplane(hypNormVec, hypConst, 'absTol', absTolVal).

Input:
  Case1:
    regular:
      hypNormArr: double[hpDims, nDims1, nDims2,...] -
          array of vectors in R^hpDims. There hpDims -
          hyperplane dimension.

  Case2:
    regular:
      hypNormArr: double[hpDims, nCols] /
          / [hpDims, nDims1, nDims2,...] /
          / [hpDims, 1] - array of vectors
          in R^hpDims. There hpDims - hyperplane dimension.
      hypConstArr: double[1, nCols] / [nCols, 1] /
          / [nDims1, nDims2,...] /
          / [nVecArrDim1, nVecArrDim2,...] -
          array of scalar.

  Case3:
    regular:
      hypNormArr: double[hpDims, nCols] /
          / [hpDims, nDims1, nDims2,...] /
          / [hpDims, 1] - array of vectors
          in R^hpDims. There hpDims - hyperplane dimension.
      hypConstArr: double[1, nCols] / [nCols, 1] /
          / [nDims1, nDims2,...] /
          / [nVecArrDim1, nVecArrDim2,...] -
          array of scalar.
      absTolValArr: double[1, 1] - value of
          absTol propeties.

    properties:
      propMode: char[1,] - property mode, the following
          modes are supported:
          'absTol' - name of absTol properties.

          note: if size of hypNormArr is
              [hpDims, nDims1, nDims2,...], then size of
              hypConstArr is [nDims1, nDims2, ...] or
              [1, 1], if size of hypNormArr [hpDims, 1],
              then hypConstArr can be any size
              [nVecArrDim1, nVecArrDim2, ...],
              in this case output variable will has
              size [nVecArrDim1, nVecArrDim2, ...].
              If size of hypNormArr is [hpDims, nCols],
              then size of hypConstArr may be
              [1, nCols] or [nCols, 1],
              output variable will has size
              respectively [1, nCols] or [nCols, 1].

Output:
  hypObjArr: hyperplane [nDims1, nDims2...] /
      / hyperplane [nVecArrDim1, nVecArrDim2, ...] -
      array of hyperplane structure hypH:
          hypH.normal - vector in R^hpDims,
          hypH.shift  - scalar.

Example:
hypNormMat = [1 1 1; 1 1 1];
hypConstVec = [1 -5 0];
hypObj = hyperplane(hypNormMat, hypConstVec);



\end{lstlisting}
\fontfamily{\familydefault}
\selectfont
\item {hyperplane.checkIsMe}
\fontfamily{pcr}
\selectfont
\begin{lstlisting}
CHECKISME - determine whether input object is hyperplane. And display
            message and abort function if input object
            is not hyperplane

Input:
  regular:
      someObjArr: any[] - any type array of objects.

Example:
hypObj = hyperplane([-2, 0]);
hyperplane.checkIsMe(hypObj)



\end{lstlisting}
\fontfamily{\familydefault}
\selectfont
\item {hyperplane.contains}
\fontfamily{pcr}
\selectfont
\begin{lstlisting}
 CONTAINS - checks if given vectors belong to the hyperplanes.

   isPosArr = CONTAINS(myHypArr, xArr) - Checks if vectors specified
       by columns xArr(:, hpDim1, hpDim2, ...) belong
       to hyperplanes in myHypArr.

 Input:
   regular:
       myHypArr: hyperplane [nCols, 1]/[1, nCols]/
           /[hpDim1, hpDim2, ...]/[1, 1] - array of hyperplanes
           of the same dimentions nDims.
       xArr: double[nDims, nCols]/[nDims, hpDim1, hpDim2, ...]/
           /[nDims, 1]/[nDims, nVecArrDim1, nVecArrDim2, ...] - array
           whose columns represent the vectors needed to be checked.

           note: if size of myHypArr is [hpDim1, hpDim2, ...], then
               size of xArr is [nDims, hpDim1, hpDim2, ...]
               or [nDims, 1], if size of myHypArr [1, 1], then xArr
               can be any size [nDims, nVecArrDim1, nVecArrDim2, ...],
               in this case output variable will has
               size [1, nVecArrDim1, nVecArrDim2, ...]. If size of
               xArr is [nDims, nCols], then size of myHypArr may be
               [nCols, 1] or [1, nCols] or [1, 1], output variable
               will has size respectively
               [nCols, 1] or [1, nCols] or [nCols, 1].

 Output:
   isPosArr: logical[hpDim1, hpDim2,...] /
       / logical[1, nVecArrDim1, nVecArrDim2, ...],
       isPosArr(iDim1, iDim2, ...) = true - myHypArr(iDim1, iDim2, ...)
       contains xArr(:, iDim1, iDim2, ...), false - otherwise.

Example:
 hypObj = hyperplane([-1; 1]);
 tempMat = [100 -1 2; 100 1 2];
 hypObj.contains(tempMat)

 ans =

      1
      0
      1




\end{lstlisting}
\fontfamily{\familydefault}
\selectfont
\item {hyperplane.contents}
\fontfamily{pcr}
\selectfont
\begin{lstlisting}
Hyperplane object of the Ellipsoidal Toolbox.


Functions:
----------
 hyperplane - Constructor of hyperplane object.
 double     - Returns parameters of hyperplane, i.e. normal vector and
              shift.
 parameters - Same function as 'double' (legacy matter).
 dimension  - Returns dimension of hyperplane.
 isempty    - Checks if hyperplane is empty.
 isparallel - Checks if one hyperplane is parallel to the other one.
 contains   - Check if hyperplane contains given point.


Overloaded operators and functions:
-----------------------------------
 eq      - Checks if two hyperplanes are equal.
 ne      - The opposite of 'eq'.
 uminus  - Switches signs of normal and shift parameters to the opposite.
 display - Displays the details about given hyperplane object.
 plot    - Plots hyperplane in 2D and 3D.




\end{lstlisting}
\fontfamily{\familydefault}
\selectfont
\item {hyperplane.dimension}
\fontfamily{pcr}
\selectfont
\begin{lstlisting}
DIMENSION - returns dimensions of hyperplanes in the array.

  dimsArr = DIMENSION(hypArr) - returns dimensions of hyperplanes
      described by hyperplane structures in the array hypArr.

Input:
  regular:
      hypArr: hyperplane [nDims1, nDims2, ...] - array
          of hyperplanes.

Output:
      dimsArr: double[nDims1, nDims2, ...] - dimensions
          of hyperplanes.

Example:
firstHypObj = hyperplane([-1; 1]);
secHypObj = hyperplane([-1; 1; 8; -2; 3], 7);
thirdHypObj = hyperplane([1; 2; 0], -1);
hypVec = [firstHypObj secHypObj thirdHypObj];
dimsVec  = hypVec.dimension()

dimsVec =

   2     5     3



\end{lstlisting}
\fontfamily{\familydefault}
\selectfont
\item {hyperplane.display}
\fontfamily{pcr}
\selectfont
\begin{lstlisting}
DISPLAY - Displays hyperplane object.

Input:
  regular:
      myHypArr: hyperplane [hpDim1, hpDim2, ...] - array
          of hyperplanes.

Example:
hypObj = hyperplane([-1; 1]);
display(hypObj)

hypObj =
size: [1 1]

Element: [1 1]
Normal:
    -1
     1

Shift:
     0

Hyperplane in R^2.



\end{lstlisting}
\fontfamily{\familydefault}
\selectfont
\item {hyperplane.double}
\fontfamily{pcr}
\selectfont
\begin{lstlisting}
DOUBLE - return parameters of hyperplane - normal vector and shift.

  [normVec, hypScal] = DOUBLE(myHyp) - returns normal vector
      and scalar value of the hyperplane.

Input:
  regular:
      myHyp: hyperplane [1, 1] - single hyperplane of dimention nDims.

Output:
  normVec: double[nDims, 1] - normal vector of the hyperplane myHyp.
  hypScal: double[1, 1] - scalar of the hyperplane myHyp.

Example:
hypObj = hyperplane([-1; 1]);
[normVec, hypScal] = double(hypObj)

normVec =

    -1
     1


hypScal =

     0



\end{lstlisting}
\fontfamily{\familydefault}
\selectfont
\item {hyperplane.getAbsTol}
\fontfamily{pcr}
\selectfont
\begin{lstlisting}
GETABSTOL - gives the array of absTol for all elements in hplaneArr

Input:
  regular:
      ellArr: hyperplane[nDim1, nDim2, ...] - multidimension array
          of hyperplane
  optional
      fAbsTolFun: function_handle[1,1] - function that apply
          to the absTolArr. The default is @min.

Output:
  regular:
      absTolArr: double [absTol1, absTol2, ...] - return absTol for
          each element in hplaneArr
  optional:
      absTol: double[1, 1] - return result of work fAbsTolFun with
          the absTolArr

Usage:
  use [~,absTol] = hplaneArr.getAbsTol() if you want get only
      absTol,
  use [absTolArr,absTol] = hplaneArr.getAbsTol() if you want get
      absTolArr and absTol,
  use absTolArr = hplaneArr.getAbsTol() if you want get only absTolArr

Example:
firstHypObj = hyperplane([-1; 1]);
secHypObj = hyperplane([-2; 5]);
hypVec = [firstHypObj secHypObj];
hypVec.getAbsTol()

ans =

   1.0e-07 *

    1.0000    1.0000



\end{lstlisting}
\fontfamily{\familydefault}
\selectfont
\item {hyperplane.isempty}
\fontfamily{pcr}
\selectfont
\begin{lstlisting}
ISEMPTY - checks if hyperplanes in H are empty.

Input:
  regular:
      myHypArr: hyperplane [nDims1, nDims2, ...] - array
          of hyperplanes.

Output:
  isPositiveArr: logical[nDims1, nDims2, ...],
      isPositiveArr(iDim1, iDim2, ...) = true - if ellipsoid
      myHypArr(iDim1, iDim2, ...) is empty, false - otherwise.

Example:
hypObj = hyperplane();
isempty(hypObj)

ans =

     1



\end{lstlisting}
\fontfamily{\familydefault}
\selectfont
\item {hyperplane.isparallel}
\fontfamily{pcr}
\selectfont
\begin{lstlisting}
ISPARALLEL - check if two hyperplanes are parallel.

  isResArr = ISPARALLEL(fstHypArr, secHypArr) - Checks if hyperplanes
      in fstHypArr are parallel to hyperplanes in secHypArr and
      returns array of true and false of the size corresponding
      to the sizes of fstHypArr and secHypArr.

Input:
  regular:
      fstHypArr: hyperplane [nDims1, nDims2, ...] - first array
          of hyperplanes
      secHypArr: hyperplane [nDims1, nDims2, ...] - second array
          of hyperplanes

Output:
  isPosArr: logical[nDims1, nDims2, ...] -
      isPosArr(iFstDim, iSecDim, ...) = true -
      if fstHypArr(iFstDim, iSecDim, ...) is parallel
      secHypArr(iFstDim, iSecDim, ...), false - otherwise.

Example:
hypObj = hyperplane([-1 1 1; 1 1 1; 1 1 1], [2 1 0]);
hypObj.isparallel(hypObj(2))

ans =

     0     1     1




\end{lstlisting}
\fontfamily{\familydefault}
\selectfont
\item {hyperplane.parameters}
\fontfamily{pcr}
\selectfont
\begin{lstlisting}
PARAMETERS - return parameters of hyperplane - normal vector and shift.

  [normVec, hypScal] = PARAMETERS(myHyp) - returns normal vector
      and scalar value of the hyperplane.

Input:
  regular:
      myHyp: hyperplane [1, 1] - single hyperplane of dimention nDims.

Output:
  normVec: double[nDims, 1] - normal vector of the hyperplane myHyp.
  hypScal: double[1, 1] - scalar of the hyperplane myHyp.

Example:
hypObj = hyperplane([-1; 1]);
[normVec, hypScal] = parameters(hypObj)

normVec =

    -1
     1


hypScal =

     0



\end{lstlisting}
\fontfamily{\familydefault}
\selectfont
\item {hyperplane.plot}
\fontfamily{pcr}
\selectfont
\begin{lstlisting}
PLOT - plots hyperplanes in 2D or 3D.


Usage:
      plot(h) - plots hyperplane H in default (red) color.
      plot(hM) -plots hyperplanes contained in hyperplane matrix.
      plot(hM1, 'cSpec1', hM2, 'cSpec1',...) - plots hyperplanes in h1 in
          cSpec1 color, hyperplanes in h2 in cSpec2 color, etc.
      plot(hM1, hM2,..., hMn, option) - plots h1,...,hn using options
          given in the option structure.

Input:
  regular:
      hMat: hyperplane[m,n] - matrix of 2D or 3D hyperplanes. All
            hyperplanes in hM must be either 2D or 3D simutaneously.
  optional:
      colorSpec: char[1,1] - specify wich color hyperplane plots will
                 have
      option: structure[1,1], containing some of follwing fields:
          option.newfigure: boolean[1,1]   - if 1, each plot command will
                                             open a new figure window.
          option.size: double[1,1] - length of the line segment in 2D, or
                                     square diagonal in 3D.
          option.center: double[1,1] - center of the line segment in 2D,
                                       of the square in 3D.
          option.width: double[1,1] - specifies the width (in points) of
                                      the line for 2D plots.
          option.color: double[1,3] - sets default colors in the form
                                      [x y z], .
          option.shade = 0-1 - level of transparency (0 - transparent, 1
                               - opaque).
          NOTE: if using options and colorSpec simutaneously, option.color
          is ignored

Output:
  regular:
      figHandleVec: double[1,n] - array with handles of figures
      hyperplanes were plotted in. Where n is number of figures.




\end{lstlisting}
\fontfamily{\familydefault}
\selectfont
\item {hyperplane.uminus}
\fontfamily{pcr}
\selectfont
\begin{lstlisting}
UMINUS - switch signs of normal vector and the shift scalar
         to the opposite.

Input:
  regular:
      inpHypArr: hyperplane [nDims1, nDims2, ...] - array
          of hyperplanes.

Output:
  outHypArr: hyperplane [nDims1, nDims2, ...] - array
      of the same hyperplanes as in inpHypArr whose
      normals and scalars are multiplied by -1.

Example:
hypObj = -hyperplane([-1; 1], 1)

hypObj =
size: [1 1]

Element: [1 1]
Normal:
     1
    -1

Shift:
    -1

Hyperplane in R^2.



\end{lstlisting}
\fontfamily{\familydefault}
\selectfont
\item {elltool.conf.Properties.parseProp}
\fontfamily{pcr}
\selectfont
\begin{lstlisting}
PARSEPROP - parses input into cell array with values of properties listed
            in neededPropNameList.
            Values are  taken from args or, if there no value for some
            property in args, in current Properties.


 Input:
   regular:
       args:cell[1,] - cell array of arguments that should be parsed.
       neededPropNameList:cell[1,] or empty cell - cell array of strings,
       containing names of parameters, that output should consist of.
       Possible properties:
               version
               isVerbose
               absTol
               relTol
               nTimeGridPoints
               ODESolverName
               isODENormControl
               isEnabledOdeSolverOptions
               nPlot2dPoints
               nPlot3dPoints
           trying to specify other properties would be regarded as an
           error.

 Output:
   varargout:cell array[1,] - cell array with values of properties listed
                              in neededPropNameList in the same order as
                              they listed in neededPropNameList

 Example:
 testAbsTol = 1;
 testRelTol = 2;
 nPlot2dPoints = 3;
 someArg = 4;
 args = {'absTol',testAbsTol, 'relTol',testRelTol,'nPlot2dPoints',nPlot2dPoints, 'someOtherArg', someArg};
 neededProp = {'absTol','relTol'};
 [absTol, relTol] = elltool.conf.Properties.parseProp(args,neededProp)

 absTol =

      1


 relTol =

      2



\end{lstlisting}
\fontfamily{\familydefault}
\selectfont
\item {elltool.conf.Properties.getPropStruct}
\fontfamily{pcr}
\selectfont
\begin{lstlisting}
Example:
elltool.conf.Properties.getConfRepoMgr.getCurConf()

ans =

                  version: '1.4dev'
                isVerbose: 0
                   absTol: 1.0000e-07
                   relTol: 1.0000e-05
          nTimeGridPoints: 200
            ODESolverName: 'ode45'
         isODENormControl: 'on'
isEnabledOdeSolverOptions: 0
            nPlot2dPoints: 200
            nPlot3dPoints: 200
                  logging: [1x1 struct]




\end{lstlisting}
\fontfamily{\familydefault}
\selectfont
\item {elltool.conf.Properties.setNTimeGridPoints}
\fontfamily{pcr}
\selectfont
\begin{lstlisting}
Example:
elltool.conf.Properties.setNTimeGridPoints(300);




\end{lstlisting}
\fontfamily{\familydefault}
\selectfont
\item {elltool.conf.Properties.setNPlot2dPoints}
\fontfamily{pcr}
\selectfont
\begin{lstlisting}
Example:
elltool.conf.Properties.setNPlot2dPoints(300);




\end{lstlisting}
\fontfamily{\familydefault}
\selectfont
\item {elltool.conf.Properties.setIsVerbose}
\fontfamily{pcr}
\selectfont
\begin{lstlisting}
Example:
elltool.conf.Properties.setIsVerbose(true);




\end{lstlisting}
\fontfamily{\familydefault}
\selectfont
\item {elltool.conf.Properties.getNPlot3dPoints}
\fontfamily{pcr}
\selectfont
\begin{lstlisting}
Example:
elltool.conf.Properties.getNPlot3dPoints();




\end{lstlisting}
\fontfamily{\familydefault}
\selectfont
\item {elltool.conf.Properties.getNPlot2dPoints}
\fontfamily{pcr}
\selectfont
\begin{lstlisting}
Example:
elltool.conf.Properties.getNPlot2dPoints();




\end{lstlisting}
\fontfamily{\familydefault}
\selectfont
\item {elltool.conf.Properties.getIsEnabledOdeSolverOptions}
\fontfamily{pcr}
\selectfont
\begin{lstlisting}
Example:
elltool.conf.Properties.getIsEnabledOdeSolverOptions();




\end{lstlisting}
\fontfamily{\familydefault}
\selectfont
\item {elltool.conf.Properties.getIsODENormControl}
\fontfamily{pcr}
\selectfont
\begin{lstlisting}
Example:
elltool.conf.Properties.getIsODENormControl();




\end{lstlisting}
\fontfamily{\familydefault}
\selectfont
\item {elltool.conf.Properties.getODESolverName}
\fontfamily{pcr}
\selectfont
\begin{lstlisting}
Example:
elltool.conf.Properties.getODESolverName();




\end{lstlisting}
\fontfamily{\familydefault}
\selectfont
\item {elltool.conf.Properties.getNTimeGridPoints}
\fontfamily{pcr}
\selectfont
\begin{lstlisting}
Example:
elltool.conf.Properties.getNTimeGridPoints();




\end{lstlisting}
\fontfamily{\familydefault}
\selectfont
\item {elltool.conf.Properties.getRelTol}
\fontfamily{pcr}
\selectfont
\begin{lstlisting}
Example:
elltool.conf.Properties.getRelTol();




\end{lstlisting}
\fontfamily{\familydefault}
\selectfont
\item {elltool.conf.Properties.getAbsTol}
\fontfamily{pcr}
\selectfont
\begin{lstlisting}
Example:
elltool.conf.Properties.getAbsTol();




\end{lstlisting}
\fontfamily{\familydefault}
\selectfont
\item {elltool.conf.Properties.getIsVerbose}
\fontfamily{pcr}
\selectfont
\begin{lstlisting}
Example:
elltool.conf.Properties.getIsVerbose();




\end{lstlisting}
\fontfamily{\familydefault}
\selectfont
\item {elltool.conf.Properties.getVersion}
\fontfamily{pcr}
\selectfont
\begin{lstlisting}
Example:
elltool.conf.Properties.getVersion();




\end{lstlisting}
\fontfamily{\familydefault}
\selectfont
\item {elltool.conf.Properties.setConfRepoMgr}
\fontfamily{pcr}
\selectfont
\begin{lstlisting}
Example:
 prevConfRepo = Properties.getConfRepoMgr();
 prevAbsTol = prevConfRepo.getParam('absTol');
 elltool.conf.Properties.setConfRepoMgr(prevConfRepo);




\end{lstlisting}
\fontfamily{\familydefault}
\selectfont
\item {elltool.conf.Properties.getConfRepoMgr}
\fontfamily{pcr}
\selectfont
\begin{lstlisting}
Example:
elltool.conf.Properties.getConfRepoMgr()

 ans =

   elltool.conf.ConfRepoMgr handle
   Package: elltool.conf

   Properties:
     DEFAULT_STORAGE_BRANCH_KEY: '_default'




\end{lstlisting}
\fontfamily{\familydefault}
\selectfont
\item {elltool.conf.Properties.init}
\fontfamily{pcr}
\selectfont
\begin{lstlisting}
Example:
 elltool.conf.Properties.init()



\end{lstlisting}
\fontfamily{\familydefault}
\selectfont
\item {elltool.conf.Properties.checkSettings}
\fontfamily{pcr}
\selectfont
\begin{lstlisting}
Example:
elltool.conf.Properties.checkSettings()




\end{lstlisting}
\fontfamily{\familydefault}
\selectfont
\item {elltool.conf.Properties.Properties}
\fontfamily{pcr}
\selectfont
\begin{lstlisting}
PROPERTIES - a static class, providing emulation of static properties for
             toolbox.



\end{lstlisting}
\fontfamily{\familydefault}
\selectfont
\item {elltool.core.GenEllipsoid.getIsGoodDir}
\fontfamily{pcr}
\selectfont
\begin{lstlisting}
Example:
firstEllObj = elltool.core.GenEllipsoid([10;0], 2*eye(2));
secEllObj = elltool.core.GenEllipsoid([0;0], [1 0; 0 0.1]);
curDirMat = [1; 0];
isOk=getIsGoodDir(firstEllObj,secEllObj,dirsMat)

isOk =

     1




\end{lstlisting}
\fontfamily{\familydefault}
\selectfont
\item {elltool.core.GenEllipsoid.display}
\fontfamily{pcr}
\selectfont
\begin{lstlisting}
Example:
ellObj = elltool.core.GenEllipsoid([5;2], eye(2), [1 3; 4 5]);
ellObj.display()
   |
   |----- q : [5 2]
   |          -------
   |----- Q : |10|19|
   |          |19|41|
   |          -------
   |          -----
   |-- QInf : |0|0|
   |          |0|0|
   |          -----



\end{lstlisting}
\fontfamily{\familydefault}
\selectfont
\item {elltool.core.GenEllipsoid.GenEllipsoid}
\fontfamily{pcr}
\selectfont
\begin{lstlisting}
GENELLIPSOID - class of generalized ellipsoids

Input:
  Case1:
    regular:
      qVec: double[nDim,1] - ellipsoid center
      qMat: double[nDim,nDim] / qVec: double[nDim,1] - ellipsoid matrix
          or diagonal vector of eigenvalues, that may contain infinite
          or zero elements

  Case2:
    regular:
      qMat: double[nDim,nDim] / qVec: double[nDim,1] - diagonal matrix or
          vector, may contain infinite or zero elements

  Case3:
    regular:
      qVec: double[nDim,1] - ellipsoid center
      dMat: double[nDim,nDim] / dVec: double[nDim,1] - diagonal matrix or
          vector, may contain infinite or zero elements
      wMat: double[nDim,nDim] - any square matrix


Output:
  self: GenEllipsoid[1,1] - created generalized ellipsoid

Example:
ellObj = elltool.core.GenEllipsoid([5;2], eye(2));
ellObj = elltool.core.GenEllipsoid([5;2], eye(2), [1 3; 4 5]);



\end{lstlisting}
\fontfamily{\familydefault}
\selectfont
\item {elltool.core.GenEllipsoid.getDiagMat}
\fontfamily{pcr}
\selectfont
\begin{lstlisting}
Example:
ellObj = elltool.core.GenEllipsoid([5;2], eye(2), [1 3; 4 5]);
ellObj.getDiagMat()

ans =

    0.9796         0
         0   50.0204




\end{lstlisting}
\fontfamily{\familydefault}
\selectfont
\item {elltool.core.GenEllipsoid.getEigvMat}
\fontfamily{pcr}
\selectfont
\begin{lstlisting}
Example:
ellObj = elltool.core.GenEllipsoid([5;2], eye(2), [1 3; 4 5]);
ellObj.getEigvMat()

ans =

    0.9034   -0.4289
   -0.4289   -0.9034




\end{lstlisting}
\fontfamily{\familydefault}
\selectfont
\item {elltool.core.GenEllipsoid.getCenter}
\fontfamily{pcr}
\selectfont
\begin{lstlisting}
Example:
ellObj = elltool.core.GenEllipsoid([5;2], eye(2), [1 3; 4 5]);
ellObj.getCenter()

ans =

     5
     2




\end{lstlisting}
\fontfamily{\familydefault}
\selectfont
\item {elltool.core.GenEllipsoid.getCheckTol}
\fontfamily{pcr}
\selectfont
\begin{lstlisting}
Example:
ellObj = elltool.core.GenEllipsoid([5;2], eye(2), [1 3; 4 5]);
ellObj.getCheckTol()

ans =

   1.0000e-09




\end{lstlisting}
\fontfamily{\familydefault}
\selectfont
\item {elltool.core.GenEllipsoid.dimension}
\fontfamily{pcr}
\selectfont
\begin{lstlisting}
Example:
firstEllObj = elltool.core.GenEllipsoid([1; 1], eye(2));
secEllObj = elltool.core.GenEllipsoid([0; 5], 2*eye(2));
ellVec = [firstEllObj secEllObj];
ellVec.dimension()

ans =

     2     2



\end{lstlisting}
\fontfamily{\familydefault}
\selectfont
\item {elltool.core.GenEllipsoid.inv}
\fontfamily{pcr}
\selectfont
\begin{lstlisting}
INV - create generalized ellipsoid whose matrix in pseudoinverse
      to the matrix of input generalized ellipsoid

Input:
  regular:
      ellObj: GenEllipsoid: [1,1] - generalized ellipsoid

Output:
  ellInvObj: GenEllipsoid: [1,1] - inverse generalized ellipsoid

Example:
ellObj = elltool.core.GenEllipsoid([5;2], [1 0; 0 0.7]);
ellObj.inv()
   |
   |----- q : [5 2]
   |          -----------------
   |----- Q : |1      |0      |
   |          |0      |1.42857|
   |          -----------------
   |          -----
   |-- QInf : |0|0|
   |          |0|0|
   |          -----




\end{lstlisting}
\fontfamily{\familydefault}
\selectfont
\item {elltool.core.GenEllipsoid.minkDiffEa}
\fontfamily{pcr}
\selectfont
\begin{lstlisting}
MINKDIFFEA - computes tight external ellipsoidal approximation for
             Minkowsky difference of two generalized ellipsoids

Input:
  regular:
      ellObj1: GenEllipsoid: [1,1] - first generalized ellipsoid
      ellObj2: GenEllipsoid: [1,1] - second generalized ellipsoid
      dirMat: double[nDim,nDir] - matrix whose columns specify
          directions for which approximations should be computed
Output:
  resEllVec: GenEllipsoid[1,nDir] - vector of generalized ellipsoids of
      external approximation of the dirrence of first and second
      generalized ellipsoids (may contain empty ellipsoids if in specified
      directions approximation cannot be computed)

Example:
firstEllObj = elltool.core.GenEllipsoid([10;0], 2*eye(2));
secEllObj = elltool.core.GenEllipsoid([0;0], [1 0; 0 0.1]);
dirsMat = [1,0].';
resEllVec  = minkDiffEa( firstEllObj, secEllObj, dirsMat)
   |
   |----- q : [10 0]
   |          -------------------
   |----- Q : |0.171573|0       |
   |          |0       |1.20557 |
   |          -------------------
   |          -----
   |-- QInf : |0|0|
   |          |0|0|
   |          -----



\end{lstlisting}
\fontfamily{\familydefault}
\selectfont
\item {elltool.core.GenEllipsoid.minkDiffIa}
\fontfamily{pcr}
\selectfont
\begin{lstlisting}
MINKDIFFIA - computes tight internal ellipsoidal approximation for
             Minkowsky difference of two generalized ellipsoids

Input:
  regular:
      ellObj1: GenEllipsoid: [1,1] - first generalized ellipsoid
      ellObj2: GenEllipsoid: [1,1] - second generalized ellipsoid
      dirMat: double[nDim,nDir] - matrix whose columns specify
          directions for which approximations should be computed
Output:
  resEllVec: GenEllipsoid[1,nDir] - vector of generalized ellipsoids of
      internal approximation of the dirrence of first and second
      generalized ellipsoids

Example:
firstEllObj = elltool.core.GenEllipsoid([10;0], 2*eye(2));
secEllObj = elltool.core.GenEllipsoid([0;0], [1 0; 0 0.1]);
dirsMat = [1,0].';
resEllVec  = minkDiffIa( firstEllObj, secEllObj, dirsMat)
   |
   |----- q : [10 0]
   |          -------------------
   |----- Q : |0.171573|0       |
   |          |0       |0.544365|
   |          -------------------
   |          -----
   |-- QInf : |0|0|
   |          |0|0|
   |          -----



\end{lstlisting}
\fontfamily{\familydefault}
\selectfont
\item {elltool.core.GenEllipsoid.minkSumEa}
\fontfamily{pcr}
\selectfont
\begin{lstlisting}
MINKSUMEA - computes tight external ellipsoidal approximation for
            Minkowsky sum of the set of generalized ellipsoids

Input:
  regular:
      ellObjVec: GenEllipsoid: [kSize,mSize] - vector of  generalized
                                          ellipsoid
      dirMat: double[nDim,nDir] - matrix whose columns specify
          directions for which approximations should be computed
Output:
  ellResVec: GenEllipsoid[1,nDir] - vector of generalized ellipsoids of
      external approximation of the dirrence of first and second
      generalized ellipsoids

Example:
firstEllObj = elltool.core.GenEllipsoid([1;1],eye(2));
secEllObj = elltool.core.GenEllipsoid([5;0],[3 0; 0 2]);
ellVec = [firstEllObj secEllObj];
dirsMat = [1 3; 2 4];
ellResVec  = minkSumEa(ellVec, dirsMat )

Structure(1)
   |
   |----- q : [6 1]
   |          -----------------
   |----- Q : |7.50584|0      |
   |          |0      |5.83164|
   |          -----------------
   |          -----
   |-- QInf : |0|0|
   |          |0|0|
   |          -----
   O

Structure(2)
   |
   |----- q : [6 1]
   |          -----------------
   |----- Q : |7.48906|0      |
   |          |0      |5.83812|
   |          -----------------
   |          -----
   |-- QInf : |0|0|
   |          |0|0|
   |          -----
   O




\end{lstlisting}
\fontfamily{\familydefault}
\selectfont
\item {elltool.core.GenEllipsoid.minkSumIa}
\fontfamily{pcr}
\selectfont
\begin{lstlisting}
MINKSUMIA - computes tight internal ellipsoidal approximation for
            Minkowsky sum of the set of generalized ellipsoids

Input:
  regular:
      ellObjVec: GenEllipsoid: [kSize,mSize] - vector of  generalized
                                          ellipsoid
      dirMat: double[nDim,nDir] - matrix whose columns specify
          directions for which approximations should be computed
Output:
  ellResVec: GenEllipsoid[1,nDir] - vector of generalized ellipsoids of
      internal approximation of the dirrence of first and second
      generalized ellipsoids

Example:
firstEllObj = elltool.core.GenEllipsoid([1;1],eye(2));
secEllObj = elltool.core.GenEllipsoid([5;0],[3 0; 0 2]);
ellVec = [firstEllObj secEllObj];
dirsMat = [1 3; 2 4];
ellResVec  = minkSumIa(ellVec, dirsMat )

Structure(1)
   |
   |----- q : [6 1]
   |          ---------------------
   |----- Q : |7.45135  |0.0272432|
   |          |0.0272432|5.81802  |
   |          ---------------------
   |          -----
   |-- QInf : |0|0|
   |          |0|0|
   |          -----
   O

Structure(2)
   |
   |----- q : [6 1]
   |          ---------------------
   |----- Q : |7.44698  |0.0315642|
   |          |0.0315642|5.81445  |
   |          ---------------------
   |          -----
   |-- QInf : |0|0|
   |          |0|0|
   |          -----
   O




\end{lstlisting}
\fontfamily{\familydefault}
\selectfont
\item {elltool.core.GenEllipsoid.plot}
\fontfamily{pcr}
\selectfont
\begin{lstlisting}
PLOT - plots ellipsoids in 2D or 3D.


Usage:
      plot(ell) - plots generic ellipsoid ell in default (red) color.
      plot(ellArr) - plots an array of generic ellipsoids.
      plot(ellArr, 'Property',PropValue,...) - plots ellArr with setting
                                               properties.

Input:
  regular:
      ellArr:  elltool.core.GenEllipsoid: [dim11Size,dim12Size,...,
               dim1kSize] - array of 2D or 3D GenEllipsoids objects.
               All ellipsoids in ellArr  must be either 2D or 3D
               simutaneously.
  optional:
      color1Spec: char[1,1] - color specification code, can be 'r','g',
                              etc (any code supported by built-in Matlab
                              function).
      ell2Arr: elltool.core.GenEllipsoid: [dim21Size,dim22Size,...,
                              dim2kSize] - second ellipsoid array...
      color2Spec: char[1,1] - same as color1Spec but for ell2Arr
      ....
      ellNArr: elltool.core.GenEllipsoid: [dimN1Size,dim22Size,...,
                               dimNkSize] - N-th ellipsoid array
      colorNSpec - same as color1Spec but for ellNArr.
  properties:
      'newFigure': logical[1,1] - if 1, each plot command will open a new .
                   figure window Default value is 0.
      'fill': logical[1,1]/logical[dim11Size,dim12Size,...,dim1kSize]  -
              if 1, ellipsoids in 2D will be filled with color.
              Default value is 0.
      'lineWidth': double[1,1]/double[dim11Size,dim12Size,...,dim1kSize]  -
               line width for 1D and 2D plots.
               Default value is 1.
      'color': double[1,3]/double[dim11Size,dim12Size,...,dim1kSize,3] -
               sets default colors in the form [x y z].
               Default value is [1 0 0].
      'shade': double[1,1]/double[dim11Size,dim12Size,...,dim1kSize]  -
               level of transparency between 0 and 1 (0 - transparent,
               1 - opaque).
               Default value is 0.4.
      'relDataPlotter' - relation data plotter object.
      Notice that property vector could have different dimensions, only
      total number of elements must be the same.
Output:
  regular:
      plObj: smartdb.disp.RelationDataPlotter[1,1] - returns the relation
      data plotter object.

Examples:
      plot([ell1, ell2, ell3], 'color', [1, 0, 1; 0, 0, 1; 1, 0, 0]);
      plot([ell1, ell2, ell3], 'color', [1; 0; 1; 0; 0; 1; 1; 0; 0]);
      plot([ell1, ell2, ell3; ell1, ell2, ell3], 'shade', [1, 1, 1; 1, 1,
      1]);
      plot([ell1, ell2, ell3; ell1, ell2, ell3], 'shade', [1; 1; 1; 1; 1;
      1]);
      plot([ell1, ell2, ell3], 'shade', 0.5);
      plot([ell1, ell2, ell3], 'lineWidth', 1.5);
      plot([ell1, ell2, ell3], 'lineWidth', [1.5, 0.5, 3]);



\end{lstlisting}
\fontfamily{\familydefault}
\selectfont
\item {elltool.core.GenEllipsoid.rho}
\fontfamily{pcr}
\selectfont
\begin{lstlisting}
Example:
ellObj = elltool.core.GenEllipsoid([1;1],eye(2));
dirsVec = [1; 0];
[resRho, bndPVec] = rho(ellObj, dirsVec)

resRho =

     2


bndPVec =

     2
     1




\end{lstlisting}
\fontfamily{\familydefault}
\selectfont
\item {elltool.reach.ReachFactory.getL0Mat}
\todo[inline]{Updated all help headers in reach}
\fontfamily{pcr}
\selectfont
\begin{lstlisting}
Example:
import elltool.reach.ReachFactory;
crm=gras.ellapx.uncertcalc.test.regr.conf.ConfRepoMgr();
crmSys=gras.ellapx.uncertcalc.test.regr.conf.sysdef.ConfRepoMgr();
rsObj =  ReachFactory('demo3firstTest', crm, crmSys, false, false);
l0Mat = rsObj.getL0Mat()

l0Mat =

     1     0
     0     1




\end{lstlisting}
\fontfamily{\familydefault}
\selectfont
\item {elltool.reach.ReachFactory.getX0Ell}
\fontfamily{pcr}
\selectfont
\begin{lstlisting}
Example:
import elltool.reach.ReachFactory;
crm=gras.ellapx.uncertcalc.test.regr.conf.ConfRepoMgr();
crmSys=gras.ellapx.uncertcalc.test.regr.conf.sysdef.ConfRepoMgr();
rsObj =  ReachFactory('demo3firstTest', crm, crmSys, false, false);
X0Ell = rsObj.getX0Ell()

X0Ell =

Center:
     0
     0

Shape Matrix:
    0.0100         0
         0    0.0100

Nondegenerate ellipsoid in R^2.




\end{lstlisting}
\fontfamily{\familydefault}
\selectfont
\item {elltool.reach.ReachFactory.getTVec}
\fontfamily{pcr}
\selectfont
\begin{lstlisting}
Example:
import elltool.reach.ReachFactory;
crm=gras.ellapx.uncertcalc.test.regr.conf.ConfRepoMgr();
crmSys=gras.ellapx.uncertcalc.test.regr.conf.sysdef.ConfRepoMgr();
rsObj =  ReachFactory('demo3firstTest', crm, crmSys, false, false);
tVec = rsObj.getTVec()

tVec =

     0    10




\end{lstlisting}
\fontfamily{\familydefault}
\selectfont
\item {elltool.reach.ReachFactory.getDim}
\fontfamily{pcr}
\selectfont
\begin{lstlisting}
Example:
import elltool.reach.ReachFactory;
crm=gras.ellapx.uncertcalc.test.regr.conf.ConfRepoMgr();
crmSys=gras.ellapx.uncertcalc.test.regr.conf.sysdef.ConfRepoMgr();
rsObj =  ReachFactory('demo3firstTest', crm, crmSys, false, false);
dim = rsObj.getDim();




\end{lstlisting}
\fontfamily{\familydefault}
\selectfont
\item {elltool.reach.ReachFactory.getLinSys}
\fontfamily{pcr}
\selectfont
\begin{lstlisting}
Example:
import elltool.reach.ReachFactory;
crm=gras.ellapx.uncertcalc.test.regr.conf.ConfRepoMgr();
crmSys=gras.ellapx.uncertcalc.test.regr.conf.sysdef.ConfRepoMgr();
rsObj =  ReachFactory('demo3firstTest', crm, crmSys, false, false);
linSys = rsObj.getLinSys();




\end{lstlisting}
\fontfamily{\familydefault}
\selectfont
\item {elltool.reach.ReachFactory.createInstance}
\fontfamily{pcr}
\selectfont
\begin{lstlisting}
Example:
import elltool.reach.ReachFactory;
crm=gras.ellapx.uncertcalc.test.regr.conf.ConfRepoMgr();
crmSys=gras.ellapx.uncertcalc.test.regr.conf.sysdef.ConfRepoMgr();
rsObj =  ReachFactory('demo3firstTest', crm, crmSys, false, false);
reachObj = rsObj.createInstance();




\end{lstlisting}
\fontfamily{\familydefault}
\selectfont
\item {elltool.reach.ReachFactory.ReachFactory}
\fontfamily{pcr}
\selectfont
\begin{lstlisting}
Example:
import elltool.reach.ReachFactory;
crm=gras.ellapx.uncertcalc.test.regr.conf.ConfRepoMgr();
crmSys=gras.ellapx.uncertcalc.test.regr.conf.sysdef.ConfRepoMgr();
rsObj =  ReachFactory('demo3firstTest', crm, crmSys, false, false);




\end{lstlisting}
\fontfamily{\familydefault}
\selectfont
\item {elltool.reach.ReachContinuous.getEllTubeUnionRel}
\fontfamily{pcr}
\selectfont
\begin{lstlisting}
Example:
aMat = [0 1; 0 0]; bMat = eye(2);
SUBounds = struct();
SUBounds.center = {'sin(t)'; 'cos(t)'};
SUBounds.shape = [9 0; 0 2];
sys = elltool.linsys.LinSysContinuous(aMat, bMat, SUBounds);
x0EllObj = ell_unitball(2);
timeVec = [0 10];
dirsMat = [1 0; 0 1]';
rsObj = elltool.reach.ReachContinuous(sys, x0EllObj, dirsMat, timeVec);
getEllTubeUnionRel(rsObj);




\end{lstlisting}
\fontfamily{\familydefault}
\selectfont
\item {elltool.reach.ReachContinuous.getEllTubeRel}
\fontfamily{pcr}
\selectfont
\begin{lstlisting}
Example:
aMat = [0 1; 0 0]; bMat = eye(2);
SUBounds = struct();
SUBounds.center = {'sin(t)'; 'cos(t)'};
SUBounds.shape = [9 0; 0 2];
sys = elltool.linsys.LinSysContinuous(aMat, bMat, SUBounds);
x0EllObj = ell_unitball(2);
timeVec = [0 10];
dirsMat = [1 0; 0 1]';
rsObj = elltool.reach.ReachContinuous(sys, x0EllObj, dirsMat, timeVec);
rsObj. getEllTubeRel();




\end{lstlisting}
\fontfamily{\familydefault}
\selectfont
\item {elltool.reach.ReachContinuous.getCopy}
\fontfamily{pcr}
\selectfont
\begin{lstlisting}
Example:
aMat = [0 1; 0 0]; bMat = eye(2);
SUBounds = struct();
SUBounds.center = {'sin(t)'; 'cos(t)'};
SUBounds.shape = [9 0; 0 2];
sys = elltool.linsys.LinSysContinuous(aMat, bMat, SUBounds);
x0EllObj = ell_unitball(2);
timeVec = [0 10];
dirsMat = [1 0; 0 1]';
rsObj = elltool.reach.ReachContinuous(sys, x0EllObj, dirsMat, timeVec);
copyRsObj = rsObj.getCopy()
copyRsObj =
Reach set of the continuous-time linear system in R^2 in the time ...
            interval [0, 10].

Initial set at time t0 = 0:
Ellipsoid with parameters
Center:
     0
     0

Shape Matrix:
     1     0
     0     1

Number of external approximations: 2
Number of internal approximations: 2




\end{lstlisting}
\fontfamily{\familydefault}
\selectfont
\item {elltool.reach.ReachContinuous.isEqual}
\fontfamily{pcr}
\selectfont
\begin{lstlisting}
ISEQUAL - checks for equality given reach set objects

Input:
  regular:
      self.
      reachObj:
          elltool.reach.ReachContinuous[1, 1] - each set object, which
           compare with self.
  optional:
      tuple: int[1, 1] - number of tuple for which will be compared.
      approxType: gras.ellapx.enums.EApproxType[1, 1] -  type of
          approximation, which will be compared.

Output:
  regular:
      ISEQUAL: logical[1, 1] - true - if reach set objects are equal.
          false - otherwise.

Example:
aMat = [0 1; 0 0]; bMat = eye(2);
SUBounds = struct();
SUBounds.center = {'sin(t)'; 'cos(t)'};
SUBounds.shape = [9 0; 0 2];
sys = elltool.linsys.LinSysContinuous(aMat, bMat, SUBounds);
x0EllObj = ell_unitball(2);
timeVec = [0 10];
dirsMat = [1 0; 0 1]';
rsObj = elltool.reach.ReachContinuous(sys, x0EllObj, dirsMat, timeVec);
copyRsObj = rsObj.getCopy();
isEqual = isEqual(rsObj, copyRsObj)

isEqual =

        1



\end{lstlisting}
\fontfamily{\familydefault}
\selectfont
\item {elltool.reach.ReachContinuous.isbackward}
\fontfamily{pcr}
\selectfont
\begin{lstlisting}
ISBACKWARD - checks if given reach set object was obtained by solving
             the system in reverse time.

Input:
  regular:
      self.

Output:
  regular:
      isBackward: logical[1, 1] - true - if self was obtained by solving
          in reverse time, false - otherwise.

Example:
aMat = [0 1; 0 0]; bMat = eye(2);
SUBounds = struct();
SUBounds.center = {'sin(t)'; 'cos(t)'};
SUBounds.shape = [9 0; 0 2];
sys = elltool.linsys.LinSysContinuous(aMat, bMat, SUBounds);
x0EllObj = ell_unitball(2);
timeVec = [10 0];
dirsMat = [1 0; 0 1]';
rsObj = elltool.reach.ReachContinuous(sys, x0EllObj, dirsMat, timeVec);
rsObj.isbackward()

ans =

     1




\end{lstlisting}
\fontfamily{\familydefault}
\selectfont
\item {elltool.reach.ReachContinuous.getInitialSet}
\fontfamily{pcr}
\selectfont
\begin{lstlisting}
GETINITIALSET - return the initial set for linear system, which is solved
                for building reach tube.

Input:
  regular:
      self.

Output:
  regular:
      x0Ell: ellipsoid[1, 1] - ellipsoid x0, which was initial set for
          linear system.

Example:
aMat = [0 1; 0 0]; bMat = eye(2);
SUBounds = struct();
SUBounds.center = {'sin(t)'; 'cos(t)'};
SUBounds.shape = [9 0; 0 2];
sys = elltool.linsys.LinSysContinuous(aMat, bMat, SUBounds);
x0EllObj = ell_unitball(2);
timeVec = [10 0];
dirsMat = [1 0; 0 1]';
rsObj = elltool.reach.ReachContinuous(sys, x0EllObj, dirsMat, timeVec);
x0Ell = rsObj.getInitialSet()

x0Ell =

Center:
     0
     0

Shape Matrix:
     1     0
     0     1

Nondegenerate ellipsoid in R^2.



\end{lstlisting}
\fontfamily{\familydefault}
\selectfont
\item {elltool.reach.ReachContinuous.getIaScaleFactor}
\fontfamily{pcr}
\selectfont
\begin{lstlisting}
GET_IASCALEFACTOR - return the scale factor for internal approximation
                    of reach tube

Input:
  regular:
      self.

Output:
  regular:
      iaScaleFactor: double[1, 1] - scale factor.

Example:
aMat = [0 1; 0 0]; bMat = eye(2);
SUBounds = struct();
SUBounds.center = {'sin(t)'; 'cos(t)'};
SUBounds.shape = [9 0; 0 2];
sys = elltool.linsys.LinSysContinuous(aMat, bMat, SUBounds);
x0EllObj = ell_unitball(2);
timeVec = [10 0];
dirsMat = [1 0; 0 1]';
rsObj = elltool.reach.ReachContinuous(sys, x0EllObj, dirsMat, timeVec);
rsObj.getIaScaleFactor()

ans =

    1.0200



\end{lstlisting}
\fontfamily{\familydefault}
\selectfont
\item {elltool.reach.ReachContinuous.getEaScaleFactor}
\fontfamily{pcr}
\selectfont
\begin{lstlisting}
GET_EASCALEFACTOR - return the scale factor for external approximation
                    of reach tube

Input:
  regular:
      self.

Output:
  regular:
      eaScaleFactor: double[1, 1] - scale factor.

Example:
aMat = [0 1; 0 0]; bMat = eye(2);
SUBounds = struct();
SUBounds.center = {'sin(t)'; 'cos(t)'};
SUBounds.shape = [9 0; 0 2];
sys = elltool.linsys.LinSysContinuous(aMat, bMat, SUBounds);
x0EllObj = ell_unitball(2);
timeVec = [10 0];
dirsMat = [1 0; 0 1]';
rsObj = elltool.reach.ReachContinuous(sys, x0EllObj, dirsMat, timeVec);
rsObj.getEaScaleFactor()

ans =

    1.0200



\end{lstlisting}
\fontfamily{\familydefault}
\selectfont
\item {elltool.reach.ReachContinuous.evolve}
\fontfamily{pcr}
\selectfont
\begin{lstlisting}

EVOLVE - computes further evolution in time of the already existing
         reach set.

Input:
  regular:
      self.

      newEndTime: double[1, 1] - new end time.

  optional:
      linSys: elltool.linsys.LinSys[1, 1] - new linear system.

Output:
  newReachObj: reach[1, 1] - reach set on time  interval
        [oldT0 newEndTime].

Example:
aMat = [0 1; 0 0]; bMat = eye(2);
SUBounds = struct();
SUBounds.center = {'sin(t)'; 'cos(t)'};
SUBounds.shape = [9 0; 0 2];
sys = elltool.linsys.LinSysContinuous(aMat, bMat, SUBounds);
dsys = elltool.linsys.LinSysDiscrete(aMat, bMat, SUBounds);
x0EllObj = ell_unitball(2);
timeVec = [0 10];
dirsMat = [1 0; 0 1]';
rsObj = elltool.reach.ReachContinuous(sys, x0EllObj, dirsMat, timeVec);
dRsObj = elltool.reach.ReachDiscrete(dsys, x0EllObj, dirsMat, timeVec);
newDRsObj = dRsObj.evolve(11);





\end{lstlisting}
\fontfamily{\familydefault}
\selectfont
\item {elltool.reach.ReachContinuous.projection}
\fontfamily{pcr}
\selectfont
\begin{lstlisting}

PROJECTION - projects the reach set self onto the orthogonal basis
             specified by the columns of matrix projMat.

Input:
  regular:
      self.
      projMat: double[nRows, nCols] - projection matrix, where nRows
          is dimension of reach set, nCols <= nRows.

Output:
  projObj: elltool.reach.IReach[1, 1] - projected reach set.

Examples:
aMat = [0 1; 0 0]; bMat = eye(2);
SUBounds = struct();
SUBounds.center = {'sin(t)'; 'cos(t)'};
SUBounds.shape = [9 0; 0 2];
sys = elltool.linsys.LinSysContinuous(aMat, bMat, SUBounds);
dsys = elltool.linsys.LinSysDiscrete(aMat, bMat, SUBounds);
x0EllObj = ell_unitball(2);
timeVec = [0 10];
dirsMat = [1 0; 0 1]';
rsObj = elltool.reach.ReachContinuous(sys, x0EllObj, dirsMat, timeVec);
dRsObj = elltool.reach.ReachRiscrete(dsys, x0EllObj, dirsMat, timeVec);
projMat = eye(2);
projObj = rsObj.projection(projMat);
dProjObj = dRsObj.projection(projMat);





\end{lstlisting}
\fontfamily{\familydefault}
\selectfont
\item {elltool.reach.ReachContinuous.get\_goodcurves}
\fontfamily{pcr}
\selectfont
\begin{lstlisting}
GET_GOODCURVES - returns the 'good curve' trajectories of the reach set.

Input:
  regular:
      self.

Output:
  goodCurvesCVec: cell[1, nPoints] of double [x, y] - array of cells,
      where each cell is array of points that form a 'good curve'.

  timeVec: double[1, nPoints] - array of time values.

Example:
aMat = [0 1; 0 0]; bMat = eye(2);
SUBounds = struct();
SUBounds.center = {'sin(t)'; 'cos(t)'};
SUBounds.shape = [9 0; 0 2];
sys = elltool.linsys.LinSysContinuous(aMat, bMat, SUBounds);
x0EllObj = ell_unitball(2);
timeVec = [0 10];
dirsMat = [1 0; 0 1]';
rsObj = elltool.reach.ReachContinuous(sys, x0EllObj, dirsMat, timeVec);
[goodCurvesCVec timeVec] = rsObj.get_goodcurves();

dsys = elltool.linsys.LinSysDiscrete(aMat, bMat, SUBounds);
dRsObj = elltool.reach.ReachDiscrete(sys, x0EllObj, dirsMat, timeVec);
[goodCurvesCVec timeVec] = dRsObj.get_goodcurves();





\end{lstlisting}
\fontfamily{\familydefault}
\selectfont
\item {elltool.reach.ReachContinuous.get\_ia}
\fontfamily{pcr}
\selectfont
\begin{lstlisting}

GET_IA - returns array of ellipsoid objects representing internal
         approximation of the  reach tube.

Input:
  regular:
      self.

Output:
  iaEllMat: ellipsoid[nAppr, nPoints] - array of ellipsoids, where nAppr
      is the number of approximations, nPoints is number of points in time
      grid.

  timeVec: double[1, nPoints] - array of time values.

Example:
aMat = [0 1; 0 0]; bMat = eye(2);
SUBounds = struct();
SUBounds.center = {'sin(t)'; 'cos(t)'};
SUBounds.shape = [9 0; 0 2];
sys = elltool.linsys.LinSysContinuous(aMat, bMat, SUBounds);
x0EllObj = ell_unitball(2);
timeVec = [0 10];
dirsMat = [1 0; 0 1]';
rsObj = elltool.reach.ReachContinuous(sys, x0EllObj, dirsMat, timeVec);
[iaEllMat timeVec] = rsObj.get_ia();





\end{lstlisting}
\fontfamily{\familydefault}
\selectfont
\item {elltool.reach.ReachContinuous.get\_ea}
\fontfamily{pcr}
\selectfont
\begin{lstlisting}

GET_EA - returns array of ellipsoid objects representing external
         approximation of the reach  tube.

Input:
  regular:
      self.

Output:
  eaEllMat: ellipsoid[nAppr, nPoints] - array of ellipsoids, where nAppr
      is the number of approximations, nPoints is number of points in time
      grid.

   timeVec: double[1, nPoints] - array of time values.

Example:
aMat = [0 1; 0 0]; bMat = eye(2);
SUBounds = struct();
SUBounds.center = {'sin(t)'; 'cos(t)'};
SUBounds.shape = [9 0; 0 2];
sys = elltool.linsys.LinSysContinuous(aMat, bMat, SUBounds);
x0EllObj = ell_unitball(2);
timeVec = [0 10];
dirsMat = [1 0; 0 1]';
rsObj = elltool.reach.ReachContinuous(sys, x0EllObj, dirsMat, timeVec);
[eaEllMat timeVec] = rsObj.get_ea();

dsys = elltool.linsys.LinSysDiscrete(aMat, bMat, SUBounds);
dRsObj = elltool.reach.ReachDiscrete(sys, x0EllObj, dirsMat, timeVec);
[eaEllMat timeVec] = dRsObj.get_ea();





\end{lstlisting}
\fontfamily{\familydefault}
\selectfont
\item {elltool.reach.ReachContinuous.get\_center}
\fontfamily{pcr}
\selectfont
\begin{lstlisting}

GET_CENTER - returns the trajectory of the center of the reach set.

Input:
  regular:
      self.

Output:
  trCenterMat: double[nDim, nPoints] - array of points that form the
      trajectory of the reach set center, where nDim is reach set
      dimentsion, nPoints - number of points in time grid.

  timeVec: double[1, nPoints] - array of time values.

Example:
aMat = [0 1; 0 0]; bMat = eye(2);
SUBounds = struct();
SUBounds.center = {'sin(t)'; 'cos(t)'};
SUBounds.shape = [9 0; 0 2];
sys = elltool.linsys.LinSysContinuous(aMat, bMat, SUBounds);
x0EllObj = ell_unitball(2);
timeVec = [0 10];
dirsMat = [1 0; 0 1]';
rsObj = elltool.reach.ReachContinuous(sys, x0EllObj, dirsMat, timeVec);
[trCenterMat timeVec] = rsObj.get_center();





\end{lstlisting}
\fontfamily{\familydefault}
\selectfont
\item {elltool.reach.ReachContinuous.get\_directions}
\fontfamily{pcr}
\selectfont
\begin{lstlisting}

GET_DIRECTIONS - returns the values of direction vectors for time grid
                 values.

Input:
  regular:
      self.

Output:
  directionsCVec: cell[1, nPoints] of double [nDim, nDir] - array of
      cells, where each cell is a sequence of direction vector values
      that correspond to the time values of the grid, where nPoints is
      number of points in time grid.

  timeVec: double[1, nPoints] - array of time values.

Example:
aMat = [0 1; 0 0]; bMat = eye(2);
SUBounds = struct();
SUBounds.center = {'sin(t)'; 'cos(t)'};
SUBounds.shape = [9 0; 0 2];
sys = elltool.linsys.LinSysContinuous(aMat, bMat, SUBounds);
x0EllObj = ell_unitball(2);
timeVec = [0 10];
dirsMat = [1 0; 0 1]';
rsObj = elltool.reach.ReachContinuous(sys, x0EllObj, dirsMat, timeVec);
[directionsCVec timeVec] = rsObj.get_directions();





\end{lstlisting}
\fontfamily{\familydefault}
\selectfont
\item {elltool.reach.ReachContinuous.get\_system}
\fontfamily{pcr}
\selectfont
\begin{lstlisting}

GET_SYSTEM - returns the linear system for which the reach set is
             computed.

Input:
  regular:
      self.

Output:
  linSys: elltool.linsys.LinSys[1, 1] - linear system object.

Example:
aMat = [0 1; 0 0]; bMat = eye(2);
SUBounds = struct();
SUBounds.center = {'sin(t)'; 'cos(t)'};
SUBounds.shape = [9 0; 0 2];
sys = elltool.linsys.LinSysContinuous(aMat, bMat, SUBounds);
x0EllObj = ell_unitball(2);
timeVec = [0 10];
dirsMat = [1 0; 0 1]';
rsObj = elltool.reach.ReachContinuous(sys, x0EllObj, dirsMat, timeVec);
linSys = rsObj.get_system()

self =
A:
     0     1
     0     0


B:
     1     0
     0     1


Control bounds:
   2-dimensional ellipsoid with center
    'sin(t)'
    'cos(t)'

   and shape matrix
     9     0
     0     2


C:
     1     0
     0     1

2-input, 2-output continuous-time linear time-invariant system of
        dimension 2:
dx/dt  =  A x(t)  +  B u(t)
 y(t)  =  C x(t)

dsys = elltool.linsys.LinSysDiscrete(aMat, bMat, SUBounds);
dRsObj = elltool.reach.ReachDiscrete(sys, x0EllObj, dirsMat, timeVec);
dRsObj.get_system();





\end{lstlisting}
\fontfamily{\familydefault}
\selectfont
\item {elltool.reach.ReachContinuous.dimension}
\fontfamily{pcr}
\selectfont
\begin{lstlisting}

DIMENSION - returns the dimension of the reach set.

Input:
  regular:
      self.

Output:
  rSdim: double[1, 1] - reach set dimension.
  sSdim: double[1, 1] - state space dimension.

Example:
aMat = [0 1; 0 0]; bMat = eye(2);
SUBounds = struct();
SUBounds.center = {'sin(t)'; 'cos(t)'};
SUBounds.shape = [9 0; 0 2];
sys = elltool.linsys.LinSysContinuous(aMat, bMat, SUBounds);
x0EllObj = ell_unitball(2);
timeVec = [0 10];
dirsMat = [1 0; 0 1]';
rsObj = elltool.reach.ReachContinuous(sys, x0EllObj, dirsMat, timeVec);
[rSdim sSdim] = rsObj.dimension()

rSdim =

         2


sSdim =

         2





\end{lstlisting}
\fontfamily{\familydefault}
\selectfont
\item {elltool.reach.ReachContinuous.cut}
\fontfamily{pcr}
\selectfont
\begin{lstlisting}
CUT - extracts the piece of reach tube from given start time to given
      end time. Given reach set self, find states that are reachable
      within time interval specified by cutTimeVec. If cutTimeVec
      is a scalar, then reach set at given time is returned.

Input:
  regular:
      self.

   cutTimeVec: double[1, 2]/double[1, 1] - time interval to cut.

Output:
  cutObj: elltool.reach.IReach[1, 1] - reach set resulting from the CUT
        operation.

Example:
aMat = [0 1; 0 0]; bMat = eye(2);
SUBounds = struct();
SUBounds.center = {'sin(t)'; 'cos(t)'};
SUBounds.shape = [9 0; 0 2];
sys = elltool.linsys.LinSysContinuous(aMat, bMat, SUBounds);
x0EllObj = ell_unitball(2);
timeVec = [0 10];
dirsMat = [1 0; 0 1]';
rsObj = elltool.reach.ReachContinuous(sys, x0EllObj, dirsMat, timeVec);
cutObj = rsObj.cut([3 5]);
dRsObj = elltool.reach.ReachDiscrete(dtsys, x0EllObj, dirsMat, timeVec);
dCutObj = dRsObj.cut([3 5]);





\end{lstlisting}
\fontfamily{\familydefault}
\selectfont
\item {elltool.reach.ReachContinuous.display}
\fontfamily{pcr}
\selectfont
\begin{lstlisting}

DISPLAY - displays the reach set object.

Input:
  regular:
      self.

Output:
  None.

Example:
aMat = [0 1; 0 0]; bMat = eye(2);
SUBounds = struct();
SUBounds.center = {'sin(t)'; 'cos(t)'};
SUBounds.shape = [9 0; 0 2];
sys = elltool.linsys.LinSysContinuous(aMat, bMat, SUBounds);
x0EllObj = ell_unitball(2);
timeVec = [0 10];
dirsMat = [1 0; 0 1]';
rsObj = elltool.reach.ReachContinuous(sys, x0EllObj, dirsMat, timeVec);
rsObj.display()

rsObj =
Reach set of the continuous-time linear system in R^2 in the time...
     interval [0, 10].

Initial set at time t0 = 0:
Ellipsoid with parameters
Center:
     0
     0

Shape Matrix:
     1     0
     0     1

Number of external approximations: 2
Number of internal approximations: 2





\end{lstlisting}
\fontfamily{\familydefault}
\selectfont
\item {elltool.reach.ReachContinuous.plot\_ia}
\fontfamily{pcr}
\selectfont
\begin{lstlisting}

PLOT_IA - plots internal approximations of 2D and 3D reach sets.

Input:
  regular:
      self.

  optional:
      colorSpec: char[1, 1] - set color to plot in following way:
                             'r' - red color,
                             'g' - green color,
                             'b' - blue color,
                             'y' - yellow color,
                             'c' - cyan color,
                             'm' - magenta color,
                             'w' - white color.

      OptStruct: struct[1, 1] with fields:
          color: double[1, 3] - sets color of the picture in the form
                [x y z].
          width: double[1, 1] - sets line width for 2D plots.
          shade: double[1, 1] in [0; 1] interval - sets transparency level
                (0 - transparent, 1 - opaque).
           fill: double[1, 1] - if set to 1, reach set will be filled with
                color.

Example:
aMat = [0 1; 0 0]; bMat = eye(2);
SUBounds = struct();
SUBounds.center = {'sin(t)'; 'cos(t)'};
SUBounds.shape = [9 0; 0 2];
sys = elltool.linsys.LinSysContinuous(aMat, bMat, SUBounds);
x0EllObj = ell_unitball(2);
timeVec = [0 10];
dirsMat = [1 0; 0 1]';
rsObj = elltool.reach.ReachContinuous(sys, x0EllObj, dirsMat, timeVec);
rsObj.plot_ia();
dsys = elltool.linsys.LinSysDiscrete(aMat, bMat, SUBounds);
dRsObj = elltool.reach.ReachDiscrete(sys, x0EllObj, dirsMat, timeVec);
dRsObj.plot_ia();





\end{lstlisting}
\fontfamily{\familydefault}
\selectfont
\item {elltool.reach.ReachContinuous.plot\_ea}
\fontfamily{pcr}
\selectfont
\begin{lstlisting}

PLOT_EA - plots external approximations of 2D and 3D reach sets.

Input:
  regular:
      self.

  optional:
      colorSpec: char[1, 1] - set color to plot in following way:
                             'r' - red color,
                             'g' - green color,
                             'b' - blue color,
                             'y' - yellow color,
                             'c' - cyan color,
                             'm' - magenta color,
                             'w' - white color.

      OptStruct: struct[1, 1] with fields:
          color: double[1, 3] - sets color of the picture in the form
                [x y z].
          width: double[1, 1] - sets line width for 2D plots.
          shade: double[1, 1] in [0; 1] interval - sets transparency level
                (0 - transparent, 1 - opaque).
           fill: double[1, 1] - if set to 1, reach set will be filled with
                 color.

Output:
  None.

Example:
aMat = [0 1; 0 0]; bMat = eye(2);
SUBounds = struct();
SUBounds.center = {'sin(t)'; 'cos(t)'};
SUBounds.shape = [9 0; 0 2];
sys = elltool.linsys.LinSysContinuous(aMat, bMat, SUBounds);
x0EllObj = ell_unitball(2);
timeVec = [0 10];
dirsMat = [1 0; 0 1]';
rsObj = elltool.reach.ReachContinuous(sys, x0EllObj, dirsMat, timeVec);
rsObj.plot_ea();
dsys = elltool.linsys.LinSysDiscrete(aMat, bMat, SUBounds);
dRsObj = elltool.reach.ReachDiscrete(sys, x0EllObj, dirsMat, timeVec);
dRsObj.plot_ea();





\end{lstlisting}
\fontfamily{\familydefault}
\selectfont
\item {elltool.reach.ReachContinuous.ReachContinuous}
\fontfamily{pcr}
\selectfont
\begin{lstlisting}
ReachContinuous - computes reach set approximation of the continuous
                  linear system for the given time interval.
Input:
    regular:
      linSys: elltool.linsys.LinSys object - given linear system
      x0Ell: ellipsoid[1, 1] - ellipsoidal set of initial conditions
      l0Mat: matrix of double - l0Mat
      timeVec: double[1, 2] - time interval; timeVec(1) must be less
           then timeVec(2)
      OptStruct: structure[1,1] in this class OptStruct doesn't matter
          anything

Output:
  regular:
    self - reach set object.

Example:
aMat = [0 1; 0 0]; bMat = eye(2);
SUBounds = struct();
SUBounds.center = {'sin(t)'; 'cos(t)'};
SUBounds.shape = [9 0; 0 2];
sys = elltool.linsys.LinSysContinuous(aMat, bMat, SUBounds);
x0EllObj = ell_unitball(2);
timeVec = [0 10];
dirsMat = [1 0; 0 1]';
rsObj = elltool.reach.ReachContinuous(sys, x0EllObj, dirsMat, timeVec);



\end{lstlisting}
\fontfamily{\familydefault}
\selectfont
\item {elltool.reach.ReachContinuous.intersect}
\fontfamily{pcr}
\selectfont
\begin{lstlisting}
INTERSECT - checks if its external (s = 'e'), or internal (s = 'i')
            approximation intersects with given ellipsoid, hyperplane
            or polytop.

Input:
  regular:
      self.

      intersectObj: ellipsoid[1, 1]/hyperplane[1,1]/polytop[1, 1].

      approxTypeChar: char[1, 1] - 'e' (default) - external approximation,
                                   'i' - internal approximation.

Output:
  isEmptyIntersect: logical[1, 1] -  true - if intersection is nonempty,
                                     false - otherwise.

Example:
aMat = [0 1; 0 0]; bMat = eye(2);
SUBounds = struct();
SUBounds.center = {'sin(t)'; 'cos(t)'};
SUBounds.shape = [9 0; 0 2];
sys = elltool.linsys.LinSysContinuous(aMat, bMat, SUBounds);
x0EllObj = ell_unitball(2);
timeVec = [0 10];
dirsMat = [1 0; 0 1]';
rsObj = elltool.reach.ReachContinuous(sys, x0EllObj, dirsMat, timeVec);
ellObj = ellipsoid([0; 0], 2*eye(2));
isEmptyIntersect = intersect(rsObj, ellObj)

sEmptyIntersect =

                1





\end{lstlisting}
\fontfamily{\familydefault}
\selectfont
\item {elltool.reach.ReachContinuous.isempty}
\fontfamily{pcr}
\selectfont
\begin{lstlisting}

ISEMPTY - checks if given reach set is an empty object.

Input:
  regular:
      self.

Output:
  isEmpty: logical[1, 1] - true - if self is empty, Ffalse - otherwise.

Example:
aMat = [0 1; 0 0]; bMat = eye(2);
SUBounds = struct();
SUBounds.center = {'sin(t)'; 'cos(t)'};
SUBounds.shape = [9 0; 0 2];
sys = elltool.linsys.LinSysContinuous(aMat, bMat, SUBounds);
dsys = elltool.linsys.LinSysContinuous(aMat, bMat, SUBounds);
x0EllObj = ell_unitball(2);
timeVec = [0 10];
dirsMat = [1 0; 0 1]';
rsObj = elltool.reach.ReachContinuous(sys, x0EllObj, dirsMat, timeVec);
dRsObj = elltool.reach.ReachRiscrete(dsys, x0EllObj, dirsMat, timeVec);
dRsObj.isempty();
rsObj.isempty()

ans =

     0





\end{lstlisting}
\fontfamily{\familydefault}
\selectfont
\item {elltool.reach.ReachContinuous.iscut}
\fontfamily{pcr}
\selectfont
\begin{lstlisting}
ISCUT - checks if given reach set object is a cut of another reach set.

Input:
  regular:
      self.

Output:
  isCut: logical[1, 1] - true - if self is a cut of the reach set,
                         false - otherwise.

Example:
aMat = [0 1; 0 0]; bMat = eye(2);
SUBounds = struct();
SUBounds.center = {'sin(t)'; 'cos(t)'};
SUBounds.shape = [9 0; 0 2];
sys = elltool.linsys.LinSysContinuous(aMat, bMat, SUBounds);
dsys = elltool.linsys.LinSysDiscrete(aMat, bMat, SUBounds);
x0EllObj = ell_unitball(2);
timeVec = [0 10];
dirsMat = [1 0; 0 1]';
rsObj = elltool.reach.ReachContinuous(sys, x0EllObj, dirsMat, timeVec);
dRsObj = elltool.reach.ReachRiscrete(dsys, x0EllObj, dirsMat, timeVec);
cutObj = rsObj.cut([3 5]);
iscut(cutObj);
cutObj = dRsObj.cut([4 8]);
iscut(cutObj);





\end{lstlisting}
\fontfamily{\familydefault}
\selectfont
\item {elltool.reach.ReachContinuous.isprojection}
\fontfamily{pcr}
\selectfont
\begin{lstlisting}

ISPROJECTION - checks if given reach set object is a projection.

Input:
  regular:
      self.

Output:
  isProj: logical[1, 1] - true - if self is projection, false - otherwise.


Example:
aMat = [0 1; 0 0]; bMat = eye(2);
SUBounds = struct();
SUBounds.center = {'sin(t)'; 'cos(t)'};
SUBounds.shape = [9 0; 0 2];
sys = elltool.linsys.LinSysContinuous(aMat, bMat, SUBounds);
dsys = elltool.linsys.LinSysDiscrete(aMat, bMat, SUBounds);
x0EllObj = ell_unitball(2);
timeVec = [0 10];
dirsMat = [1 0; 0 1]';
rsObj = elltool.reach.ReachContinuous(sys, x0EllObj, dirsMat, timeVec);
dRsObj = elltool.reach.ReachRiscrete(dsys, x0EllObj, dirsMat, timeVec);
projMat = eye(2);
projObj = rsObj.projection(projMat);
isprojection(projObj);
projObj = dRsObj.projection(projMat);
isprojection(projObj);





\end{lstlisting}
\fontfamily{\familydefault}
\selectfont
\item {elltool.reach.ReachDiscrete.evolve}
\fontfamily{pcr}
\selectfont
\begin{lstlisting}

EVOLVE - computes further evolution in time of the already existing
         reach set.

Input:
  regular:
      self.

      newEndTime: double[1, 1] - new end time.

  optional:
      linSys: elltool.linsys.LinSys[1, 1] - new linear system.

Output:
  newReachObj: reach[1, 1] - reach set on time  interval
        [oldT0 newEndTime].

Example:
aMat = [0 1; 0 0]; bMat = eye(2);
SUBounds = struct();
SUBounds.center = {'sin(t)'; 'cos(t)'};
SUBounds.shape = [9 0; 0 2];
sys = elltool.linsys.LinSysContinuous(aMat, bMat, SUBounds);
dsys = elltool.linsys.LinSysDiscrete(aMat, bMat, SUBounds);
x0EllObj = ell_unitball(2);
timeVec = [0 10];
dirsMat = [1 0; 0 1]';
rsObj = elltool.reach.ReachContinuous(sys, x0EllObj, dirsMat, timeVec);
dRsObj = elltool.reach.ReachDiscrete(dsys, x0EllObj, dirsMat, timeVec);
newDRsObj = dRsObj.evolve(11);





\end{lstlisting}
\fontfamily{\familydefault}
\selectfont
\item {elltool.reach.ReachDiscrete.projection}
\fontfamily{pcr}
\selectfont
\begin{lstlisting}

PROJECTION - projects the reach set self onto the orthogonal basis
             specified by the columns of matrix projMat.

Input:
  regular:
      self.
      projMat: double[nRows, nCols] - projection matrix, where nRows
          is dimension of reach set, nCols <= nRows.

Output:
  projObj: elltool.reach.IReach[1, 1] - projected reach set.

Examples:
aMat = [0 1; 0 0]; bMat = eye(2);
SUBounds = struct();
SUBounds.center = {'sin(t)'; 'cos(t)'};
SUBounds.shape = [9 0; 0 2];
sys = elltool.linsys.LinSysContinuous(aMat, bMat, SUBounds);
dsys = elltool.linsys.LinSysDiscrete(aMat, bMat, SUBounds);
x0EllObj = ell_unitball(2);
timeVec = [0 10];
dirsMat = [1 0; 0 1]';
rsObj = elltool.reach.ReachContinuous(sys, x0EllObj, dirsMat, timeVec);
dRsObj = elltool.reach.ReachRiscrete(dsys, x0EllObj, dirsMat, timeVec);
projMat = eye(2);
projObj = rsObj.projection(projMat);
dProjObj = dRsObj.projection(projMat);





\end{lstlisting}
\fontfamily{\familydefault}
\selectfont
\item {elltool.reach.ReachDiscrete.plot\_ia}
\fontfamily{pcr}
\selectfont
\begin{lstlisting}

PLOT_IA - plots internal approximations of 2D and 3D reach sets.

Input:
  regular:
      self.

  optional:
      colorSpec: char[1, 1] - set color to plot in following way:
                             'r' - red color,
                             'g' - green color,
                             'b' - blue color,
                             'y' - yellow color,
                             'c' - cyan color,
                             'm' - magenta color,
                             'w' - white color.

      OptStruct: struct[1, 1] with fields:
          color: double[1, 3] - sets color of the picture in the form
                [x y z].
          width: double[1, 1] - sets line width for 2D plots.
          shade: double[1, 1] in [0; 1] interval - sets transparency level
                (0 - transparent, 1 - opaque).
           fill: double[1, 1] - if set to 1, reach set will be filled with
                color.

Example:
aMat = [0 1; 0 0]; bMat = eye(2);
SUBounds = struct();
SUBounds.center = {'sin(t)'; 'cos(t)'};
SUBounds.shape = [9 0; 0 2];
sys = elltool.linsys.LinSysContinuous(aMat, bMat, SUBounds);
x0EllObj = ell_unitball(2);
timeVec = [0 10];
dirsMat = [1 0; 0 1]';
rsObj = elltool.reach.ReachContinuous(sys, x0EllObj, dirsMat, timeVec);
rsObj.plot_ia();
dsys = elltool.linsys.LinSysDiscrete(aMat, bMat, SUBounds);
dRsObj = elltool.reach.ReachDiscrete(sys, x0EllObj, dirsMat, timeVec);
dRsObj.plot_ia();





\end{lstlisting}
\fontfamily{\familydefault}
\selectfont
\item {elltool.reach.ReachDiscrete.plot\_ea}
\fontfamily{pcr}
\selectfont
\begin{lstlisting}

PLOT_EA - plots external approximations of 2D and 3D reach sets.

Input:
  regular:
      self.

  optional:
      colorSpec: char[1, 1] - set color to plot in following way:
                             'r' - red color,
                             'g' - green color,
                             'b' - blue color,
                             'y' - yellow color,
                             'c' - cyan color,
                             'm' - magenta color,
                             'w' - white color.

      OptStruct: struct[1, 1] with fields:
          color: double[1, 3] - sets color of the picture in the form
                [x y z].
          width: double[1, 1] - sets line width for 2D plots.
          shade: double[1, 1] in [0; 1] interval - sets transparency level
                (0 - transparent, 1 - opaque).
           fill: double[1, 1] - if set to 1, reach set will be filled with
                 color.

Output:
  None.

Example:
aMat = [0 1; 0 0]; bMat = eye(2);
SUBounds = struct();
SUBounds.center = {'sin(t)'; 'cos(t)'};
SUBounds.shape = [9 0; 0 2];
sys = elltool.linsys.LinSysContinuous(aMat, bMat, SUBounds);
x0EllObj = ell_unitball(2);
timeVec = [0 10];
dirsMat = [1 0; 0 1]';
rsObj = elltool.reach.ReachContinuous(sys, x0EllObj, dirsMat, timeVec);
rsObj.plot_ea();
dsys = elltool.linsys.LinSysDiscrete(aMat, bMat, SUBounds);
dRsObj = elltool.reach.ReachDiscrete(sys, x0EllObj, dirsMat, timeVec);
dRsObj.plot_ea();





\end{lstlisting}
\fontfamily{\familydefault}
\selectfont
\item {elltool.reach.ReachDiscrete.get\_system}
\fontfamily{pcr}
\selectfont
\begin{lstlisting}

GET_SYSTEM - returns the linear system for which the reach set is
             computed.

Input:
  regular:
      self.

Output:
  linSys: elltool.linsys.LinSys[1, 1] - linear system object.

Example:
aMat = [0 1; 0 0]; bMat = eye(2);
SUBounds = struct();
SUBounds.center = {'sin(t)'; 'cos(t)'};
SUBounds.shape = [9 0; 0 2];
sys = elltool.linsys.LinSysContinuous(aMat, bMat, SUBounds);
x0EllObj = ell_unitball(2);
timeVec = [0 10];
dirsMat = [1 0; 0 1]';
rsObj = elltool.reach.ReachContinuous(sys, x0EllObj, dirsMat, timeVec);
linSys = rsObj.get_system()

self =
A:
     0     1
     0     0


B:
     1     0
     0     1


Control bounds:
   2-dimensional ellipsoid with center
    'sin(t)'
    'cos(t)'

   and shape matrix
     9     0
     0     2


C:
     1     0
     0     1

2-input, 2-output continuous-time linear time-invariant system of
        dimension 2:
dx/dt  =  A x(t)  +  B u(t)
 y(t)  =  C x(t)

dsys = elltool.linsys.LinSysDiscrete(aMat, bMat, SUBounds);
dRsObj = elltool.reach.ReachDiscrete(sys, x0EllObj, dirsMat, timeVec);
dRsObj.get_system();





\end{lstlisting}
\fontfamily{\familydefault}
\selectfont
\item {elltool.reach.ReachDiscrete.get\_goodcurves}
\fontfamily{pcr}
\selectfont
\begin{lstlisting}
GET_GOODCURVES - returns the 'good curve' trajectories of the reach set.

Input:
  regular:
      self.

Output:
  goodCurvesCVec: cell[1, nPoints] of double [x, y] - array of cells,
      where each cell is array of points that form a 'good curve'.

  timeVec: double[1, nPoints] - array of time values.

Example:
aMat = [0 1; 0 0]; bMat = eye(2);
SUBounds = struct();
SUBounds.center = {'sin(t)'; 'cos(t)'};
SUBounds.shape = [9 0; 0 2];
sys = elltool.linsys.LinSysContinuous(aMat, bMat, SUBounds);
x0EllObj = ell_unitball(2);
timeVec = [0 10];
dirsMat = [1 0; 0 1]';
rsObj = elltool.reach.ReachContinuous(sys, x0EllObj, dirsMat, timeVec);
[goodCurvesCVec timeVec] = rsObj.get_goodcurves();

dsys = elltool.linsys.LinSysDiscrete(aMat, bMat, SUBounds);
dRsObj = elltool.reach.ReachDiscrete(sys, x0EllObj, dirsMat, timeVec);
[goodCurvesCVec timeVec] = dRsObj.get_goodcurves();





\end{lstlisting}
\fontfamily{\familydefault}
\selectfont
\item {elltool.reach.ReachDiscrete.get\_ia}
\fontfamily{pcr}
\selectfont
\begin{lstlisting}

GET_IA - returns array of ellipsoid objects representing internal
         approximation of the  reach tube.

Input:
  regular:
      self.

Output:
  iaEllMat: ellipsoid[nAppr, nPoints] - array of ellipsoids, where nAppr
      is the number of approximations, nPoints is number of points in time
      grid.

  timeVec: double[1, nPoints] - array of time values.

Example:
aMat = [0 1; 0 0]; bMat = eye(2);
SUBounds = struct();
SUBounds.center = {'sin(t)'; 'cos(t)'};
SUBounds.shape = [9 0; 0 2];
sys = elltool.linsys.LinSysContinuous(aMat, bMat, SUBounds);
x0EllObj = ell_unitball(2);
timeVec = [0 10];
dirsMat = [1 0; 0 1]';
rsObj = elltool.reach.ReachContinuous(sys, x0EllObj, dirsMat, timeVec);
[iaEllMat timeVec] = rsObj.get_ia();





\end{lstlisting}
\fontfamily{\familydefault}
\selectfont
\item {elltool.reach.ReachDiscrete.get\_ea}
\fontfamily{pcr}
\selectfont
\begin{lstlisting}

GET_EA - returns array of ellipsoid objects representing external
         approximation of the reach  tube.

Input:
  regular:
      self.

Output:
  eaEllMat: ellipsoid[nAppr, nPoints] - array of ellipsoids, where nAppr
      is the number of approximations, nPoints is number of points in time
      grid.

   timeVec: double[1, nPoints] - array of time values.

Example:
aMat = [0 1; 0 0]; bMat = eye(2);
SUBounds = struct();
SUBounds.center = {'sin(t)'; 'cos(t)'};
SUBounds.shape = [9 0; 0 2];
sys = elltool.linsys.LinSysContinuous(aMat, bMat, SUBounds);
x0EllObj = ell_unitball(2);
timeVec = [0 10];
dirsMat = [1 0; 0 1]';
rsObj = elltool.reach.ReachContinuous(sys, x0EllObj, dirsMat, timeVec);
[eaEllMat timeVec] = rsObj.get_ea();

dsys = elltool.linsys.LinSysDiscrete(aMat, bMat, SUBounds);
dRsObj = elltool.reach.ReachDiscrete(sys, x0EllObj, dirsMat, timeVec);
[eaEllMat timeVec] = dRsObj.get_ea();





\end{lstlisting}
\fontfamily{\familydefault}
\selectfont
\item {elltool.reach.ReachDiscrete.get\_directions}
\fontfamily{pcr}
\selectfont
\begin{lstlisting}

GET_DIRECTIONS - returns the values of direction vectors for time grid
                 values.

Input:
  regular:
      self.

Output:
  directionsCVec: cell[1, nPoints] of double [nDim, nDir] - array of
      cells, where each cell is a sequence of direction vector values
      that correspond to the time values of the grid, where nPoints is
      number of points in time grid.

  timeVec: double[1, nPoints] - array of time values.

Example:
aMat = [0 1; 0 0]; bMat = eye(2);
SUBounds = struct();
SUBounds.center = {'sin(t)'; 'cos(t)'};
SUBounds.shape = [9 0; 0 2];
sys = elltool.linsys.LinSysContinuous(aMat, bMat, SUBounds);
x0EllObj = ell_unitball(2);
timeVec = [0 10];
dirsMat = [1 0; 0 1]';
rsObj = elltool.reach.ReachContinuous(sys, x0EllObj, dirsMat, timeVec);
[directionsCVec timeVec] = rsObj.get_directions();





\end{lstlisting}
\fontfamily{\familydefault}
\selectfont
\item {elltool.reach.ReachDiscrete.get\_center}
\fontfamily{pcr}
\selectfont
\begin{lstlisting}

GET_CENTER - returns the trajectory of the center of the reach set.

Input:
  regular:
      self.

Output:
  trCenterMat: double[nDim, nPoints] - array of points that form the
      trajectory of the reach set center, where nDim is reach set
      dimentsion, nPoints - number of points in time grid.

  timeVec: double[1, nPoints] - array of time values.

Example:
aMat = [0 1; 0 0]; bMat = eye(2);
SUBounds = struct();
SUBounds.center = {'sin(t)'; 'cos(t)'};
SUBounds.shape = [9 0; 0 2];
sys = elltool.linsys.LinSysContinuous(aMat, bMat, SUBounds);
x0EllObj = ell_unitball(2);
timeVec = [0 10];
dirsMat = [1 0; 0 1]';
rsObj = elltool.reach.ReachContinuous(sys, x0EllObj, dirsMat, timeVec);
[trCenterMat timeVec] = rsObj.get_center();





\end{lstlisting}
\fontfamily{\familydefault}
\selectfont
\item {elltool.reach.ReachDiscrete.display}
\fontfamily{pcr}
\selectfont
\begin{lstlisting}

DISPLAY - displays the reach set object.

Input:
  regular:
      self.

Output:
  None.

Example:
aMat = [0 1; 0 0]; bMat = eye(2);
SUBounds = struct();
SUBounds.center = {'sin(t)'; 'cos(t)'};
SUBounds.shape = [9 0; 0 2];
sys = elltool.linsys.LinSysContinuous(aMat, bMat, SUBounds);
x0EllObj = ell_unitball(2);
timeVec = [0 10];
dirsMat = [1 0; 0 1]';
rsObj = elltool.reach.ReachContinuous(sys, x0EllObj, dirsMat, timeVec);
rsObj.display()

rsObj =
Reach set of the continuous-time linear system in R^2 in the time...
     interval [0, 10].

Initial set at time t0 = 0:
Ellipsoid with parameters
Center:
     0
     0

Shape Matrix:
     1     0
     0     1

Number of external approximations: 2
Number of internal approximations: 2





\end{lstlisting}
\fontfamily{\familydefault}
\selectfont
\item {elltool.reach.ReachDiscrete.dimension}
\fontfamily{pcr}
\selectfont
\begin{lstlisting}

DIMENSION - returns the dimension of the reach set.

Input:
  regular:
      self.

Output:
  rSdim: double[1, 1] - reach set dimension.
  sSdim: double[1, 1] - state space dimension.

Example:
aMat = [0 1; 0 0]; bMat = eye(2);
SUBounds = struct();
SUBounds.center = {'sin(t)'; 'cos(t)'};
SUBounds.shape = [9 0; 0 2];
sys = elltool.linsys.LinSysContinuous(aMat, bMat, SUBounds);
x0EllObj = ell_unitball(2);
timeVec = [0 10];
dirsMat = [1 0; 0 1]';
rsObj = elltool.reach.ReachContinuous(sys, x0EllObj, dirsMat, timeVec);
[rSdim sSdim] = rsObj.dimension()

rSdim =

         2


sSdim =

         2





\end{lstlisting}
\fontfamily{\familydefault}
\selectfont
\item {elltool.reach.ReachDiscrete.cut}
\fontfamily{pcr}
\selectfont
\begin{lstlisting}
CUT - extracts the piece of reach tube from given start time to given
      end time. Given reach set self, find states that are reachable
      within time interval specified by cutTimeVec. If cutTimeVec
      is a scalar, then reach set at given time is returned.

Input:
  regular:
      self.

   cutTimeVec: double[1, 2]/double[1, 1] - time interval to cut.

Output:
  cutObj: elltool.reach.IReach[1, 1] - reach set resulting from the CUT
        operation.

Example:
aMat = [0 1; 0 0]; bMat = eye(2);
SUBounds = struct();
SUBounds.center = {'sin(t)'; 'cos(t)'};
SUBounds.shape = [9 0; 0 2];
sys = elltool.linsys.LinSysContinuous(aMat, bMat, SUBounds);
x0EllObj = ell_unitball(2);
timeVec = [0 10];
dirsMat = [1 0; 0 1]';
rsObj = elltool.reach.ReachContinuous(sys, x0EllObj, dirsMat, timeVec);
cutObj = rsObj.cut([3 5]);
dRsObj = elltool.reach.ReachDiscrete(dtsys, x0EllObj, dirsMat, timeVec);
dCutObj = dRsObj.cut([3 5]);





\end{lstlisting}
\fontfamily{\familydefault}
\selectfont
\item {elltool.reach.ReachDiscrete.getCopy}
\fontfamily{pcr}
\selectfont
\begin{lstlisting}
GETCOPY - create a new copy of Self reach object.

Input:
  self: reach[1, 1] - reach set object, copy of which should be create.

Output:
    newReachObj - reach set object.

Example:
adMat = [0 1; -1 -0.5];
bdMat = [0; 1];
udBoundsEllObj  = ellipsoid(1);
dtsys = elltool.linsys.LinSysDiscrete(adMat, bdMat, udBoundsEllObj);
x0EllObj = ell_unitball(2);
timeVec = [0 10];
dirsMat = [1 0; 0 1]';
dRsObj = elltool.reach.ReachDiscrete(dtsys, x0EllObj, dirsMat, timeVec);
newDRsObj = dRsObj.getCopy();



\end{lstlisting}
\fontfamily{\familydefault}
\selectfont
\item {elltool.reach.ReachDiscrete.ReachDiscrete}
\fontfamily{pcr}
\selectfont
\begin{lstlisting}
ReachDiscrete - computes reach set approximation of the discrete linear
                system for the given time interval.


Input:
    linSys: elltool.linsys.LinSys object - given linear system
    x0Ell: ellipsoid[1, 1] - ellipsoidal set of initial conditions
    l0Mat: matrix of double - l0Mat
    timeVec: double[1, 2] - time interval
    OptStruct: struct[1, 1] - structure with
    fields:
        approximation: int[1, 1] - field, which mean the following values
         for type approximation:
          = 0 for external,
          = 1 for internal,
          = 2 for both (default).
        save_all: logical [1, 1] - field, which
          = 1 if save intermediate calculation data,
          = 0 (default) if delete intermediate calculation data.
        minmax: logical[1, 1] - field, which:
          = 1 compute minmax reach set,
          = 0 (default) compute maxmin reach set.
        This option makes sense only for discrete-time systems with
        disturbance.

self = ReachDiscrete(linSys, x0Ell, l0Mat,timeVec, Options, prop) is the
same as self = ReachDiscrete(linSys, x0Ell, l0Mat, timeVec, Options), but
with "Properties"  specified in  prop.
In other cases "Properties" are taken from current values stored in
elltool.conf.Properties

As "Properties" we understand here such list of ellipsoid properties:
        absTol
        relTol
        nPlot2dPoints
        Plot3dPoints
        nTimeGridPoints

Output:
  regular:
      self - reach set object.

Example:
adMat = [0 1; -1 -0.5];
bdMat = [0; 1];
udBoundsEllObj  = ellipsoid(1);
dtsys = elltool.linsys.LinSysDiscrete(adMat, bdMat, udBoundsEllObj);
x0EllObj = ell_unitball(2);
timeVec = [0 10];
dirsMat = [1 0; 0 1]';
dRsObj = elltool.reach.ReachDiscrete(dtsys, x0EllObj, dirsMat, timeVec);




\end{lstlisting}
\fontfamily{\familydefault}
\selectfont
\item {elltool.reach.ReachDiscrete.intersect}
\fontfamily{pcr}
\selectfont
\begin{lstlisting}
INTERSECT - checks if its external (s = 'e'), or internal (s = 'i')
            approximation intersects with given ellipsoid, hyperplane
            or polytop.

Input:
  regular:
      self.

      intersectObj: ellipsoid[1, 1]/hyperplane[1,1]/polytop[1, 1].

      approxTypeChar: char[1, 1] - 'e' (default) - external approximation,
                                   'i' - internal approximation.

Output:
  isEmptyIntersect: logical[1, 1] -  true - if intersection is nonempty,
                                     false - otherwise.

Example:
aMat = [0 1; 0 0]; bMat = eye(2);
SUBounds = struct();
SUBounds.center = {'sin(t)'; 'cos(t)'};
SUBounds.shape = [9 0; 0 2];
sys = elltool.linsys.LinSysContinuous(aMat, bMat, SUBounds);
x0EllObj = ell_unitball(2);
timeVec = [0 10];
dirsMat = [1 0; 0 1]';
rsObj = elltool.reach.ReachContinuous(sys, x0EllObj, dirsMat, timeVec);
ellObj = ellipsoid([0; 0], 2*eye(2));
isEmptyIntersect = intersect(rsObj, ellObj)

sEmptyIntersect =

                1





\end{lstlisting}
\fontfamily{\familydefault}
\selectfont
\item {elltool.reach.ReachDiscrete.isempty}
\fontfamily{pcr}
\selectfont
\begin{lstlisting}

ISEMPTY - checks if given reach set is an empty object.

Input:
  regular:
      self.

Output:
  isEmpty: logical[1, 1] - true - if self is empty, Ffalse - otherwise.

Example:
aMat = [0 1; 0 0]; bMat = eye(2);
SUBounds = struct();
SUBounds.center = {'sin(t)'; 'cos(t)'};
SUBounds.shape = [9 0; 0 2];
sys = elltool.linsys.LinSysContinuous(aMat, bMat, SUBounds);
dsys = elltool.linsys.LinSysContinuous(aMat, bMat, SUBounds);
x0EllObj = ell_unitball(2);
timeVec = [0 10];
dirsMat = [1 0; 0 1]';
rsObj = elltool.reach.ReachContinuous(sys, x0EllObj, dirsMat, timeVec);
dRsObj = elltool.reach.ReachRiscrete(dsys, x0EllObj, dirsMat, timeVec);
dRsObj.isempty();
rsObj.isempty()

ans =

     0





\end{lstlisting}
\fontfamily{\familydefault}
\selectfont
\item {elltool.reach.ReachDiscrete.iscut}
\fontfamily{pcr}
\selectfont
\begin{lstlisting}
ISCUT - checks if given reach set object is a cut of another reach set.

Input:
  regular:
      self.

Output:
  isCut: logical[1, 1] - true - if self is a cut of the reach set,
                         false - otherwise.

Example:
aMat = [0 1; 0 0]; bMat = eye(2);
SUBounds = struct();
SUBounds.center = {'sin(t)'; 'cos(t)'};
SUBounds.shape = [9 0; 0 2];
sys = elltool.linsys.LinSysContinuous(aMat, bMat, SUBounds);
dsys = elltool.linsys.LinSysDiscrete(aMat, bMat, SUBounds);
x0EllObj = ell_unitball(2);
timeVec = [0 10];
dirsMat = [1 0; 0 1]';
rsObj = elltool.reach.ReachContinuous(sys, x0EllObj, dirsMat, timeVec);
dRsObj = elltool.reach.ReachRiscrete(dsys, x0EllObj, dirsMat, timeVec);
cutObj = rsObj.cut([3 5]);
iscut(cutObj);
cutObj = dRsObj.cut([4 8]);
iscut(cutObj);





\end{lstlisting}
\fontfamily{\familydefault}
\selectfont
\item {elltool.reach.ReachDiscrete.isprojection}
\fontfamily{pcr}
\selectfont
\begin{lstlisting}

ISPROJECTION - checks if given reach set object is a projection.

Input:
  regular:
      self.

Output:
  isProj: logical[1, 1] - true - if self is projection, false - otherwise.


Example:
aMat = [0 1; 0 0]; bMat = eye(2);
SUBounds = struct();
SUBounds.center = {'sin(t)'; 'cos(t)'};
SUBounds.shape = [9 0; 0 2];
sys = elltool.linsys.LinSysContinuous(aMat, bMat, SUBounds);
dsys = elltool.linsys.LinSysDiscrete(aMat, bMat, SUBounds);
x0EllObj = ell_unitball(2);
timeVec = [0 10];
dirsMat = [1 0; 0 1]';
rsObj = elltool.reach.ReachContinuous(sys, x0EllObj, dirsMat, timeVec);
dRsObj = elltool.reach.ReachRiscrete(dsys, x0EllObj, dirsMat, timeVec);
projMat = eye(2);
projObj = rsObj.projection(projMat);
isprojection(projObj);
projObj = dRsObj.projection(projMat);
isprojection(projObj);





\end{lstlisting}
\fontfamily{\familydefault}
\selectfont
\item {elltool.linsys.LinSysDiscrete.display}
\todo[inline]{Updated all help headers in linsys}
\fontfamily{pcr}
\selectfont
\begin{lstlisting}

DISPLAY - displays the details of linear system object.

Input:
  regular:
      self: elltool.linsys.ILinSys[1, 1] - linear system.

Output:
  None.

Example:
aMat = [0 1; 0 0]; bMat = eye(2);
SUBounds = struct();
SUBounds.center = {'sin(t)'; 'cos(t)'};
SUBounds.shape = [9 0; 0 2];
sys = elltool.linsys.LinSysContinuous(aMat, bMat, SUBounds);
sys.display()





\end{lstlisting}
\fontfamily{\familydefault}
\selectfont
\item {elltool.linsys.LinSysDiscrete.LinSysDiscrete}
\fontfamily{pcr}
\selectfont
\begin{lstlisting}
LINSYSDISCRETE - constructor of discrete linear system object.

Discrete-time linear system:
          x[k+1]  =  A[k] x[k]  +  B[k] u[k]  +  G[k] v[k]
            y[k]  =  C[k] x[k]  +  w[k]

Input:
  regular:
      atInpMat: double[nDim, nDim]/cell[nDim, nDim] - matrix A.

      btInpMat: double[nDim, kDim]/cell[nDim, kDim] - matrix B.

      uBoundsEll: ellipsoid[1, 1]/struct[1, 1] - control bounds
          ellipsoid.

      gtInpMat: double[nDim, lDim]/cell[nDim, lDim] - matrix G.

      distBoundsEll: ellipsoid[1, 1]/struct[1, 1] - disturbance bounds
          ellipsoid.

      ctInpMat: double[mDim, nDim]/cell[mDim, nDim]- matrix C.

      noiseBoundsEll: ellipsoid[1, 1]/struct[1, 1] - noise bounds
          ellipsoid.

      discrFlag: char[1, 1] - if discrFlag set:
           'd' - to discrete-time linSys
           not 'd' - to continuous-time linSys.

Output:
  self: elltool.linsys.LinSysDiscrete[1, 1] - discrete linear system.

Example:
for k = 1:20
   atMat = {'0' '1 + cos(pi*k/2)'; '-2' '0'};
   btMat =  [0; 1];
   uBoundsEllObj = ellipsoid(4);
   gtMat = [1; 0];
   distBounds = 1/(k+1);
   ctVec = [1 0];
   lsys = elltool.linsys.LinSysDiscrete(atMat, btMat,...
       uBoundsEllObj, gtMat,distBounds, ctVec);
end




\end{lstlisting}
\fontfamily{\familydefault}
\selectfont
\item {elltool.linsys.LinSysDiscrete.isEqual}
\fontfamily{pcr}
\selectfont
\begin{lstlisting}

ISEQUAL - produces produces logical array the same size as
          self/compLinSysArr (if they have the same).
          isEqualArr[iDim1, iDim2,...] is true if corresponding
          linear systems are equal and false otherwise.

Input:
  regular:
      self: elltool.linsys.ILinSys[nDims1, nDims2,...] -  an array of
           linear systems.
      compLinSysArr: elltool.linsys.ILinSys[nDims1,...nDims2,...] - an
           array of linear systems.

Output:
  isEqualArr: elltool.linsys.LinSys[nDims1, nDims2,...] - an array of
      logical values.
      isEqualArr[iDim1, iDim2,...] is true if corresponding linear systems
      are equal and false otherwise.

Examples:
aMat = [0 1; 0 0]; bMat = eye(2);
SUBounds = struct();
SUBounds.center = {'sin(t)'; 'cos(t)'};
SUBounds.shape = [9 0; 0 2];
sys = elltool.linsys.LinSysContinuous(aMat, bMat, SUBounds);
newSys = sys.getCopy();
isEqual = sys.isEqual(newSys)

isEqual =

     1
dsys = elltool.linsys.LinSysDiscrete(aMat, bMat, SUBounds);
newDSys = sys.getCopy();
isEqual = dsys.isEqual(newDSys)

isEqual =

     1




\end{lstlisting}
\fontfamily{\familydefault}
\selectfont
\item {elltool.linsys.LinSysDiscrete.getCopy}
\fontfamily{pcr}
\selectfont
\begin{lstlisting}

GETCOPY - gives array the same size as linsysArr with with copies of
          elements of self.

Input:
  regular:
      self: elltool.linsys.ALinSys[nDims1, nDims2,...] - an array of
            linear systems.

Output:
  copyLinSysArr: elltool.linsys.LinSys[nDims1, nDims2,...] -  an array of
     copies of elements of self.

Examples:
aMat = [0 1; 0 0]; bMat = eye(2);
SUBounds = struct();
SUBounds.center = {'sin(t)'; 'cos(t)'};
SUBounds.shape = [9 0; 0 2];
sys = elltool.linsys.LinSysContinuous(aMat, bMat, SUBounds);
newSys = sys.getCopy();
dsys = elltool.linsys.LinSysDiscrete(aMat, bMat, SUBounds);
newDSys = dsys.getCopy();





\end{lstlisting}
\fontfamily{\familydefault}
\selectfont
\item {elltool.linsys.LinSysDiscrete.getAbsTol}
\fontfamily{pcr}
\selectfont
\begin{lstlisting}

GETABSTOL - gives array the same size as linsysArr with values of absTol
            properties for each hyperplane in hplaneArr.


Input:
  regular:
      self: elltool.linsys.LinSys[nDims1, nDims2,...] - an array of linear
            systems.

Output:
  absTolArr: double[nDims1, nDims2,...] - array of absTol properties for
      linear systems in self.

Examples:
aMat = [0 1; 0 0]; bMat = eye(2);
SUBounds = struct();
SUBounds.center = {'sin(t)'; 'cos(t)'};
SUBounds.shape = [9 0; 0 2];
sys = elltool.linsys.LinSysContinuous(aMat, bMat, SUBounds);
sys.getAbsTol();
dsys = elltool.linsys.LinSysDiscrete(aMat, bMat, SUBounds);
dsys.getAbsTol();




\end{lstlisting}
\fontfamily{\familydefault}
\selectfont
\item {elltool.linsys.LinSysDiscrete.islti}
\fontfamily{pcr}
\selectfont
\begin{lstlisting}

ISLTI - checks if linear system is time-invariant.

Input:
  regular:
      self: elltool.linsys.LinSys[nDims1, nDims2,...] - an array of linear
            systems.

Output:
  isLtiMat: logical[nDims1, nDims2,...] -array such that it's element at
      each position is true if corresponding linear system is
      time-invariant, and false otherwise.

Examples:
aMat = [0 1; 0 0]; bMat = eye(2);
SUBounds = struct();
SUBounds.center = {'sin(t)'; 'cos(t)'};
SUBounds.shape = [9 0; 0 2];
sys = elltool.linsys.LinSysContinuous(aMat, bMat, SUBounds);
isLtiArr = sys.islti();
dsys = elltool.linsys.LinSysDiscrete(aMat, bMat, SUBounds);
isLtiArr = dsys.islti();





\end{lstlisting}
\fontfamily{\familydefault}
\selectfont
\item {elltool.linsys.LinSysDiscrete.isempty}
\fontfamily{pcr}
\selectfont
\begin{lstlisting}

ISEMPTY - checks if linear system is empty.

Input:
  regular:
      self: elltool.linsys.LinSys[nDims1, nDims2,...] - an array of linear
            systems.

Output:
  isEmptyMat: logical[nDims1, nDims2,...] - array such that it's element at
      each position is true if corresponding linear system is empty, and
      false otherwise.

Examples:
aMat = [0 1; 0 0]; bMat = eye(2);
SUBounds = struct();
SUBounds.center = {'sin(t)'; 'cos(t)'};
SUBounds.shape = [9 0; 0 2];
sys = elltool.linsys.LinSysContinuous(aMat, bMat, SUBounds);
sys.isempty()

ans =

     0
dsys = elltool.linsys.LinSysDiscrete(aMat, bMat, SUBounds);
dsys.isempty();





\end{lstlisting}
\fontfamily{\familydefault}
\selectfont
\item {elltool.linsys.LinSysDiscrete.hasnoise}
\fontfamily{pcr}
\selectfont
\begin{lstlisting}

HASNOISE - checks if linear system has unknown bounded noise.

Input:
  regular:
      self: elltool.linsys.LinSys[nDims1, nDims2,...] - an array of linear
            systems.

Output:
  isNoiseMat: logical[nDims1, nDims2,...] - array such that it's element at
      each position is true if corresponding linear system has noise, and
      false otherwise.

Examples:
aMat = [0 1; 0 0]; bMat = eye(2);
SUBounds = struct();
SUBounds.center = {'sin(t)'; 'cos(t)'};
SUBounds.shape = [9 0; 0 2];
sys = elltool.linsys.LinSysContinuous(aMat, bMat, SUBounds);
sys.hasnoise()

ans =

     0
dsys = elltool.linsys.LinSysDiscrete(aMat, bMat, SUBounds);
dsys.hasnoise();





\end{lstlisting}
\fontfamily{\familydefault}
\selectfont
\item {elltool.linsys.LinSysDiscrete.hasdisturbance}
\fontfamily{pcr}
\selectfont
\begin{lstlisting}

HASDISTURBANCE - checks if linear system has unknown bounded
                 isturbance.

Input:
  regular:
      self: elltool.linsys.LinSys[nDims1, nDims2,...] - an array of
            linear systems.
  optional:
      isMeaningful: logical[1,1] - if true(default), treat constant
                    disturbance vector as absence of disturbance

Output:
  isDisturbanceArr: logical[nDims1, nDims2,...] - array such that it's
      element at each position is true if corresponding linear system
      has disturbance, and false otherwise.

Examples:
aMat = [0 1; 0 0]; bMat = eye(2);
SUBounds = struct();
SUBounds.center = {'sin(t)'; 'cos(t)'};
SUBounds.shape = [9 0; 0 2];
sys = elltool.linsys.LinSysContinuous(aMat, bMat, SUBounds);
sys.hasdisturbance()

ans =

     0
dsys = elltool.linsys.LinSysDiscrete(aMat, bMat, SUBounds);
dsys.hasdisturbance();





\end{lstlisting}
\fontfamily{\familydefault}
\selectfont
\item {elltool.linsys.LinSysDiscrete.dimension}
\fontfamily{pcr}
\selectfont
\begin{lstlisting}

DIMENSION - returns dimensions of state, input, output and disturbance
            spaces.
Input:
  regular:
      self: elltool.linsys.LinSys[nDims1, nDims2,...] - an array of
            linear systems.

Output:
  stateDimArr: double[nDims1, nDims2,...] - array of state space
      dimensions.

  inpDimArr: double[nDims1, nDims2,...] - array of input dimensions.

  outDimArr: double[nDims1, nDims2,...] - array of output dimensions.

  distDimArr: double[nDims1, nDims2,...] - array of disturbance
        dimensions.

Examples:
aMat = [0 1; 0 0]; bMat = eye(2);
SUBounds = struct();
SUBounds.center = {'sin(t)'; 'cos(t)'};
SUBounds.shape = [9 0; 0 2];
sys = elltool.linsys.LinSysContinuous(aMat, bMat, SUBounds);
[stateDimArr, inpDimArr, outDimArr, distDimArr] = sys.dimension()

stateDimArr =

     2


inpDimArr =

     2


outDimArr =

     2


distDimArr =

     0

dsys = elltool.linsys.LinSysDiscrete(aMat, bMat, SUBounds);
dsys.dimension();





\end{lstlisting}
\fontfamily{\familydefault}
\selectfont
\item {elltool.linsys.LinSysDiscrete.getNoiseBoundsEll}
\fontfamily{pcr}
\selectfont
\begin{lstlisting}

Input:
  regular:
      self: elltool.linsys.ILinSys[1, 1] - linear system.

Output:
  noiseEll: ellipsoid[1, 1]/struct[1, 1] - noise bounds ellipsoid.

Examples:
aMat = [0 1; 0 0]; bMat = eye(2);
SUBounds = struct();
SUBounds.center = {'sin(t)'; 'cos(t)'};
SUBounds.shape = [9 0; 0 2];
sys = elltool.linsys.LinSysContinuous(aMat, bMat, SUBounds);
dsys = elltool.linsys.LinSysDiscrete(aMat, bMat, SUBounds);
noiseEll = dsys.getNoiseBoundsEll()

noiseEll =

     []





\end{lstlisting}
\fontfamily{\familydefault}
\selectfont
\item {elltool.linsys.LinSysDiscrete.getCtMat}
\fontfamily{pcr}
\selectfont
\begin{lstlisting}

Input:
  regular:
      self: elltool.linsys.ILinSys[1, 1] - linear system.

Output:
  cMat: double[cMatDim, cMatDim]/cell[cMatDim, cMatDim] - matrix C.

Examples:
aMat = [0 1; 0 0]; bMat = eye(2);
SUBounds = struct();
SUBounds.center = {'sin(t)'; 'cos(t)'};
SUBounds.shape = [9 0; 0 2];
sys = elltool.linsys.LinSysContinuous(aMat, bMat, SUBounds);
dsys = elltool.linsys.LinSysDiscrete(aMat, bMat, SUBounds);
cMat = sys.getCtMat()

cMat =

     1     0
     0     1





\end{lstlisting}
\fontfamily{\familydefault}
\selectfont
\item {elltool.linsys.LinSysDiscrete.getDistBoundsEll}
\fontfamily{pcr}
\selectfont
\begin{lstlisting}

Input:
  regular:
      self: elltool.linsys.ILinSys[1, 1] - linear system.

Output:
  distEll: ellipsoid[1, 1]/struct[1, 1] - disturbance bounds ellipsoid.

Examples:
aMat = [0 1; 0 0]; bMat = eye(2);
SUBounds = struct();
SUBounds.center = {'sin(t)'; 'cos(t)'};
SUBounds.shape = [9 0; 0 2];
sys = elltool.linsys.LinSysContinuous(aMat, bMat, SUBounds);
dsys = elltool.linsys.LinSysDiscrete(aMat, bMat, SUBounds);
distEll = sys.getDistBoundsEll();





\end{lstlisting}
\fontfamily{\familydefault}
\selectfont
\item {elltool.linsys.LinSysDiscrete.getGtMat}
\fontfamily{pcr}
\selectfont
\begin{lstlisting}

Input:
  regular:
      self: elltool.linsys.ILinSys[1, 1] - linear system.

Output:
  gMat: double[gMatDim, gMatDim]/cell[gMatDim, gMatDim] - matrix G.

Examples:
aMat = [0 1; 0 0]; bMat = eye(2);
SUBounds = struct();
SUBounds.center = {'sin(t)'; 'cos(t)'};
SUBounds.shape = [9 0; 0 2];
sys = elltool.linsys.LinSysContinuous(aMat, bMat, SUBounds);
dsys = elltool.linsys.LinSysDiscrete(aMat, bMat, SUBounds);
gMat = sys.getGtMat();





\end{lstlisting}
\fontfamily{\familydefault}
\selectfont
\item {elltool.linsys.LinSysDiscrete.getUBoundsEll}
\fontfamily{pcr}
\selectfont
\begin{lstlisting}

Input:
  regular:
      self: elltool.linsys.ILinSys[1, 1] - linear system.

Output:
  uEll: ellipsoid[1, 1]/struct[1, 1] - control bounds ellipsoid.

Examples:
aMat = [0 1; 0 0]; bMat = eye(2);
SUBounds = struct();
SUBounds.center = {'sin(t)'; 'cos(t)'};
SUBounds.shape = [9 0; 0 2];
sys = elltool.linsys.LinSysContinuous(aMat, bMat, SUBounds);
dsys = elltool.linsys.LinSysDiscrete(aMat, bMat, SUBounds);
uEll = dsys.getUBoundsEll();





\end{lstlisting}
\fontfamily{\familydefault}
\selectfont
\item {elltool.linsys.LinSysDiscrete.getBtMat}
\fontfamily{pcr}
\selectfont
\begin{lstlisting}

Input:
  regular:
      self: elltool.linsys.ILinSys[1, 1] - linear system.

Output:
  bMat: double[bMatDim, bMatDim]/cell[bMatDim, bMatDim] - matrix B.

Examples:
aMat = [0 1; 0 0]; bMat = eye(2);
SUBounds = struct();
SUBounds.center = {'sin(t)'; 'cos(t)'};
SUBounds.shape = [9 0; 0 2];
sys = elltool.linsys.LinSysContinuous(aMat, bMat, SUBounds);
dsys = elltool.linsys.LinSysDiscrete(aMat, bMat, SUBounds);
bMat = dsys.getBtMat();





\end{lstlisting}
\fontfamily{\familydefault}
\selectfont
\item {elltool.linsys.LinSysDiscrete.getAtMat}
\fontfamily{pcr}
\selectfont
\begin{lstlisting}

Input:
  regular:
      self: elltool.linsys.ILinSys[1, 1] - linear system.

Output:
  aMat: double[aMatDim, aMatDim]/cell[nDim, nDim] - matrix A.

Examples:
aMat = [0 1; 0 0]; bMat = eye(2);
SUBounds = struct();
SUBounds.center = {'sin(t)'; 'cos(t)'};
SUBounds.shape = [9 0; 0 2];
sys = elltool.linsys.LinSysContinuous(aMat, bMat, SUBounds);
dsys = elltool.linsys.LinSysDiscrete(aMat, bMat, SUBounds);
aMat = dsys.getAtMat();





\end{lstlisting}
\fontfamily{\familydefault}
\selectfont
\item {elltool.linsys.LinSysContinuous.display}
\fontfamily{pcr}
\selectfont
\begin{lstlisting}

DISPLAY - displays the details of linear system object.

Input:
  regular:
      self: elltool.linsys.ILinSys[1, 1] - linear system.

Output:
  None.

Example:
aMat = [0 1; 0 0]; bMat = eye(2);
SUBounds = struct();
SUBounds.center = {'sin(t)'; 'cos(t)'};
SUBounds.shape = [9 0; 0 2];
sys = elltool.linsys.LinSysContinuous(aMat, bMat, SUBounds);
sys.display()





\end{lstlisting}
\fontfamily{\familydefault}
\selectfont
\item {elltool.linsys.LinSysContinuous.LinSysContinuous}
\fontfamily{pcr}
\selectfont
\begin{lstlisting}
LINSYSCONTINUOUS - Constructor of continuous linear system object.

Continuous-time linear system:
          dx/dt  =  A(t) x(t)  +  B(t) u(t)  +  G(t) v(t)
           y(t)  =  C(t) x(t)  +  w(t)

Input:
  regular:
      atInpMat: double[nDim, nDim]/cell[nDim, nDim] - matrix A.

      btInpMat: double[nDim, kDim]/cell[nDim, kDim] - matrix B.

      uBoundsEll: ellipsoid[1, 1]/struct[1, 1] - control bounds
            ellipsoid.

      gtInpMat: double[nDim, lDim]/cell[nDim, lDim] - matrix G.

      distBoundsEll: ellipsoid[1, 1]/struct[1, 1] - disturbance
            bounds ellipsoid.

      ctInpMat: double[mDim, nDim]/cell[mDim, nDim]- matrix C.

      noiseBoundsEll: ellipsoid[1, 1]/struct[1, 1] - noise bounds
            ellipsoid.

      discrFlag: char[1, 1] - if discrFlag set:
             'd' - to discrete-time linSys,
             not 'd' - to continuous-time linSys.


Output:
  self: elltool.linsys.LinSysContinuous[1, 1] - continuous linear
            system.

Example:
aMat = [0 1; 0 0]; bMat = eye(2);
SUBounds = struct();
SUBounds.center = {'sin(t)'; 'cos(t)'};
SUBounds.shape = [9 0; 0 2];
sys = elltool.linsys.LinSysContinuous(aMat, bMat, SUBounds);




\end{lstlisting}
\fontfamily{\familydefault}
\selectfont
\item {elltool.linsys.LinSysContinuous.isEqual}
\fontfamily{pcr}
\selectfont
\begin{lstlisting}

ISEQUAL - produces produces logical array the same size as
          self/compLinSysArr (if they have the same).
          isEqualArr[iDim1, iDim2,...] is true if corresponding
          linear systems are equal and false otherwise.

Input:
  regular:
      self: elltool.linsys.ILinSys[nDims1, nDims2,...] -  an array of
           linear systems.
      compLinSysArr: elltool.linsys.ILinSys[nDims1,...nDims2,...] - an
           array of linear systems.

Output:
  isEqualArr: elltool.linsys.LinSys[nDims1, nDims2,...] - an array of
      logical values.
      isEqualArr[iDim1, iDim2,...] is true if corresponding linear systems
      are equal and false otherwise.

Examples:
aMat = [0 1; 0 0]; bMat = eye(2);
SUBounds = struct();
SUBounds.center = {'sin(t)'; 'cos(t)'};
SUBounds.shape = [9 0; 0 2];
sys = elltool.linsys.LinSysContinuous(aMat, bMat, SUBounds);
newSys = sys.getCopy();
isEqual = sys.isEqual(newSys)

isEqual =

     1
dsys = elltool.linsys.LinSysDiscrete(aMat, bMat, SUBounds);
newDSys = sys.getCopy();
isEqual = dsys.isEqual(newDSys)

isEqual =

     1




\end{lstlisting}
\fontfamily{\familydefault}
\selectfont
\item {elltool.linsys.LinSysContinuous.getCopy}
\fontfamily{pcr}
\selectfont
\begin{lstlisting}

GETCOPY - gives array the same size as linsysArr with with copies of
          elements of self.

Input:
  regular:
      self: elltool.linsys.ALinSys[nDims1, nDims2,...] - an array of
            linear systems.

Output:
  copyLinSysArr: elltool.linsys.LinSys[nDims1, nDims2,...] -  an array of
     copies of elements of self.

Examples:
aMat = [0 1; 0 0]; bMat = eye(2);
SUBounds = struct();
SUBounds.center = {'sin(t)'; 'cos(t)'};
SUBounds.shape = [9 0; 0 2];
sys = elltool.linsys.LinSysContinuous(aMat, bMat, SUBounds);
newSys = sys.getCopy();
dsys = elltool.linsys.LinSysDiscrete(aMat, bMat, SUBounds);
newDSys = dsys.getCopy();





\end{lstlisting}
\fontfamily{\familydefault}
\selectfont
\item {elltool.linsys.LinSysContinuous.getAbsTol}
\fontfamily{pcr}
\selectfont
\begin{lstlisting}

GETABSTOL - gives array the same size as linsysArr with values of absTol
            properties for each hyperplane in hplaneArr.


Input:
  regular:
      self: elltool.linsys.LinSys[nDims1, nDims2,...] - an array of linear
            systems.

Output:
  absTolArr: double[nDims1, nDims2,...] - array of absTol properties for
      linear systems in self.

Examples:
aMat = [0 1; 0 0]; bMat = eye(2);
SUBounds = struct();
SUBounds.center = {'sin(t)'; 'cos(t)'};
SUBounds.shape = [9 0; 0 2];
sys = elltool.linsys.LinSysContinuous(aMat, bMat, SUBounds);
sys.getAbsTol();
dsys = elltool.linsys.LinSysDiscrete(aMat, bMat, SUBounds);
dsys.getAbsTol();




\end{lstlisting}
\fontfamily{\familydefault}
\selectfont
\item {elltool.linsys.LinSysContinuous.islti}
\fontfamily{pcr}
\selectfont
\begin{lstlisting}

ISLTI - checks if linear system is time-invariant.

Input:
  regular:
      self: elltool.linsys.LinSys[nDims1, nDims2,...] - an array of linear
            systems.

Output:
  isLtiMat: logical[nDims1, nDims2,...] -array such that it's element at
      each position is true if corresponding linear system is
      time-invariant, and false otherwise.

Examples:
aMat = [0 1; 0 0]; bMat = eye(2);
SUBounds = struct();
SUBounds.center = {'sin(t)'; 'cos(t)'};
SUBounds.shape = [9 0; 0 2];
sys = elltool.linsys.LinSysContinuous(aMat, bMat, SUBounds);
isLtiArr = sys.islti();
dsys = elltool.linsys.LinSysDiscrete(aMat, bMat, SUBounds);
isLtiArr = dsys.islti();





\end{lstlisting}
\fontfamily{\familydefault}
\selectfont
\item {elltool.linsys.LinSysContinuous.isempty}
\fontfamily{pcr}
\selectfont
\begin{lstlisting}

ISEMPTY - checks if linear system is empty.

Input:
  regular:
      self: elltool.linsys.LinSys[nDims1, nDims2,...] - an array of linear
            systems.

Output:
  isEmptyMat: logical[nDims1, nDims2,...] - array such that it's element at
      each position is true if corresponding linear system is empty, and
      false otherwise.

Examples:
aMat = [0 1; 0 0]; bMat = eye(2);
SUBounds = struct();
SUBounds.center = {'sin(t)'; 'cos(t)'};
SUBounds.shape = [9 0; 0 2];
sys = elltool.linsys.LinSysContinuous(aMat, bMat, SUBounds);
sys.isempty()

ans =

     0
dsys = elltool.linsys.LinSysDiscrete(aMat, bMat, SUBounds);
dsys.isempty();





\end{lstlisting}
\fontfamily{\familydefault}
\selectfont
\item {elltool.linsys.LinSysContinuous.hasnoise}
\fontfamily{pcr}
\selectfont
\begin{lstlisting}

HASNOISE - checks if linear system has unknown bounded noise.

Input:
  regular:
      self: elltool.linsys.LinSys[nDims1, nDims2,...] - an array of linear
            systems.

Output:
  isNoiseMat: logical[nDims1, nDims2,...] - array such that it's element at
      each position is true if corresponding linear system has noise, and
      false otherwise.

Examples:
aMat = [0 1; 0 0]; bMat = eye(2);
SUBounds = struct();
SUBounds.center = {'sin(t)'; 'cos(t)'};
SUBounds.shape = [9 0; 0 2];
sys = elltool.linsys.LinSysContinuous(aMat, bMat, SUBounds);
sys.hasnoise()

ans =

     0
dsys = elltool.linsys.LinSysDiscrete(aMat, bMat, SUBounds);
dsys.hasnoise();





\end{lstlisting}
\fontfamily{\familydefault}
\selectfont
\item {elltool.linsys.LinSysContinuous.hasdisturbance}
\fontfamily{pcr}
\selectfont
\begin{lstlisting}

HASDISTURBANCE - checks if linear system has unknown bounded
                 isturbance.

Input:
  regular:
      self: elltool.linsys.LinSys[nDims1, nDims2,...] - an array of
            linear systems.
  optional:
      isMeaningful: logical[1,1] - if true(default), treat constant
                    disturbance vector as absence of disturbance

Output:
  isDisturbanceArr: logical[nDims1, nDims2,...] - array such that it's
      element at each position is true if corresponding linear system
      has disturbance, and false otherwise.

Examples:
aMat = [0 1; 0 0]; bMat = eye(2);
SUBounds = struct();
SUBounds.center = {'sin(t)'; 'cos(t)'};
SUBounds.shape = [9 0; 0 2];
sys = elltool.linsys.LinSysContinuous(aMat, bMat, SUBounds);
sys.hasdisturbance()

ans =

     0
dsys = elltool.linsys.LinSysDiscrete(aMat, bMat, SUBounds);
dsys.hasdisturbance();





\end{lstlisting}
\fontfamily{\familydefault}
\selectfont
\item {elltool.linsys.LinSysContinuous.dimension}
\fontfamily{pcr}
\selectfont
\begin{lstlisting}

DIMENSION - returns dimensions of state, input, output and disturbance
            spaces.
Input:
  regular:
      self: elltool.linsys.LinSys[nDims1, nDims2,...] - an array of
            linear systems.

Output:
  stateDimArr: double[nDims1, nDims2,...] - array of state space
      dimensions.

  inpDimArr: double[nDims1, nDims2,...] - array of input dimensions.

  outDimArr: double[nDims1, nDims2,...] - array of output dimensions.

  distDimArr: double[nDims1, nDims2,...] - array of disturbance
        dimensions.

Examples:
aMat = [0 1; 0 0]; bMat = eye(2);
SUBounds = struct();
SUBounds.center = {'sin(t)'; 'cos(t)'};
SUBounds.shape = [9 0; 0 2];
sys = elltool.linsys.LinSysContinuous(aMat, bMat, SUBounds);
[stateDimArr, inpDimArr, outDimArr, distDimArr] = sys.dimension()

stateDimArr =

     2


inpDimArr =

     2


outDimArr =

     2


distDimArr =

     0

dsys = elltool.linsys.LinSysDiscrete(aMat, bMat, SUBounds);
dsys.dimension();





\end{lstlisting}
\fontfamily{\familydefault}
\selectfont
\item {elltool.linsys.LinSysContinuous.getNoiseBoundsEll}
\fontfamily{pcr}
\selectfont
\begin{lstlisting}

Input:
  regular:
      self: elltool.linsys.ILinSys[1, 1] - linear system.

Output:
  noiseEll: ellipsoid[1, 1]/struct[1, 1] - noise bounds ellipsoid.

Examples:
aMat = [0 1; 0 0]; bMat = eye(2);
SUBounds = struct();
SUBounds.center = {'sin(t)'; 'cos(t)'};
SUBounds.shape = [9 0; 0 2];
sys = elltool.linsys.LinSysContinuous(aMat, bMat, SUBounds);
dsys = elltool.linsys.LinSysDiscrete(aMat, bMat, SUBounds);
noiseEll = dsys.getNoiseBoundsEll()

noiseEll =

     []





\end{lstlisting}
\fontfamily{\familydefault}
\selectfont
\item {elltool.linsys.LinSysContinuous.getCtMat}
\fontfamily{pcr}
\selectfont
\begin{lstlisting}

Input:
  regular:
      self: elltool.linsys.ILinSys[1, 1] - linear system.

Output:
  cMat: double[cMatDim, cMatDim]/cell[cMatDim, cMatDim] - matrix C.

Examples:
aMat = [0 1; 0 0]; bMat = eye(2);
SUBounds = struct();
SUBounds.center = {'sin(t)'; 'cos(t)'};
SUBounds.shape = [9 0; 0 2];
sys = elltool.linsys.LinSysContinuous(aMat, bMat, SUBounds);
dsys = elltool.linsys.LinSysDiscrete(aMat, bMat, SUBounds);
cMat = sys.getCtMat()

cMat =

     1     0
     0     1





\end{lstlisting}
\fontfamily{\familydefault}
\selectfont
\item {elltool.linsys.LinSysContinuous.getDistBoundsEll}
\fontfamily{pcr}
\selectfont
\begin{lstlisting}

Input:
  regular:
      self: elltool.linsys.ILinSys[1, 1] - linear system.

Output:
  distEll: ellipsoid[1, 1]/struct[1, 1] - disturbance bounds ellipsoid.

Examples:
aMat = [0 1; 0 0]; bMat = eye(2);
SUBounds = struct();
SUBounds.center = {'sin(t)'; 'cos(t)'};
SUBounds.shape = [9 0; 0 2];
sys = elltool.linsys.LinSysContinuous(aMat, bMat, SUBounds);
dsys = elltool.linsys.LinSysDiscrete(aMat, bMat, SUBounds);
distEll = sys.getDistBoundsEll();





\end{lstlisting}
\fontfamily{\familydefault}
\selectfont
\item {elltool.linsys.LinSysContinuous.getGtMat}
\fontfamily{pcr}
\selectfont
\begin{lstlisting}

Input:
  regular:
      self: elltool.linsys.ILinSys[1, 1] - linear system.

Output:
  gMat: double[gMatDim, gMatDim]/cell[gMatDim, gMatDim] - matrix G.

Examples:
aMat = [0 1; 0 0]; bMat = eye(2);
SUBounds = struct();
SUBounds.center = {'sin(t)'; 'cos(t)'};
SUBounds.shape = [9 0; 0 2];
sys = elltool.linsys.LinSysContinuous(aMat, bMat, SUBounds);
dsys = elltool.linsys.LinSysDiscrete(aMat, bMat, SUBounds);
gMat = sys.getGtMat();





\end{lstlisting}
\fontfamily{\familydefault}
\selectfont
\item {elltool.linsys.LinSysContinuous.getUBoundsEll}
\fontfamily{pcr}
\selectfont
\begin{lstlisting}

Input:
  regular:
      self: elltool.linsys.ILinSys[1, 1] - linear system.

Output:
  uEll: ellipsoid[1, 1]/struct[1, 1] - control bounds ellipsoid.

Examples:
aMat = [0 1; 0 0]; bMat = eye(2);
SUBounds = struct();
SUBounds.center = {'sin(t)'; 'cos(t)'};
SUBounds.shape = [9 0; 0 2];
sys = elltool.linsys.LinSysContinuous(aMat, bMat, SUBounds);
dsys = elltool.linsys.LinSysDiscrete(aMat, bMat, SUBounds);
uEll = dsys.getUBoundsEll();





\end{lstlisting}
\fontfamily{\familydefault}
\selectfont
\item {elltool.linsys.LinSysContinuous.getBtMat}
\fontfamily{pcr}
\selectfont
\begin{lstlisting}

Input:
  regular:
      self: elltool.linsys.ILinSys[1, 1] - linear system.

Output:
  bMat: double[bMatDim, bMatDim]/cell[bMatDim, bMatDim] - matrix B.

Examples:
aMat = [0 1; 0 0]; bMat = eye(2);
SUBounds = struct();
SUBounds.center = {'sin(t)'; 'cos(t)'};
SUBounds.shape = [9 0; 0 2];
sys = elltool.linsys.LinSysContinuous(aMat, bMat, SUBounds);
dsys = elltool.linsys.LinSysDiscrete(aMat, bMat, SUBounds);
bMat = dsys.getBtMat();





\end{lstlisting}
\fontfamily{\familydefault}
\selectfont
\item {elltool.linsys.LinSysContinuous.getAtMat}
\fontfamily{pcr}
\selectfont
\begin{lstlisting}

Input:
  regular:
      self: elltool.linsys.ILinSys[1, 1] - linear system.

Output:
  aMat: double[aMatDim, aMatDim]/cell[nDim, nDim] - matrix A.

Examples:
aMat = [0 1; 0 0]; bMat = eye(2);
SUBounds = struct();
SUBounds.center = {'sin(t)'; 'cos(t)'};
SUBounds.shape = [9 0; 0 2];
sys = elltool.linsys.LinSysContinuous(aMat, bMat, SUBounds);
dsys = elltool.linsys.LinSysDiscrete(aMat, bMat, SUBounds);
aMat = dsys.getAtMat();





\end{lstlisting}
\fontfamily{\familydefault}
\selectfont
\item {elltool.linsys.LinSysFactory.create}
\fontfamily{pcr}
\selectfont
\begin{lstlisting}
CREATE returns linear system object.

Continuous-time linear system:
          dx/dt  =  A(t) x(t)  +  B(t) u(t)  +  G(t) v(t)
           y(t)  =  C(t) x(t)  +  w(t)

Discrete-time linear system:
          x[k+1]  =  A[k] x[k]  +  B[k] u[k]  +  G[k] v[k]
            y[k]  =  C[k] x[k]  +  w[k]

Input:
  regular:
  regular:
      atInpMat: double[nDim, nDim]/cell[nDim, nDim] - matrix A.

      btInpMat: double[nDim, kDim]/cell[nDim, kDim] - matrix B.

      uBoundsEll: ellipsoid[1, 1]/struct[1, 1] - control bounds
          ellipsoid.

      gtInpMat: double[nDim, lDim]/cell[nDim, lDim] - matrix G.

      distBoundsEll: ellipsoid[1, 1]/struct[1, 1] - disturbance bounds
          ellipsoid.

      ctInpMat: double[mDim, nDim]/cell[mDim, nDim]- matrix C.

      noiseBoundsEll: ellipsoid[1, 1]/struct[1, 1] -  noise bounds
         ellipsoid.

      discrFlag: char[1, 1] - if discrFlag set:
          'd' - to discrete-time linSys
          not 'd' - to continuous-time linSys.

Output:
  linSys: elltool.linsys.LinSysContinuous[1, 1]/
      elltool.linsys.LinSysDiscrete[1, 1] - linear system.

Examples:
aMat = [0 1; 0 0]; bMat = eye(2);
SUBounds = struct();
SUBounds.center = {'sin(t)'; 'cos(t)'};
SUBounds.shape = [9 0; 0 2];
sys = elltool.linsys.LinSysFactory.create(aMat, bMat,SUBounds);




\end{lstlisting}
\fontfamily{\familydefault}
\selectfont
\item {elltool.linsys.LinSysFactory.LinSysFactory}
\fontfamily{pcr}
\selectfont
\begin{lstlisting}
Factory class of linear system objects of the Ellipsoidal Toolbox.



\end{lstlisting}
\fontfamily{\familydefault}
\selectfont
\item {elltool.linsys.LinSysFactory.empty}
\fontfamily{pcr}
\selectfont
\begin{lstlisting}



\end{lstlisting}
\fontfamily{\familydefault}
\selectfont
\end{enumerate}
