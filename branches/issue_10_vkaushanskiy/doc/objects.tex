\section{ellOptions}
Functions of the {\it Ellipsoidal Toolbox} can be called with
user-specified values of certain global parameters. These parameters
are stored in the global structure {\tt ellOptions}, which is kept in the
MATLAB workspace as global variable. This structure is initialized with
default values of parameters upon the first call to almost any function
of the {\it ET}. Before  execution, the {\it ET} routine checks
if global structure {\tt ellOptions} already exists, and if not, it calls
the function {\tt ellipsoids\_init} that performs the initialization.

Here we list the fields of {\tt ellOptions} structure along with their
default values.
\begin{itemize}
\item {\tt version = '1.1.1'} - current version of {\it ET}. A this time
this parameter is harmless in the sense that it is not used by any of the
routines. In the future, however, it may be used for version compatibility.
\item {\tt verbose = 1} - if set to $0$, makes all the calls to {\it ET}
routines silent, and no information except errors is displayed. Otherwise,
it is assumed to be $1$. Currently, there are no other levels of verbosity.
\item {\tt abs\_tol = 1e-9} - absolute tolerance.
\item {\tt rel\_tol = 1e-7} - relative tolerance.
\item {\tt time\_grid = 200} - density of the time grid for the
continuous time reach set computation.
This parameter directly affects the number of ellipsoids to
be stored in the {\tt reach} object.
\item {\tt ode\_solver = 1} - specifies the ODE solver for continuous time
reach set computation: $1$ = 'RK45', $2$ = 'RK23', $3$ = 'Adams'.
\item {\tt norm\_control = 'on'} - switches on and off the norm control
in the ODE solver. When turned on, it slows down the computation, but improves
the accuracy.
\item {\tt ode\_solver\_options = 0} - when set to $0$, calls the ODE solver
without any additional options like norm control. It makes the computation
faster but less accurate. Otherwise, it is assumed to be $1$, and only in this
case the previous option makes a difference.
\item {\tt nlcp\_solver = 0} - specifies which gradient method implementation
to use. If set to $0$, it is {\tt ell\_nlfnlc}, which comes with {\it ET};
if set to $1$, it is {\tt fmincon}, which is part of MATLAB Optimization Toolbox.
\item {\tt plot2d\_grid = 200} - specifies number of points used to plot a
2D ellipsoid. This parameter also affects the quality of 2D reach tube
and reach set plots.
\item {\tt plot3d\_grid = 200} - the number of points used to plot
a 3D ellipsoid is calculated as $\frac{{\tt plot3d\_grid}^2}{2}$.
This parameter also affects the quality of 3D reach set plots.
\item {\tt sdpsettings} - the settings used by YALMIP optimization toolbox.
\end{itemize}
These parameters can be modified by editing {\it ellipsoids/ellipsoids\_init.m}
file. After you finished editing, for changes to take effect, type

{\tt >> clear global ellOptions;}

If you would like to change certain parameters temporarily, without
modifying the {\it ellipsoids\_init.m} file, say, turn the verbosity off,
you should type

{\tt >> global ellOptions;
\newline
>> ellOptions.verbose = 0;}

and proceed with your work. Before you modify {\tt ellOptions} structure
this way however, make sure it is initialized.



\section{ellipsoid}
The main object of the {\it Ellipsoidal Toolbox}, {\tt ellipsoid}, is
very simple. In accordance with definition \ref{ellipsoiddef}, it contains
two fields:
\begin{itemize}
\item {\tt center} - $n$-dimensional vector specifying the center
of the ellipsoid;
\item {\tt shape} - $(n\times n)$-dimensional symmetric positive
semidefinite matrix.
\end{itemize}
These fields cannot be accessed by the user directly. Their values
can be obtained through {\tt ellipsoid/parameters} function, but they cannot
be modified except by some allowed operation with the {\tt ellipsoid}
object. For the list of {\tt ellipsoid} methods, see appendix A.1.



\section{hyperplane}
According to  definition \ref{hyperplanedef}, the hyperplane object
contains two fields:
\begin{itemize}
\item {\tt normal} - $n$-dimensional vector specifying the normal to
the hyperplane ($c$ in \ref{hyperplane});
\item {\tt shift} - the scalar ($\gamma$ in \ref{hyperplane}).
\end{itemize}
These fields cannot be accessed by the user directly. Their values
can be obtained through {\tt hyperplane/parameters} function, but they cannot
be modified other than by some allowed operation with the {\tt hyperplane}
object. For the list of {\tt hyperplane} methods, see appendix A.2.

In some {\it ET} functions, for example, in {\tt ellipsoid/intersection\_ea}
and {\tt ellipsoid/intersection\_ia}, the hyperplane specifies the halfspace.
It is assumed that the halfspace is
\[ \{ x\in{\bf R}^n ~|~ \langle{\tt normal},x\rangle\leq{\tt shift}\}. \]



\section{linsys}
{\it Ellipsoidal Toolbox}  supports both types of linear (affine)
dynamical systems: continuous-time,
\begin{eqnarray*}
\dot{x}(t) & = & A(t)x(t) + B(t)u(t) + G(t)v(t),\\
y(t) & = & C(t)x(t) + D(t)u(t) + w(t);
\end{eqnarray*}
and discrete-time,
\begin{eqnarray*}
x[k+1] & = & A[k]x[k] + B[k]u[k] + G[k]v[k], \\
y[k] & = & C[k]x[k] + D[k]u[k] + w[k].
\end{eqnarray*}
Both can be time-invariant (have constant matrices $A$, $B$, $G$, $C$, $D$)
or time-variant.
\newline
The {\tt linsys} object contains the fields:
\begin{itemize}
\item {\tt A} - $(n\times n)$-dimensional matrix $A$ of type {\tt double}
if constant, or {\tt cell} to symbolically represent $A(t)$ or $A[k]$.
\item {\tt B} - $(n\times m)$-dimensional matrix $B$, {\tt double} if constant,
{\tt cell} if symbolic.
\item {\tt control} - ellipsoidal bounds on control $u$,
either an {\tt ellipsoid}
object of dimension $m$, or structure {\tt U} with fields {\tt U.center} and
{\tt U.shape} to represent the ellipsoid that depends on time. For example,
\newline
{\tt >> U.center = [0; 1];}
\newline
{\tt >> U.shape = \{'4' 'cos(t)'; 'cos(t)' '1'\};}
\newline
defines ellipsoid $\EE(p, P(t))$ with $p=\left[\begin{array}{c}
0\\
1\end{array}\right]$ and $P(t) = \left[\begin{array}{cc}
4 & \cos(t)\\
\cos(t) & 1\end{array}\right]$.
\item {\tt G} - $(n\times d)$-dimensional matrix $G$ of type {\tt double}
if constant, or {\tt cell} if symbolic. Can be empty if the system has
no disturbance or affine term.
\item {\tt disturbance} - ellipsoidal bounds on disturbance $v$, either an
{\tt ellipsoid} object of dimension $d$, or structure {\tt V} with
fields {\tt V.center} and {\tt V.shape} for symbolic representation of
ellipsoid, similar to the {\tt control} field.
This field can be also a single $d$-dimensional vector - constant or symbolic -
to represent an affine term.
\item {\tt C} - $(r\times n)$-dimensional matrix $C$ of type {\tt double}
if constant, or {\tt cell} if symbolic.
\item {\tt D} - $(r\times m)$-dimensional matrix $D$ of type {\tt double}
if constant, or {\tt cell} if symbolic. Can be empty.
\item {\tt noise} - ellipsoidal bounds on the noise $w$, either an
{\tt ellipsoid} object of dimension $r$, or structure {\tt W} with
fields {\tt W.center} and {\tt W.shape} for symbolic representation of
ellipsoid, similar to the {\tt control} and {\tt disturbance} fields.
\newline
This field can be also a single $r$-dimensional vector, constant or symbolic,
to represent an affine term.
\item {\tt lti} - $1$ if the system is time-invariant, $0$ - otherwise.
\item {\tt dt} - $1$ if the system is discrete-time, $0$ - otherwise.
\item {\tt constantbounds} - indicates if the bounds on control, disturbance
and noise are constant.
\end{itemize}
The fields of {\tt linsys} object can be accessed but cannot be modified
directly by the user. The only way to modify these fields is through
{\tt linsys/linsys} constructor.
For the list of {\tt linsys} methods, see appendix A.3.



\section{reach}
The {\tt reach} object represents the reach (or backward reach)
set of an affine system. It contains the fields:
\begin{itemize}
\item {\tt system} - the description of the system, for which the reach set
is computed, in the form of {\tt linsys} object.
\item {\tt t0} - initial time value.
\item {\tt X0} - the set of initial (or terminating, in case of backward
reachability) conditions in the form of {\tt ellipsoid}
object.
\item {\tt initial\_directions} - matrix whose columns represent the values
of direction vector $l$ (see Chapter \ref{ch_reachability}),
for which the ellipsoidal approximations of the reach set are computed.
\item {\tt time\_values} - time interval, for which the reach set is computed,
is split into the number of segments specified by the global
{\tt ellOptions.time\_grid} parameter. This field contains the values
of the time grid. If the last value of this array is less than the value
of {\tt t0}, then the reach set is in fact backward reach set.
\item {\tt center\_values} - matrix whose columns are values of the reach set
center trajectory evaluated at times specified by {\tt time\_values}.
\item {\tt l\_values} - array of directions vectors evaluated at times
specified by {\tt time\_values}.
\item {\tt ea\_values} - array of the shape matrices of the external ellipsoids
evaluated at times specified by {\tt time\_values}.
\item {\tt ia\_values} - array of the shape matrices of the internal ellipsoids
evaluated at times specified by {\tt time\_values}.
\item {\tt projection\_basis} - if the reach set is projected onto the given
orthonormal basis, the columns of this field are the basis vectors, otherwise,
this field is empty.
\item {\tt calc\_data} - this field is empty unless the reach set is computed
with the {\tt save\_all} option set to $1$. This field then contains the
intermediate calculation data, which can be used for the approximation
refinement. For more detail, see description of the function
{\tt refine} in appendix A.4.
\end{itemize}
These fields can be accessed and modified only through {\tt reach} methods.
For the list of {\tt reach} methods, see Appendix A.4.





